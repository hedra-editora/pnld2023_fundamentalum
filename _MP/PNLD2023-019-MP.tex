\documentclass[11pt]{extarticle}
\usepackage{manualdoprofessor}
\usepackage{fichatecnica}
\usepackage{lipsum,media9}
\usepackage[justification=raggedright]{caption}
\usepackage[one]{bncc}
\usepackage[araucaria]{../edlab}
\usepackage{marginnote}
\usepackage{pdfpages}

\newcommand{\AutorLivro}{Tereza Yamashita}
\newcommand{\TituloLivro}{Troca de pele}
%\newcommand{\Tema}{}
\newcommand{\Genero}{Conto; crônica; novela}
%\newcommand{\imagemCapa}{./images/PNLD2022-001-01.jpeg}
\newcommand{\issnppub}{XXX-XX-XXXXX-XX-X}
\newcommand{\issnepub}{XXX-XX-XXXXX-XX-X}
% \newcommand{\fichacatalografica}{PNLD0001-00.png}
\newcommand{\colaborador}{Ana Lancman}

\begin{document}

\title{\TituloLivro}
\author{\AutorLivro}
\def\authornotes{\colaborador}

\date{}
\maketitle

\tableofcontents


\begin{abstract}


O presente manual tem como objetivo oferecer a vocês uma orientação sobre a obra \textit{Troca de pele}. A partir deste manual, vocês poderão incentivar a prática da leitura aos estudantes e proporcionar um conteúdo enriquecedor. Apresentamos aqui sugestões de atividades a serem realizadas antes, durante e após a leitura do livro, com propostas que buscam introduzir os gêneros literários e aprofundar as discussões trazidas pelas obras. Vocês encontrarão também informações sobre o autor, sobre o gênero e sobre os temas trabalhados ao longo do livro. Ao fim do manual, apresentamos sugestões de livros, artigos e sites selecionados para enriquecer a sua experiência de leitura e, consequentemente, a de seus estudantes.

A obra \textit{Troca de pele}, de Tereza Yamashita, se passa na cidade de Xadrez, um lugar em que tudo é preto, branco ou cinza. A personagem principal é Zilá, uma merendeira adorada pelos alunos da escola em que trabalha e que sonha em ter seu próprio filho para criar. Ela é casada com Jamil e os dois não conseguem ter filhos. A vida de Zilá muda completamente quando ela conhece o jovem Éric em uma situação inesperada. O menino é diferente de todos os outros da cidade e os ensina a respeitar as diferenças e celebrar a diversidade.

O livro aborda a questão do combate ao preconceito de maneira envolvente, poética e imaginativa. Sua leitura pode ser uma fonte de debate extremamente potente para a sala de aula, ao tratar de temas como racismo, convivência familiar e direitos da criança e do adolescente. Além disso, por ser um livro mais longo e dividido em vários capítulos, será um incentivo para a prática da leitura. Com o acompanhamento do professor, será possível aprofundar a compreensão do enredo, que possui diversas personagens e reviravoltas. Esperamos que as atividades sugeridas e o material indicado sejam proveitosos em sala de aula! 

\end{abstract}

\section{Sobre o livro} 

\paragraph{O livro} \textit{Troca de pele}, de Tereza Yamashita, é ambientado em uma cidade fictícia chamada Xadrez. É uma cidade em que tudo é preto, branco ou cinza --- não existem cores e, na teoria, também não existe racismo. 

Um esquilo chamado Carambola é o narrador e conta a história de Zilá, que é merendeira de um colégio. Ela é adorada pelas crianças da escola em que trabalha e tem vontade de ter o seu próprio filho. Ela é casada com Seu Jamil e os dois não conseguem ter filhos.
Acompanhamos a jornada de Zilá, desde o momento em que caminha entristecida por não tem um filho para chamar de seu, até um passeio no parque que a leva para uma viagem fantástica. 

Zilá conhece a Guardiã das Cores, da Mata e dos Sonhos, Crevax, que lhe promete um filho. Tudo parece ter sido um sonho, mas ela volta ao parque e cava um buraco à procura da guardiã. No buraco, ela encontra um garoto, Éric, órfão que fugiu de um orfanato. Éric começa a lhe contar a sua história, repleta de violência e abandono. Ele vivia em uma casa de trabalhos forçados e tinha fugido, sem ter para onde ir. É nesse momento que conhece Zilá.

A merendeira e seu marido acolhem o menino e cuidam dele. Aos poucos, Éric passa a fazer parte da família e é adotado oficialmente. Éric é um garoto colorido, ou seja, não é preto ou branco que nem os outros, por isso sofre o preconceito de colegas e moradores da cidade. Os únicos que o defendem são seus novos pais e seu esquilo Carambola.

Por um período, Éric procura formas de mudar de cor e tentar ficar branco ou preto. De repente, ele e seu colega da escola, Paulinho, descobrem uma água mágica no parque. Seu amigo bebe a água, que o torna colorido. Então, eles passam por experiências parecidas de preconceito, o que os aproxima. Essa iniciativa influencia outras crianças, que também querem beber a água mágica. A cidade de Xadrez começa a se colorir, apesar da resistência de diversas pessoas. Em um ato heroico, Éric consegue demonstrar a importância de respeitar o próximo e valorizar a diversidade.

\section{Sobre os autores}

\paragraph{A autora} Tereza Yamashita nasceu em 1965, em São Paulo. Formou-se em Artes Plásticas pela Universidade Mackenzie. Trabalha com design gráfico e especializou-se em capas e projetos gráficos para livros. Publicou contos em diversos jornais e revistas do país, como as revistas \textit{Et Cetera} (Curitiba), \textit{Mininas} (Belo Horizonte), \textit{Ciência Hoje das Crianças} (Rio de Janeiro), \textit{Puçanga}, \textit{Nova Escola}, \textit{Toca} e \textit{Peteca} (São Paulo) e os jornais \textit{Rascunho} (Curitiba) e \textit{Folhinha de S.Paulo} (São Paulo). É ilustradora e escritora e publicou mais de dez livros para crianças e adolescentes, sendo diversos deles em co-autoria com Luiz Bras.

\paragraph{O ilustrador} Marcelo Pitel é ilustrador e designer. Desde 2000, atua no mercado artístico e editorial. Algumas de suas influências artísticas são J. Carlos, Saul Bass e Cássio Loredano. Iniciou a carreira no departamento de arte da revista \textit{Vogue Brasil}. Desde então trabalhou para diversas outras revistas. Também atuou na área educacional ao ilustrar projetos para instituições como \textit{Fundação Unibanco} e \textit{Fundação Carlos Chagas}. Além disso, trabalhou como designer para a \textit{Folha de São Paulo}, \textit{Rico Lins+Studio}, \textit{Editora Globo}, entre outros.

\Image{Foto da autora (Arquivo pessoal)}{PNLD2023-019-02.png}

\section{Sobre o gênero}

\paragraph{O gênero} O gênero deste livro é \textit{conto; crônica; novela}.

\paragraph{Descrição} O que define um gênero narrativo é o fato de, não importa
qual seja sua forma, eles \textit{contarem uma história}.
As especificidades do \textit{como} esta história será contada é que
qualificaram os tipos de gênero narrativo, que podem ser: conto, crônica, novela,
epopeia, romance ou fábula. 

Toda narrativa possui, necessariamente, um narrador, uma personagem, um enredo,
um tempo e um espaço. O narrador, ou narradora, pode ser onisciente, literalmente
\textit{que tudo sabe}, observador ou personagem --- categorias que não são excludentes.
O discurso elaborado por este narrador ou narradora pode ser direto, indireto ou indireto livre 
--- ou seja, ele ou ela pode aparecer mais diretamente ou mais indiretamente; no último caso,
sua voz se mistura à das personagens da história.

O narrador \textbf{não é necessariamente} a voz do autor. É errada a afirmação
de que o autor fala através do narrador de uma história. É bastante comum,
há algum tempo na história literária, sobretudo desde os pré-modernistas, que 
o narrador represente justamente o contrário do que pensa o autor. Neste caso, 
utilizam-se elementos como a \textbf{ironia} para sugerir que o autor \textit{não é confiável}.

Já as personagens variam quanto a sua \textbf{profundidade}. Há personagens planas, ou
personagens-tipo, e personagens redondas, ou complexas. Personagens planas
são facilmente repetíveis pois se amparam em lugares-comuns da cultura, como
o vilão, o herói, a vítima, o palhaço, tudo isso com marcações de gênero e espécie ---
o herói tradicionalmente é um homem, a vítima, uma mulher, e o vilão, uma figura que 
se afasta da humanidade por alguma razão, às vezes sobrenatural. 
Personagens redondos, por outro lado, estão mais próximos das \textit{pessoas reais}.
Uma personagem complexa pode ser, em um dado momento da narrativa, vilã, e em 
outro, heroína. É importante notar como as visões de mundo, um traço cultural e 
portanto relativo, influenciam na caracterização das personagens, planas 
ou redondas, de uma história.

O tempo de uma narrativa pode ser cronológico ou psicológico.
No tempo cronológico, o enredo segue a ordem ``normal'' dos acontecimentos,
aquela marcada pelo relógio e pelo calendário. Os acontecimentos vêm um após o 
outro, e se delimita muito bem \textit{passado}, \textit{presente} e \textit{futuro}.
Já no tempo psicológico, segue-se uma ordem \textit{subjetiva} dos acontecimentos, 
e portanto, \textit{não linear}, já que a influência emocional e psíquica 
da subjetividade afeta a racionalidade do tempo cronológico. 

O espaço, por fim, é o lugar onde se passa a narrativa. Dependendo do caso, 
ele pode funcionar mais como um pano de fundo, sem muita interferência
no enredo, ou mais ativamente, aproximando-se das características das personagens
e influenciando no desenrolar da trama. 

O último aspecto de um gênero narrativo que podemos abordar é sua 
\textit{extensão}. Dentre os elementos que distinguem um subgênero 
de outro, é fundamental o tamanho da história: uma crônica e um conto são \textit{necessariamente}
curtos, ao passo que uma epopeia e um romance são longos. Uma novela
está no ponto intermediário entre um romance e um conto.
Ainda poderíamos falar dos registros de cada subgênero: 
a epopeia é originalmente um subgênero \textit{oral}, versificado, e metrificado,
já o romance é tradicionalmente \textit{escrito} em prosa. 
Desde meados do século \textsc{xviii}, no entanto, o estabelecimento
dos gêneros e subgêneros narrativos torna-se cada vez menos rígido,
com as características cada vez mais fluidas e intercomunicativas.

Como o presente livro contém uma narrativa \textit{curta},
finalizamos com as palavras de Luiza Vilma Pires a respeito do
subgênero:

\begin{quote}
sob o nome de narrativa curta, estão situadas obras que apresentam uma trama 
um pouco mais complexa, que ocorre em diversos espaços e em uma temporalidade 
que pode ser de vários dias, semanas ou meses. Entretanto a função das ilustrações 
continua as mesmas, são complementares à história e contribuem para sua compreensão. 
Os temas relacionam-se a vivência infantis (brincadeiras, passeios, pequenas aventuras), 
a aspectos ligados à interioridade das personagens (busca de identidade, insegurança, 
medos) ou a relações interpessoais (desentendimentos familiares, entre amigos, solidariedade).\footnote{“Narrativas infantis”, de Luiza Vilma Pires Vale. In \textsc{saraiva}, J. A. (Org.) \textit{Literatura e alfabetização: do plano do choro ao plano da ação}. Porto Alegre: Artmed, 2001.} 
\end{quote} 

\section{Atividades}

\subsection{Pré-leitura}

\BNCC{EF05HI01}
%Identificar +Relacionar: ``cultura, povo e o espaço geográfico ocupado'' 
\BNCC{EF05HI04}
%Identificar +Relacionar: ``cidadania, diversidade e direitos humanos'' 
\BNCC{EF05HI05}
%Identificar +Relacionar: ``cidadania, direitos dos povos e das sociedades'' 
\BNCC{EF05GE02}
%Identificar: diferenças e desigualdades étnico-culturais;

\subsubsection{Atividade 1}

\paragraph{Tema} O combate ao preconceito no Brasil.

\paragraph{Conteúdo} Apresentação dos direitos à igualdade no país a partir da Constituição Federal.

\paragraph{Objetivo} Conscientizar os estudantes de seus direitos e deveres e incentivar a defesa da diversidade no ambiente escolar.

\paragraph{Justificativa} Como a obra \textit{Troca de pele} aborda o tema do racismo e do preconceito, é importante introduzir esse debate em sala de aula para que os alunos tenham uma maior compreensão das questões trazidas pelo texto.

\paragraph{Metodologia} Primeiramente, proponha um debate em sala de aula sobre a Constituição Federal de 1988, que pode ser acessada no link \url{http://www.planalto.gov.br/ccivil_03/constituicao/constituicao.htm}.

Sugestões de questões a serem feitas para os alunos:

\begin{itemize}

\item O que são direitos?

\item Como podemos nos informar sobre nossos direitos?

\item Vocês sabem o que é a constituição? 

\item Qual a importância de cada país ter uma constituição?

\end{itemize}

A partir desse debate, apresente brevemente um histórico sobre a Constituição Brasileira de 1988 e sua importância. Mostre a Constituição para os alunos, através da versão \textit{online} ou impressa. Nesse momento, será essencial acessar o Artigo 3, Inciso \textsc{iv} e ler de forma conjunta: 

\begin{quote}Constituem objetivos fundamentais da República Federativa do Brasil:

\textsc{iv} -- promover o bem de todos, sem preconceitos de origem, raça, sexo, cor, idade e quaisquer outras formas de discriminação.\end{quote}

Peça que os alunos comentem se já conheciam esse artigo da constituição e o que compreendem sobre esse trecho. 

Sugestões de questões para fomentar o debate:

\begin{itemize}

\item O que é discriminação?

\item Vocês sabiam que existem diferentes tipos de preconceitos?

\item Como podemos respeitar a diversidade e combater preconceitos?

\end{itemize}

Em seguida, leia de forma conjunta o Artigo 5, inciso \textsc{xlii}:

\begin{quote}Art. 5º Todos são iguais perante a lei, sem distinção de qualquer natureza, garantindo-se aos brasileiros e aos estrangeiros residentes no País a inviolabilidade do direito à vida, à liberdade, à igualdade, à segurança e à propriedade, nos termos seguintes:

\textsc{xlii} -- prática do racismo constitui crime inafiançável e imprescritível, sujeito à pena de reclusão, nos termos da lei.\end{quote}

Nesse momento, o tema da questão racial deve ser aprofundado. Trabalhe o conceito de \textbf{racismo} com os estudantes e qual a importância de ter conhecimento de que essa prática é crime segundo a constituição. Aborde o tema do combate ao racismo e pergunte aos alunos quais as formas de prevenir que esse tipo de crime aconteça.

\paragraph{Tempo estimado} Duas aulas de 50 minutos.

\subsubsection{Atividade 2}

\BNCC{EF04GE01}
%Identificar: histórias familiares da comunidade com elementos indígenas, afro-brasileiras;
\BNCC{EF04GE04}
%Identificar: analisar a interdependência do campo e da cidade;
\BNCC{EF04GE06}
%Identificar: erritórios étnico-culturais existentes no Brasil;

\paragraph{Tema} As comunidades quilombolas no Brasil.

\paragraph{Conteúdo} Apresentação de documentário sobre crianças que vivem em comunidades quilombolas e pesquisa sobre as práticas culturais do quilombo.

\paragraph{Objetivo} Aprofundar o tema da questão racial trazida pela atividade 1 de pré-leitura, com um exemplo prático de defesa da diversidade no país.

\paragraph{Justificativa} Muitas pessoas não conhecem a origem dos quilombos e a importância de defender seus territórios atualmente. Através dessa atividade, será possível escutar crianças que vivem nessas comunidades e entrar em contato com sua rotina. 

\paragraph{Metodologia} Sugere-se que seja exibido em sala de aula o documentário \textit{Disque Quilombola}, de 2012. É um documentário curto, de 12 minutos, que pode ser visualizado gratuitamente no \href{https://youtu.be/GStv-f_bcfU}{Youtube}. No vídeo, filmado em diversos locais do estado do Espírito Santo, crianças das comunidades quilombolas São Cristóvão e Angelim e do morro São Benedito conversam sobre suas experiências e as peculiaridades dos lugares em que vivem. Através do vídeo podemos conhecer mais sobre o cotidiano dos quilombos e sua importância para a diversidade cultural brasileira. 

\Image{Comemoração no Parque Memorial Quilombo dos Palmares, no dia da Consciência Negra (Ministério da Cultura; CC-BY-2.0)}{PNLD2023-019-08.png}

Após assistirem ao documentário, acesse com os alunos o \href{https://cpisp.org.br/mapa-dos-quilombos-geografia-da-resistencia/}{Mapa dos Quilombos}, desenvolvido pela Comissão Pró-Índio, uma organização que atua junto das populações indígenas e quilombolas para garantir seus direitos. Pode ser interessante que primeiramente os estudantes naveguem livremente pelo mapa, observando a distribuição das comunidades quilombolas nos estados brasileiros.

\Image{Quilombo do Vão de Almas, Cavalcante-GO. Dona Domingas Francisco Maia prepara a refeição para sua neta Gabryela Fernandes Pereira. (Ministério do Desenvolvimento Social; CC-BY-SA-2.0)}{PNLD2023-019-09.png}

Por fim, peça que os estudantes façam uma breve pesquisa sobre a história, costumes e práticas culturais do quilombo mais próximo do lugar em que vivem. A atividade poderá ser feita junto de professores de História e Geografia. Caso vivam em um quilombo, podem pesquisar sobre uma comunidade próxima e quais as semelhanças e diferenças com a sua própria. A observação conjunta do mapa será complementada pela lista de comunidades quilombolas disponível no \href{https://cpisp.org.br/direitosquilombolas/observatorio-terras-quilombolas/}{Observatório das Terras Quilombolas}, também desenvolvido pela Comissão Pró-Índio. 

\paragraph{Tempo estimado} Duas aulas de 50 minutos.

\subsection{Leitura}

\BNCC{EF35LP06}
%Identificar: relações entre partes de um texto; Compreensão de texto; ; 
\BNCC{EF35LP03}
%Identificar: a ideia central do texto; Compreensão de texto; 
\BNCC{EF35LP04}
%Identificar: informações implícitas; Compreensão de texto; 
\BNCC{EF35LP26}
%Ler: +sozinho; narrativa com personagem; Compreensão de texto

\paragraph{Tema} O enredo de \textit{Troca de pele}.

\paragraph{Conteúdo} Roteiro para leitura acompanhada e debate sobre as principais questões trazidas pela narrativa.

\paragraph{Objetivo} Incentivar os estudantes a identificar os diferentes momentos da narrativa e ampliar seu entendimento sobre as questões trazidas pela obra.

\paragraph{Justificativa} Por ser um livro mais longo, é interessante que o professor apresente um roteiro de leitura de \textit{Troca de pele} para os estudantes e dedique uma série de aulas para o acompanhamento deste roteiro. Ao realizar uma leitura acompanhada, será possível trabalhar a compreensão do texto de forma detalhada e discutir com profundidade os temas que aparecem ao longo da narrativa, como o preconceito, a desigualdade social e as crianças em situações de vulnerabilidade social. 

\paragraph{Metodologia} Como a narrativa de \textit{Troca de pele} é dividida em vários capítulos, sugere-se que sejam trabalhados os capítulos em blocos, a partir dos seguintes tópicos:

\subsubsection{O início da obra}

No primeiro capítulo, é apresentada aos leitores a personagem Zilá, uma merendeira escolar que deseja ter um filho. Também entramos em contato com os elementos centrais da cidade de Xadrez:

\begin{quote}A cidade de Xadrez, como a chamavam, era preta, branca e cinza. Lá não existiam as cores (o verde, o amarelo, o vermelho, o azul, nenhuma cor). Lá era tudo no preto, no branco e no cinza. Tudo monocromático: assim era a cidade de Xadrez com seus habitantes pretos, brancos e em tons de cinza. Mas apesar da pele dos habitantes ser de tons diferentes, na cidade não havia racismo. \footnote{Página 11 e 12 de \textit{Troca de pele}}\end{quote}

Pergunte aos alunos o que compreenderam sobre o nome da cidade. Peça que identifiquem a metáfora contida. O que o autor quis dizer com esse nome? Poderá ser usado um tabuleiro de xadrez para apresentar o básico sobre o jogo e para que possam ter uma visualização conjunta das cores do tabuleiro. 

\subsubsection{O protagonismo de Dona Zilá}

Nos capítulos \textsc{ii} e \textsc{iii}, conhecemos mais sobre a vida de Dona Zilá. Nesse momento, poderão ser recuperados elementos discutidos na atividade de pré-leitura. A intenção da autora, ao tratar da questão racial, pode auxiliar no combate à discriminação? Qual a importância da personagem principal ser uma mulher negra? Qual a classe social de Dona Zilá e Seu Jamil na cidade de Xadrez?

\Image{Ilustração do livro, página 9}{PNLD2023-019-04.png}

\subsubsection{A vulnerabilidade social de Éric}

Nos capítulos \textsc{iv}, \textsc{v} e \textsc{vi}, Zilá conhece Éric em uma situação improvável. O garoto rapidamente começa a contar sua triste história para a merendeira. Sugere-se que o seguinte trecho seja lido coletivamente em sala de aula:

\begin{quote}
— Você sabia que eu moro na rua? Aqui e ali? Não sei quem são os meus pais. Ah, sabia que eu acabei de fugir de um orfanato?

— Como assim, você fugiu? — perguntou Zilá, sem entender nada.

— É, no orfanato era tudo muito chato. Eu vivia trabalhando feito um burro de carga. A diretora de lá não nos deixava ir à escola, de jeito nenhum. Ela dizia que trabalhar era o certo, que o nosso futuro estava no trabalho duro. Certo só na cabeça desparafusada dela! — disse Éric com olhar furioso. 

— Nossa! Você nunca foi à escola? Quantos anos você tem? — perguntou Zilá, muito triste com toda essa história.

— Eu acho que já tenho uns nove ou dez anos. Não sei direito, não. Foi o que ouvi a Marli falar.

— Quem é a Marli? — Zilá perguntou.

— A diretora do orfanato. Ela era meio louca, às vezes ameaçava e deixava a gente sem comer e dizia que era o nosso castigo.
\footnote{Página 27 e 28 de \textit{Troca de pele}}.

\end{quote}

Este trecho é extremamente importante para a compreensão da narrativa e também para abordar o tema da desigualdade social. Éric é um menino que passa por situações de extrema violência e abandono. Este é um tema delicado, mas que deve ser discutido com os alunos. É possível dedicar uma aula ao assunto, com o auxílio dos professores de História e Geografia. Use o caso de Éric para expor os distintos tipos de abuso que uma criança pode sofrer no Brasil. 

\Image{Ilustração do livro, página 39}{PNLD2023-019-05.png}

Deve ser abordada a questão do trabalho infantil. Sugere-se utilizar o comunicado publicado pela \textsc{unicef} com o título \textit{Trabalho infantil aumenta pela primeira vez em duas décadas e atinge um total de 160 milhões de crianças e adolescentes no mundo}. Pode ser acessado no \href{https://www.unicef.org/brazil/comunicados-de-imprensa/trabalho-infantil-aumenta-pela-primeira-vez-em-duas-decadas-e-atinge-um-total-de-160-milhoes-de-criancas-e-adolescentes-no-mundo}{site da \textsc{unicef}}. A partir do comunicado, é interessante explicar a proibição do trabalho infantil para menores de 14 anos e as especificidades do contexto brasileiro.

\subsubsection{O acolhimento familiar}

Nos capítulos \textsc{vii}, \textsc{viii} e \textsc{ix}, acompanhamos Éric doente na casa de Dona Zilá e Seu Jamil. Zilá cuida de Éric como se fosse seu filho e se apega ao garoto. Nessa parte, é emocionante notar o acolhimento da criança em uma nova família. Éric tinha sido abandonado e estava em um local extremamente violento. Em sua nova casa, recebe alimento, cuidado e afeto. 

Discuta com os estudantes sobre a importância da família para a garantia dos direitos básicos das crianças. A situação de Éric pode ser um exemplo do acolhimento em uma família que não era a sua originalmente. Lembre-se se considerar a abrangência do que uma família pode significar no contexto específico de cada aluno. 

Peça que os alunos identifiquem o que caracteriza Zilá, Jamil e Éric como família. Aborde o tema da família funcional e a prática da escuta e diálogo no lugar do autoritarismo e violência.

\subsubsection{O preconceito racial na cidade de Xadrez}

Entre os capítulos \textsc{x} e \textsc{xv}, Zilá e Jamil levam Éric para conhecer a cidade de Xadrez e o apresentam aos moradores da cidade. O garoto tenta se aproximar de outras crianças e também passeia pela cidade. Em diversas cenas, fica evidente o preconceito sofrido por Eric.

Peça que os alunos identifiquem nesses capítulos os episódios de exclusão e preconceito. Os estudantes podem comentar sobre esses episódios e tentar propor soluções práticas que poderiam ter evitado a violência em cada caso. Pode ser interessante recuperar conceitos discutidos na atividade de pré-leitura, como o combate à discriminação e a criminalização do racismo.

\Image{Ilustração do livro, página 66}{PNLD2023-019-06.png}

\subsubsection{A água mágica}

Entre os capítulos \textsc{xvi} e \textsc{xvii}, acompanhamos as aventuras de Éric, Carambola e Paulinho em busca da água mágica. Peça que os estudantes identifiquem diferentes significados para a metáfora presente na água. O que esses personagens desejam com essa água? Quais as consequências desta água para a cidade de Xadrez?

\subsubsection{Conclusão}

A partir do capítulo \textsc{xviii}, é encaminhado um encerramento para a narrativa. Finalizada a leitura, proponha um debate com os alunos sobre a solução proposta ao final do livro. Qual a importância do gesto de Éric ao doar sangue para a criança que está à beira da morte? Por que Éric afirma que \textit{(...) nesta cidade as coisas vão mesmo mudar. Para melhor.}\footnote{Página 118 de \textit{Troca de pele}}? Explore também o título da obra para que os estudantes possam ter uma compreensão geral do livro. Qual a metáfora contida na ``troca de pele''? 

Por fim, peça que os estudantes desenhem a cidade de Xadrez antes e depois da aparição de Éric. Explore o tema da cidade colorida como símbolo de defesa da diversidade.

\paragraph{Tempo estimado} Um bimestre.

\subsection{Pós-leitura}

\BNCC{EF04GE03}
%Identificar: poder público municipal e participação da população; Conselhos;
\BNCC{EF35LP15}
%Falar: Debate; opinar e defender ponto de vista; Vida cotidiana; 
\BNCC{EF05LP18}
%Produzir: vídeo para vlogs; ``influencer'';
\BNCC{EF05LP19}
%Falar: discussão; argumentar oralmente; ``roda''; Opinião; Internet; 

\paragraph{Tema} O Estatuto da Criança e do Adolescente.

\paragraph{Conteúdo} Leitura conjunta de artigos do Estatuto da Criança e do Adolescente (\textsc{eca}) e realização em grupo de vídeo sobre sua origem e seu conteúdo.

\paragraph{Objetivo} Aproximar os estudantes do conhecimento acerca de seus direitos e deveres.

\paragraph{Justificativa} Em \textit{Troca de pele}, são abordados diversos temas cruciais para os direitos das crianças e adolescentes, como o direito à educação, saúde e moradia, assim como a proteção familiar. Nessa atividade, será possível aprofundar essas questões com um exemplo prático do contexto brasileiro.

\paragraph{Metodologia} É importante apresentar um contexto histórico do \textsc{eca} e sua importância no Brasil. Leia artigos selecionados do Estatuto, relacionados com a trajetória de Éric e Dona Zilá. O estatuto pode ser acessado gratuitamente no \href{https://www.gov.br/mdh/pt-br/assuntos/noticias/2021/julho/trinta-e-um-anos-do-estatuto-da-crianca-e-do-adolescente-confira-as-novas-acoes-para-fortalecer-o-eca/ECA2021_Digital.pdf}{site do Governo Brasileiro}. 

\Image{É essencial que as crianças e adolescentes conheçam seus direitos. (Tony Winston/Agência Brasília; CC BY 2.0)}{PNLD2023-019-10.png}

Sugere-se a leitura conjunta dos seguintes artigos:

\begin{quote}É dever de todos velar pela dignidade da criança e do adolescente, pondo-os a salvo de qualquer tratamento desumano,
violento, aterrorizante, vexatório ou constrangedor.\footnote{Estatuto da Criança e do Adolescente. Página 20, Capítulo II, Art. 18.}\end{quote} 

\begin{quote}A criança e o adolescente têm o direito de ser educados e cuidados sem o uso de castigo físico ou de tratamento cruel ou degradante, como formas de correção, disciplina, educação ou qualquer outro pretexto, pelos pais, pelos integrantes da família
ampliada, pelos responsáveis, pelos agentes públicos executores de medidas socioeducativas ou por qualquer pessoa encarregada
de cuidar deles, tratá-los, educá-los ou protegê-los.\footnote{Estatuto da Criança e do Adolescente. Página 20, Capítulo II, Art. 18-A.}\end{quote} 

\begin{quote}É direito da criança e do adolescente ser criado e educado no seio de sua família e, excepcionalmente, em família substituta, assegurada a convivência familiar e comunitária, em ambiente que garanta seu desenvolvimento integral.\footnote{Estatuto da Criança e do Adolescente. Página 21, Capítulo III, Art. 18-A.}\end{quote} 

\begin{quote}É proibido qualquer trabalho a menores de quatorze anos de idade, salvo na condição de aprendiz.\footnote{Estatuto da Criança e do Adolescente. Página 45, Capítulo V, Art. 60.}\end{quote} 

Converse com os estudantes sobre estes artigos, auxiliando a sua compreensão. É importante ser convidativo e abrir espaço para os alunos exporem suas dúvidas. O cuidado é essencial nessa atividade, uma vez que é um tema sensível abordado em uma linguagem que pode ser desafiadora.

Na aula seguinte, deverá ser produzido um vídeo sobre o Estatuto da Criança e do Adolescente. Peça que os alunos se dividam em grupos de até cinco pessoas. Eles deverão montar um breve roteiro que apresente o que é o \textsc{eca}, quando surgiu e qual sua importância no Brasil. Cada grupo deverá escolher um artigo do Estatuto e explicar o seu significado. Acompanhe os grupos e auxilie na produção audiovisual, que poderá ser feita com celulares ou o material que estiver disponível na escola. Caso não tenham como filmar, sugere-se que sejam feitas breves apresentações teatrais, que poderão ser apresentadas para o resto da turma.

\paragraph{Tempo estimado} Quatro aulas de 50 minutos.

\section{Sugestão de referências complementares}

\subsection{Filmes}

\begin{itemize}

\item \textit{Ana}. Dirigido por Vitória Felipe dos Santos, 2017.

Curta-metragem que conta a história de Ana, uma menina negra que não se reconhece como negra. Jeannette é uma professora refugiada com dificuldades de adaptação no Brasil. Vítimas de racismo, elas descobrem juntas um modo de transformar a si mesmas. Disponível no \href{https://youtu.be/MO1f8n3gMG8}{Youtube}.

\item \textit{Disque quilombola}. Dirigido por David Reeks, 2012.

Crianças do Espírito Santo conversam de um jeito divertido sobre como é a vida em uma comunidade quilombola e em um morro na cidade de Vitória. Por meio de uma genuína brincadeira infantil, os dois grupos falam de suas raízes e desvelam o quanto a infância tem mais semelhanças do que diferenças. Disponível no \href{https://youtu.be/GStv-f_bcfU}{Youtube}.

\item \textit{O menino e o mundo}. Dirigido por Alê Abreu, 2013.

O filme é uma animação emocionante, que conta a história de Cuca, um menino que vive em um mundo distante, numa pequena aldeia no interior de seu mítico país. Sofrendo com a falta do pai, que parte em busca de trabalho na desconhecida capital, Cuca deixa sua aldeia e sai mundo afora à procura dele. Durante sua jornada, Cuca descobre uma sociedade marcada pela pobreza, exploração de trabalhadores e falta de perspectivas.

\end{itemize}

\subsection{Links}

\begin{itemize}

\item \textit{Comissão Pró-Índio}

No site da ONG, é possível conhecer os trabalhos realizados junto de comunidades indígenas e quilombolas, assim como um extenso banco de dados sobre os processos de homologação de suas terras. Acesse em \url{https://cpisp.org.br/}.

\item \textit{Comunidade Quilombola Angelim I}

A Comunidade Quilombola Angelim I, localizada em Itaúnas (Espírito Santo) é uma das comunidades participantes do documentário \textit{Disque quilombola}. Em seu site, é possível conhecer as práticas da comunidade e ver fotos do espaço. Acesse em \url{https://itaunas.org.br/quilombola-angelim}.

\item \textit{Instituto Socioambiental}

O Instituto Socioambiental é uma ONG que trabalha junto de povos originários e monitora áreas protegidas, pelo viés do direito socioambiental. Conheça suas iniciativas em \url{https://www.socioambiental.org/pt-br}.

\end{itemize}

\subsection{Museus}

\begin{itemize}

\item \textit{Museu Afro-Brasil}

O museu, localizado no Parque Ibirapuera, em São Paulo, possui um acervo de mais de 6 mil obras que representam os universos culturais africanos e afro-brasileiros.

Endereço: Av. Pedro Álvares Cabral — Parque Ibirapuera, Portão 10, São Paulo/SP.

\item \textit{Museu Afro-Brasileiro}

O \textsc{mafro} - Museu Afro-brasileiro da Universidade Federal da Bahia está sediado em Salvador e possui um acervo de mais de 1100 peças de cultura material africana e afro-brasileira.

Endereço: Largo do Terreiro de Jesus s/n, Prédio da Faculdade de Medicina da Bahia, Centro Histórico de Salvador, Bahia.

\end{itemize}

\section{Bibliografia comentada}

\begin{itemize}

\item \textsc{almeida}, Silvio. \textit{Racismo estrutural}. São Paulo: Jandaíra, 2019.

Livro essencial de Silvio Almeida, advogado, filósofo e professor universitário, em que define os príncipios de raça e racismo pelo viés estrutural. O conceito de racismo é abordado de forma relacional ao Estado, aos direitos humanos, à história e economia brasileiras.

\item \textsc{evaristo}, Conceição. \textit{Olhos d'água}. São Paulo: Pallas, 2014.

Obra impactante de Conceição Evaristo, uma das autoras brasileiras mais relevantes da atualidade. Os contos deste livro expressam a força da mulher negra, sua conexão com a ancestralidade e as vivências de violência comuns a tantas trajetórias.

\item \textsc{de jesus}, Carolina Maria. \textit{Quarto de despejo — Diário de uma favelada}. São Paulo: Ática, 1992.

O diário de Carolina Maria de Jesus é um documento histórico, ao retratar a pobreza na cidade de São Paulo dos anos 1960. A autora escreveu uma obra imprescindível e apesar das adversidades, sustentou sua narrativa e representou uma quebra de paradigma na literatura brasileira.

\item \textsc{ribeiro}, Djamila. \textit{Pequeno manual antirracista}. São Paulo: Companhia das Letras, 2019. 

Ensaio da filósofa, professora e militante feminista e antirracista brasileira Djamila Ribeiro. Na obra, a filósofa defende que o racismo é um desafio para toda a sociedade brasileira devido ao passado escravocrata que o país possui. 

\item \textsc{tenório}, Jeferson. \textit{O avesso da pele}. São Paulo: Companhia das Letras, 2020.

Livro que ganhou o Prêmio Jabuti na categoria Romance Literário em 2021, a obra de Tenório é uma lição sobre racismo através de um viés sensível e envolvente. 

\end{itemize}

\end{document}