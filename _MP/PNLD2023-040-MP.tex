% \Image{Capa do livro (; )}{PNLD2022-001-01.png}
% \Image{Ilustração do livro (Acorde/Manuella Silveira; Acorde)}{PNLD2022-001-04.png}
% \Image{Ilustração do livro (Acorde/Manuella Silveira; Acorde)}{PNLD2022-001-05.png}
% \Image{Ilustração do livro (Acorde/Manuella Silveira; Acorde)}{PNLD2022-001-06.png}

\documentclass[11pt]{extarticle}
\usepackage{manualdoprofessor}
\usepackage{fichatecnica}
\usepackage{lipsum,media9}
\usepackage[justification=raggedright]{caption}
\usepackage[one]{bncc}
\usepackage[nmenosum]{../edlab}
\usepackage{marginnote}
\usepackage{pdfpages}

\newcommand{\AutorLivro}{Maria do Rosário Lustosa da Cruz}
\newcommand{\TituloLivro}{O reisado: festejando o nascimento de Jesus}
\newcommand{\Tema}{Diversão e aventura}
\newcommand{\Genero}{Cordel}
%\newcommand{\imagemCapa}{./images/PNLD2022-001-01.jpeg}
\newcommand{\issnppub}{XXX-XX-XXXXX-XX-X}
\newcommand{\issnepub}{XXX-XX-XXXXX-XX-X}
% \newcommand{\fichacatalografica}{PNLD0001-00.png}
\newcommand{\colaborador}{Paulo Pompermaier}

\begin{document}

\title{\TituloLivro}
\author{\AutorLivro}
\def\authornotes{\colaborador}

\date{}
\maketitle

\tableofcontents


\begin{abstract}

Este material tem a intenção de contribuir para que você desenvolva um trabalho aprofundado com a obra \textit{O reisado: festejando o nascimento de Jesus} em sala de aula.
Você encontrará informações sobre o autor, sobre o gênero e também 
algumas propostas de trabalho para a sala de aula que você poderá explorar livremente, 
da forma que considerar mais apropriada para os seus estudantes.

A autora do livro, Maria do Rosário Lustosa da Cruz, envolveu-se com o universo artístico e com a cultura popular brasileira desde pequena. Nascida em Juazeiro do Norte, dedica-se desde os anos 2000 à pesquisa e divulgação da arte popular brasileira. É membro de diversas academias e instituições culturais dedicadas ao universo do cordel, publicando seus próprios cordéis e coletâneas para divulgação dessa arte milenar.

Em seu livro \textit{O reisado: festejando o nascimento de Jesus}, ela utiliza-se do tradicional formato do cordel para narrar o nascimento de Jesus e o surgimento, a partir desse acontecimento, das festança do reisado. Cheio de ilustrações e descrições que deixam ver uma festa do reisado, o livro permite uma infinidade de trabalhos com os estudantes: explorar a narrativa religiosa, a relação com a cultura cristã e árabe, os elementos da festa popular, as características do cordel etc. 

\SideImage{Típica capa de um cordel feita a partir da gravura em madeira, a xilogravura (CC-BY-SA-4.0)}{PNLD2023-004-06.jpg}

Ao longo do manual, todos esses aspectos serão explorados e relacionados a sugestões de atividades. Com isso, objetiva-se oferecer algumas ideias e inspirações para um trabalho que pode ser desenvolvido tanto a curto, quanto a médio e longo prazo. Sinta-se à vontade para personalizar a aula e torná-la sua, aplicando seus conhecimentos, sua 
personalidade e aproveite para fortalecer seu vínculo com a turma.

Boa aula!

\end{abstract}

\section{Sobre o livro}
Por meio da poesia de cordel, o livro narra o nascimento do menino Jesus, sua relação com a festança do reisado nascida no Egito e suas formas de realização no Brasil.

A história pode ser dividida em duas partes: na primeira, ouvimos a história de Maria e José, que vão para Belém, na atual Palestina, para participar do censo populacional. Lá vem à luz Jesus de Nazaré e, por meio da poesia, ouvimos a história dos três reis magos que vão visitá-lo e presenteá-lo.

Na segunda parte, a autora começa a falar dos reisados, que são festas para celebrar o nascimento de Jesus. Ela conta como foi uma celebração surgida no Egito que, com a colonização, chegou ao Brasil e adquiriu tons e características próprias da nossa nação.

Ela descreve então os diversos aspectos que constituem essa celebração popular: suas três modalidades (reisado de congo, de baile e de couro), a constituição do ciclo de reis, a dinâmica das danças, sua realização tradicional no Ceará, as personagens, vestimentas, as narrativas, o jogo de espadas, sua relação com a peça do Divino Espírito Santo, os alimentos das festanças etc.

A cada dupla de páginas há ilustrações que exploram esses aspectos abordados na narrativa. Além de dar cor e forma aos elementos narrados, as imagens ajudam a visualizar as dinâmicas do reisado, explorando as posições das danças, os momentos decisivos da narrativa (como a batalha dos reis pela rainha) e as vestimentas que constituem tão tradicional festa popular.


\section{Sobre a autora}


%532 caracteres
\paragraph{A autora}
A poeta e cordelista Maria do Rosário Lustosa da Cruz nasceu em Juazeiro do Norte, Ceará, em 2 de novembro de 1953. Desde pequena se envolveu com o meio artístico de sua região, participando de programas de rádio, cinema e peças de teatro. Cantava no programa de calouro da Rádio Iracema, participou do filme \textit{Padre Cícero: os milagres de Juazeiro} (1976) e foi integrante do Grupo Teatral Willian Shakespeare, com o qual participou da 1ª Bienal de Artes de Juazeiro do Norte com a peça \textit{Soraia, Posto 2}.

Sua aventura na escrita de cordéis, no entanto, se deu um pouco mais tarde. Aos 39 anos começou a escrever poesia de cordel e, oito anos depois, em 2000, já se dedicava integralmente à arte popular. Dessa época data sua participação no Projeto Letras Vivas da Lira Nordestina (2005), que tinha como objetivo resgatar a memória e a produção da literatura de cordel.

Em 2011, junto ao professor Renato Dantas, fez parte do projeto de publicar 100 cordéis para celebrar os 100 anos de Juazeiro do Norte, o que deu origem à obra \textit{100 Anos de Juazeiro Registrados no Cordel}. Entre 2014 e 2015 publicou mais dois livros dedicados ao tema: \textit{Tempo de Saudade no embalo do cordel} e \textit{Crato na Literatura de Cordel}.

Desde 2001 a autora é membro do Instituto Cultural do Vale Caririense (\textsc{icvc}). Dois anos depois ingressou na Academia dos Cordelistas da Cidade do Crato (\textsc{acc}) e, em 2014, no Instituto Cultural do Cariri (\textsc{icc}) e na Academia de Xilógrafos e Cordelistas do Cariri. A partir de 2011, passou a ocupar também uma posição na Associação dos Poetas de Barbalha (\textsc{apb}).

Rosário Lustosa formou-se em Pedagogia em 2003, no ano seguinte fez uma pós-graduação em Língua Portuguesa e Arte Educação, com pesquisa em Literatura de Cordel, e em 2014 cursou Assistência Social.

\paragraph{O ilustrador}

\Image{Disposição dos folhetos em uma típica feira de cordéis (Diego Dacal; CC-BY-SA-4.0)}{PNLD2023-004-07.jpg}

\section{Sobre o gênero}

%55 caracteres
\paragraph{O gênero} O gênero deste livro é \textit{poesia de cordel}. 


Para uma primeira definição de poesia enquanto gênero literário, poder"-se"-ia recorrer à definição do professor Domingos Paschoal Cegalla, para quem ``poesia é a linguagem subjetiva, carregada de emoção e sentimento, com ritmo melódico constante, bela e indefinível como o mundo interior do poeta visa a um efeito estético''.\footnote{\textsc{cegalla}, Domingos Paschoal. \textit{Novíssima Gramática da Língua Portuguesa}. São Paulo: Companhia Editora Nacional, 2008, p.\,640}

Aprofundando um pouco essa definição, o crítico Antonio Candido expande a definição de poesia ao diferenciá"-la do verso.
Para o crítico, a poesia enquanto ato criador do artista independe da forma métrica do verso, que passa a ser apenas um dos registros possíveis do poético:

\begin{quote}
A poesia não se confunde necessariamente com o verso, muito menos com o verso metrificado. Pode haver poesia em prosa e poesia em verso livre. [\ldots]
Pode ser feita em verso muita coisa que não é poesia.\footnote{\textsc{candido}, Antonio. \textit{O estudo analítico do poema}. São Paulo: Terceira leitura, 1993, p.\,13--14.}
\end{quote}

Delineada, de forma breve e geral, a forma poética, pode"-se pensar agora em seus três gêneros básicos: lírico, épico e dramático.
Para o crítico Anatol Rosenfeld, a lírica é o gênero mais subjetivo, no qual uma voz central exprime um estado de alma traduzido em orações poéticas.
Seria a expressão de emoções e experiências vividas, ``a plasmação imediata das vivências intensas de um Eu no encontro com o mundo, sem que se interponham eventos distendidos no tempo (como na Épica e na Dramática)''.\footnote{\textsc{rosenfeld}, Anatol. \textit{O teatro épico}. São Paulo: Perspectiva, 2006, p.\,22.}

Devido a essa característica central da lírica, a expressão de um estado emocional, Rosenfeld considera que o eu"-lírico, nesse gênero, não se delineia enquanto um personagem. Embora possa evocar personagens e narrar acontecimentos, a lírica entendida enquanto gênero puro afasta"-se sobremaneira da apreensão objetiva do mundo, que não existe independente da subjetividade intensa que o apreende e exprime. Assim, na lírica prevalece a fusão entre o sujeito e o objeto, que serve mais a realçar os estados profundos de alma do poeta.
Sobre os aspectos formais do gênero, Rosenfeld nota:

\begin{quote}
À intensidade expressiva, à concentração e ao caráter ``imediato'' do poema lírico, associa"-se, como traço estilístico importante, o uso do ritmo e da musicalidade das palavras e dos versos. De tal modo se realça o valor da aura conotativa do verbo que este muitas vezes chega a ter uma função mais sonora que lógico"-denotativa. A isso se liga a preponderância da voz do presente que indica a ausência de distância, geralmente associada ao pretérito. Este caráter do imediato, que se manifesta na voz do presente, não é, porém, o de uma atualidade que se processa e distende através do tempo (como na Dramática) mas de um momento ``eterno''.\footnote{Ibidem, p.\,23.}
\end{quote}

No caso específico da poesia de cordel, dizem os especialistas, é uma poesia escrita para
ser lida, enquanto o repente ou o desafio é a poesia feita oralmente, que mais tarde pode
ser registrada por escrito. Essa divisão é muito esquemática. Por exemplo, o
cordel, mesmo sendo escrito e impresso para ser lido, costumava ser lido em
voz alta e desfrutado por outros ouvintes além do leitor. A poesia popular,
praticada principalmente no Nordeste do Brasil, tem muita influência da
linguagem oral, aproveita muito da língua coloquial praticada nas ruas e na
comunicação cotidiana. 

Naturalmente, portanto, pode"-se considerar a poesia narrativa do cordel uma
forma de poesia mais compartilhada e desfrutada coletivamente, o que dá também
uma grande ressonância social. Muitos dos temas do cordel são originários das
tradições populares e eruditas da Europa medieval e moderna. Outros temas são
retirados de tradições orientais, como neste \textit{História de
Aladim e a lâmpada maravilhosa}. O personagem Aladim pertence ao \textit{Livro das mil
e uma noites}, um dos famosos conjuntos de histórias de todos os tempos. Também
encontramos temas retirados das novelas de cavalaria medievais e das narrativas
bíblicas. Ao lado destes temas mais literários, encontram"-se os temas locais,
quase sempre narrados na forma de crônicas de coisas realmente acontecidas,
como em outro famoso cordel de Patativa intitulado  “Padre Henrique e o dragão da maldade”, que fala de um causo verídico e contemporâneo ao poeta. Também há, entre sua profícua produção literária, as histórias
fantásticas, que se valem das tradições semirreligiosas, ligadas à experiência
com o mundo espiritual. 

Os grandes poemas de cordel são perfeitamente metrificados e rimados. A métrica
e a rima são recursos que favorecem a memorização e tradicionalmente se costuma
dizer que são resquícios de uma cultura oral, na qual toda a tradição e
sabedoria são sabidas de cor.  


\paragraph{O sertão geográfico e cultural}

O sertão tem mitos culturais próprios. Contemporaneamente, o sertão evoca
principalmente o sofrimento resignado daqueles que padecem a falta de chuva e
de boas safras na lavoura. Evoca a experiência histórica de uma região
empobrecida, embora tenha sido geradora de riquezas, como o cacau e cana de
açúcar, ambos bens muito valiosos. 

O sertão formou também o seu imaginário por meio de grandes personalidades e
uma pujante expressão artística. Além do cordel, o sertão viu nascer ritmos tão
importantes quanto o forró e o baião. Produziu artistas tão expressivos quanto
Luiz Gonzaga, grande cantor da vida do sertanejo em canções como “Asa branca”.
Um escultor como Mestre Vitalino criou toda uma tradição de representação da
vida e dos hábitos sertanejos em miniaturas de barro. A gravura popular, que
sempre acompanha os folhetos de cordel, também floresceu em diversos pontos e
ficou mais famosa em Juazeiro do Norte, no Ceará, e em Caruaru, no estado de
Pernambuco. 

Dentre os grande mitos do sertão, está certamente o do cangaço com seu líder
histórico, mas também mítico, Virgulino Ferreira, o Lampião. Até hoje as
opiniões se dividem: para alguns foi uma grande homem, para outros um bandido
impiedoso. 

Uma figura muito presente na cultura nordestina é o Padre Cícero Romão,
considerado beato pela Igreja Católica. Consta que teria feito milagres e
dedicado sua vida aos pobres. 

\paragraph{Variação linguística}

A linguística moderna usa o termo “idioleto” para marcar grupos distintos no
interior de uma língua. Um idioleto pode ser a fala peculiar de uma região, de
um grupo étnico ou de uma dada profissão. 

Uma das grandes forças da poesia popular do Nordeste se origina em sua forma
muito própria de falar, com um ritmo muito diferente dos falares do sul, e
também muito diferentes entre si, pois percebe"-se a diferença entre os falares
de um baiano, um cearense e um pernambucano, por exemplo.

Além desse aspecto rítmico, quase sempre também há palavras peculiares a certas
regiões. 


\section{Atividades}

\subsection{Pré-leitura}

\subsubsection{Atividade 1}

\BNCC{EF15AR03}
\BNCC{EF15LP15}

\paragraph{Tema} A literatura de cordel e suas características estéticas e culturais.

\paragraph{Conteúdo} Introduzir a literatura de cordel aos estudantes, apresentando diferentes títulos e incentivando que busquem características semelhantes que permitam caracterizar determinadas obras como literatura de cordel.

\paragraph{Objetivo} Fornecer aos estudantes subsídios para pensar as distintas matrizes artísticas que compõem a cultura brasileira, com enfoque nas características e peculiaridades da literatura de cordel.

\paragraph{Justificativa} A literatura de cordel é um patrimônio da cultura brasileira. Publicada no formato de folheto, geralmente traz para o universo escrito casos e histórias já comuns na tradição oral popular. Entre suas características típicas estão a construção do texto por meio de rimas e a utilização de gravuras, normalmente impressas por meio da xilogravura, que caracterizam as obras com seu tracejado muito próprio. Conhecer a literatura de cordel, portanto, não é apenas entrar em contato com uma importante matriz cultural do Nordeste, mas com uma das mais importantes e características correntes artísticas que constituem o Brasil.

\paragraph{Metodologia} Para uma primeira aproximação da obra poética de Rosário Lustosa, recomenda-se ao professor explorar o formato do folheto de cordel e suas principais características, familiarizando os alunos no universo do cordel no qual estão entrando. 
Para uma primeira abordagem, podem-se levantar questões que instiguem os alunos a pensar nas peculiaridades do cordel.
Algumas perguntas para orientar essa primeira atividade poderiam ser:

\begin{itemize}
\item Alguém já viu ou ouviu falar sobre a literatura de cordel?

\item Se sim, quais características conseguem observar?

\item Se não, o que vem à cabeça ao pensar em ``cordel''?

\item Alguém já ouviu falar no Reisado?

\item De qual região do país imaginam que vem a literatura de cordel?
\end{itemize}


Após verificar o conhecimento prévio que os alunos têm sobre o cordel, comece a explorar alguns temas e características comuns a essa literatura. Algumas imagens de diferentes cordéis e xilogravuras podem ser projetadas na sala, do modo como tipicamente são dispostos nas feiras. Outra opção é levar alguns cordéis para os alunos lerem e manusearem na sala, pensando-se que, atualmente, muitas bancas e jornaleiros vendem cordéis a um baixo custo.
Concomitante à exposição imagética dos cordéis, fale sobre suas principais características, tais como: o uso de xilogravuras e a forma de talhar o desenho na matriz de madeira; a relação do cordel com a oralidade; a composição rimada e sua relação com a memorização da história; os temas, personagens e enredos frequentes a essa literatura; o papel dos cordelistas na cultura nordestina etc.

Após essa introdução, converse com os alunos sobre as informações novas que aprenderam e levante um debate em torno de alguns pontos centrais, tais como:

\begin{itemize}
\item Identificar o folheto de cordel e suas características;

\item Diferenciar um folheto de cordel e um livro;

\item Aproximar o poema do folheto e a fala;

\item Identificar quais são as histórias típicas do folheto de cordel;

\item Perceber as características regionais do folheto de cordel.
\end{itemize}

Posteriormente, o professor pode coletar essas informações junto aos alunos e abordá-las durante as aulas sobre o livro. Assim, as características gerais do cordel podem ser relacionadas com as particularidades do poema de Rosário Lustosa, que foge ao modelo tradicional.


\paragraph{Tempo estimado} Duas aulas de 50 minutos.


\subsubsection{Atividade 2}

\BNCC{EF02GE04}
\BNCC{EF03GE02}
\BNCC{EF02ER02}
\BNCC{EF02ER05}
\BNCC{EF03ER01}
\BNCC{EF03ER03}

\paragraph{Tema} Relacionando a literatura à geografia: Belém e Egito.

\paragraph{Conteúdo} Explorar a relação da história narrada por Rosário Lustrosa com a tradição literária Oriental, explorando os hábitos, características e particularidades de Belém e do Egito, onde se inicia a história.

\paragraph{Objetivo} Aprofundar a compreensão geográfica do estudante a partir da literatura. Mostrar como diferentes culturas se misturam e atravessam na tradicional contação de histórias dos cordelistas. 

\paragraph{Justificativa} Como bem demonstrou o escritor paraibano Ariano Suassuna com sua pesquisa armorial, a cultura nordestina é fortemente influenciada por mitos, lendas e folclores medievais europeus. Com a colonização, muitos elementos formadores do caldo cultural medieval foram importados para a colônia portuguesa e, como eram transmitidos oralmente, aqui se continuou essa tradição. Cavaleiros, criaturas mágicas, feiticeiros e bobos visionários, típicos das lendas medievais, abundam também nas narrativas dos cordéis, misturados às crenças, personagens e tradições locais. A cultura cristã e suas primeiras tradições, igualmente, participam desse ambiente cultural, afinal a Península Ibérica é fortemente influenciada pelo cristianismo e pelas tradições orientais. Vale ressaltar que ela passou por séculos de domínio mouro e, mesmo após a fundação de Portugal pelo rei Afonso Henriques, os costumes e tradições desenvolvidos por séculos naquele território não foram extintos pela ocupação portuguesa, mas misturaram-se ao conjunto da vida portuguesa.

A partir da obra de Rosário Lustrosa, portanto, coloca-se uma oportunidade para introduzir o aluno na cultura de tão distante região, aprofundando sua apreensão sobre a tradição do reisado, sua relação com o cristianismo e sobre as diversas culturas que se cruzam e entrelaçam na criação artística.

\paragraph{Metodologia} O poema de Rosário Lustrosa se inicia em Belém, na atual Palestina, passando pelo Egito antes de chegar às terras brasileiras. O educador pode, inicialmente, projetar imagens da cidade em sala de aula, ressaltando seus elementos característicos: as vestimentas, compostas de burcas, turbantes, coletes, gorro Kufi etc.; as construções com seus ladrilhos coloridos e as características janelas, cúpulas e abóbodas em estilo árabe; a forte presença de regiões desérticas na narrativa etc. É interessante, igualmente, projetar um mapa, ou usar um mapa-múndi de papel, para mostrar a localização de Belém e do Egito no globo terrestre. 

Após essa breve apresentação de Belém e do Egito e de algumas características dos povos árabes, converse com os alunos sobre as semelhanças e diferenças que observam entre o que aprenderam e a vida no Brasil. 
Pode-se fazer perguntas como:

\begin{enumerate}
\item Quais diferenças vocês percebem entre essa terra que acabaram de conhecer e o Brasil?

\item Como vocês acham que é a vida das pessoas lá? 

\item Se fôssemos falar do Brasil para uma pessoa que nunca veio para cá, quais características vocês ressaltariam?

\item Quais elementos ilustram melhor nosso país?

\end{enumerate}

A partir das repostas dos alunos, aconselha-se a ir se aproximando da obra, facilitando a introdução da atividade de leitura. Fale um pouco sobre a história do menino Jesus, sobre a lenda dos três reis magos, o simbolismo de seus presentes, sua relação com datas celebrativas como o Natal e o Dia de Reis e pergunte qual a proximidade e familiaridade das crianças com essas narrativas fundantes de nossa cultura.

Como os alunos já foram introduzidos à literatura de cordel na primeira atividade, nesse momento pode-se relacionar mais intimamente a cultura árabe com a cultura nordestina, levantando hipóteses com os alunos que expliquem a relação de uma lenda cristã com o território árabe e sua vinda para o Brasil. A riqueza da cultura, justamente, vem da inter-relação entre diferentes países e tradições: explore esses pontos de contato com os alunos.

Por fim, relacione os temas debatidos durante a atividade com o cotidiano das crianças: 

\begin{enumerate}
\item Conseguem identificar alguma influência árabe no seu cotidiano?

\item Alguém já viu uma mesquita perto de sua casa? Ou uma pessoa de turbante ou burca?

\item Já viram restaurantes com culinária oriental?

\item Percebem outros aspectos, nos seus lugares de vivência, que são marcados pela cultura árabe?
\end{enumerate}

Se não conhecerem ou identificarem elementos da cultura árabe em seu cotidiano, pode-se perguntar sobre outras culturas e povos que possam ter notado. Assim, a partir da história do nascimento de Jesus e do reisado, coloca-se em debate o sincretismo do Brasil e as contribuições das mais diferentes culturas para constituir o que hoje entendemos como Brasil.


\paragraph{Tempo estimado} Duas aulas de 50 minutos.

\subsection{Leitura}

\BNCC{EF12EF12}
\BNCC{EF12LP19}
\BNCC{EF12LP07}
\BNCC{EF02LP28}
\BNCC{EF03LP27}


\paragraph{Tema} O enredo de \textit{O reisado: festejando o nascimento de Jesus}.

\paragraph{Conteúdo} Exercícios de leitura compartilhada do poema de cordel.

\paragraph{Objetivo} Através de uma leitura conjunta, objetiva-se despertar o interesse do aluno pelo poema e aproximá-lo de algumas características: a forma como as estrofes são encadeadas em rimas; os traços típicos da linguagem do cordel; a estrutura do enredo.

\paragraph{Justificativa} De forma geral, as pessoas que estão se iniciando no universo da leitura têm mais dificuldade em ler poesia do que prosa. Isso porque a poesia, acredita-se, tem um texto mais elíptico, cifrado, formal e, portanto, de assimilação mais difícil do que o texto em prosa. A partir de uma leitura acompanhada entre o professor e os alunos, facilita-se a apreensão da narrativa poética. Outro aspecto importante é que a poesia de cordel é, por excelência, para ser declamada. Logo, a leitura em voz alta permite captar melhor as sonoridades do poema, as rimas, jogos de palavras e outros elementos rítmicos fundamentais para a poesia de cordel.

\paragraph{Metodologia} A proposta é fazer uma leitura alternada do professor e dos alunos. O professor pode fazer uma primeira leitura integral, para apresentar a narrativa e elucidar possíveis dúvidas, como algumas palavras que podem apresentar maior dificuldade de compreensão, como manjedoura, regalia, mirra, lapinha, sequilho, mucunzár, guriabá, cangaceiro etc. 

Em seguida, pode-se fazer propriamente a leitura alternada, em que o professor lê a primeira estrofe e, em sequência, solicita a um aluno que leia a estrofe seguinte, até que todos tenham lido ou que o poema se encerre.
Explore a percepção dos traços típicos da linguagem do cordel: elementos de ritmo e rima, vocabulários característicos, o uso de palavras específicas para obedecer ao ritmo e à melodia. Fale sobre a relação entre essa estrutura rimada e o contexto dos cordelistas, que normalmente decoravam as histórias e, assim, utilizavam as rimas para auxiliar a memória e a fixação do texto da mente.

Por fim, pode-se explorar mais detidamente a estrutura do enredo. Fale sobre os diferentes elementos que o compõem:

\begin{enumerate}
\item A introdução do ambiente e contexto histórico (viagem para participar do censo demográfico);

\item A introdução das personagens histórias;

\item A chegada dos reis magos;

\item A constituição do reisado para celebrar a narrativa religiosa;

\item A migração do reisado para o Egito;

\item A colonização que traz o reisado para o Brasil;

\item A formação do reisado no Brasil;

\item Suas principais características narradas;

\item As regiões em que acontecem;

\item As informações transmitidas pela história.
\end{enumerate}

A partir desses elementos, os alunos começam a tomar contato e familiaridade com a estrutura de um texto ficcional, e como diferentes situações e personagens são invocadas para gerar o movimento da história e tramar a narrativa. Incentive que os alunos interajam com a obra, apelando para seus gostos e impressões.
Pode-se fazer perguntas como:

\begin{itemize}
\item De qual personagem vocês mais gostaram? Por que?

\item Qual a cena mais emocionante da história?

\item Como deve ser um reisado no Egito?

\item Quais elementos vocês mais gostaram do reisado?

\item Já viram algum?

\item Gostariam de participar de uma festa dessas?
\end{itemize}

Dessa forma, estimula-se a apreensão dos alunos do enredo e da estrutura do poema em cordel.
Após a leitura, o professor pode ainda reproduzir em sala algum vídeo de um reisado tradicional, ressaltando seus aspectos em relação com as descrições presentes no poema.

\paragraph{Tempo estimado} Duas aulas de 50 minutos.


\subsection{Pós-leitura}

\subsubsection{Atividade 1}

\BNCC{EF12LP05}
\BNCC{EF35LP28}
\BNCC{EF03LP13}
\BNCC{EF15AR01}
\BNCC{EF35LP12}

\paragraph{Tema} Explorando novas palavras.

\paragraph{Conteúdo} Produção escrita, no formato cordel, em torno da pesquisa de palavras que os estudantes não conheciam.

\paragraph{Objetivo} Estimular a expressão escrita e refletir sobre o vocabulário da língua portuguesa.

\paragraph{Justificativa} A partir da apreensão poética do cordel, passa-se ao momento de aplicar alguns conhecimentos trabalhados ao longo das aulas na produção de um cordel do próprio estudante. Assim, pode-se relacionar o universo artístico dos cordéis, com suas características como os versos rimados, com a produção escrita (ou declamação, para os estuantes ainda não alfabetizados) e algumas de suas características esperadas em aluno de 1º, 2º e 3º anos. Essa é uma forma de desenvolver o conhecimento gramatical e linguístico de forma afetiva e artística, pois relaciona os conhecimentos aprendidos sobre o cordel com as regras gramaticais e as próprias experiências dos estudantes, que podem servir de base para a criação de sua pequena história em cordel.


\paragraph{Metodologia} Já mais familiarizados com as características do cordel, os alunos devem passar então à produção de um texto nesse formato. Solicite, primeiro, que os estudantes elenquem palavras da narrativa que não conheciam (manjedoura, regalia, mirra, lapinha, sequilho, mucunzár, guriabá, cangaceiro etc.). Auxilie-os em uma pesquisa no dicionários para verificar o significado de tais palavras.

Em seguida, peça que escolham alguma das cenas ou festas presente na narrativa (o Natal, o Dia de Reis, as modalidades do reisado, o jogo de espadas etc.) que mais lhes chamaram a atenção.

Solicite, então, que façam um pequeno poema utilizando alguma das palavras novas que aprenderam e uma cena da qual gostaram. Pode ser apenas uma sextilha (uma estrofe de seis versos), que tente utilizar as palavras novas e algum dos elementos da narrativa. Se o estudante ainda não for alfabetizado, pode-se auxiliá-lo na tarefa. Ele pode, por exemplo, contar o que está pensando e o professor ajuda na redação. 

A palavra e o elemento narrativo são apenas os motes para inciar a composição, que pode explorar diversos elementos caros aos estudantes. A partir disso, eles podem ir para diferentes lugares para se inspirar. Algumas sugestões:

\begin{itemize}
\item Trabalhar com a redação de um fato autobiográfico do aluno, usando algo visto ou vivido pelo estudante como mote para sua sextilha. Escrever, por exemplo, sobre um Natal que passou com sua família;

\item Recriar algum elemento da história vista em aula, pensando, por exemplo, em um elemento que o estudante gostaria de incluir em seu reisado, ou em diferentes ambientes (o bairro ou a cidade do aluno, por exemplo) nos quais a história poderia passar;

\item Partir de algum acontecimento de interesse social, divulgado em \textsc{tv}, rádio, mídia impressa e digital, para criar uma narrativa em cima de uma fato real, como era tradição entre os poetas cordelistas.
\end{itemize}

O tema da redação não precisa ser fechado, mas é interessante que os alunos se norteiem pelas novas palavras aprendidas em sala de aula e pelos elementos do reisado pelos quais se interessaram.


\subsubsection{Atividade 1}
\BNCC{EF12EF01}
\BNCC{EF12EF02}
\BNCC{EF12EF11}
\BNCC{EF12EF12}
\BNCC{EF35EF01}

\paragraph{Tema} Recriando um reisado em sala de aula.

\paragraph{Conteúdo} Trabalhar a expressão corporal dos estudantes na encenação de um reisado em sala de aula.

\paragraph{Objetivo} Explorar as características do reisado abordadas no livro; auxiliar os estudantes na apreensão de tais características por meio do trabalho corporal; estimular a cooperação entre os colegas; ressaltar as diferentes brincadeiras, danças e jogos que constituem o patrimônio cultural brasileiro.

\paragraph{Justificativa} Por meio da recriação de brincadeiras e jogos da cultura popular, o estudante desenvolve um senso de comunidade e região, reconhecendo e respeitando as diferenças entre si e seus colegas e entre as diversas culturas que compõem o Brasil.

O uso da linguagem corporal na atividade é importante pois auxilia esse processo, trazendo para o universo lúdico e afetivo a compreensão da obra e da importância das danças, jogos e outras expressões populares para a constituição cultural do país.

Assim, os elementos constitutivos do reisado abordados no livro (ritmo, espaço, gestos, vestimentas, narrativa etc.) serão melhor apreendidos pelo estudante, que vai testá-los no próprio corpo e com os colegas. Por fim, trata-se de uma forma de valorizar os diferentes elementos e matrizes culturais (europeias, indígenas, africanas, asiáticas etc.) que constituem nosso patrimônio histórico cultural.

\paragraph{Metodologia} Após a leitura do livro, o professor pode auxiliar os estudantes a elencar as características de um reisado apresentadas na obra. Pode-se dividir a turma em grupos, ficando cada um responsável por anotar e listar os elementos de determinados eixos temáticos escolhidos pelo professor.
Por exemplo:

\begin{itemize}
\item As comidas;

\item As vestimentas;

\item As etapas da dança;

\item Os elementos narrativos;

\item As personagens.
\end{itemize}

É interessante abordar e ressaltar os elementos narrativos que constituem o reisado, como: a formação do casal, a presença da peça do divino, o jogo de espadas, a luta entre os reisados para salvar a rainha, a hora de guerrear, os comandos do mestre do reisado etc.

Em seguida, o professor pode projetar em sala de aula o vídeo de um reisado típico do Nordeste (talvez de Cariri, Ceará, que é mencionado pela narrativa). Peça, então, que os estudantes identifiquem no vídeo os elementos que observaram e listaram da obra. Estabeleça comparações e relações, auxiliando que as crianças visualizem aquilo que antes fora lido e tornara-se apenas uma imagem mental.

Por fim, o professor pode reservar a última aula da atividade para montar um reisado em sala. Solicite aos estudantes, com uma semana de antecedência, que tragam para a próxima aula elementos para montar um figurino (papéis coloridos, papel crepon, confetes, fitas de diferentes cores, roupas de que gostam, papelão e cartolina para montar coroas e espadas etc.). Elabore em sala esse figuro com os estudantes, inspirando-se nas descrições do livro e nos vídeos assistidos em sala.

Por fim, divida a turma entre os personagens presentes na narrativa. Fale um pouco sobre o papel de cada um na dança e os ajude a encenar o reisado.

\paragraph{Tempo estimado} Três aulas de 50 minutos.


\section{Sugestões de referências complementares}


\begin{itemize}
\item \textsc{diegues júnior}, Daniel. \textit{Literatura popular em verso}. Estudos. Belo Horizonte: Itatiaia, 1986. 

\item \textsc{marco}, Haurélio. \textit{Breve história da literatura de cordel}. São Paulo: Claridade, 2010.

\item \textsc{tavares}, Braulio. \textit{Contando histórias em versos. Poesia e romanceiro popular no Brasil}. São Paulo: 34, 2005.

\item \textsc{tavares}, Braulio. \textit{Os martelos de trupizupe}. Natal: Edições Engenho de Arte, 2004.
\end{itemize}

\section{Bibliografia comentada}

\begin{itemize}
\item \textsc{brasil}. Ministério da Educação. \textit{Base Nacional Comum Curricular}. Brasília, 2018.

Consultar a \textsc{bncc} é essencial para criar atividades para a turma. Além de especificar 
quais habilidades precisam ser desenvolvidas em cada ano, é fonte de informações sobre 
o processo de aprendizagem infantil. 

 
\item \textsc{van der linden}, Sophie. \textit{Para ler o livro ilustrado}. São Paulo: Cosac Naify, 2011.

Livro sobre as particularidades do livro ilustrado, que apresenta as diferenças entre o livro ilustrado e o livro com ilustração. 
\end{itemize}

\end{document}
