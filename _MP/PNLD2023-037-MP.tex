\documentclass[11pt]{extarticle}
\usepackage{manualdoprofessor}
\usepackage{fichatecnica}
\usepackage{lipsum,media9}
\usepackage[justification=raggedright]{caption}
\usepackage[one]{bncc}
\usepackage[ayllon]{../edlab}
\usepackage{marginnote}
\usepackage{pdfpages}

\newcommand{\AutorLivro}{Lucas Kröeff}
\newcommand{\TituloLivro}{O galo Gogó}
\newcommand{\Tema}{Diversão e aventura}
\newcommand{\Genero}{Poesia}
%\newcommand{\imagemCapa}{./images/PNLD2022-001-01.jpeg}
\newcommand{\issnppub}{XXX-XX-XXXXX-XX-X}
\newcommand{\issnepub}{XXX-XX-XXXXX-XX-X}
% \newcommand{\fichacatalografica}{PNLD0001-00.png}
\newcommand{\colaborador}{Paulo Pompermaier}

\begin{document}

\title{\TituloLivro}
\author{\AutorLivro}
\def\authornotes{\colaborador}

\date{}
\maketitle

\tableofcontents

\begin{abstract}
Este material tem a intenção de contribuir para que você consiga desenvolver um trabalho aprofundado com a obra \textit{O galo Gogó} em sala de aula.
Você encontrará informações sobre o autor, sobre o gênero e também 
algumas propostas de trabalho para a sala de aula que você poderá explorar livremente, 
da forma que considerar mais apropriada para os seus estudantes.

O autor do livro, Lucas Kröeff, é um destacado designer e artista contemporâneo.
Em sua obra para o público juvenil, cria uma narrativa nos moldes de uma parlenda popular, aliando textos rápidos e ritmados a ilustrações coloridas e divertidas, tendo como pano de fundo uma figura típica do Brasil: o galo que canta o nascer do dia.

Com sua bagagem de artista plástico, o autor cria um livro que não comunica apenas pelo texto, mas também pela imagem. Como a história do galo Gogó transita entre determinados eixos que se repetem (o canto do raiar do dia, seus movimentos corporais em cima do telhado, seus sonhos durante a noite e o dia seguinte com a mesma rotina), o livro permite diversas leituras, pois depende do movimento criado pela
sequência das imagens apresentadas ao leitor. As diferentes versões possíveis 
para a história permitem que o leitor e o mediador da leitura estejam 
no mesmo nível, pois não há um código escrito a ser decifrado. 

Além de uma introdução no gênero poético das parlendas, o livro desenvolve importantes capacidades e competências do jovem leitor:
observação e imaginação e, logo, debate e troca de ideias durante 
a leitura. A exploração das ilustrações do livro com seus alunos contribuirá 
para o enriquecimento do repertório da criança: desde o vocabulário até o 
olhar artístico que também pode ser afinado ao longo do trabalho.

As imagens são poderosas formas de comunicação, 
muito presentes na sociedade em que vivemos, desde outdoors e manuais de 
instrução até as inúmeras telas com que temos contato. Como interpretamos 
imagens o tempo inteiro, não conseguimos perceber com clareza o quanto 
interpretar uma imagem é uma atividade complexa. Ler imagens com competência, 
perceber seus recursos e nuances é parte importante do processo de apreensão, 
leitura e compreensão do mundo e de nossa existência. Antes de ler textos 
verbais, lemos textos visuais e os interpretamos a partir de nossas vivências, 
emoções e percepções.

Para além da narrativa, as imagens de \textit{O galo Gogó} 
apresentam às crianças uma linguagem artística complexa.
As possibilidades são infinitas: explorar as cores, as formas, o posicionamento dos personagens na página e até mesmo a opinião e os sentimentos das crianças sobre as imagens 
aprofundarão a leitura, aumentarão o repertório e incentivarão o desenvolvimento do vocabulário e da fluidez do discurso.

Ao longo do manual, todos esses aspectos serão explorados e relacionados a sugestões de atividades. Com isso, objetiva-se oferecer algumas ideias e inspirações para um trabalho que pode ser desenvolvido tanto a curto, quanto a médio e longo prazo. Sinta-se à vontade para personalizar a aula e torná-la sua, aplicando seus conhecimentos, sua 
personalidade e aproveite para fortalecer seu vínculo com a turma.
Boa aula!

\end{abstract}

\section{Sobre o livro}
\textit{O galo Gogó} é um livro de literatura infantil composto por textos 
verbais e não verbais. Nele os leitores acompanham a rotina de um galo chamado
Gogó. Com muitas ilustrações e textos, a cada página é
apresentado um momento do dia, desde o amanhecer, quando
Gogó acorda e canta seu ``Cocoricó!'' até a hora de ir dormir e sonhar
em ser cantor. 

Com este livro, os estudantes podem
trabalhar diversos temas, como o conhecimento dos
animais da fauna local e o respeito à biodiversidade, as formas geométricas das ilustrações,
as transformações da natureza (amanhecer e anoitecer) e, principalmente, a forma poética da parlenda, que aparece aqui em íntima relação com um animal emblemático da cultura brasileira.

Conhecido por sua importância para a manutenção do galinheiro, nesta história
o galo é apresentado em seu aspecto mais simbólico: aquele que canta logo
que o sol nasce e acorda as pessoas para o dia. Ele está, por isso, associado
ao sol e ao trabalho. Tal associação, mais comum sobretudo na área rural, onde estes bichos
existem em maior quantidade, é feita na obra por meio das ilustrações e da parlenda, abordando não apenas a poesia da escrita, como aquela criada e transmitida pela imagem.


\section{Sobre o autor}


\paragraph{O autor} Lucas-K, nome artístico de Lucas de Mesquita Kröeff, é um artista visual brasileiro. Bacharel em Design pela Universidade Federal de Minas Gerais (\textsc{ufmg}) e em Artes Visuais pela Cambridge School of Arts (Ruskin School), na Inglaterra, desenvolve seus trabalhos numa interface livre entre iniciativas independentes, instituições de arte e editoras de livros, produzindo instalações, livros, vídeos e experiências coletivas.
Lucas Kröeff explora, em seus trabalhos, a relação entre história, política, imaginação coletiva e a construção da sua própria subjetividade, num processo de construção de redes de intercâmbio, assim como de procedimentos sistemáticos de organização da vida quotidiana.

%313 caracteres
\paragraph{Publicações} Como artista gráfico, Lucas Kröeff desenvolve capas de livros e coleções de livros usando o alfabeto como ferramenta de desenho conceitual. Ele já fez capas de livros de autores como Masha Alyokhina, Félix Guattari, Harriet Jacobs, Walter Benjamim, Franz Kafka, Fernando Pessoa, Celso Favaretto, Paul D. Escott, Baudelaire, Maquiavel, H.P. Lovecraft, Fernand Deligny, Stéphane Mallarmé e outros.

%358 caracteres
\paragraph{Currículo} Lucas Kröeff já teve trabalhos apresentados na 11ª Bienal de Arquitetura de São Paulo, Museu de Tecnologia de Cambridge, Cinemateca de São Paulo, Museu de Minas e Metal, \textsc{iiix} Festival Internacional de Videoarte de Barcelona e ARCOMadrid, entre outras instituições e galerias de arte. Em 2015 recebeu o Prêmio de Arte da Sustentabilidade em Cambridge (Reino Unido). Publica livros há mais de 10 anos.
Foi cofundador e participou de vários coletivos de arte, incluindo a Atpress em Londres, bem como nos grupos \textsc{mapa}, \textsc{banca} e Quadradocirculo no Brasil.


\SideImage{O gênero poético incentiva a curiosidade e a imaginação. (LACMA/Remedios Varo; CC BY-NC 2.0)}{PNLD2023-037-07.png}

\section{Sobre o gênero}

%55 caracteres
\paragraph{O gênero} O gênero deste livro é \textit{poesia}. 


Para uma primeira definição de poesia enquanto gênero literário, poder"-se"-ia recorrer à definição do professor Domingos Paschoal Cegalla, para quem ``poesia é a linguagem subjetiva, carregada de emoção e sentimento, com ritmo melódico constante, bela e indefinível como o mundo interior do poeta visa a um efeito estético''.\footnote{\textsc{cegalla}, Domingos Paschoal. \textit{Novíssima Gramática da Língua Portuguesa}. São Paulo: Companhia Editora Nacional, 2008, p.\,640}

Aprofundando um pouco essa definição, o crítico Antonio Candido expande a definição de poesia ao diferenciá"-la do verso.
Para o crítico, a poesia enquanto ato criador do artista independe da forma métrica do verso, que passa a ser apenas um dos registros possíveis do poético:

\begin{quote}
A poesia não se confunde necessariamente com o verso, muito menos com o verso metrificado. Pode haver poesia em prosa e poesia em verso livre. [\ldots]
Pode ser feita em verso muita coisa que não é poesia.\footnote{\textsc{candido}, Antonio. \textit{O estudo analítico do poema}. São Paulo: Terceira leitura, 1993, p.\,13--14.}
\end{quote}

Delineada, de forma breve e geral, a forma poética, pode"-se pensar agora em seus três gêneros básicos: lírico, épico e dramático.
Para o crítico Anatol Rosenfeld, a lírica é o gênero mais subjetivo, no qual uma voz central exprime um estado de alma traduzido em orações poéticas.
Seria a expressão de emoções e experiências vividas, ``a plasmação imediata das vivências intensas de um Eu no encontro com o mundo, sem que se interponham eventos distendidos no tempo (como na Épica e na Dramática)''.\footnote{\textsc{rosenfeld}, Anatol. \textit{O teatro épico}. São Paulo: Perspectiva, 2006, p.\,22.}

Devido a essa característica central da lírica, a expressão de um estado emocional, Rosenfeld considera que o eu"-lírico, nesse gênero, não se delineia enquanto um personagem. Embora possa evocar personagens e narrar acontecimentos, a lírica entendida enquanto gênero puro afasta"-se sobremaneira da apreensão objetiva do mundo, que não existe independente da subjetividade intensa que o apreende e exprime. Assim, na lírica prevalece a fusão entre o sujeito e o objeto, que serve mais a realçar os estados profundos de alma do poeta.
Sobre os aspectos formais do gênero, Rosenfeld nota:

\begin{quote}
À intensidade expressiva, à concentração e ao caráter ``imediato'' do poema lírico, associa"-se, como traço estilístico importante, o uso do ritmo e da musicalidade das palavras e dos versos. De tal modo se realça o valor da aura conotativa do verbo que este muitas vezes chega a ter uma função mais sonora que lógico"-denotativa. A isso se liga a preponderância da voz do presente que indica a ausência de distância, geralmente associada ao pretérito. Este caráter do imediato, que se manifesta na voz do presente, não é, porém, o de uma atualidade que se processa e distende através do tempo (como na Dramática) mas de um momento ``eterno''.\footnote{Ibidem, p.\,23.}
\end{quote}

\includepdf[nup=2x3, 				% grid
			%offset=-15mm -5mm, 	% posição
			scale=.8, 				% tamanho da página
            delta=4mm 4mm, 			
            frame,
            pages={9-14}]{./pdfs/\jobname_MIOLO.pdf}

\section{Atividades}

\subsection{Pré-leitura}
\BNCC{EF12LP18}
\BNCC{EF15LP10}

\paragraph{Tema} A poesia das palavras e das imagens.

\paragraph{Conteúdo} Explorar as características e estrutura da poesia e apresentar algumas formas de sua manifestação, como através da música, das artes plásticas e da literatura. Em \textit{O galo Gogó}, a poesia não se manifesta apenas através do texto escrito, mas também do texto imagético: em cada momento do dia, Gogó apresenta diferentes expressões corporais, cores e formas, refletindo as diferentes experiências pelas quais passa no cotidiano. Essas diferentes dimensões da narrativa são construídas a partir das ilustrações, criando-se uma poética visual.


\paragraph{Objetivo} Preparar os alunos para a leitura do livro e apresentar duas formas de manifestações artísticas poéticas: o desenho e a escrita. Aprender gêneros textuais e artísticos.

\paragraph{Justificativa} A poesia comumente é associada somente à literatura, por isso a ideia é ampliar a visão dos alunos sobre poesia e suas diferentes manifestações, mostrando como uma história pode ser narrada somente através de desenhos. Em \textit{O galo Gogó}, é possível fazer uma leitura considerando-se apenas as imagens, e é muito importante que os alunos saibam interpretar imagens e relacioná-las aos acontecimentos da narrativa. O mundo contemporâneo é permeado de imagens, saber observá-las e analisá-las criticamente faz com que a experiência literária relacione-se diretamente ao cotidiano e às vivências dos estudantes. 

\paragraph{Metodologia} O professor pode iniciar a aula perguntando aos alunos se alguém sabe o que é poesia e promover um pequeno debate acerca do tema, explorando o significado que dão à palavra.

Para esse momento, pode utilizar alguns elementos do próprio enredo para explorar a criação poética e sua relação com a sonoridade das palavras.
Por exemplo, “Cocoricó” é uma expressão recorrente no livro, e é uma figura de linguagem, uma onomatopeia para imitar o canto do galo. Explore as figuras de linguagem com os estudantes, fale sobre sua importância para a criação poética, sobre a cadência e o ritmo das palavras, evidente na separação silábica de co-co-ri-có.

Além de abordar o conceito de poesia e explorar a compreensão dos alunos, é essencial que o educador estimule as crianças para que façam indagações sobre o tema. Pode-se apresentar as diferentes manifestações poéticas, como filmes, músicas, artes plásticas e literatura, perguntar quais dessas expressões artísticas eles gostam mais, quais consumiram recentemente, como percebem a poesia em determinados contextos de produção artística a partir do que foi exposto sobre esse gênero textual.

Por fim, quando todos estiverem contextualizados sobre o que é poesia e como ela apresenta diversas formas de manifestação, o professor pode pedir para que os alunos façam um desenho sobre o que compreenderam da ``poesia''. O desenho é livre e a ideia é que se inspirem nas aulas para soltar a imaginação em torno do tema. Podem, assim, tanto representar uma ideia materializada da poesia como fazer uma criação propriamente poética.

\paragraph{Tempo estimado} Duas aulas de 50 minutos.


\subsection{Leitura}
\BNCC{EF01LP16}
\BNCC{EF01LP19}
\BNCC{EF01LP06}
\BNCC{EF01LP08}
\BNCC{EF02LP02}

\paragraph{Tema} Leitura do livro e compreensão de sua dimensão poética.

\paragraph{Conteúdo} Expor aos alunos o livro como uma obra poética que envolve os desenhos e a parlenda. Explicitar o que é o gênero parlenda e evidenciar seu uso coloquial no cotidiano dos alunos. Em \textit{O galo Gogó}, a personagem faz um jogo entre o lúdico e o cotidiano, o que vai ser explorado na atividade. Também serão trabalhados os elementos textuais do livro, como separação de sílabas, a rima e o ritmo.

\paragraph{Objetivo} Demonstrar como as imagens do livro despertam a criatividade e a imaginação e como a estrutura da escrita traz ritmo para leitura através da rima e da separação entre as sílabas. Explicar como o emprego dos acentos gráficos muda o som da palavra.

\paragraph{Justificativa} Trazer a ludicidade para a aprendizagem torna o processo muito mais significativo para as crianças, pois elas brincam aprendendo. O momento de leitura e de explicação de conceitos se torna ainda mais rico quando a emoção e a subjetividade dos alunos é relacionada ao conteúdo. No livro \textit{O galo Gogó}, a personagem sente diferentes emoções ao longo do dia: Gogó canta, dança e sonha, possibilitando uma aproximação das características da narrativa com a vida dos próprios alunos. Essa é uma maneira de trazer a obra para mais perto das crianças, produzindo identificação e maior interesse.


\paragraph{Metodologia} Antes de iniciar a leitura do conteúdo gráfico do livro, o professor pode dialogar com os alunos sobre a personagem Gogó, incentivando a leitura e a interpretação  através das imagens. O educador fará perguntas sobre o que os alunos acham que Gogó está fazendo, como está fazendo e por que está fazendo. Independente do texto, as imagens transmitem por si só uma narrativa, expressam as ações e emoções de Gogó. Essa pode ser uma primeira aproximação dos estudantes com o livro, pois podem identificar os elementos da narrativa a partir da disposição e do conteúdo das imagens.

Em seguida explique o que é uma parlenda, suas características e particularidades, utilizando exemplos de parlendas populares, como abordadas em mais profundidade na atividade de pós-leitura. Incentive a imaginação explorando a ludicidade dos alunos.  Quando a leitura começar, explore a oralidade, explicando conceitos como ritmo e rima. No seguinte trecho, por exemplo:

\begin{verse}
Quando canta,\\
Gogó dança\\
Gogó bate as asas\\
que quase voa\\
Gogó canta e dança\\
e depois quer descansar
\end{verse}

Pode-se explicar como a repetição de ``Gogó'' e de verbos como ``canta'' e ``dança'' imprimem ritmo à narrativa, assim como a repetição em sequência de palavras terminadas em -a, em ``canta'' e ``dança''.

Fale também sobre as quebras silábicas. Na onomatopeia ``cocoricó'', as sílabas são grafadas separadamente, respeitando a divisão silábica da língua portuguesa: ``co co ri có''. Demonstre para os alunos como os espaços em branco entre as sílabas fazem com que a leitura ganhe um ritmo diferente. Explore o ritmo criado pela separação das sílabas, incentivando que os alunos leiam outros trechos da narrativa empregando uma mesma leitura pausada que respeite a divisão das sílabas. Por exemplo: ``Go-gó can-ta, dan-ça e de-pois quer des-can-sar''.

Através da separação silábica, também pode-se explicar o uso dos acentos diacríticos e a modificação fonética decorrente. Quando Gogó canta, em ``co co ri có'', por exemplo, mostre como os dois primeiros ``co'' têm um som mais fechado, em comparação com o som aberto do último ``có'' em decorrência do acento agudo. Também pode-se explorar a mudança de som provocada pelo acento no nome do próprio personagem, composto por duas sílabas iguais mas com um primeiro uso fechado (go) e o segundo mais aberto em decorrência do acento (gó).

Explore esses elementos enquanto faz a leitura dialogada, solicitando sempre a opinião dos estudantes sobre o trecho lido e o ponto debatido a partir desse trecho, acolhendo dúvidas e incentivando perguntas.

\paragraph{Tempo estimado} Duas aulas de 50 minutos.

\subsection{Pós-leitura}
\BNCC{EF01LP19}
\BNCC{EF01LP22}

\paragraph{Tema} Parlendas.

\paragraph{Conteúdo} Parlendas populares. Fluência de leitura, decodificação de signos e formação do leitor.


\paragraph{Objetivo} Relacionar o conteúdo da aula anterior com parlendas populares conhecidas pelas crianças e promover a literacia familiar.


\paragraph{Justificativa} O  conteúdo de aprendizagem que ultrapassa o espaço escolar e que faz parte do dia a dia dos alunos torna a situação de ensino aprendizagem muito mais rica. Portanto, relacionar a história do livro com parlendas populares e envolver os pais nesta atividade potencializa a assimilação e acomodação do que foi trabalhado em sala de aula. 

\paragraph{Metodologia} Após os alunos estudarem o gênero literário parlenda e ler em sala de aula o livro \textit{O galo Gogó}, o professor deverá utilizar as parlendas mais populares, que atravessam gerações, para a atividade de pós-leitura.

Retome o conceito de parlenda (rimas infantis, geralmente curtas e divertidas, para memorizar algo, escolher alguém, brincar com as palavras etc) e pergunte para os alunos se eles conhecem alguma. Escolha quatro parlendas populares que provavelmente os pais das crianças conhecem. Um exemplo clássico é:

\begin{verse}
Um, dois, feijão com arroz,\\
Três, quatro, feijão no prato,\\
Cinco, seis, falar inglês,\\
Sete, oito, comer biscoito,\\
Nove, dez, comer pastéis
\end{verse}

Leia as parlendas escolhidas em sala de aula com os alunos e estimule-os a comparar as características das parlendas populares com a do livro. Incentive que as crianças troquem informações e impressões sobre as parlendas. Por fim, oriente-as a explicar para os pais o que é uma parlenda e a escolher uma para memorizar e recitar para os familiares.

Orientem as crianças a fazer as seguintes perguntas para os responsáveis:

\begin{itemize}
\item Você já conhecia essa parlenda?

\item Ela fez parte da sua infância?

\item Você brincou utilizando essa parlenda?

\item Como?

\item Em qual contexto aprendeu a parlenda?
\end{itemize}

Peça aos alunos que registrem as respostas dos pais. Para os que não são alfabetizados, peça para que contem o que ouviram ou solicite que os responsáveis ajudem a escrever as repostas. Explique para os pais o objetivo da atividade e sua importância.
Quando todas as crianças tiverem as respostas dos seus familiares, estimule a troca de informações na sala. Os estudantes podem expor as parlendas que seus pais conheciam, falar como eles tomaram contato com elas, qual era sua função e o contexto sociocultural no qual elas circulavam. Compare o ritmo das parlendas trazidas para a sala de aula com o ritmo da leitura de \textit{O galo Gogó}. Assim, através da experiência familiar, os alunos vão montar um panorama das parlendas, entendendo os diferentes contextos e funções que essa forma poética ocupava na geração de seus pais.
Consulte no link \href{https://blog.ataba.com.br/parlendas/}{as dezoito parlendas} para o professor escolher e trabalhar em sala com os estudantes. 

\paragraph{Tempo estimado} Duas aulas de 50 minutos.

\section{Sugestões de referências complementares}


\begin{itemize}
\item \textsc{andrade}, Mário de. \textit{Aspectos do folclore brasileiro}. Rio de Janeiro: Global editora, 2019.

\item \textsc{brasil}. Ministério da Educação. \href{http://alfabetizacao.mec.gov.br/images/conta-pra-mim/livros/versao_digital/parlendas_versao_digital.pdf}{\textit{Parlendas}}. Brasília, 2020. 

\item \textsc{cascudo}, Câmara. \textit{Folclore do Brasil}. São Paulo: Global edições, 2017.

\item \_\_\_\_\_\_. \emph{Morfologia do conto maravilhoso}. Rio de Janeiro: Forense Universitária, 2006. 
\end{itemize}

\section{Bibliografia comentada}

\subsection{Livros}

\begin{itemize}
\item \textsc{brasil}. Ministério da Educação. Base Nacional Comum Curricular. Brasília, 2018.

Consultar a \textsc{bncc} é essencial para criar atividades para a turma. Além de especificar 
quais habilidades precisam ser desenvolvidas em cada ano, é fonte de informações sobre 
o processo de aprendizagem infantil. 

\item \textsc{massi}, Augusto (org.). \emph{Artes e ofícios da poesia}. Porto Alegre:
  Artes e Ofícios, 1991.

Depoimentos de 29 poetas contemporâneos sobre seus respectivos processos criativos. Entre os nomes selecionados, estão Adélia Prado, Alice Ruiz, José Paulo Paes, Manoel de Barros e
Sebastião Uchoa Leite. O depoimento de Orides Fontela incluído no livro se intitula ``Na trilha do trevo''.

\item \textsc{van der linden}, Sophie. Para ler o livro ilustrado. São Paulo: Cosac Naify, 2011.

Livro sobre as particularidades do livro ilustrado, que apresenta as diferenças entre o livro ilustrado e o livro com ilustração. 
\end{itemize}

\subsection{Audiovisual}

\begin{itemize}
\item \href{https://www.youtube.com/watch?v=EznwhYEC9tE}{\textit{Cocoricó. \textsc{dvd} com 28 clipes musicais}}.  

O \textsc{dvd} traz 28 canções do grupo musical Cocoricó. Todas têm como personagens os galos e galinhas no galinheiro. Uma forma lúdica e divertida de abordar o assunto do livro em sala de aula.

\item Mundo Bita. \href{https://www.youtube.com/watch?v=cjONzZPJONc}{Música} \textit{Fazendinha}. 

Música que narra o amanhecer de uma fazenda a partir do nascer do sol e do canto do galo. Outra forma lúdica de explorar as atividades proporcionadas pelo canto do galo que desperta a vida ao sol que surge no horizonte.
\end{itemize}

\end{document}

