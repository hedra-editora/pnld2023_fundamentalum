\documentclass[11pt]{extarticle}
\usepackage{manualdoprofessor}
\usepackage{fichatecnica}
\usepackage{lipsum,media9}
\usepackage[justification=raggedright]{caption}
\usepackage[one]{bncc}
\usepackage[acorde]{../edlab}
\usepackage{marginnote}
\usepackage{pdfpages}
\usepackage[printwatermark]{xwatermark}
% \newwatermark[pagex=2]{\includegraphics[scale=3.3]{watermarks/test-a.png}}	% página específica
% %\newwatermark[oddpages]{\includegraphics{watermarks/test-a.png}}			% páginas ímpars
% %\newwatermark[evenpages]{\includegraphics{watermarks/test-a.png}}			% págimas pares
% \newwatermark[allpages]{\includegraphics[scale=3.3]{watermarks/test-b.png}}

% \pagecolor{cyan!0!magenta!10!yellow!28!black!28!}

\newcommand{\AutorLivro}{Antonio Prata}
\newcommand{\TituloLivro}{Esconde esconde}
\newcommand{\Genero}{Poesia}
%\newcommand{\imagemCapa}{./images/PNLD0001-01.png}
\newcommand{\issnppub}{978-65-99441-24-0}
\newcommand{\issnepub}{978-65-99441-27-1}
% \newcommand{\fichacatalografica}{PNLD0001-00.png}
\newcommand{\colaborador}{Gabriela Karam}

\begin{document}

\title{\TituloLivro}
\author{\AutorLivro}
\def\authornotes{\colaborador}

\date{}
\maketitle

%\begin{abstract}\addcontentsline{toc}{section}{Carta ao professor}
%\pagebreak

\tableofcontents


\begin{abstract}

Caras e caros educadores,

Desenvolvemos este material com a intenção de auxiliá-los na utilização de jogos que possam ser usados com a turma para que torne o aprofundamento da leitura mais lúdico e divertido. Contamos com vocês para mergulharmos juntos nessa história cheia de brincadeiras com palavras e rimas espirituosas. Vocês encontrarão aqui também algumas propostas de trabalho para a sala de aula que poderão explorar livremente, da forma que considerarem mais apropriada para os seus estudantes. \textit{Esconde Esconde} é um livro escrito por \textit{Antonio Prata} e ilustrado por \textit{Talita Hoffmann}, a mesma dupla dos infantis \textit{Jacaré, não!}e \textit{A menina que morava no chuveiro}. 

\Image{Ilustração do livro (Imagem do livro, pag. 6)}{PNLD2023-035-02.png}

O livro é uma imersão no universo da linguística. O autor propõe a seguinte brincadeira: procurar palavras dentro de palavras.
\textit{Já reparou que, às vezes, dentro de uma palavra, outra palavra se esconde? Aqui, ó, do bumbum do esc eis que surge um onde.}


\end{abstract}


 \section{Sobre o livro}
 \section{Sobre o autor}
 \section{Sobre o gênero}

E por aí vai! Antonio Prata, que é  filho dos também escritores Mário Prata e Marta Góes, já publicou vários livros e é roteirista e autor contratado pela Rede Globo, onde colaborou em várias novelas, entre elas Bang Bang, de seu pai Mário Prata e Carlos Lombardi, e a célebre Avenida Brasil. Em 2015 escreveu o piloto da série \textit{Os Experientes}, dirigido por Kiko e Fernando Meirelles, que venceu o prêmio APCA de melhor série de televisão e foi finalista do Emmy Awards. Em 2012, foi incluído na edição brasileira da revista Granta como um dos vinte melhores escritores nacionais com menos de 40 anos.
%PNLD2023-035-03

Antonio Prata, que desde 2010 também escreve aos domingos no caderno da Folha de São Paulo, vem se aventurando já há um tempo no universo infantil. E agora, em Esconde Esconde, parte para uma empreitada linguística: com um texto bem sacado e ilustrações super imaginativas, o livro convida, com muito humor, crianças e adultos a pensarem e tomarem consciência da composição das palavras em português. Um exercício que, aliás, continua fora do livro, depois que se acaba a leitura. O livro trata de maneira lúdica a respeito de um tema teórico clássico da linguística de \textit{Ferdinand de Saussure}: a arbitrariedade do signo linguístico, ou seja, o fato de que a relação entre o som (significante) e o sentido (significado) de uma palavra é arbitrária, um acaso. A formação de palavras é um dos temas mais importantes para os estudantes do Ensino Fundamental como um todo, visto que o entendimento do português depende do começo: saber os radicais das palavras, entender os afixos que são adicionados a eles e, assim, desenvolver uma lógica muito importante para compreender diversos textos. Por conta disso, a obra gera uma série de possibilidades de atividades para os alunos, e de modo muito divertido. 
%PNLD2023-035-04.

Além disso, vale ressaltar que as ilustrações de Hoffmann são muito certeiras por conta das cores e detém uma identificação muito clara com as crianças e os pré-adolescentes. A genialidade da obra está na fusão entre uma narrativa que gera muito interesse e que, ao mesmo tempo, tem rimas e jogos de palavras muito ricos para o entendimento da língua portuguesa. Ao longo desse manual, sugestões para atividades e algumas referências serão compartilhadas para que a leitura e o trabalho com o livro sejam mais proveitosos e aprofundados. Usando e abusando de formas lúdicas para se contar uma história, verificamos no livro a presença de respostas para perguntas que as crianças costumam fazer, como por exemplo: será que um pássaro voa daqui até a Argentina se quiser? Com maestria, autor e ilustradora proporcionam uma série de questionamentos para as crianças, além de aflorar muito a criatividade dos pequenos. 
%PNLD2023-035-05

Com isso, objetiva-se oferecer algumas ideias e inspirações para um trabalho que pode ser desenvolvido tanto a curto quanto a médio e longo prazos. Esperamos que este manual abra caminhos para outras atividades que surgirão no decorrer da aula. Sinta se a vontade para usufruir da melhor forma que lhe aprouver e boa aula!


\section{Proposta de Atividades}
\subsection{Pré Leitura}
\subsubsection{Atividade 1}

\BNCC{EF03LP10}% Reconhecer prefixos e sufixos produtivos na formação de palavras derivadas de substantivos, de adjetivos e de verbos, utilizando-os para compreender palavras e para formar novas palavras.

Antes de mergulhar no universo da obra \textit{Esconde Esconde}, o educador pode estimular os alunos a explorarem as palavras de uma forma geral. Para isso, antes de mais nada, pergunte e debata com os estudantes sobre a formação de palavras através das letras e dos sons. Logo em seguida, proponha um caça palavras entre eles e, com isso, acabará provocando uma intimidade com as letras e a formação de palavras. 
%PNLD2023-035-06


Outro jogo muito interessante e que também provoca algo bastante parecido com o anterior a respeito da desenvoltura com a nossa língua é a Forca. 
%PNLD2023-035-07

Algumas dicas para que esse jogo fique divertido e estimulante para os estudantes: 


\begin{itemize}
\item Divida a turma em grupos de cinco estudantes cada
\item Escolha palavras aleatórias dentro do universo dos estudantes para que não desestimule a turma
\item Coloque apenas as linhas de cada letra com a formação da palavra no quadro escolar
\item Inicie o jogo para a descoberta da palavra através das letras escolhidas. Cada grupo participa de uma vez, mas organize para que a turma inteira participe
\item Aguce a perspicácia deles, desperte o interesse pela palavra e, conforme o jogo for fluindo, escolha palavras mais difíceis para incentivá-los cada vez mais

Uma vez findadas as rodadas, aproveite as palavras trabalhadas para trazer o tema de formação de palavras: a anexação de afixos e os tipos de palavras produzidos com isso, bem como a função do radical. Com eles, identifique o radical das palavras e os afixos existentes. É um bom momento para que o professor traga à tona os tipos de formação: por composição ou por derivação. Depois te trazer a teoria, é interessante colocá-los para praticar. E um jogo muito interessante para isso é o Stop:
%PNLD2023-035-08

\item Solicite papel e caneta para todos os estudantes
\item Defina as categorias de acordo com a idade da turma. Por exemplo: Nome, Fruta, Cor, Cidade
\item Escolha uma letra para que todas as pessoas comecem a preencher
\item A primeira pessoa que terminar de preencher tudo primeiro grita Stop e todo mundo para de escrever
\end{itemize}

Aproveite para debater depois dos jogos sobre as dificuldades e as facilidades dentro das atividades propostas, perguntando a eles quais processos de formação de palavras apareceram ao longo do processo. Por fim, forneça aos estudantes subsídios para pensar sobre as palavras e como elas nos ajudam em uma comunicação mais clara com nós mesmos e com o mundo, fortalecendo, assim, não só a nossa auto estima, como a noção de cidadania.

\subsubsection{Atividade 2}

\BNCC{EF15AR01}% Identificar e apreciar formas distintas das artes visuais tradicionais e contemporâneas, cultivando a percepção, o imaginário, a capacidade de simbolizar e o repertório imagético.
\BNCC{EF15AR04}% Experimentar diferentes formas de expressão artística (desenho, pintura, colagem, quadrinhos, dobradura, escultura, modelagem, instalação, vídeo, fotografia etc.), fazendo uso sustentável de materiais, instrumentos, recursos e técnicas convencionais e não convencionais.

Quando mergulhamos no mundo das palavras, um universo infinito de imagens surgem nas nossas cabeças, provocando signos que fazem com que elas digam a que vieram. No livro \textit{Esconde esconde}, os desenhos que ilustram o texto fazem parte integrante da história. A pessoa que executa uma ilustração de um livro, é uma das personagens principais na literatura, em especial na literatura infantil. É ela quem dá imagem, cor e formas à ideia do texto, ou até mesmo conta a história por meio das ilustrações. 

Portanto, essa segunda atividade tem como objetivo desenvolver o aprofundamento do contato da palavra através de imagens. Selecione algumas das palavras usadas na atividade anterior e estimule a criação de imagens relacionadas a elas. Peça que usem desenhos ou colagens, preocupando-se com a sustentabilidade dos materiais usados. Aproveite para apresentar algumas ilustrações de livros diversos. Não se limite só nas ilustrações mais concretas, provoque um pouco mais os instintos e mostre algumas mais abstratas para que os estudantes ampliem sua observação de leitura de imagens. Acreditamos que depois dessas atividades, os estudantes já estarão prontos para uma leitura comprometida e ao mesmo tempo divertida. 
%PNLD2023-035-09.


\section{Leitura}

\BNCC{EF02LP26}% Ler e compreender, com certa autonomia, textos literários, de gêneros variados, desenvolvendo o gosto pela leitura.
\BNCC{EF15LP04}% Identificar o efeito de sentido produzido pelo uso de recursos expressivos gráfico-visuais em textos multissemióticos.

Antes de iniciar a leitura, coloque as cadeiras em roda para que os estudantes possam se ver e obter aquela sensação de histórias em volta da fogueira. Como esse livro tem uma presença muito forte das ilustrações, que estão lado a lado com a narrativa, sugerimos que a leitura seja compartilhada e que, a cada página, a pessoa que estiver lendo apresente para a turma as ilustrações de forma que os desenhos possam ser vistos e apreciados por todo mundo. 
%PNLD2023-035-10

Uma outra sugestão seria que o professor realizasse uma primeira leitura integral para apresentar a narrativa e elucidar possíveis dúvidas, como algumas palavras que podem apresentar maior dificuldade de compreensão. Sanar as questões das palavras dentro de outras e como as ilustrações vão colocando essas palavras num lugar lúdico e fantástico. Em seguida, pode-se fazer uma demonstração apenas das ilustrações para a descoberta da riqueza de detalhes que acompanham cada página. Essa forma de leitura acompanhada entre o professor e a turma facilita a apreensão da narrativa e do estudo das palavras em questão.


Incentive que os estudantes interajam com a obra, trazendo para seus gostos e impressões. Pode-se fazer perguntas como:

\begin{itemize}
\item Qual estrofe vocês mais gostaram?
\item Alguma dessas palavras vocês acharam difícil de identificar dentro da outra? Qual?
\item Qual parte vocês acharam mais interessante? Por que?
\item Qual ilustração vocês tiveram mais dificuldade de entender? Por quê?
\item Quais foram os processos de formação de palavras mais recorrentes?
\end{itemize}

Dessa forma, estimula-se a apreensão do enredo, do conteúdo da formação de palavras e das ilustrações que estão completamente em consonância com a proposta do livro.

\section{Pós-leitura}

\BNCC{EF15LP18}% Relacionar texto com ilustrações e outros recursos gráficos.
\BNCC{EF12LP05}% Planejar e produzir, em colaboração com os colegas e com a ajuda do professor, (re)contagens de histórias, poemas e outros textos versificados (letras de canção, quadrinhas, cordel), poemas visuais, tiras e histórias em quadrinhos, dentre outros gêneros do campo artístico-literário, considerando a situação comunicativa e a finalidade do texto.

No próprio livro \textit{Esconde Esconde}, o autor propõe uma atividade que envolvem outras novas palavras para que os leitores brinquem da mesma forma que o livro ofereceu na narrativa. Divida a turma em grupos e, utilizando o quadro escolar, escreva as palavras propostas no final do livro. Em seguida, para que todo mundo participe, escolha uma palavra e peça para que um grupo descubra quais palavras mais se escondem entre ela e circule com uma caneta ou giz colorido para enfatizar bem a palavra encontrada, e assim sucessivamente até as palavras acabarem. Peça que identifiquem se houver formação de palavras por composição ou derivação. Ao final desta atividade, proponha que os mesmos grupos tentem encontrar outras palavras que tenham palavras escondidas. Refaça o exercício anterior, desta vez com as palavras que os grupos propuseram. Esse exercício estimula a busca por palavras novas e o aumento do vocabulário das crianças, assim como a proximidade com a nossa língua portuguesa que tem muito a nos oferecer.

Uma outra atividade que poderá ser desenvolvida é a criação de uma história assim como o livro se propõe. Contudo, dessa vez, utilizando essa nova lista de palavras. Se for necessário, leia junto com eles novamente o livro para que as ideias se iluminem. Incentive e aguce a imaginação dos estudantes apresentando narrativas possíveis com as palavras propostas. Lembre que é importante que o professor acompanhe o processo da escrita, tirando dúvidas e ajudando em possíveis dificuldades. Relembre que eles podem usar ilustrações para ajudar a contar essa história e estimule a realização de desenhos que comuniquem conforme a preferência de cada grupo.

Dessa forma, além de desenvolver a prática da escrita, o aluno consegue organizar melhor suas impressões da obra. Pode, assim, equilibrar os aspectos que mais lhe agradaram durante a leitura, bem como o que apreendeu a partir dos debates e dos exercícios desenvolvidos em sala de aula. Como forma de orientar e estimular as (os) estudantes, a (o) professora (r) pode relembrar as atividades em torno da tarefa de pré-leitura, na qual diversas palavras foram reveladas e como a construção de uma narrativa através de palavras pode ser rica e atrativa para leitoras (res).

\section{Sugestões de referências complementares}

\subsection{Livros} 

\begin{itemize}
\item \textsc{rylant}, Cynthia. \textit{A velhinha que dava nome às coisas}. São Paulo: Brinque-Book, 2002.

Livro que conta a história de uma velhinha que já não tinha nenhum amigo, pois todos eles haviam morrido. Por isso, ela começa a dar nome às coisas que durariam mais que ela: sua casa, seu carro, sua poltrona, até o dia em que um cachorrinho apare no seu portão.

\item \textsc{bandeira}, Pedro. \textit{Mais respeito, eu sou criança!}. São Paulo: Moderna, 2009.

Livro que fala sobre a necessidade de as crianças expressarem seus sentimentos e suas memórias por meio da via literária.

\end{itemize}

\section{Bibliografia comentada}
\subsection{Livros}

\begin{itemize}
\item \textsc{brasil}. Ministério da Educação. Base Nacional Comum Curricular. Brasília, 2018.

Consultar a \textsc{bncc} é essencial para criar atividades para a turma. Além de especificar quais habilidades precisam ser desenvolvidas em cada ano, é fonte de informações sobre o processo de aprendizagem infantil. 

\item \textsc{meireles}, Cecília. \textit{Ou isto ou aquilo}. São Paulo: Global Editora, 2012.

O tema do livro é que a vida é feita de escolhas que muitas vezes são difíceis de resolver, o cotidiano marcado pela dúvida e pela dificuldade de decisão, de forma poética e brincando com os pronomes demonstrativos.

\item \textsc{bojunga}, Lygia. \textit{A bolsa amarela}. São Paulo: Casa Lygia Bojunga, 2013.

A bolsa amarela é o romance de uma menina que entra em conflito consigo mesma e com a família ao reprimir três grandes vontades que ela esconde numa bolsa amarela.

\item \textsc{porto}, Cristina. \textit{O diário escondido da Serafina}. São Paulo: Ática, 2021.

Livro que traz a história de Serafina, uma menina que tem seu diário e deixa escondido na casa de seu avô.

\item \textsc{coelho}, Nelly Novaes. Literatura infantil, teoria, análise, didática. 1ª ed. São Paulo: Moderna, 2000.

Livro que fala sobre os espaços da literatura infantil na contemporaneidade e a importância de as crianças estarem ligadas ao seu imaginário pela via literária.

\item \textsc{paixão}, Fernando. \textit{Poesia a gente inventa}. São Paulo: FTD Educação, 2019.

Livro que traz cinco poemas para crianças, vinculados às ilustrações.

\end{itemize}

\subsection{\textit{Sites}}

\begin{itemize}
\item Artigo "A Importância da Leitura dos Contos de Fadas na Educação Infantil", por Ana Maria da Silva. Disponível em: \url{https://siteantigo.portaleducacao.com.br/conteudo/artigos/educacao/a-importancia-da-leitura-dos-contos-de-fadas-na-educacao-infantil/30151}. 
Acesso em 20 dez. de 2021.

No artigo, a autora fala sobre a importância da construção do imaginário pela via da literatura para as crianças, trazendo elementos que analisam o mundo pós-moderno e os espaços que a literatura infantil, principalmente os contos, devem ter.

\item Reportagem "Biografia para crianças, a palavra de quem faz" Disponível em: \url{http://www.multirio.rj.gov.br/index.php/leia/reportagens-artigos/reportagens/3059-biografias-para-criancas-a-palavra-de-quem-faz}. Acesso em 24 dez. de 2021

Reportagem que fala sobre como contar histórias biográficas para crianças.

\end{itemize}

\subsection{\textit{Filmes}}

\begin{itemize}
\item \textit{O menino maluquinho}. Dirigido por Helvécio Ratton, 1995.

No final dos anos 1960, o Menino Maluquinho é um garoto normal, feliz e bem cuidado por sua família que, enquanto aproveita a infância brincando na rua com a turma, observa o mundo que o cerca e aprende a lidar com a vida.

\item \item \textit{Meu pé de laranja lima}. Dirigido por Marcos Bernstein, 2013.

Zezé é um garoto de oito anos que, apesar de levado, tem um bom coração. Ele leva uma vida bem modesta, devido ao fato de seu pai estar desempregado há bastante tempo, e tem o costume de ter longas conversas com um pé de laranja lima que fica no quintal de sua casa. Até que, um dia, conhece Portuga, um senhor que passa a ajudá-lo e logo se torna seu melhor amigo.


\end{itemize}
\end{document} 
