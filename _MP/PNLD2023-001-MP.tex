\documentclass[11pt]{extarticle}
\usepackage{manualdoprofessor}
\usepackage{fichatecnica}
\usepackage{lipsum,media9}
\usepackage[justification=raggedright]{caption}
\usepackage[one]{bncc}
\usepackage[araucaria]{../edlab}
\usepackage{marginnote}
\usepackage{pdfpages}

\newcommand{\AutorLivro}{Bart Mertens}
\newcommand{\TituloLivro}{Ninguém e eu}
\newcommand{\Tema}{Descoberta de si: Personagens/sujeitos líricos vivenciando a percepção do corpo; sentimentos; ações e linguagem}
\newcommand{\Genero}{Poesia; poema; trava-línguas; parlendas; adivinhas; provérbios; quadrinhas e congêneres}
%\newcommand{\imagemCapa}{./images/PNLD2022-001-01.jpeg}
\newcommand{\issnppub}{XXX-XX-XXXXX-XX-X}
\newcommand{\issnepub}{XXX-XX-XXXXX-XX-X}
% \newcommand{\fichacatalografica}{PNLD0001-00.png}
\newcommand{\colaborador}{Ana Lancman}
\begin{document}

\title{\TituloLivro}
\author{\AutorLivro}
\def\authornotes{\colaborador}

\date{}
\maketitle

%\begin{abstract}\addcontentsline{toc}{section}{Carta ao professor}
%\pagebreak

\tableofcontents
\pagebreak

\begin{abstract}

Caro professor,\medskip

O presente manual tem como objetivo oferecer uma orientação sobre a obra \textit{Ninguém e eu}. A partir deste manual, você poderá incentivar a prática da leitura aos estudantes e proporcionar um conteúdo enriquecedor. Apresentamos aqui sugestões de atividades a serem realizadas antes, durante e após a leitura do livro, com propostas que buscam introduzir os gêneros literários e aprofundar as discussões trazidas pelas obras. Você encontrará informações sobre o autor, sobre o gênero e sobre os temas trabalhados ao longo do livro. Ao fim do manual, você encontrará também sugestões de livros, artigos e sites selecionados para enriquecer a sua experiência de leitura e, consequentemente, a de seus estudantes.

A obra \textit{Ninguém e eu} foi escrita por Bart Mertens e ilustrada por Benjamin Leroy. É um livro ilustrado que trata de temas como solidão, tempo, descobertas e mudanças. \textit{Ninguém e eu} conta a história de ``Ninguém'', um menino que estava sozinho sem ninguém para conversar e brincar. De repente ele encontra ``Algo'', uma bola para brincar de pega-pega, mas com o passar do tempo continua se sentindo sozinho. Até que conhece ``Eu'', uma menina que tinha perdido ``Algo'' e também buscava companhia.

Esta narrativa trata da importância de estar junto de amigos e é uma reflexão sobre aprender a conviver com o outro. 
A obra também aborda a importância das brincadeiras tradicionais na formação das crianças, uma vez que é através da vontade de brincar que as personagens do livro se encontram e fazem companhia uma à outra. Esperamos que as atividades sugeridas e o material indicado sejam proveitosos em sala de aula. 

\end{abstract}

\section{Sobre o livro}

\paragraph{O livro} \textit{Ninguém e eu}, de Bart Mertens, é uma obra sensível e com grande potência poética. É um texto que 
proporciona forte identificação, ao tratar de temas tão humanos como `sentir-se sozinho', `conectar-se com seus desejos', `buscar o que está faltando' e `encontrar companhias ao longo da vida'. A obra também trata da passagem do tempo, em que as marcas temporais se misturam e se confundem de um momento para outro, variando entre ``horas, dias, semanas, anos e séculos''\footnote{Ver p.\,6 de \textit{Ninguém e eu}}.

Através de brincadeiras com pronomes, o texto cria um efeito muito interessante. Ao princípio, pode causar estranhamento, mas ao mergulharmos na história, compreendemos facilmente. O texto e a ilustração trabalham juntos e apresentam um visual gráfico impecável, que enriquece ainda mais a experiência. 

\SideImage{O gênero poético incentiva a curiosidade e a imaginação. (LACMA/Remedios Varo; CC BY-NC 2.0)}{PNLD2023-001-07.png}

\paragraph{Descrição} \textit{Ninguém e eu} apresenta ``Ninguém'', um menino que se sente extremamente sozinho. Cheio de tédio, não tinha nenhum lugar pra ir e ninguém para conversar e brincar. De repente, surge ``Algo'', uma bola para brincar de pega-pega. ``Ninguém'' e ``Algo'' se divertem muito juntos, mas com o passar do tempo ``Ninguém'' continua se sentindo sozinho. Até que ``Algo'' parte e ``Ninguém'' decide segui-lo. Os dois percorrem um longo caminho juntos, que os leva até ``Eu'', uma menina que tinha perdido ``Algo'' muito tempo atrás. Ela também buscava companhia e quando conhece ``Ninguém'', os dois se apaixonam à primeira vista e decidem se acompanhar para não se sentirem mais sozinhos.

\section{Sobre os autores}

\paragraph{O autor} Bart Mertens nasceu em 1980 na cidade de Halles, na Bélgica. \textit{Ninguém e eu} é o seu segundo livro voltado para o público infantil. Participou ao longo de cinco anos como ator de um dos principais grupos de teatro de bonecos de Flandres, um trabalho que, segundo ele, o fazia voltar a ser criança. 

\paragraph{O ilustrador} Benjamin Leroy nasceu em 1980 na cidade de Neerpelt, na Bélgica. Trabalhou durante alguns anos com animação e estudou de ilustração em Hasselt. Formou-se em 2003 e desde então trabalha com livros ilustrados.

%\Image{Foto do autor (Arquivo pessoal)}{PNLD2023-001-02.png}

\SideImage{Foto do ilustrador (Arquivo pessoal)}{PNLD2023-001-03.png}

\section{Sobre o gênero}

\paragraph{O gênero} O gênero deste livro é o da \textit{poesia}. 


\paragraph{Descrição} Um dos meios mais expressivos de comunicação e inovação da linguagem, a poesia é uma das mais antigas formas de arte literária, anterior até mesmo à escrita, pois existe desde a tradição oral. Ela combina palavras, significados, sonoridades, ritmos e, muitas vezes, também imagens para permitir uma experiência estética. A linguagem poética é condensada e emotiva e busca trabalhar a língua de forma que o leitor experimente as palavras de uma forma nova. Na maior parte das vezes, a poesia é dividida em versos que, juntos, são chamados de estrofes. O ponto de vista do autor e sua visão pessoal do mundo estão muito presentes nesse tipo de texto e, justamente por essa particularidade, a experiência da leitura de uma poesia é extremamente individual e subjetiva.

\paragraph{Interação} Esse gênero é um grande aliado na formação do leitor. O olhar da criança para o mundo é, em essência, um olhar poético, calcado na curiosidade pelo mundo. A poesia é a forma perfeita de valorizar esse olhar e incentivar que a criança brinque com as palavras, observe os sons e experimente novos ritmos. Por sua liberdade e criatividade, a poesia tem potencial para estabelecer um diálogo único com os pequenos leitores. A presença de fantasias, imagens, repetição e símbolos permite uma maior identificação, pois a criança ainda está construindo seu mundo interior e experimenta a vida de forma diferente do adulto. 

\paragraph{Competências} 
O caráter polissêmico do texto poético pode e deve ser explorado no ambiente escolar, assim como a dimensão lúdica da linguagem e as suas possibilidades. A própria estrutura do poema já produz aprendizado: ela seduz e desafia o leitor, apresenta ritmos, efeitos sonoros e, ao mesmo tempo, apresenta novas vivências, oferecendo possibilidades para a criança simbolizar suas próprias experiências. Cada dupla de páginas do livro \textit{Ninguém e eu} apresenta composições de versos e ilustrações. Assim, a leitura da poesia se faz em paralelo com a observação de uma ilustração que sugere caminhos de sentido e interpretação à criança. A leitura do poema, realizada pelo educador, aumenta o repertório do aluno, incentiva o desenvolvimento do vocabulário e da fluidez do discurso. A associação entre a aquisição da linguagem e a poesia, ademais, permite explorar múltiplas competências ao mesmo tempo, pois relaciona os princípios linguísticos à linguagem poética, introduzindo o aluno no universo lúdico e artístico da poesia.

\section{Atividades}

\subsection{Pré-leitura}

\BNCC{EF02HI02}
%Produzir +Descrever: ``práticas e papéis sociais em diferentes comunidades"
\BNCC{EF02HI03}
%Produzir +Registrar: ``situações cotidianas relacionadas a mudança, pertencimento e memória'' 
\BNCC{EF02HI08}
%Produzir +Escrever: ``histórias da família e/ou da comunidade'' 
\BNCC{EF02GE02}
%Identificar +Comparar: ``costumes e tradições de diferentes populações do bairro'' 
\BNCC{EF02ER03}
%Produzir +Registrar: ``memórias pessoais, familiares e escolares (fotos, músicas, narrativas, álbuns)'' 

\paragraph{Tema} As brincadeiras de antigamente e de hoje em dia.

\paragraph{Conteúdo} Conversa em sala de aula sobre brincadeiras e pesquisa com familiares.

\paragraph{Objetivo} Incentivar os alunos a pesquisar sobre os costumes e tradições de sua família. Proporcionar uma reflexão sobre a importância de brincar e as mudanças que as brincadeiras tiveram ao longo do tempo.

\Image{Crianças brincando com carrinho de rolimã. (Flickr/Festival Contato; CC BY 2.0)}{PNLD2023-001-10.png}
\paragraph{Justificativa} É direito das crianças e adolescentes ter a liberdade para brincar,\footnote{A legislação brasileira reconhece explicitamente o direito de brincar, tanto na Constituição Federal (1988), artigo 227, quanto no Estatuto da Criança e do Adolescente – ECA (1990), artigos 4º e 16. Outros direitos guardam direta relação com o brincar, tais como o direito ao lazer (art. 4º), direito à liberdade e à participação (art. 16), comum à pessoa em desenvolvimento (art. 71). Ver 
\href{http://primeirainfancia.org.br/eca-e-o-direito-de-brincar-por-marilena-flores-martins-do-ipa-brasil/}{O direito de brincar, por Marilena Flores Martins}.
} 
praticar esportes e divertir-se. Com essa atividade, os alunos poderão compreender que as brincadeiras sempre estiveram presentes ao longo do tempo e que passaram por transformações, acompanhando as trajetórias de suas famílias e as mudanças histórico-geográficas de seus bairros e cidades. 

\paragraph{Metodologia} Comece a atividade com uma conversa com os alunos sobre suas brincadeiras favoritas. Cada estudante pode escolher uma brincadeira que gosta e explicar por que é sua preferida. Sugere-se que sejam feitas perguntas aos alunos para fomentar o debate, tais como:

\begin{itemize}

\item Você costuma brincar na rua?

\item A brincadeira que você escolheu é para brincar na rua ou em casa?

\item Você sabe quais brincadeiras eram mais frequentes antigamente?

\item Alguém da sua família já te apresentou uma brincadeira que você não conhecia? Como foi?

\end{itemize}


Em seguida, será realizada uma breve pesquisa por parte de cada aluno com seus familiares. Eles vão entrevistar as pessoas próximas de sua família sobre as brincadeiras mais frequentes de sua infância. Na aula seguinte, proponha que os estudantes façam uma tabela comparativa das semelhanças e diferenças entre as brincadeiras de antigamente e as de hoje em dia.

\Image{Crianças brincando de futebol na rua (Arley Cruzper; CC-BY-SA-4.0)}{PNLD2023-001-08.png}

A partir dessa reflexão e com o auxílio dos professores de História e Geografia, cada aluno produzirá um texto sobre a transformação das brincadeiras ao longo do tempo. Incentive que seja explorada na redação o tema da brincadeira de rua, suas permanências e mudanças em cada contexto familiar e geográfico. Podem ser identificados distintos contextos sociais e diferenças culturais a partir dos textos, a ser conversado em sala de aula no fim da atividade.

\paragraph{Tempo estimado} Duas aulas de 50 minutos.

\subsection{Leitura}

\BNCC{EF15LP16} 
%Ler: narrativas, contos, crônicas; +grupo, +professor e +sozinho; Mundo imaginário;
\BNCC{EF15LP04}
%Identificar: ``interpretação de imagem'', recursos gráfico-visuais, multissemióticos;
\BNCC{EF15LP11} 
%Identificar: +Falar: conversação; ``roda de conversa'', ``discussão";
\BNCC{EF02LP17}
%Identificar: experiências pessoais, informalidade, ``cagegorias temporais: antes, ontem, hoje, amanhã; 
\BNCC{EF02LP17}
%Identificar: experiências pessoais, informalidade, ``cagegorias temporais: antes, ontem, hoje, amanhã;
\BNCC{EF02MA18}
%Identificar: indicar duração de intervalos de tempos;

\paragraph{Tema} O enredo de \textit{Ninguém e eu} e seus aspectos gramaticais.

\paragraph{Conteúdo} Compreensão de texto e dos recursos gráficos-visuais, além de estudo sobre termos gramaticais relevantes para a compreensão da obra.

\paragraph{Objetivo} Incentivar os estudantes a identificar os diferentes momentos da narrativa e ampliar seu entendimento sobre as questões trazidas pela obra.

\paragraph{Justificativa} A obra \textit{Ninguém e eu} oferece uma leitura impactante e proporciona reflexões que podem ser aprofundadas de forma efetiva em uma atividade de leitura conjunta. O professor também poderá auxiliar na compreensão da brincadeira feita com os pronomes ao longo do livro, que poderá ser um material enriquecedor para a sala de aula.

\paragraph{Metodologia} Proponha uma primeira leitura conjunta em sala de aula. Note que a brincadeira com os pronomes pode causar certo estranhamento com os alunos. Trabalhe primeiramente a interpretação de texto de forma coletiva, em que poderão ser feitas questões como:

\begin{itemize}

\item O que acontece nessa história? 

\item Se puderem separar a história em três momentos, quais seriam?

\item Quem são os personagens?

\item Escolham uma página que melhor represente cada um dos personagens.

\item Qual o momento mais marcante da narrativa para vocês?

\end{itemize}

Em seguida, será feita uma nova leitura do texto, dessa vez silenciosa e individual. Nesse momento proponha um exercício de gramática, em que serão identificados os pronomes utilizados como substantivos próprios. Trabalhe os tipos de pronomes a partir das páginas do livro. É importante que os estudantes compreendam que os pronomes indefinidos ``Ninguém'' e ``Algo'' e o pronome pessoal ``Eu'' são nomes de personagens ou objetos animados. 

\Image{Ilustração do livro, página 10}{PNLD2023-001-04.png}

\Image{Ilustração do livro, página 21}{PNLD2023-001-05.png}

Outro tópico interessante a ser trabalhado com os alunos é a percepção temporal através dos jogos de palavras realizadas na narrativa. Peça que todos leiam novamente a página 16 do livro. Aborde a questão da verossimilhança de terem se passado ``horas, dias, semanas, séculos'' ao longo da história. Sugere-se que seja feito um debate acerca da utilização das categorias temporais e seus efeitos poéticos.

\Image{Ilustração do livro, página 16}{PNLD2023-001-06.png}

\paragraph{Tempo estimado} Duas aulas de 50 minutos.

\subsection{Pós-leitura}

\BNCC{EF12EF02}
%Identificar: brincadeiras e os jogos populares com corporal, visual, oral e escrita;
\BNCC{EF15AR24}
%Produzir: experimentar brinquedos, brincadeiras, jogos, danças, canções e histórias;
\BNCC{EF02HI01}
%Identificar +Reconhecer: ``aproximação e separação de pessoas em espaços sociais'' 
\BNCC{EF02CI03}
%Identificar: ``prevenção de acidentes domésticos"; ``ex: objetos cortantes'' 

\paragraph{Tema} O ``Algo'' que falta para cada um.

\paragraph{Conteúdo} Oficina de brinquedos.

\paragraph{Objetivo} Exercitar a criatividade e aprofundar as discussões feitas na atividade de pré-leitura e leitura.

\paragraph{Justificativa} A partir de uma reflexão sobre a importância da brincadeira e de estar junto, essa atividade proporcionará um momento lúdico para os estudantes. Poderão utilizar materiais artísticos e reciclados para criar objetos próprios e em seguida compartilhar com os colegas, em um momento de convivência em grupo.

\paragraph{Metodologia} Sugira para os estudantes que criem o seu próprio ``Algo'' que estava faltando. Será uma atividade criativa, em que os alunos inventarão seus próprios brinquedos. Ofereça os materiais artísticos que tiverem disponíveis em sala de aula e peça para que os alunos tragam objetos reciclados de sua casa.

\Image{Brinquedo feito com materiais recicláveis. (Flickr/Tom B; CC BY-NC-SA 2.0)}{PNLD2023-001-09.png}

No primeiro momento, deixe os estudantes criarem livremente. Circule em sala de aula para acompanhar suas criações e auxilie no que for necessário. Oriente os alunos para que os objetos criados sejam seguros de brincar e manusear, evitando que usem objetos cortantes ou materiais tóxicos. Devem terminar de construir seus brinquedos no início da aula seguinte, caso necessário.

Em seguida, será o momento de compartilhamento de objetos. Divida a sala em grupos pequenos e proponha que os estudantes coloquem seus brinquedos no centro da roda. Cada um pode brincar com o objeto do outro e ver como esses brinquedos podem se misturar em uma só brincadeira. Pode ser interessante que os estudantes inventem um jogo que una esses diferentes brinquedos. Podem formular as regras do jogo e qual a função de cada brinquedo.

Ao final da atividade, reúna todos os alunos para uma roda de conversa, em que poderão compartilhar como foi a experiência de dividir seus próprios brinquedos e brincar com os dos outros. Caso os grupos tenham inventado jogos com esses brinquedos, poderão apresentar para os colegas. Pode ser interessante abordar o tema da convivência em grupo a partir da brincadeira.

\paragraph{Tempo estimado} Duas aulas de 50 minutos.

\section{Sugestão de referências complementares}

\subsection{Filmes}

\begin{itemize}

\item \textit{Coraline e o mundo secreto}. Dirigido por Henry Selick, 2009.

A animação conta a história de Coraline, uma menina que se sente entediada em sua nova casa. De repente, ela encontra uma porta secreta que é um portal para uma versão melhorada de sua própria vida. Rapidamente ela percebe que esse mundo aparentemente perfeito torna-se assustador.

\item \textit{Toy Story}. Dirigido por John Lasseter, 1995.

A clássica animação apresenta um mundo em que os brinquedos têm vida própria quando seus donos estão dormindo. 

\end{itemize}

\subsection{Livros}

\begin{itemize}

\item \textit{Brincadeiras de todos os tempos}. Anna Claudia Ramos, Editora Larousse, 2005.

O livro, indicado para a faixa etária de 6 a 8 anos, conta sobre brincadeiras atemporais que envolvem avós e netos.

\item \textit{Manual de Brincadeiras}. Secretaria Municipal de Educação, Prefeitura de São Paulo, 2006.

Este manual apresenta diversas brincadeiras infantis e orienta os professores sobre os aspectos pedagógicos relacionados a cada uma. Disponível no link \url{http://portal.sme.prefeitura.sp.gov.br/Portals/1/Files/15107.pdf}.

\end{itemize}

\subsection{Museus}

\begin{itemize}

\item \href{https://museucatavento.org.br/}{Catavento – Museu de Ciências}

Localizado no centro da cidade de São Paulo, o museu é dividido em quatro seções: universo, vida, engenho e sociedade. É um museu educativo e interativo.

Endereço: Av Mercúrio, s/n - Pq Dom Pedro II, Centro, São Paulo - SP

\item \href{http://www.museudosbrinquedos.org.br/}{Museu dos Brinquedos}

Localizado em Belo Horizonte em uma casa que é patrimônio histórico, o museu apresenta 800 brinquedos em exposição permanente e possui mais de 5 mil brinquedos no acervo.

Endereço: Avenida Afonso Pena, 2564, Funcionários – Belo Horizonte (MG)

\end{itemize}

\subsection{Vídeos}

\begin{itemize}

\item \textit{10 curiosidades sobre brinquedos e brincadeiras de antigamente}, do canal Tempo Junto. Disponível no \href{https://www.youtube.com/watch?v=oGzmQm7RZGQ&ab_channel=Tempojunto}{Youtube}. 

Neste vídeo, é apresentada a história de diversos brinquedos e brincadeiras de antigamente.

\item \textit{Brinquedos com materiais recicláveis}, do canal Gerando Raízes. Disponível no \href{https://www.youtube.com/watch?v=qeNmZrRDfrw&ab_channel=GERANDORAIZES}{Youtube}. 

Neste vídeo o professor Gustavo Carneiro ensina a fazer três brinquedos diferentes a partir de materiais recicláveis. Pode ser uma inspiração para a atividade de pós-leitura.

\item \textit{Como fazer ioiô com materiais recicláveis}, de Terezinha da Souza e Silva. Disponível no \href{https://www.youtube.com/watch?v=iWO3JFAAvJc&ab_channel=TerezinhadeSouzadaSilva}{Youtube}. 

Neste tutorial Terezinha ensina a fazer um ioiô com tampinhas de garrafa \textsc{pet}, \textsc{eva}, parafuso e barbante. Também pode ser útil para a atividade de pós-leitura. 

\end{itemize}

\section{Bibliografia comentada}

\begin{itemize}
\item \textsc{brasil}. Ministério da Educação. Base Nacional Comum Curricular. Brasília, 2018.

Consultar a \textsc{BNCC} é essencial para criar atividades para a turma. Além de especificar 
quais habilidades precisam ser desenvolvidas em cada ano, é fonte de informações sobre 
o processo de aprendizagem infantil. 
 
\item \textsc{brasil}. Ministério da Educação. Secretaria de Alfabetização. PNA Política Nacional de Alfabetização/Secretaria 
de Alfabetização. Brasília: \textsc{mec, sealf}, 2019.

Um guia fundamental para trabalhar a alfabetização de estudantes.

\item \textsc{van der linden}, Sophie. Para ler o livro ilustrado. São Paulo: Cosac Naify, 2011.

Livro sobre as particularidades do livro ilustrado, que apresenta as diferenças entre o livro ilustrado e o livro com ilustração. 
\end{itemize}

\end{document}
