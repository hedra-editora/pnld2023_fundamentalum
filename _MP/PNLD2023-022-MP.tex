\documentclass[11pt]{extarticle}
\usepackage{manualdoprofessor}
\usepackage{fichatecnica}
\usepackage{lipsum,media9}
\usepackage[justification=raggedright]{caption}
\usepackage[one]{bncc}
\usepackage[acorde]{../edlab}
\usepackage{marginnote}
\usepackage{pdfpages}

\newcommand{\AutorLivro}{Klévisson Viana}
\newcommand{\TituloLivro}{Lampião... Era o cavalo do tempo atrás da besta da vida}
\newcommand{\Tema}{O mundo natural e social}
\newcommand{\Genero}{Cordel em quadrinhos}
%\newcommand{\imagemCapa}{./pdf/capa.jpg}
\newcommand{\issnppub}{XXX-XX-XXXXX-XX-X}
\newcommand{\issnepub}{XXX-XX-XXXXX-XX-X}
% \newcommand{\fichacatalografica}{PNLD0001-00.png}
\newcommand{\colaborador}{Renier Silva}

\begin{document}

\title{\TituloLivro}
\author{\AutorLivro}
\def\authornotes{\colaborador}

\date{}
\maketitle

\tableofcontents


\begin{abstract}

Caros professores, 

gostaríamos de apresentar a obra \textit{Lampião... Era o cavalo do tempo atrás da besta da vida},
de Klévisson Viana. Publicada pela primeira vez em 1998 e reeditada pela Editora Hedra, 
trata-se de um \textbf{cordel em quadrinhos} que conta a última batalha de Lampião, aquela que 
o levou à morte, depois de anos de atividade no cangaço sendo perseguido pelas autoridades governamentais,
tanto regionais quanto federais. 

Klévisson Viana, o autor da obra, responsável tanto pela parte escrita quanto pela visual, 
é já há algum tempo consagrado no estilo. Natural de Quixeramobim, no interior do estado do Ceará,
tem em seu repertório tanto histórias em quadrinho quanto livros mais tradicionais e cordéis. 
Suas obras já foram publicadas em países como França, Portugal, Holanda e Turquia. 
Em \textit{Lampião... Era o cavalo do tempo atrás da besta da vida}, você terá a oportunidade de trabalhar com 
os alunos do Ensino Fundamental \textsc{i} um texto ao mesmo tempo lúdico, criativo e informativo. 

Dividido em duas partes, o livro apresenta, primeiro, a história da última batalha de Lampião
numa estrutura de \textit{flashback}, cujo tempo presente é o do narrador sertanejo. 
No decorrer da narrativa, aparecem referências diretas a passagens históricas, como o estabelecimento 
do Estado Novo de Getúlio Vargas, que propunha o fim do coronelismo e do cangaço,
bem como a figuras como o líder messiânico Antônio Conselheiro e outros cangaceiros. 
Mais à frente, na segunda parte do livro, são apresentados elementos da cultura sertaneja nordestina, numa tentativa de oferecer,
segundo o autor, um retrato de sua cultura, em primeira mão, 
sem partir de referências intermediárias e, na maior parte das vezes, redutoras
em seus preconceitos, como o cinema estadunidense. Instrumentos do quotidiano, como a viola sertaneja,
a urupema, o alguidar, além da própria figura do \textbf{cangaceiro} são constituídas verbal e visualmente. 

Esperamos, professor ou professora, que este material sirva como um guia 
para seu trabalho em sala de aula. Já contamos, no entanto, com as adaptações
que surgirão organicamente na recepeção de vocês, que possuem 
trajetórias e escolhas didáticas específicas, bem como no contato com os 
alunos, que tanto têm a oferecer para o enriquecimento da experiência didática.

Boa aula!

\end{abstract}

\section{Sobre o livro}

\textit{Lampião... Era o cavalo do tempo atrás da besta da vida} é uma versão em quadrinhos da 
história do lendário cangaceiro pernambucano Virgulino Ferreira, mais conhecido pela alcunha de \textit{Lampião},
que, diz-se, faz referência a sua habilidade em manejar o rifle,
o que, nas noites escuras da caatinga, garantia grandes clarões.

Nascido em 1898 em Serra Talhada, sertão do estado de Pernambuco, Lampião, \textit{o Rei do Cangaço}, 
foi ao mesmo tempo herói e contraventor. O maior líder do banditismo brasileiro do século \textsc{xx},
representava, para as elites, um perigo brutal a ser extinto, e para o povo das
classes oprimidas, a honra e a insubmissão frente às autoridades, símbolos das injustiças sociais. 

Virgulino trabalhou, até os 21 anos, como artesão. Era alfabetizado e usava óculos para leitura, 
características bastante incomuns para a região sertaneja e pobre onde habitava.
Sua entrada no cangaço se explica pelas disputas que sua família travava com outras famílias locais, 
em geral por limites de terras. Em decorrência disto, seu pai foi morto em confronto com a polícia, em 1919. 
Virgulino, então, jurou vingança, e, junto de mais dois irmãos, passou a integrar o grupo do cangaceiro Sinhô Pereira.

Tornou-se líder do bando, em 1922, e, em 1930, se juntou afetivamente a sua icônica companheira Maria Bonita, 
na Bahia. No mesmo ano, foi retratado no \textit{The New York Times} como um ``moderno Robin Hood''. Esse mesmo jornal também noticiaria, futuramente, seu assassinato. Em 1936, seu cotidiano na caatinga foi fotografado e 
filmado por Benjamin Abrahão Botto. 

\textit{Lampião... Era o cavalo do tempo atrás da besta da vida} é dividido em duas partes: na primeira, é apresentado o episódio da morte de Lampião através 
de uma técnica narrativa de «mise en abîme», na qual uma história é contada dentro de outra --- dois sertanejos sentam-se para conversar e relembram o acontecido; já na segunda parte, a terminologia e o acervo do universo 
dos objetos específicos do sertão nordestino -- tanto os que foram vistos na história quanto outros complementares -- são apresentados ao leitor. 

A linguagem escrita neste quadrinho é importante pois não segue a norma-padrão da língua portuguesa, mas
está mais próxima de uma transcrição da variante sertaneja do português falado na região do Nordeste brasileiro. 
A intenção do autor nesta obra é fornecer um produto cultural autêntico da cultura sertaneja e cabocla
brasileira, de modo que seja cada vez mais desnecessário recorrer a referências externas e estereotipadas. 

\reversemarginpar
\marginparwidth=5cm


\section{Sobre o autor}

\paragraph{Klévisson Viana}

Nascido em Quixeramobim, no estado do Ceará, Klévisson Viana, nome artístico de Antônio 
Clévisson Viana Lima, é um poeta popular, escritor, cordelista, roteirista, cartunista, 
xilogravador, editor e presidente da \textsc{aestrofe} --- Associação de Escritores, 
Trovadores e Folheteiros do Estado do Ceará. É também membro da \textsc{ablc} --- Academia 
Brasileira de Literatura de Cordel (\textsc{rj}). Coordena o projeto editorial da 
Tupynanquim Editora, na qual já publicou cerca de mil obras de quase uma centena de autores. 

Como autor, Klévisson Viana publicou mais de 30 livros e quase 200 folhetos de 
Literatura de Cordel. Seus trabalhos fluíram pelos quadrinhos, pela televisão e por 
adaptações para o teatro. Destaca-se o folheto \textit{A quenga e o delegado}, transformado 
em episódio da série Brava Gente da Rede Globo. Tem trabalhos publicados em diversas 
editoras nacionais e internacionais como Chandeigne --- Paris (\textsc{fr}), Editora Leya 
--- Lisboa (\textsc{pt}), Editora Hedra --- São Paulo (\textsc{br}), Nova Alexandria --- 
São Paulo (\textsc{br}), Editora Demócrito Rocha --- Ceará (\textsc{br}), 
Editora Amarilys --- São Paulo (\textsc{br}), Edelbra --- Porto Alegre (\textsc{br}), 
Nova Alexandria --- São Paulo (\textsc{br}), dentre outras. Tem outras obras publicadas 
em antologias na Turquia, Israel, Bélgica, Itália e Holanda. 


\Image{Edição francesa de um livro em cordel de Klévisson Viana.(Arquivo do site do autor.)}{PNLD2023_022_04.jpg}

Dentre sua extensa obra podemos encontrar os livros \textit{Sertão menino}, de 2008, 
\textit{Abecedário dos bichos}, de 2013, \textit{O Guarani em cordel}, de 2013, e 
\textit{Miolo da rapadura}, de 2017; os álbuns em quadrinhos \textit{O mundo do Cajulino}, 
de 1993, \textit{Lampião... Era o cavalo do tempo atrás da besta da vida}, de 1999, 
\textit{Admirável riso novo}, de 2004, e \textit{O cangaceiro do futuro e o jumento espacial}, 
de 2017; e os folhetos de cordel \textit{A chegada de Ariano Suassuna no céu}, 
\textit{Carta de um jumento a Jô Soares}, \textit{Cinco anos do São Paulo capital do Nordeste}, 
\textit{A triste partida de Patativa do Assaré}, \textit{O cordelista na França}, e 
\textit{Seu Lunga --- o homem mais zangado do mundo, volumes \textsc{i}, \textsc{ii} e \textsc{iii}}.

Klévisson Viana tem prêmios importantes no currículo. Foi vencedor seis vezes consecutivas do 
\textsc{pnbe} --- Programa Nacional da Biblioteca Escolar, do \textsc{mec}, recebeu três 
vezes o Troféu \textsc{hq} Mix, uma vez do \textsc{pnaic}, Programa Nacional de Alfabetização 
na Idade Certa, do \textsc{mec} e "Prêmio Jabuti de Literatura" concedido anualmente pela 
Câmara Brasileira do Livro (\textsc{cbl}), dentre outros. 

Klévisson Viana coordena eventos culturais, ministra palestras, oficinas e recitais em 
todo o Brasil e já levou sua arte a países como França, Portugal, México, Cabo Verde e Costa Rica. 


\section{Sobre o gênero}

%55 caracteres
\paragraph{O gênero} O gênero deste livro é \textit{cordel em quadrinhos}. 


Alguns estudiosos defendem que o termo \textit{cordel} venha de Portugal, onde os \textit{folhetos} 
eram vendidos em feiras pendurados em barbantes, em cordões que se chamavam cordéis. Já para
outros, o cordel era assim chamado porque as brochuras eram encadernadas com barbantes. 
No Brasil, porém, não se falava em cordel. Somente a partir dos anos 1960, com a persistência 
dos pesquisadores europeus pelo nome, os poetas passaram a ser chamados de 
cordelistas. Para o público mais popular, no entanto, ele continua sendo chamado de \textit{romance} e 
\textit{folheto}.

São chamadas de \textit{cordel} as histórias curtas em versos rimados de personagens 
lendárias impressas em cadernos, geralmente artesanais, com ilustrações feitas sob a técnica da 
xilogravura, e comercializadas originalmente em feiras livres do Nordeste do Brasil. 
Sua origem remete às cantigas portuguesas medievais trazidas pelos colonos.
O cordel não tem nem um limite nem uma receita pronta. É o verso da 
nossa tradição popular brasileira. Hoje, o gênero versa sobre qualquer 
assunto, aborda qualquer temática e reflete o mundo do nosso tempo:

\begin{quote}
Não há um só grande acontecimento local, nacional, ou mesmo mundial que não tenha sido tratado 
pela literatura de cordel. O folheto mostra a realidade, mais do que os grandes meios de comunicação, 
porque não é atrelado a coisa alguma. É independente e é a opinião do autor. Não tem interesse em 
grupos econômicos, nem tem patrocinadores. Por isso, critica e aborda, como nenhum outro meio. 
Sendo honesto em suas abordagens, é natural que o cordel se sinta ameaçado --- da mesma forma que 
a televisão e o rádio ameaçaram o jornal impresso.\footnote{``Klévisson Viana --- Cordel para os intelectuais e folheto para o povo.'' Entrevista para o jornal \textit{A nova democracia, ano \textsc{i}, nº 8, abril de 2003.}}
\end{quote}

O cordel aqui em questão carrega influências de outro gênero literário, também
verbo-visual: a história em quadrinhos, ou \textsc{hq}. A \textsc{hq} é um 
gênero que trabalha ao mesmo tempo a linguagem verbal e a visual; trata-se, portanto, de 
uma \textbf{narrativa gráfica}. Não há hierarquia entre o texto e a ilustração: nem 
o texto é mera legenda da imagem, nem a imagem, mera ilustração do texto; são dois elementos 
de uma mesma obra, que deve ser lida como um todo.

Ambas as formas literárias exercitam a imaginação e a criatividade das crianças e dos jovens 
quando bem utilizadas. Podem servir de reforço à leitura e constituem uma linguagem altamente dinâmica. 
São linguagens que, ainda que de uma origem longínqua, são adequadas à nossa era devido à fluidez, 
à intensidade e sobretudo à abertura à inovação que as constitui.

\paragraph{A xilogravura}

Tanto o cordel quanto as histórias em quadrinhos têm algo em comum:
a presença de imagens. Ao se pensar em cordel, logo se pensa em \textit{xilogravura}, 
mas a xilogravura não surgiu com a literatura de cordel. Ela começou a fazer
parte dos folhetos a partir da década de 1950. Tradicionalmente, trata-se
de uma matriz de madeira que imita um clichê de chumbo. O clichê em si 
já é uma imitação da xilogravura, \textbf{uma técnica milenar dos egípcios
e chineses}: recorta-se uma figura em relevo sobre uma madeira. A figura 
em relevo imprime, como um carimbo sobre um papel em branco, e as partes
cortadas são os sulcos onde a tinta não aparece. 

A xilogravura entrou na vida da literatura de cordel como uma alternativa 
ao poeta sem recursos para ilustrar a capa de um folheto. Ela entrou indiretamente na 
estrutura do folheto e o público não se identificou de imediato. Atualmente, 
o público intelectualizado, o estudioso ou o turista que gostam de folheto preferem a capa com a xilogravura; o público 
tradicional, pro sua vez, gosta mais da capa com desenhos ou fotografias. 
Os autores e editores tentam sempre agradar a todos, trabalhando tanto com a xilogravura 
como com desenhos, figuras, etc.

\subsection{Pré-leitura}

\BNCC{EF35LP10}
\BNCC{EF35LP11}

O livro que o professor ou professora tem em mãos é especial:
aqui, serão trabalhados outros aspectos da linguagem para além do texto.
A narrativa da última batalha de Lampião é contada por meio de descrições, diálogos
e ilustrações. Mas a principal característica é que o registro da linguagem não é
o padrão, aquele que comumente está nos livros, mas um registro \textit{oral}.
O discurso do narrador e os diálogos e falas das personagens reproduzem o falar
do sertanejo nordestino, o mesmo utilizado pela personagem histórica que dá nome 
ao livro, Lampião, bem como pelo autor, seu conterrâneo. 
Como se trata de uma tradução de registros, do oral para o verbal, o professor
ou professora deve estar atento às possíveis dificuldades que serão encontradas pelos alunos. 

Ancorados na habilidade \textsc{ef35lp10} da \textsc{bncc}, professor e professora devem apresentar aos
alunos o universo dos gêneros orais, deixando clara sua importância equivalente aos
gêneros escritos. Para isso, uma distinção entre os dois deve ser feita. 
As atividades a seguir propõem um percurso para este trabalho de aproximação. 


\paragraph{Atividade 1.1}


\paragraph{Tema} Escrita e oralidade.


\paragraph{Conteúdo} Diferenças entre os registros da linguagem escrita e oral. 


\paragraph{Justificativa} É importante que os alunos sejam capazes de
identificar as características dos gêneros orais. Para isso, 
eles precisam estar cientes de que a oralidade é tão importante quanto a escrita,
e desfazer o preconceito de superioridade da última em relação à primeira.


\paragraph{Metodologia} Preferencialmente em roda, faça as seguintes perguntas aos alunos: 

\begin{itemize}
\item Para que serve a escrita?
\item O que é mais importante, falar ou escrever? 
\end{itemize}

Anote suas respostas e comentários na lousa. Incentive-os a discutir acerca do tema. 

Num segundo momento, conduza a discussão para que entendam que tanto o registro oral 
quanto o escrito são importantes para a comunicação humana. Apresente a escrita
como uma \textbf{padronização} da linguagem. Dê o exemplo da língua portuguesa,
idioma oficial de nove países, alguns deles de dimensões continentais como o Brasil, e apresente
todas as características que favorecem a \textbf{diversidade}, sobretudo na fala,
nos sotaques, mas também nos vocabulários usados em cada um desses países e suas regiões. Use como exemplo as variantes do português falado no Brasil.
Dê exemplos e peça outros, sempre com cuidado para não cair em esteriótipos xenofóbicos. 

Aproveite para apresentar o conceito de \textbf{xenofobia}: ``medo, aversão'' a ``estranhos'', em geral, imigrantes
de outros países ou outras regiões de um mesmo país. Dê alguns exemplos concretos de 
xenofobia, utilizando notícias de jornal, por exemplo. 

Mostre um vídeo de um português com um sotaque típico de Portugal falando, pode ser um trecho de 
telejornal. Depois, apresente uma versão transcrita da fala do mesmo. 
A partir daí, proponha a seguinte reflexão:

\begin{itemize}
\item O que foi mais fácil de entender, a primeira versão apenas oral ou a segunda,
com um texto escrito? 
\end{itemize}

A proposta é que os alunos entendam a função da escrita enquanto ferramente facilitadora
da comunicação. No entanto, ela não pode tomar o lugar da fala, ou de outra linguagem,
como a visual, como veremos na próxima atividade.

\paragraph{Atividade 1.2}

Em sequência à discussão sobre oralidade e escrita, pergunte aos alunos se seria possível,
com certeza absoluta, extrair informações acerca do locutor, a pessoa que está falando, 
somente a partir do texto transcrito. Faça o mesmo exercício da atividade anterior, 
só que com a ordem inversa: apresente transcrições de discursos de falantes de português de diferentes
origens: brasileiros do Nordeste e do Sul, moçambicanos e portugueses. Peça que leiam \textbf{em silêncio}. 
Após a leitura, pergunte de onde eles acham que são os autores desses textos. 
Depois, apresente a versão oralizada dos discursos e refaça a mesma pergunta. 
Espera-se que os alunos tenham bem mais facilidade desta vez, já que a fala e seus elementos, 
como prosódia e sotaque, dão informações acerca do falante. 
Finalize a atividade ressaltando a importância de ambos os registros, tanto o oral quanto o escrito.
Enquanto a escrita está mais ligada à padronização, a fala permite maior expressão das diferenças estudadas. 
Um registro não substitui o outro, mas os dois podem ser úteis quando trabalhados \textbf{em conjunto}.


\paragraph{Atividade 1.3}

No último momento das atividades de pré-leitura, o professor ou a professora deve 
mostrar aos alunos que, assim como a língua falada e a língua escrita são complementares
e auxiliares entre si, a imagem e o texto também o são. Esta atividade deve desembocar 
na aproximação dos alunos com a linguagem das histórias em quadrinhos e dos cordéis,
gêneros aos quais se relaciona o livro trabalhado em sala. 

Ainda que as palavras, escritas ou mesmo faladas, possam aproximar quem fala de quem escuta,
para que ambos entendam do que se está falando, nem sempre elas são suficientes para uma comunicação efetiva.
No caso das primeiras atividades, nas quais a transcrição do discurso não era suficiente para que
identificar os falantes, às vezes é necessário partir para uma 
outra linguagem, como a visual, por exemplo. 

Apresente aos alunos as seguintes palavras, e pergunte o que eles acham que significa:

\begin{itemize}
\item Urupema
\item Pé-de-bode
\item Alguidar
\item Viola nordestina
\end{itemize}

Então, mostre as seguintes ilustrações, retiradas do livro.
%No PDF, as imagens não apareceram na sequência

\Image{Ilustrações da segunda parte do livro dedicada a apresentar elementos da cultura sertaneja nordestina.}{PNLD2023_022_01.jpg} 

\Image{Ilustrações da segunda parte do livro dedicada a apresentar elementos da cultura sertaneja nordestina.}{PNLD2023_022_02.jpg} 

Se desconhecerem as palavas, os alunos não conseguirão imaginar o que elas significam. As imagens servião, nesse caso, vêm para 
\textbf{traduzir} o sentido. No caso de \textbf{viola nordestina}, os alunos 
que desconhecem esse objeto específico podem aproximar-se do significado, conhecendo o que é uma viola, mesmo sem os detalhes que a diferenciam do instrumento padrão. 
Procure deixar clara essa diferença para os alunos. Eles devem perceber que as funções
da imagem e do texto verbal variam dependendo do contexto em que estão. Às vezes, 
não entendemos nada do que está sendo dito ou escrito, mas a imagem nos explica,
e outras, até temos alguma noção do que se trata, e a imagem vem para \textbf{enriquecer}
o sentido. 

\Image{Ilustrações da segunda parte do livro dedicada a apresentar elementos da cultura sertaneja nordestina.}{PNLD2023_022_03.jpg}

Por fim, apresente-lhes os cordéis e as histórias em quadrinhos, gêneros nos quais 
essas diferentes linguagens estão sempre em contato produzindo, \textbf{juntas},
o sentido. Nenhuma é mais importante que a outra: as ilustrações, as xilogravuras
têm um papel fundamental nessas obras, bem como os diálogos e descrições escritos.

Finalize perguntando quais \textsc{hq}s e cordéis os alunos conhecem e diga que trabalharão, em seguida,
com a obra \textit{Lampião... Era o cavalo do tempo atrás da besta da vida}.

\paragraph{Tempo estimado} Quatro aulas de 50 minutos.

\subsection{Leitura}

\BNCC{EF35LP03}
\BNCC{EF35LP04}
\BNCC{EF35LP05}
\BNCC{EF35LP06}
\BNCC{EF35LP16}

\paragraph{Tema} Narrador e tempo.


\paragraph{Conteúdo} Análise da estrutura da narrativa a partir dos conceitos de \textit{narrador} e \textit{temporalidade}.


\paragraph{Justificativa} Os alunos devem ser capazes de identificar o sentido global do texto
e os elementos estruturais que levam a tal compreensão. Neste caso, quais são os marcadores temporais 
e narrativos implícitos ou explícitos no texto literário. 


\paragraph{Metodologia} Para a leitura do quadrinho, o professor ou a professora deve fazer uma leitura conjunta, 
em voz alta com os alunos em sala de aula.

\BNCC{EF09GE03}

Inicie a leitura chamando a atenção para o tema discutido na Atividade 1.2 da pré-leitura,
que diz respeito à \textbf{diversidade de falares}, e a importância de respeitar as variantes de uma língua. Ainda que o tom dos quadrinhos tenha algo de humorístico,
o professor ou a professora deve tomar cuidado de, caso se trate de uma sala de aula
numa região fora do Nordeste, não cair em esteriótipos ridicularizadores.

Conforme a leitura decorre, apresente perguntas norteadoras que dizem respeito à \textbf{estrutura}
do livro, como para tratar da questão do narrador:

\begin{itemize}
\item Quem está narrando essa história? 
\item Qual fala é do narrador e qual fala é das personagens?
\end{itemize}

Para trabalhar a compreensão do texto, pergunte:

\begin{itemize}
\item O narrador que conta a batalha de Lampião parece estar do lado dos policiais que queriam matá-lo
ou do lado de Lampião? 
\item Como vocês perceberam isso?
\end{itemize}

Chame a atenção para diferentes \textbf{temporalidades} da narrativa: existe o tempo presente, do narrador,
e o da batalha de Lampião, que acontece no passado. Quando este último evento é narrado, ainda que 
se trate de um evento já ocorrido, os verbos são conjugados no presente. 
Explique que se trata de uma técnica narrativa (\textit{flashback} ou \textit{mise en abîme}) na qual uma história
está dentro de outra. Os alunos devem entender que, ao invés da voz que conta a história se manter 
sob o controle do narrador, que a contaria no passado, o narrador desaparece e dá lugar às próprias 
personagens da história, que falam como se o evento estivesse acontecendo naquele momento. 
Por isso, trata-se de ``uma história dentro de outra'', pois existem ``dois presentes'', e não
um presente falando do passado. 

Outro aspecto que pode ser trabalhado durante a leitura é o contexto histórico da história de Lampião. Logo no início do livro, são apresentadas expressões conceituais como \textbf{Estado Novo} e \textbf{Coronelismo}, além do nome do presidente \textbf{Getúlio Vargas}. O professor ou a professora, em diálogo com o professor ou professora
de História, que pode aproveitar o ensejo para dar continuidade ao trabalho em suas aulas,
pode pedir que os alunos realizem uma pesquisa na internet acerca desses termos.


\paragraph{Tempo estimado} Quatro aulas de 50 minutos.

 
\subsection{Pós-leitura}

\paragraph{Atividade 1}

\BNCC{EF15AR04}
\BNCC{EF15AR05}

\paragraph{Tema} Produção de cordeis em quadrinho.

\paragraph{Conteúdo} Elaboração de um material autoral a partir da leitura
do cordel em quadrinhos.

\paragraph{Justificativa} Incentivar a produção, após a leitura, de diferentes formas de 
expressão artísticas; neste caso, o \textbf{cordel em quadrinhos}. A produção 
deve levar em conta o uso sustentável dos materiais, instrumentos, recursos e técnicas,
convencionais ou não.


\paragraph{Metodologia}

Depois que o quadrinho foi lido, o professor ou a professora deve incentivar os alunos
a produzir uma história em quadrinho. Obrigatoriamente, devem ser realizadas uma parte
escrita e outra, visual. Os alunos podem trabalhar em grupos e dividir as atividades
entre os que se sentem mais à vontade com cada uma das áreas.

Em diálogo com o professor ou a professora de Artes, pode haver uma oficina de \textbf{xilogravura}
adaptada ao ambiente da sala de aula. Originalmente, a arte de talhar em \textit{madeira} padrões
reproduzíveis em papel pode ser adaptada a outros materiais. Para entender esse processo, assista a uma oficina de xilogravura em isopor no link a seguir: \url{https://youtu.be/9CEjFNPpaF4}  

Os grupos podem escolher narrar qualquer história que seja de seu interesse,
desde que sigam as premissas básicas da concomitância de linguagens verbal e visual, e 
narrativa de uma personagem relevante. 

Ao final da atividade, que deve durar cerca de quatro aulas de cinquenta minutos,
 o professor ou a professora pode realizar uma feira de cordéis e quadrinhos
 seguindo os moldes tradicionais: os livretos pendurados em varais.
 Neste caso, a exposição pode ser feita em uma área comum da escola para o acesso de outros alunos
 de outras turmas. Outra opção é expor o resultado na Festa Junina da escola, para que não só a comunidade
 escolar, mas também os familiares e vizinhos possam usufruir seus trabalhos.


\paragraph{Tempo estimado} Quatro aulas de 50 minutos.

\paragraph{Atividade 2}

\BNCC{EF15AR06}

\paragraph{Tema} Feira de cordéis.

\paragraph{Conteúdo} Organização de uma feira de cordéis seguindo o modelo tradicional.

\paragraph{Justificativa} Após o trabalho individual dos alunos ou dos grupos,
é importante que haja uma interação entre a turma na qual os envolvidos
deverão compartilhar impressões e apreciações acerca das obras.

\paragraph{Metodologia} Após a realização dos cordéis em quadrinhos,
 o professor ou a professora deve animar a realização de uma feira de cordéis 
  seguindo os moldes tradicionais: os livretos pendurados em varais.
 Neste caso, a exposição pode ser feita em uma área comum da escola para o acesso de outros alunos
 de outras turmas. OUtra opção é expor o resultado na Festa Junina da escola, para que não só a comunidade
 escolar, mas também os familiares e vizinhos possam usufruir seus trabalhos.


\paragraph{Tempo estimado} Duas aulas de 50 minutos.
 
\Image{Klévisson Viana performando numa feira de cordéis na França.(Arquivo do site do autor.)}{PNLD2023_022_05.jpg}


\section{Sugestões de referências complementares}

\subsection{Músicas} 

\begin{itemize}
	\item ``Perseguição'', de Sérgio Ricardo. 

Trilha sonora do filme \textit{Deus e o Diabo na Terra do Sol}, de Glauber Rocha. 

Um trecho da música é citado por uma das crianças que \textit{brincam de cangaço} enquanto os mais velhos contam a história
da última batalha de Lampião.

	\item Álbum musical \textit{Nordeste: Cordel, Repente, Canção}. Disponível em: \url{https://www.youtube.com/watch?v=wS6jzcZcc6U}. Último acesso em 14 de dezembro de 2021.
\end{itemize}

\subsection{Filmes}

\begin{itemize}

	\item \textit{Deus e o Diabo na Terra do Sol}, de 1964, de Glauber Rocha. Disponível em: \url{https://www.youtube.com/watch?v=RyTnX_yl1bw}. Último acesso em: 14 de dezembro de 2021.

	\item \textit{Nordeste: Cordel, Repente, Canção.} Disponível em: \url{https://www.youtube.com/watch?v=xFOZxwBcUmo}. Último acesso em: 14 de dezembro de 2021.

\end{itemize}

\subsection{Livros}

\begin{itemize}
	\item \textit{A chegada de Lampião no inferno}, de José Pacheco da Rocha. 

O cordel continua a história de Lampião após a sua morte, narrada em \textit{Lampião... Era o cavalo do tempo atrás da besta da vida}, quando ele teria descido ao Inferno e fora recebido com assombro pelo próprio Diabo. 

	\item \textsc{van der linden}, Sophie. \textit{Para ler o livro ilustrado}. São Paulo: Cosac Naify, 2011.

Livro sobre as particularidades do livro ilustrado, que apresenta as diferenças entre o livro ilustrado e o livro com ilustração. 
\end{itemize}

\begin{itemize}
\item \textsc{sardelich}, Maria Emilia. Leitura de Imagens, Cultura Visual e Prática Educativa. 
In: \textit{Cadernos de Pesquisa}. V.36, n.128, p.451-472, mai/ago.2006. Disponível em: \url{https://www.scielo.br/pdf/cp/v36n128/v36n128a09}. 
Acesso em 29 abr 2021. 

Artigo acadêmico que discorre sobre a importância de trabalhar cultura 
visual na educação na sociedade contemporânea. 

\item \textsc{pranke}, Marha Elfrida. Organização dos espaços da sala de aula na Educação Infantil. Disponível em: \url{http://centraldeinteligenciaacademica.blogspot.com/2016/04/organizacao-dos-espacos-da-sala-de-aula.html}. Acesso em 04 mai 2021. 

Artigo acadêmico que discorre sobre a importância da rotina e de criar ambientes dentro da sala de aula na Educação Infantil.  
\end{itemize}

\subsection{Artigos}

\begin{itemize}
\item ``Lampião e outros cangaceiros sob as lentes de Benjamin Abrahão''. Disponível em \url{https://brasilianafotografica.bn.gov.br/?p=9527}.

Com registros do fotógrafo sírio Benjamin Abrahão Calil Botto (1901--1938), a Brasiliana Fotográfica lembra Lampião, 
Virgulino Ferreira da Silva (c.\,1898--1938), o rei do cangaço, e seu bando. A iconografia produzida por Benjamin --- registros fotográficos e filme --- não é a única sobre o cangaço, mas por sua extensão contribuiu enormemente para o conhecimento da história dos cangaceiros no Brasil. É uma comprovação visual da marcante estética dos bandoleiros da caatinga e os trouxe para os jornais e à imaginação popular.

\item ``Lampião, bandido de marketing''. Disponível em \url{http://www.historianet.com.br/conteudo/default.aspx?codigo=863}.

Nos últimos 70 anos a imagem de Lampião tem sofrido transformações maiores que as suas proezas. Antes de ser vítima do covarde massacre de Angicos, em 1938, na língua da imprensa ele oscilou entre o Terror do Nordeste e o Rei do Cangaço. O que não queria dizer, já então, mesmo quando era o "rei", uma imagem positiva, porque o seu reinado era o cangaço, a vida marginal dos cangaceiros, os bandidos do sertão nordestino. Depois, em anos de ascensão da democracia no Brasil, a sua imagem viajou do ponto de injustiçado, do rebelde sem partido, de um aliado, portanto, até o ponto do marginal que serviu ao poder de atraso dos "coronéis", os senhores feudais do Nordeste. Em 1978, com a cara de couro curtido de macho, serviu de contraponto e ironia para um jornal dirigido ao público homossexual, O Lampião da Esquina.

\end{itemize}


\end{document}

