\documentclass[11pt]{extarticle}
\usepackage{manualdoprofessor}
\usepackage{fichatecnica}
\usepackage{lipsum,media9}
\usepackage[justification=raggedright]{caption}
\usepackage[one]{bncc}
\usepackage[araucaria]{../edlab}
\usepackage{marginnote}
\usepackage{pdfpages}

\newcommand{\AutorLivro}{Leonardo da Vinci}
\newcommand{\TituloLivro}{A semente e a torre}
%\newcommand{\Tema}{}
\newcommand{\Genero}{Lendas; mitos; fábula}
%\newcommand{\imagemCapa}{./images/PNLD2022-001-01.jpeg}
\newcommand{\issnppub}{XXX-XX-XXXXX-XX-X}
\newcommand{\issnepub}{XXX-XX-XXXXX-XX-X}
% \newcommand{\fichacatalografica}{PNLD0001-00.png}
\newcommand{\colaborador}{Ana Lancman}

\begin{document}

\title{\TituloLivro}
\author{\AutorLivro}
\def\authornotes{\colaborador}

\date{}
\maketitle

\tableofcontents


\begin{abstract}

O presente manual tem como objetivo oferecer uma orientação ao professor sobre a obra \textit{A semente e a torre}. A partir deste manual, os professores poderão incentivar a prática da leitura aos estudantes e proporcionar um conteúdo enriquecedor. Apresentamos aqui sugestões de atividades a serem realizadas antes, durante e após a leitura do livro, com propostas que buscam introduzir os gêneros literários e aprofundar as discussões trazidas pelas obras. Você encontrará informações sobre o autor, sobre o gênero e sobre os temas trabalhados ao longo do livro. Ao fim do manual, você encontrará também sugestões de livros, artigos e sites selecionados para enriquecer a sua experiência de leitura e, consequentemente, a de seus estudantes.

\textit{A semente e a torre} é uma fábula escrita por Leonardo da Vinci, que trata de uma estranha amizade: uma semente e uma torre. Na história, a semente é transportada para o alto da torre por um acaso: uma gralha que a encontra e a leva ao alçar vôo. Uma vez lá em cima, a semente fica encaixada entre tijolos e pede ao muro para ficar lá. O muro aceita a sua presença, que com o tempo toma uma proporção maior do que o esperado. Esta obra pode ser uma metáfora para a natureza tomando vida própria, mas acima de tudo, fala sobre o imprevisível da vida.

A partir dessa obra, será possível trabalhar \textsc{bncc}'s de diversas áreas, desde Artes, Geografia e História, além de Língua Portuguesa. É um texto que abre margem para debate sobre a relação dos seres humanos com a natureza. Nas atividades, será explorada também a criatividade dos estudantes através de encenações teatrais. Esperamos que as atividades sugeridas e o material indicado sejam proveitosos em sala de aula! 

\end{abstract}

\section{Sobre o livro}

\paragraph{O livro} \textsc{A semente e a torre} é uma das histórias escritas por Leonardo da Vinci, que tinha o hábito de escrever histórias em caderninhos quando trabalhava na corte do rei da França, Francisco \textsc{i}. Dizem que chegou a escrever mais de trinta mil páginas deles. Nessa adaptação para um livro ilustrado, a narrativa torna-se visualmente encantadora e nos convida a mergulhar na história.

\paragraph{Descrição} A fábula começa com um embate de aves, uma gralha e um bem-te-vi. De repente, a gralha encontra uma semente e voa com ela na boca. A semente é deixada no alto de uma torre e cai entre tijolos de um muro, onde fica encaixada. Ela pede então ao muro que a deixe ficar lá. A semente conta a história de como chegou até o alto, porque que foi arrancada dos galhos de seu pai ela gralha e não pôde cumprir o seu destino: "crescer sob suas terras férteis e suas folhas caídas"\footnote{Página 14 de \textit{A semente e a torre}}. 

O muro fica comovido e aceita o seu pedido, sem prever o que aconteceria com o passar do tempo. A semente começa a brotar e suas raízes penetram pelos espaços entre os tijolos. As raízes crescem tanto que chegam até a torre e a envolvem. Nesse momento, o muro parece se arrepender de sua decisão e se pergunta o porquê de ter deixado a semente ficar. Ele percebe que a semente parecia inofensiva e que, no fim das contas, foi sua própria solidariedade que o destruiu. Por fim, vemos novamente o bem-te-vi, que recua com os gritos da gralha e decide voar para outro lugar. É com a imagem da gralha em uma posição agressiva que termina o livro.

\section{Sobre os autores}

\SideImage{Busto de Leonardo da Vinci, em Roma (Itália) (Krzysztof Golik; CC-BY-SA-4.0)}{PNLD2023-013-02.png}

\paragraph{O autor} Leonardo da Vinci foi um dos maiores pintores e escultores de todos os tempos. Ele também foi escritor, cientista, matemático, engenheiro, anatomista e botânico. Nasceu na Itália, em uma pequena cidade chamada Vinci, na região da Florença, em 1452.

Da Vinci foi figura marcante no Renascimento, período da história da Europa, de meados do século \textsc{xiv} até o fim do século  \textsc{xvi}. Foi um movimento cultural, econômico e político, em que houve uma forte revalorização das referências da Antiguidade Clássica. Com sua obra \textit{O homem vitruviano}, Da Vinci sintetizou o ideário renascentista em que o racionalismo e o individualismo foram colocados no centro.

\Image{O "Homem Vitruviano", de 1492. (Leonardo da Vinci; Domínio público)}{PNLD2023-013-10.png}

Como artista, foi responsável por dois dos quadros mais importantes da história: a \textit{Mona Lisa} e a \textit{A última ceia}. Como inventor, projetou máquinas voadoras, equipamentos de mergulho e até mesmo um helicóptero, mas nenhum deles saiu do papel. Ele faleceu com 67 anos em Amboise, no Reino da França, deixando um legado inestimável em diversas áreas do conhecimento.

\paragraph{O ilustrador} Ulisses Garcez nasceu em 1979, em São Paulo --- onde vive até hoje. Formou-se em Artes Visuais com habilitação em Multimídia e Intermídia na Escola de Comunicações e Artes(\textsc{eca}) da Universidade de São Paulo(\textsc{usp}), em 2006. Ele é artista plástico, ilustrador, quadrinista e professor de artes visuais. Foi editor de quadrinhos por mais de uma década na revista Sociedade Alternativa. Publicou diversas histórias em quadrinho e participou de várias exposições, individuais e coletivas.

\SideImage{Foto do ilustrador (Arquivo pessoal)}{PNLD2023-013-03.png}

\section{Sobre o gênero}

\paragraph{O gênero} O gênero deste livro é \textit{Lendas; mitos; fábula}. 

\paragraph{Descrição} A lenda e o mito são narrativas fantasiosas transmitida pela tradição oral através dos tempos. De caráter fantástico, as lendas e os mitos combinam fatos reais e históricos com fatos que não têm comprovação de acontecimento, a não ser pela palavra dos que sobraram para contar a história. As lendas e mitos de uma sociedade são fundamentais para que entendamos quem são essas pessoas e no que acreditam, bem como suas tradições. Uma lenda é verdadeira até que se prove o contrário. Com exemplos bem definidos em todos os países do mundo, as lendas e os mitos de um povo geralmente fornecem explicações plausíveis, e até certo ponto aceitáveis, para coisas que não têm explicações científicas comprovadas, como acontecimentos misteriosos ou sobrenaturais.

Neste caso, a obra trata-se de uma \textbf{fábula}. A fábula é uma narrativa curta em que os personagens principais geralmente são animais personificados.  Estes animais apresentam características humanas, tais como a fala e traços de personalidade. Essas personagens podem ser também objetos animados ou deuses. Em cada história há uma "lição de moral": uma mensagem de cunho educativo que busca conscientizar o leitor. A fábula veio do conto, mas se diferencia pela centralidade dos personagens animais e pelo intuito de concluir a história com um ensinamento. É uma história que pode ser contada em prosa ou em versos. 

Sobre a origem da fábula, Douglas Tufano afirma que:

\begin{quote} A fábula teria nascido provavelmente na Ásia Menor e daí teria passado pelas ilhas gregas, chegando ao continente helênico. Há registros sobre fábulas egípcias e hindus, mas sua criação é atribuída à Grécia, pois é onde a fábula passa a ser considerada como um tipo específico de criatividade dentro da teoria literária. 

Na Grécia, os primeiros exemplos de fábula datam do século VIII a.C. Isso nos mostra, é claro, que Esopo não foi o inventor do gênero, mas sim o mais conhecido fabulista na Antiguidade como autor e narrador dessas pequenas histórias.\footnote{\textsc{tufano}, Douglas. Esopo --- Fábulas completas. São Paulo: Moderna, 2015.}\end{quote}

Esopo foi um autor da Grécia Antiga a quem são atribuídas algumas das mais famosas fábulas, como \textit{A raposa e o cacho de uvas} e \textit{A galinha de ovos de ouro}. Diversas de suas histórias foram recontadas por La Fontaine, que é também um dos mais clássicos fabulistas já existentes. 

\section{Atividades}

\subsection{Pré-leitura}

\subsubsection{Atividade 1}

\BNCC{EF15LP19}
%Falar: "com suas próprias palavras";
\BNCC{EF15LP11} 
%Identificar: +Falar: conversação; "roda de conversa", "discussão";
\BNCC{EF15AR23}
%Identificar: relação entre diferentes/diversas linguagens artísticas

\paragraph{Tema} O que é uma fábula?

\paragraph{Conteúdo} Apresentação sobre o gênero fábula e conversa com os alunos sobre seus conhecimentos prévios.

\paragraph{Objetivo} Preparar a leitura de \textit{A semente e a torre} através de uma introdução ao gênero fábula e apresentar algumas fábulas conhecidas.

\paragraph{Justificativa} Para introduzir o livro \textit{A semente e a torre}, é importante apresentar as características fundamentais da \textbf{fábula} aos alunos. As fábulas são composições literárias curtas em que animais ou objetos aparecem como personagens, com qualidades humanas. Além disso, oferecem uma moral ao final do texto, uma mensagem com caráter educativo.

\Image{Uma das principais características da fábula são animais personificados como personagens. (Milo Winter; Domínio público)}{PNLD2023-013-07.png}

\paragraph{Metodologia} Comece a atividade com uma conversa com os alunos sobre seus conhecimentos prévios acerca das fábulas. Sugere-se que sejam feitas perguntas aos alunos para fomentar o debate, tais como:

\begin{itemize}

\item O que é uma fábula?

\item Quais as características mais importantes de uma fábula?

\item Quais fábulas vocês conhecem?

\item Vocês sabem quem foi Esopo?

\end{itemize}

Anote os elementos trazidos pelos estudantes na lousa e, a partir desses apontamentos, trabalhe sobre o gênero da fábula, sua origem e características principais. Uma boa abordagem pode ser a apresentação de fábulas conhecidas, como \textit{A cigarra e a formiga}, \textit{A lebre e a tartaruga} e \textit{A galinha dos ovos de ouro}.

\SideImage{A obra representa o fabulista grego Esopo (Diego Velázquez; Domínio público)}{PNLD2023-013-08.png}

É possível complementar o conteúdo com o vídeo \href{https://youtu.be/lGm0nfoRBiI}{Aprenda a origem e as características das fábulas}, do canal \textit{Lu ensina}. É um vídeo curto que apresenta o gênero através de ilustrações, de forma didática.

\paragraph{Tempo estimado} Uma aula de 50 minutos.

\subsubsection{Atividade 2}

\BNCC{EF03HI05}
%Identificar +Perceber: "marcos históricos e seus significados"
\BNCC{EF15AR07}
%Identificar: museus, galerias, instituições, artistas, artesãos, curadores etc;

\paragraph{Tema} A obra de Leonardo da Vinci.

\paragraph{Conteúdo} Apresentação sobre a vida e a obra de uma das figuras mais importantes do Renascimento.

\paragraph{Objetivo} Ampliar o repertório dos alunos com conteúdo sobre a trajetória de Da Vinci. Identificar, a partir de seus trabalhos, a interface entre distintas áreas do conhecimento.

\paragraph{Justificativa} Para preparar a leitura, é importante que os estudantes conheçam as obras mais importantes de Da Vinci e reconheçam a relevância de sua figura em áreas como História, Artes e Ciência. É provável que alguns da sala conheçam sua figura ou pelo menos já tenham visto o quadro da \textit{Mona Lisa}, cujo cenário histórico poderá ser de grande interesse dos alunos.

\Image{Quadro \textit{A última ceia}, de 1498. (Leonardo da Vinci; Domínio público)}{PNLD2023-013-09.png}

\paragraph{Metodologia} Faça uma apresentação sobre Leonardo Da Vinci. É interessante que seja feito um breve contexto histórico de sua vida, com o auxílio dos professores de História, Geografia e Artes. Podem ser apresentadas obras do autor que simbolizem a diversidade de seus ofícios, uma vez que Da Vinci foi cientista, matemático, engenheiro, anatomista, pintor, escultor e botânico, além de escritor. Explique os motivos dele ter ficado tão conhecido.

Uma plataforma interessante a ser explorada é o \textit{Google Arts \& Culture} disponível no \href{https://artsandculture.google.com/entity/\%2Fm\%2F04lg6?hl=pt-BR\&col=RGB_0E182D}{link}. Nela, é possível visualizar obras de Da Vinci com ótima qualidade e de forma gratuita. Lá também está disponível um \textit{tour online} na cidade de Florença, percorrendo os caminhos feitos por Da Vinci, disponível no \href{https://artsandculture.google.com/story/sQURHljckUgVZQ?hl=pt-BR}{link}.

Sugere-se também que seja exibido o episódio \textit{Leonardo da Vinci}, que faz parte da série Pequenos Ilustres. É um desenho animado curto que apresenta de forma divertida a vida do inventor. Pode ser acessado gratuitamente no \href{https://youtu.be/0hwJzpATtjg}{Youtube}.

\paragraph{Tempo estimado} Uma aula de 50 minutos.

\subsection{Leitura}

\subsubsection{Atividade 1}

\BNCC{EF15LP02}
%Identificar: "Interpretação de texto", recepção, gênero, tema, imagens, e "partes da obra": índice;
\BNCC{EF15LP11}
%Identificar: +Falar: conversação; "roda de conversa", "discussão";
\BNCC{EF15LP16} 
%Ler: narrativas, contos, crônicas; +grupo, +professor e +sozinho; Mundo imaginário;
\BNCC{EF15LP18}
%Identificar: texto e ilustração;
\BNCC{EF15AR20}
%Produzir: improvisações teatrais;

\paragraph{Tema} Compreensão do enredo de \textit{A semente e a torre}.

\paragraph{Conteúdo} Leitura conjunta da obra em sala de aula, roda de conversa sobre o conteúdo da obra e resolução conjunta do conflito entre "semente" e "muro".

\paragraph{Objetivo} Incentivar que os estudantes se envolvam com a obra e que possam refletir sobre as diversas questões trazidas pela obra.

\paragraph{Justificativa} Por ser uma fábula, \textit{A semente e a torre} apresenta uma mensagem ao final do livro. Ao ler a obra de forma conjunta e com auxílio do professor, a apreensão dos significados da história é enriquecida. Além disso, ao longo da atividade, os alunos serão convidados a participar do debate e expressar seu posicionamento.

\paragraph{Metodologia} Será realizada a leitura integral do livro em sala de aula, com a intermediação do professor. Primeiramente, leia a fábula de forma conjunta, convidando alunos que se voluntariem para ler algumas páginas. É possível que haja uma pausa para contemplar as ilustrações, que são recheadas de detalhes a serem observados. Sugere-se que, ao ser finalizada essa primeira leitura, seja realizada uma nova leitura individual. Para esse segundo momento, o professor pode propor aos estudantes que anotem quais cenas foram mais marcantes para eles. Também peça que prestem atenção em algumas questões, como as personagens que aparecem e quais os distintos momentos e espaços da narrativa.

Em uma roda, proponha que os alunos contem como entenderam a história. Poderá ser feita uma conversa livre, em que os estudantes poderão expressar quais sensações tiveram ao longo da leitura. 

\Image{Ilustração do livro, página 7}{PNLD2023-013-04.png}

Sugestões de questões a serem feitas para fomentar o debate:

\begin{itemize}

\item O que as primeiras ilustrações do livro mostram?

\item Quais elementos da história podemos observar antes mesmo da primeira frase?

\item Quem são os personagens?

\item Quais personagens se movem e quais são imóveis?

\item O que acontece com a semente?

\item Qual é o pedido que a semente faz para o muro?

\item Qual é a pergunta que o muro faz a si mesmo no final do livro?

\item Quais as mensagens que o livro traz?

\end{itemize}

Em seguida, divida a sala em dois. Metade da classe deverá se colocar do lado da semente e a outra metade tomar o partido do muro. Primeiro, peça que cada grupo escreva em uma cartolina as características principais entre a personagem da semente e do muro. A proposta é que se baseiem na história, mas poderão trazer referências de sementes e muros que sejam próximas a eles.

\Image{Ilustração do livro, página 12}{PNLD2023-013-05.png}

\Image{Ilustração do livro, página 18}{PNLD2023-013-06.png}

Agora mais próximos de seus personagens, poderão improvisar uma conversa que poderia acontecer entre o muro e a semente após o fim do livro. Poderão imaginar o que poderia ter acontecido depois do muro perceber que foi "invadido" pelas raízes. A ideia é que defendam seu ponto de vista e elaborem o ocorrido para que possam chegar em um diálogo comum. Um aluno de cada grupo deverá falar por vez, intercalando entre "sementes" e "muros". Se possível, devem apresentar uma proposta que resolva a questão apresentada no final da fábula.

\paragraph{Tempo estimado} Duas aulas de 50 minutos.

\subsubsection{Atividade 2}

\BNCC{EF02GE04}
%Identificar +Diferenciar: "hábitos e relações com a natureza em diferentes lugares" 
\BNCC{EF03GE04}
%Identificar: processos naturais e históricos que mudem a paisagem;
\BNCC{EF15AR04} 
%Produzir: desenho, pintura, colagem, quadrinhos, dobradura, escultura, modelagem...;

\paragraph{Tema} A natureza com vida própria.

\paragraph{Conteúdo} Ensaio artístico que retrate a ação da natureza sobre construções arquitetônicas.

\paragraph{Objetivo} Incentivar os estudantes a identificarem os processos naturais que mudam a paisagem ao longo do tempo.

\paragraph{Justificativa} Um dos temas que pode ser abordado a partir da fábula \textit{A semente e a torre} é a ação do tempo e a forma como a natureza toma conta dos espaços que foram construídos pelos seres humanos. Nessa atividade, os alunos poderão refletir sobre a questão ambiental e exercitar sua criatividade.

\paragraph{Metodologia} Inicie a atividade propondo uma conversa livre com os alunos sobre a ação da semente e das raízes sobre o muro. É interessante trazer a questão do meio ambiente e o impacto que os seres humanos podem causar através das construções de casas e prédios. Pergunte aos alunos se eles têm a lembrança de ter visto alguma construção com a presença de raízes.

Em seguida, sugere-se que sejam apresentadas as fotografias de Jonk, fotógrafo francês que possui um projeto chamado \textit{Naturalia}. Ele viaja o mundo visitando locais abandonados e registrando a ação da natureza sobre as construções. É possível acessar gratuitamente em seu \href{https://www.jonk-photography.com/en/naturalia/}{site}.

Para a aula seguinte, os alunos deverão trazer um retrato de um local que tenha marcas da natureza. Pode ser um muro em sua rua ou algum lugar abandonado que encontraram na internet. O retrato poderá ser uma foto, um desenho, uma pintura, uma colagem \ldots{} É um momento de exercitar a criatividade e ampliar perspectivas. Cada aluno deverá apresentar o seu retrato para o resto da sala. Poderão falar qual foi o local retratado e quais as ações da natureza identificadas.

\paragraph{Tempo estimado} Duas aulas de 50 minutos.

\subsection{Pós-leitura}

\BNCC{EF02LP27}
%Produzir +Reescrever: textos narrativos "com suas próprias palavras
\BNCC{EF15LP05}
%Produzir: planejar "partes do texto", para quem escreve, tema, "adequação";
\BNCC{EF15AR21}
%Produzir: imitação e o faz de conta; encenar com música, imagem; 

\paragraph{Tema} Fábula encenada.

\paragraph{Conteúdo} Redação de uma fábula e apresentação teatral.

\paragraph{Objetivo} Aprofundar os estudos sobre o gênero fábula e aprimorar a prática da escrita, além de incentivar a expressão artística da linguagem teatral.

\paragraph{Justificativa} Ao reescrever a moral da história com suas próprias palavras, os estudantes poderão aprofundar seu conhecimento sobre a obra. Além disso, na convivência em grupo, poderão exercer o trabalho em equipe e praticar a expressão corporal por meio do teatro.

\paragraph{Metodologia} Proponha aos estudantes que se dividam em grupos pequenos.

\begin{itemize}

\item Cada grupo deverá redigir uma moral da história segundo o que interpretaram no fábula \textit{A semente e a torre}.

\item Em seguida, poderão adaptar essa moral para uma nova fábula. Esta fábula deve apresentar novos personagens. Peça para os alunos escolherem dois objetos ou animais quaisquer que venham a cabeça. É um momento dos alunos usarem sua criatividade, a partir dos temas discutidos na atividade de pré-leitura e leitura.

\item A partir do texto dessa nova fábula, cada grupo montará uma breve encenação teatral. Poderão utilizar materiais artísticos disponíveis na escola ou trazer de casa para montar um cenário e figurino. Acompanhe os estudantes, com o auxílio do professor de Artes, e oriente no que for necessário. 

\item Cada grupo apresentará para o restante a sua encenação teatral. Peça que os estudantes colaborem coletivamente para que adaptar os espaços em sala de aula ou o local da escola escolhido para as apresentações. 

\end{itemize}

Por fim, reúna todos os alunos para um debate sobre os temas abordados nas cenas e as mensagens finais escolhidas por cada grupo. É interessante recuperar as questões trazidas pela fábula de Leonardo da Vinci e propor um debate relacionando as distintas formas literárias e suas múltiplas interpretações.

\paragraph{Tempo estimado} Quatro aulas de 50 minutos.

\section{Sugestão de referências complementares}

\subsection{Audiovisual}

\begin{itemize}

\item \textit{Eu, Leonardo}. Dirigido por Jesus Garces Lambert, 2019.

A viagem se passa principalmente na mente de Da Vinci, onde o gênio encontra os artistas, os homens de poder e os estudantes.

\item \textit{O cão e a raposa}. Dirigido por Art Stevens, Richard Rich e Ted Berman, 1981.

O filme é uma fabula animada, que conta a história de dois amigos improváveis, uma raposa vermelha chamada Tod e um cão de caça chamado Copper. Os dois lutam para preservar sua amizade apesar de seus instintos emergentes e das pressões sociais ao redor exigindo que sejam adversários. 

\item \textit{Wall-e}. Dirigido por Andrew Stanton, 2008.

A história segue um robô chamado WALL·E, criado no ano de 2100 para limpar a Terra coberta por lixo. Ele se apaixona por um outro robô, chamado EVA, que tem a missão de encontrar pelo menos uma planta na superfície do planeta Terra. Poderá ser um material complementar para a atividade 2 de leitura.

\end{itemize}

\subsection{Links}

\begin{itemize}

\item \textit{Naturalia}, de Jonk. Projeto de fotografia que apresenta locais abandonados ao redor do mundo que foram absorvidos pela natureza. Disponível no \href{https://www.jonk-photography.com/en/naturalia/}{site do fotógrafo}. 

\item \textit{Google Arts \& Culture}. Plataforma que apresenta obras de arte em alta qualidade e \textit{tours online} em diferentes museus do mundo. Acesse gratuitamente o \href{https://artsandculture.google.com/}{site}. 

\end{itemize}

\subsection{Museus}

\begin{itemize}

\item \href{https://museucatavento.org.br/}{Catavento – Museu de Ciências}

Localizado no centro da cidade de São Paulo, o museu é dividido em quatro seções: universo, vida, engenho e sociedade. É um museu educativo e interativo.

Endereço: Av Mercúrio, s/n - Pq Dom Pedro II, Centro, São Paulo - SP

\item {Museu do Amanhã}

Localizado no centro histórico do Rio de Janeiro, é um museu de ciências que foi inaugurado em 2015. Possui diversos ambientes audiovisuais e tecnologias interativas com o intuito de imaginar futuros possíveis.

Endereço: Praça Mauá, 1 - Centro, Rio de Janeiro - RJ.

\end{itemize}

\subsection{Bibliografia comentada}

\begin{itemize}

\item \textsc{japiassu}, Ricardo. Metodologia do ensino de teatro. Campinas: Papirus, 2003.

Obra voltada para o ensino de práticas teatrais de 1º a 4º ano do Ensino Fundamental.

\item \textsc{isaacson}, Walter. Leonardo da Vinci. São Paulo: Intrínseca, 2017.

Biografia definitiva de Leonardo da Vinci, a partir de páginas dos impressionantes cadernos que o escritor manteve ao longo de boa parte da vida.

\item \textsc{tufano}, Douglas. Esopo --- Fábulas completas. São Paulo: Moderna, 2015.
 
Essa reunião de fábulas apresenta uma introdução sobre o gênero e notas explicativas.

\end{itemize}

\end{document}