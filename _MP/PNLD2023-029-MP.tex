\documentclass[11pt]{extarticle}
\usepackage{manualdoprofessor}
\usepackage{fichatecnica}
\usepackage{lipsum,media9}
\usepackage[justification=raggedright]{caption}
\usepackage[one]{bncc}
\usepackage[ayllon]{../edlab}
\usepackage{marginnote}
\usepackage{pdfpages}

\newcommand{\AutorLivro}{Monteiro Lobato}
\newcommand{\TituloLivro}{O plano da Emília e outros textos}
\newcommand{\Tema}{Diversão e aventura}
\newcommand{\Genero}{Conto}
%\newcommand{\imagemCapa}{./images/PNLD2022-001-01.jpeg}
\newcommand{\issnppub}{XXX-XX-XXXXX-XX-X}
\newcommand{\issnepub}{XXX-XX-XXXXX-XX-X}
% \newcommand{\fichacatalografica}{PNLD0001-00.png}
\newcommand{\colaborador}{Paulo Pompermaier}

\begin{document}

\title{\TituloLivro}
\author{\AutorLivro}
\def\authornotes{\colaborador}

\date{}
\maketitle

\tableofcontents

\section{Carta ao professor}

\begin{abstract}
Este material tem a intenção de contribuir para que você desenvolva um trabalho aprofundado com a obra \textit{O plano da Emília e outros textos} em sala de aula.
Você encontrará informações sobre o autor, sobre o gênero e também 
algumas propostas de trabalho para a sala de aula que você poderá explorar livremente, 
da forma que considerar mais apropriada para os seus estudantes.

Fazendeiro, escritor para crianças e adultos, editor, empresário,
defensor do petróleo nacional: a intensidade com que Monteiro Lobato
experienciou as várias faces de sua vida transparece na vitalidade de
seus contos, frutos de sua sensibilidade, observação crítica,
conhecimentos literários e trabalho intelectual e artístico.

Nascido José Bento Renato Monteiro Lobato, tal é sua importância para a literatura infantil que o dia de seu nascimento, 18 de abril de 1882, ficou consagrado como Dia Nacional do Livro Infantil. Sua obra foi uma das primeiras no país a se dedicar explicitamente ao universo infantil, criando uma literatura brasileira dedicada às crianças e formando gerações de leitores.

Muitos se formaram em contato com sua obra não apenas através da literatura, mas também da televisão, com a famosa série \textit{Sítio do Picapau amarelo}. O sítio e seus personagens são figuras incontornáveis do universo literário infantil brasileiro, tamanha sua influência que perdura até hoje. Nos 24 textos reunidos em \textit{O plano da Emília e outros textos}, o leitor tem um contato privilegiado com o universo infantil de Lobato.
Além de todo mérito literário das histórias, essa seleção percorre praticamente trinta anos de produção do autor, evidenciando os principais motivos, personagens e enredos constitutivos de sua literatura.

Ao longo do manual, todos esses aspectos serão explorados e relacionados a sugestões de atividades. Com isso, objetiva-se oferecer algumas ideias e inspirações para um trabalho que pode ser desenvolvido tanto a curto, quanto a médio e longo prazo. Sinta-se à vontade para personalizar a aula e torná-la sua, aplicando seus conhecimentos, sua 
personalidade e aproveite para fortalecer seu vínculo com a turma.
Boa aula!
\end{abstract}

\section{Sobre o livro}
A obra \textit{O plano da Emília e outros textos} reúne 24 textos do escritor paulista.
Trata-se tanto de contos como de capítulos extraídos de suas principais obras e que funcionam como narrativas independentes. A seleção tenta abarcar os trabalhos mais representativos de Lobato, funcionando como uma espécie de introdução à vasta produção literária do autor.

As histórias de \textit{O plano da Emília e outros textos} foram colhidas de dez obras escritas entre 1921 e 1947. São elas:

\begin{itemize}
\item\textit{O Saci} (1921);
\item\textit{Fábulas} (1922);
\item\textit{Reinações de Narizinho} (1931);
\item\textit{Viagem ao céu} (1932);
\item\textit{Emília no País da Gramática} (1934);
\item\textit{O Pica-pau Amarelo} (1939);
\item\textit{O Minotauro} (1939);
\item\textit{A reforma da natureza} (1941);
\item\textit{Os doze trabalhos de Hércules} (1944);
\item\textit{Histórias diversas} (1947).
\end{itemize}

Com essa seleção, o estudante percorre praticamente trinta anos de produção literária do autor, podendo acompanhar não apenas os enredos deliciosos, mas também a forma como seu estilo evoluiu e como tratou a narrativa infantil de diferentes maneiras ao longo de sua trajetória. Nos textos da década de 1920, por exemplo, percebemos a preocupação do autor em recuperar temas folclóricos brasileiros, como a figura do saci e de outros animais típicos das fábulas. Na década de 1930 percorremos o universo de Narizinho, Emília e de outros personagens eternizados na série televisiva \textit{Sítio do Pica-pau Amarelo}.
Apesar de explorar os temas brasileiros, o autor não deixou de investigar os grandes mitos e narrativas da cultura ocidental, preocupação que pode ser notada em seus trabalhos de finais da década de 1930 e início e 1940, como \textit{O Minotauro} e \textit{Os doze trabalhos de Hércules}. Com esse panorama, o jovem leitor é introduzido no universo literário de Monteiro Lobato e tem a oportunidade de ler e fruir as principais narrativas do escritor nascido em Taubaté.



\section{Sobre o autor}


%532 caracteres
\paragraph{O autor}
Fazendeiro, escritor para crianças e adultos, editor, empresário,
defensor do petróleo nacional: a intensidade com que Monteiro Lobato
experienciou as várias faces de sua vida transparece na vitalidade de
seus contos, frutos de sua sensibilidade, observação crítica,
conhecimentos literários e trabalho intelectual e artístico.

José Bento Renato Monteiro Lobato nasceu em Taubaté, São Paulo, a 18 de
abril de 1882, que ficou consagrado como Dia Nacional do Livro Infantil,
e faleceu em São Paulo, a 4 de julho de 1948.

Escritor de literatura infantojuvenil, contista, jornalista, editor,
tradutor, pintor e fotógrafo, aos onze anos mudou seu nome para José
Bento, por causa das iniciais gravadas no castão da bengala do pai,
\textsc{j.b.m.l}. Apesar de sua inclinação para as artes plásticas, cursou a
Faculdade de Direito do Largo São Francisco, em São Paulo, por imposição
do avô, o Visconde de Tremembé. Formado em 1904, voltou a Taubaté, onde
foi nomeado promotor público interino, transferido, em 1907, para
Areias, São Paulo. Enviou artigos para \emph{A Tribuna}, de Santos,
traduções para o jornal \emph{O Estado de S. Paulo} e caricaturas para a
revista \emph{Fon-Fon!}, do Rio de Janeiro. Em 1911 herdou, com as duas
irmãs, a fazenda do avô. Publicou, em 1914, os artigos ``Velha praga'' e
``Urupês'' em \emph{O Estado de S. Paulo}, criando o personagem Jeca
Tatu. Em 1917, vendeu a fazenda e se mudou para São Paulo.

\Image{O escritor Monteiro Lobato em foto da década de 1920. (CC BY-NC 2.0)}{PNLD2023-029-02.jpg}

Escreveu em \emph{O Estado de S. Paulo} o artigo ``A propósito da
Exposição de Malfatti'' (``Paranoia ou mistificação?''), de crítica
contra as vanguardas, abrindo polêmica com os modernistas. Em 1918,
estreou com o livro de contos \emph{Urupês}, que esgotou 30 mil
exemplares entre 1918 e 1925, e comprou a \emph{Revista do Brasil},
lançando as bases da indústria editorial no país. \emph{Cidades mortas},
originalmente publicado em 1919, numa edição da \emph{Revista do
Brasil}, reúne os primeiros escritos de Lobato, ainda estudante em
Taubaté, e contos que escreveu antes de viajar a Nova York para ocupar
um posto no Consulado brasileiro.

Criando uma rede de distribuição, com vendedores autônomos e
consignatários, revolucionou o mercado livreiro. Em 1920, fundou a
editora Monteiro Lobato \& Cia, que publicou obras de Lima Barreto, Léo
Vaz, Oswald de Andrade, Ribeiro Couto, Menotti del Picchia, Guilherme de
Almeida, Oliveira Viana e Amadeu Amaral, entre muitos outros. No mesmo
ano, lançou \emph{A menina do Narizinho Arrebitado}, primeira da série
de histórias com que Lobato criou a literatura brasileira dedicada às
crianças, formando gerações de leitores. Em 1924, com capital ampliado e
nova denominação, Companhia Gráfico"-Editora Monteiro Lobato, sua editora
monta o maior parque gráfico da América Latina. Porém, no ano seguinte,
dificuldades financeiras o levam a vender a \emph{Revista do Brasil} e
liquidar a editora. Mudou"-se para o Rio de Janeiro e fundou a Companhia
Editora Nacional.

Adido comercial em Nova York de 1927 até 1930, voltou ao Brasil com
ideias para a exploração de ferro e petróleo. Fundou empresas de
prospecção, mas, contrariando interesses multinacionais e fazendo
oposição, em artigos e entrevistas, ao governo Vargas, foi preso por
seis meses em 1941. Recebeu indulto depois de cumprir metade da pena,
mas o governo mandou apreender e queimar seus livros infantis.

Em 1944, Lobato recusou indicação para a Academia Brasileira de Letras.
Em 1946, tornou-se sócio da editora Brasiliense. Embarcou para a
Argentina e fundou em Buenos Aires a Editorial Acteon, retornando no ano
seguinte a São Paulo.

Suas principais publicações são:

\begin{enumerate}
\item Livros para crianças: \emph{O Saci} (1921); \emph{Fábulas} (1922);
\emph{Reinações de Narizinho} (1931); \emph{Viagem ao céu} (1932);
\emph{Caçadas de Pedrinho} (1933); \emph{História do Mundo para as}
\emph{Crianças} (1933); \emph{Emília no País da Gramática} (1934);
\emph{Aritmética da Emília} (1935); \emph{Memórias da Emília} (1936);
\emph{O Poço do Visconde} (1937); \emph{O Picapau Amarelo} (1939);
\emph{A Reforma da Natureza} (1941); \emph{A Chave do Tamanho} (1942);
\emph{Os doze trabalhos de Hércules}, dois volumes (1944);

\item Livros para adultos: \emph{Urupês} (1918); \emph{Cidades
mortas} (1919); \emph{Ideias de Jeca Tatu} (1919); \emph{Negrinha}
(1920); \emph{O macaco que se fez homem} (1923); \emph{Mundo da lua}
(1923); \emph{O presidente negro/O choque das raças} (1926);
\emph{Ferro} (1931); \emph{América} (1932); \emph{O escândalo do
petróleo} (1936); \emph{A barca de Gleyre: quarenta anos de
correspondência literária entre Monteiro Lobato e Godofredo Rangel}
(1944).
\end{enumerate}

Lobato traduziu e adaptou diversas obras, entre as quais: \emph{Da
história da filosofia}, de Will Durand; \emph{Memórias}, de André
Maurois; \emph{Por quem os sinos dobram}, de Ernest Hemingway;
\emph{Crepúsculo dos ídolos e Anticristo}, de Friedrich Nietzsche;
\emph{Robinson Crusoé}, de Daniel Defoe; \emph{Mogli, o menino lobo}, de
Rudyard Kipling; \emph{Aventuras de Tom Sawyer}, de Mark Twain;
\emph{Pollyana}, de Eleanor H. Porter; \emph{Moby Dick}, de Herman
Melville; \emph{Tarzan}, de Edgar Rice Burroughs.

\paragraph{A organizadora}
Ieda Lebensztayn é crítica literária, pesquisadora e ensaísta,
preparadora e revisora de livros. Mestre em Teoria Literária e
Literatura Comparada e doutora em Literatura Brasileira pela
Universidade de São Paulo. Fez dois pós-doutorados: no Instituto de
Estudos Brasileiros, \textsc{ieb-usp}, sobre a correspondência de
Graciliano Ramos (Fapesp 2010/12034-9); e na Biblioteca Brasiliana
Mindlin / Faculdade de Filosofia, Letras e Ciências Humanas,
\textsc{bbm/fflch-usp}, a respeito da recepção literária de Machado de
Assis (\textsc{cnp}q 166032/2015-8). Autora de \emph{Graciliano Ramos e
a} Novidade\emph{: o astrônomo do inferno e os meninos impossíveis} (São
Paulo: Hedra, 2010). Organizou, com Thiago Mio Salla, os livros
\emph{Cangaços} e \emph{Conversas}, de Graciliano Ramos, publicados em
2014 pela Record. E, com Hélio Guimarães, os dois volumes de
\emph{Escritor por escritor: Machado de Assis segundo seus pares} --
1908-1939; 1939-2008 (São Paulo: Imprensa Oficial do Estado de São
Paulo, 2019). Colabora no caderno ``Aliás'' de \emph{O Estado de S.
Paulo}.

\Image{O que define um gênero narrativo é o fato de, não importa qual seja sua forma, contar uma história. (Dorothe/Px Here; Domínio público)}{PNLD2023-029-07.png}


\section{Sobre o gênero}

%55 caracteres
\paragraph{O gênero} O gênero deste livro é \textit{conto}. 

\begin{quote}
O conto é, do ângulo dramático, unívoco, univalente. [\ldots]
Etimologicamente preso à linguagem teatral,
``drama'' significava ``ação''. E com o tempo passou a designar
toda peça destinada à representação. Na época romântica, dado o
princípio da fusão de gêneros, entendia-se por drama o misto de
tragédia e comédia. Transferido para a prosa de ficção, o termo
``drama'' entrou a significar ``conflito'', ``atrito''. Nesse caso,
``ação'' ``conflito'' se tonaram equivalentes, uma vez que toda
ação pressupõe conflito, e este, promove a ação, ou por meio dela
se manifesta; em suma, ambos se implicam mutuamente.

O conto é, pois, uma narrativa unívoca, univalente: constitui
uma \textit{unidade dramática}, uma \textit{célula dramática}, visto gravitar ao
redor de um só conflito, um só drama, uma só ação. Caracteriza-se,
assim, por conter \textit{unidade de ação}, tomada esta como a sequência de atos praticados pelos protagonistas, ou de acontecimentos de
que participam. A ação pode ser externa, quando as personagens se
deslocam no espaço e no tempo, e interna, quando o conflito se
localiza em sua mente.\footnote{\textsc{moisés}, Massaud. \textit{A criação literária}. São Paulo: Cultrix, 2006, p.\,40.}
\end{quote}

Partindo da definição de Massaud Moisés sobre o conto, evidencia"-se a principal característica desse gênero literário: a unidade de conflito, condensada em ações que se completam em um único enredo. Ao conto, ainda seguindo Moisés, aborrecem as divagações e os excessos, pois há uma concentração de efeitos e pormenores essenciais, em sua brevidade, para o bom funcionamento do conto.
Cada construção, cada palavra nesse gênero tem sua razão de existir, pois integra a economia global da narrativa.

Apesar da brevidade de sua forma, o conto desdobra"-se em muitas direções e implicações, e o faz a partir de elementos restritos: a unidade dramática, como já mencionada, assim como a presença de poucas personagens e a limitação espacial e temporal. Um ótimo exemplo é o conto ``Missa do galo'', de Machado de Assis, em que o narrador, Nogueira, conta a sua experiência de uma única noite na companhia de sua hospedeira, D.\,Conceição. Apesar de unidade temporal (a noite que antecede a Missa do galo), espacial (uma sala na casa de D.\,Conceição) e da redução dramática, basicamente, à interação entre duas personagens, Conceição e Nogueira, esse conto desdobra"-se em muitas direções. A companhia de Conceição desperta a sexualidade de Nogueira, e seu impacto é tão profundo que o narrador relembra aos leitores esse acontecimento de sua juventude. As intenções da anfitriã, narradas e, logo, distorcidas pela memória de Nogueira, também são ambíguas, levantando as mais diversas questões e interpretações.

Como reflete o escritor argentino Julio Cortázar, o conto consegue, de forma muito concisa, despertar ``uma realidade infinitamente mais vasta que a do seu mero argumento'', influindo ``em nós com uma força que nos faria suspeitar da modéstia do seu conteúdo aparente, da brevidade do seu texto''.\footnote{\textsc{cortázar}, Julio. \textit{Valise de cronópio}. São Paulo: Editora Perspectiva, 2008, p.\,155.}

Apesar da aparente banalidade do argumento, o conto abre essa possibilidade de desenvolver o tema em profundidade, em contraposição à aparente concisão narrativa. Realiza plenamente, assim, o que Cortázar define como o gênero do conto:

\begin{quote}
Um escritor argentino, muito amigo do boxe, dizia"-me que nesse combate que se trava entre um texto apaixonante e o leitor, o romance ganha sempre por pontos, enquanto que o conto deve ganhar por \textit{knock"-out}. É verdade, na medida em que o romance acumula progressivamente seus efeitos no leitor, enquanto que um bom conto é incisivo, mordente, sem trégua desde as primeiras frases. Não se entenda isto demasiado literalmente, porque o bom contista é um boxeador muito astuto, e muitos dos seus golpes iniciais podem parecer pouco eficazes quando, na realidade, estão minando já as resistências mais sólidas do adversário.
Tomem os senhores qualquer grande conto que seja de sua preferência, e analisem a primeira página. Surpreender"-me"-ia se encontrassem elementos gratuitos, meramente decorativos. O contista sabe que não pode proceder acumulativamente, que não tem o tempo por aliado; seu único recurso é trabalhar em profundidade, verticalmente, seja para cima ou para baixo do espaço literário.\footnote{Ibid., p.\,152.}
\end{quote}

\Image{Capa da primeira edição de \textit{A menina do narizinho arrebitado}, da década de 1920. (CC BY-NC 2.0)}{PNLD2023-029-03.png}

\section{Atividades}

\subsection{Pré-leitura}

\paragraph{Tema} Pesquisa bibliográfica.

\paragraph{Conteúdo} Prática de leitura e investigação bibliográfica. Nome do autor, data de nascimento, obra e vida, características da época em que nasceu e as brincadeiras típicas deste tempo.  


\paragraph{Objetivo} Explorar a capacidade investigativa dos alunos e prepará-los para a leitura do livro fazendo com que se aproximem do autor e de sua obra.


\paragraph{Justificativa} Pesquisar sobre o autor e obra traz uma aproximação das crianças com o contexto de produção do livro, seu autor e as características de seus escritos como o gênero, as personagens, o tempo e o espaço onde se passa, os elementos folclóricos das narrativas etc. 

\paragraph{Metodologia} No primeiro momento o professor deverá indagar os alunos com as seguintes perguntas:


\begin{itemize}
\item Alguém conhece Monteiro Lobato?

\item Quais histórias ele escreveu?

\item Já assistiram ou leram o \textit{Sítio do Picapau Amarelo}?

\item Quais brincadeiras vocês reconhecem nessas histórias?
\end{itemize}

Promova um debate sobre o autor, participe como mediador e facilitador do diálogo.

Na segunda parte da atividade, monte com os alunos uma série de questionamentos sobre a vida e obra do autor. Cada aluno terá seus próprios materiais para escrever as perguntas, o educador pode ditar ou escrever no quadro.

Crie eixos de perguntas e explique aos alunos o objetivo de cada eixo.
Por exemplo, um dos eixos pode ser sobre a vida do autor:

\begin{itemize}
\item Estado onde nasceu;

\item Profissões que exerceu;

\item Data de nascimento;

\item Características socioculturais da época.
\end{itemize}

Faça o mesmo com os outros eixos, busque esmiuçar as possibilidades de perguntas dentro dos eixos, peça a participação dos alunos na construção do questionário. O número de perguntas deve ser definido pelo professor, de acordo com as possibilidades de tempo e recursos tecnológicos. 

Depois de terminar o questionário, os alunos poderão realizar a pesquisa em casa ou na escola, sendo preferencial no ambiente escolar. Caso seja feito em casa, oriente os pais sobre os objetivos da tarefa e como a criança deverá executá-la. Além disso, é preciso demonstrar aos alunos fontes seguras para realizar a pesquisa, assim como orientar os pais sobre este protocolo. 


\paragraph{Tempo estimado} Duas aulas de 50 minutos.

\SideImage{Ilustração da primeira edição de \textit{A menina do narizinho arrebitado}. Evidencia o caráter fabular das narrativas. (CC BY-NC 2.0)}{PNLD2023-029-04.png}

\subsection{Leitura}
\BNCC{EF05LP04}

\paragraph{Tema} Leitura dialogada e análise do enredo.

\paragraph{Conteúdo} Contos do livro de Monteiro Lobato; identificação de aspectos constitutivos dos textos: diálogos, enredo, espaço/tempo onde a história se passa, o gênero e acentuação gráfica.

\paragraph{Objetivo} Promover a leitura dialogada, o conhecimento de aspectos literários e de língua portuguesa.

\paragraph{Justificativa} Através da análise do texto e da leitura dialogada, os alunos terão a possibilidade de aprender em conjunto aspectos gramaticais e de interpretação de texto.


\paragraph{Metodologia} Após a realização da pesquisa de pré-leitura, os alunos já terão muito mais proximidade com o autor. Pergunte como foi a experiência de realizar a pesquisa e quais aspectos eles consideraram mais importantes.

Depois da conversa distribua os livros e descreva como a atividade será realizada. Os alunos deverão ler partes dos contos como se estivessem passando uma leitura no teatro, cinema ou \textsc{tv}.

Estabeleça quem será o narrador e quem serão as personagens em cada rodada de leitura. Estimule os alunos a entrar nas personagens colocando entonação e intenção em suas falas. Planeje antecipadamente a ordem dos leitores para a leitura ter fluidez. Faça pausas nos diálogos para associar e evidenciar características das narrativas, tais como:

\begin{enumerate}
\item A introdução das personagens;

\item Tempo/espaço narrativo;

\item O motor do enredo;

\item O desfecho da narrativa;

\item A moral da história;
\end{enumerate}

É importante sempre dar espaço e abertura para que, além de identificar os elementos do enredo, os alunos possam emitir opiniões e juízos sobre os contos lidos: as personagens que mais gostaram, as histórias mais divertidas, algum elemento particular que chamou a atenção deles. Também podem ser exploradas as características gramaticais do texto, associando-as com outros conteúdos programáticos do ano. 

Todos os alunos devem participar da leitura, obedecendo a estrutura: narrador e personagens. O livro deve ser lido de maneira integral, dessa forma as leituras poderão ter continuação em várias aulas seguintes, obedecendo o mesmo método e técnica. Todos devem participar da leitura dramática do livro.


\paragraph{Tempo estimado} Três aulas de 50 minutos.

\subsection{Pós-leitura}

\paragraph{Tema} Debate sobre a pesquisa bibliográfica e produção de um conto.

\paragraph{Conteúdo} Produção textual do gênero conto e reflexão sobre aspectos bibliográficos do autor.


\paragraph{Objetivo} Estimular a produção textual e o diálogo crítico sobre a pesquisa e o conteúdo dos contos.

\paragraph{Justificativa} A produção textual e o diálogo sobre as atividades produzidas a partir do contato com o livro possibilitam ao aluno não só o papel de leitor e receptor de conteúdo, mas também de participante ativo na produção de seu próprio conhecimento. Tal postura privilegia a identificação com o autor e obra, tornando o contato com a literatura mais significativo e próximo da realidade do aluno.

\Image{Os alunos deverão se atentar para as características da personalidade da personagem. (CC BY-NC 2.0)}{PNLD2023-029-05.png}

\paragraph{Metodologia} Quando a leitura dos contos terminar, o professor retomará o conteúdo da atividade de pré-leitura, a pesquisa bibliográfica. Os alunos formarão grupos de até quatro pessoas para discutir aspectos de suas pesquisas, apontar diferenças e semelhanças de escrita, de conteúdo e as dificuldades da investigação.

Após a reunião, os alunos produzirão uma carta de apresentação e recomendação de Monteiro Lobato com o seguinte tema: ‘’Por que você vai gostar das histórias de Monteiro Lobato?’’. Cada grupo criará sua carta e, na sequência, os grupos trocarão as cartas entre si.

Outro desdobramento interessante para a atividade é selecionar, com os alunos, os personagens que consideraram mais significativos nos contos. 
Depois dessa etapa, reúna os nomes e realize um sorteio, no qual cada aluno sorteie um personagem, como Emília, Tia Anastácia, Dona Benta, Rabicó, Rã, Pedrinho etc.

Quando o sorteio terminar, explique a dinâmica da atividade: os alunos deverão se atentar para as características da personalidade da personagem: Tia Anastácia é cozinheira, vive no sítio com Dona Benta; tio Barnabé é muito legal e divertido, e por aí vai. 
Cada um deverá estudar o personagem que sorteou e produzir um pequeno conto com uma história original que pode ser ambientada em qualquer lugar que o aluno quiser e envolver personagens que não necessariamente estão no livro de Monteiro, desde que consigam, de alguma forma, incluir o personagem que sortearam e estudaram na sua própria narrativa. Explique os aspectos do gênero conto e auxilie os alunos na produção do texto.


\paragraph{Tempo estimado} Duas aulas de 50 minutos.

\section{Sugestões de referências complementares}

\subsection{Audiovisual}

\begin{itemize}

\item \textit{O comprador de fazendas}, filme adaptado do conto de Monteiro
Lobato, do volume \emph{Urupês} (1918). Direção de Alberto Pieralisi.
São Paulo, Companhia Cinematográfica Maristela, 1950. Comédia, P\&B.

O filme, disponível no \href{https://www.youtube.com/watch?v=LcdfdfD9_Bs}{Youtube}, 
narra a história de um fazendeiro arruinado do Vale do Paraíba, em
São Paulo, que decidindo vender sua propriedade, coloca um anúncio no jornal.
Com a chegada do possível comprador, arregimenta"-se uma série de eventos para 
alçar a fazenda. 
\end{itemize}

\subsection{\emph{Sites}}

\begin{itemize}

\item Dicionário Aulete digital

\href{http://www.aulete.com.br/}{Dicionário Aulete da língua portuguesa} para 
consultar termos e palavras usados na literatura de outras épocas.

\item  Site oficial de Monteiro Lobato

O site \href{https://lobato.com.vc//}{Lobato com você}, em formato de 
blog, apresenta a vida e obra do autor Monteiro Lobato 
reunindo matérias de jornais e entrevistas com outros escritores e aficionados 
por sua obra. 

\end{itemize}

\section{Bibliografia comentada}

\begin{itemize}
\item  \textsc{andrade}, Mário de. ``Contos e contistas'' {[}1938{]}. In: \emph{O
empalhador de passarinho}. 3. ed. São Paulo; Brasília: Martins; \textsc{inl},
1972, pp. 5-8. 

Motivado por uma pesquisa da \emph{Revista Acadêmica} em
busca dos dez melhores contos brasileiros, o artigo reflete sobre esse
gênero literário.

\item  \textsc{carpeaux}, Otto Maria. ``Obras-primas desconhecidas do conto
brasileiro'', \emph{A Manhã}, ``Letras e Artes'', Rio de Janeiro, 10
abr. 1949; \emph{Folha da Manhã}, Quarto Caderno, São Paulo, 15 maio
1949, pp. 14-5. In: \textsc{ramos}, Graciliano. \emph{Conversas}. Organização de
Thiago Mio Salla e Ieda Lebensztayn. Rio de Janeiro: Record, 2014, p.
207--213. 

A conversa entre o crítico e o escritor convida os leitores a
conhecerem não só diversos contistas, como também critérios de
construção artística para avaliar os textos preferidos.

\item \textsc{lobato}, Monteiro \& \textsc{rangel}, Godofredo. \emph{A barca de Gleyre: quarenta
anos de correspondência literária}. São Paulo: Brasiliense, 1968.

O livro apresenta os largos anos de correspondência ativa de Monteiro Lobato com Godofredo 
Rangel, percorrendo o período da República Velha até o governo Dutra.

\item \textsc{propp}, Vladimir. \emph{Morfologia do conto maravilhoso}. Tradução de
Jasna Paravich Sarhan. Organização e prefácio de Boris Schnaiderman. 2.
ed. Rio de Janeiro: Forense Universitária, 2006. 

Propp se dedica à descrição de contos populares russos, formados por esquemas narrativos
constantes, em busca de conhecer sua estrutura e de definir o conto
maravilhoso.
\end{itemize}

\end{document}

