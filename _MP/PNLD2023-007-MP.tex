\documentclass[11pt]{extarticle}
\usepackage{manualdoprofessor}
\usepackage{fichatecnica}
\usepackage{lipsum,media9}
\usepackage[justification=raggedright]{caption}
\usepackage[one]{bncc}
\usepackage[nmenosum]{../edlab}
\usepackage{marginnote}
\usepackage{pdfpages}

\newcommand{\AutorLivro}{Barbie Furtado}
\newcommand{\TituloLivro}{O dia em que o mundo parou}
%\newcommand{\Tema}{}
\newcommand{\Genero}{Conto; crônica; novela}
%\newcommand{\imagemCapa}{./images/PNLD2022-001-01.jpeg}
\newcommand{\issnppub}{978-65-86941-80-7}
\newcommand{\issnepub}{XXX-XX-XXXXX-XX-X}
% \newcommand{\fichacatalografica}{PNLD0001-00.png}
\newcommand{\colaborador}{Ana Lancman}
\begin{document}

\title{\TituloLivro}
\author{\AutorLivro}
\def\authornotes{\colaborador}

\date{}
\maketitle

%\begin{abstract}\addcontentsline{toc}{section}{Carta ao professor}
%\pagebreak

\tableofcontents

\pagebreak

\begin{abstract}

O presente manual tem como objetivo oferecer uma orientação ao professor sobre a obra \textit{O dia em que o mundo parou}. A partir deste manual, os professores poderão incentivar a prática da leitura aos estudantes e proporcionar um conteúdo enriquecedor. Apresentamos aqui sugestões de atividades a serem realizadas antes, durante e após a leitura do livro, com propostas que buscam introduzir os gêneros literários e aprofundar as discussões trazidas pelas obras. Você encontrará informações sobre o autor, sobre o gênero e sobre os temas trabalhados ao longo do livro. Ao fim do manual, você encontrará também sugestões de livros, artigos e sites selecionados para enriquecer a sua experiência de leitura e, consequentemente, a de seus estudantes.

\textit{O dia em que o mundo parou} foi escrito por Barbie Furtado e ilustrado por Emanuel Oliveira. O livro trata de Clara, uma menina de nove anos, e seu pai que vão pescar em um dia que o sol nasce do lado oposto. Tudo se torna estranho muito rapidamente, quando os moradores do povoado percebem que os seres da natureza agem de forma inesperada. Ao longo da história, acompanhamos a jornada de Clara para tentar entender o que está acontecendo com o mundo.

É uma obra que emociona ao abordar temas urgentes como a relação dos seres humanos com a natureza e a poluição do meio ambiente. 

Esperamos que as atividades sugeridas e o material indicado sejam proveitosos em sala de aula! 

\end{abstract}


\section{Sobre o livro}

Dividido em cinco capítulos, este livro é uma narrativa curta. \textit{O dia em que o mundo parou} conta a história de Clara, uma menina de nove anos que acompanha seu pai, José, enquanto ele pesca. Normalmente, sempre encontram peixes no mar, mas naquele dia não conseguem pescar nenhum. Clara se questiona porque os peixes desapareceram, intrigada. Ambos percebem também que o dia está mais escuro do que o normal, mesmo antes do amanhecer. Subitamente, a menina acredita ter pescado algo grande e quando consegue puxar a vara, descobre que é um saco de lixo. Inconformada com a poluição no mar, ela critica fervorosamente as pessoas que jogam lixo no oceano. 

De repente, descobrem que o sol está nascendo no lado contrário, no lado oeste. Todos vão em direção ao sol para tentar descobrir o que aconteceu. A partir daí, tudo parece muito estranho na cidade: um cachorro chamado Miojo gira ao redor do próprio rabo sem parar, as formigas andam de frente pra trás e os pássaros sumiram.

Os moradores do povoado se dirigem então para a Capela São Pedro dos Pescadores, buscando acalento. Todos param para ouvir o discurso de Dona Mundica, uma senhora que teria poderes mágicos, segundo o que a comunidade especulava. Ela argumenta que os seres humanos tem a mania de pensar que são donos da natureza e vivem sem se importar com as consequências dos rastros que deixam para o planeta. A fala da senhora emociona a todos, que decidem sair para ver o pôr do sol e aplaudir. A lua aparece no céu e decidem esperar o dia seguinte para ver o que poderia acontecer. Por fim, Clara reflete que, a partir daquele momento, ela e as pessoas em seu entorno poderiam agir pensando nos efeitos de suas ações para a natureza.

\section{Sobre os autores}

\paragraph{A autora} Barbie Furtado nasceu em Fortaleza. Fez a graduação em Letras Português--Inglês e o mestrado em Linguística, ambos pela Universidade Federal do Ceará (\textsc{ufc}). Trabalha como tradutora e professora de inglês. Seu conto ``A menina dos olhos dourados'' foi publicado na na coletânea \textit{Contos de Travessia} e o conto ``Loop'' na coletânea \textit{Farol}. Em 2018, estudou roteiro na \textit{New York Film Academy}, em Nova York, \textsc{eua}. 

\SideImage{A autora Barbie Furtado (Arquivo pessoal)}{PNLD2023-007-02.png}

\paragraph{O ilustrador} Emanuel Oliveira nasceu e vive em Fortaleza. Faz fanzines desde adolescente, para trocar pelos quadrinhos de amigos. Formou-se em faculdade de Artes Visuais pelo Instituto Federal de Educação, Ciência e Tecnologia do Ceará (\textsc{ifce}). Realizou trabalhos com pintura, colagem e escultura, mas seu foco principal sempre foi o desenho. Além de \textit{O dia em que o mundo parou}, ilustrou os livros ``De um jeito que não era'' e o ``O segredo de Joãozinho''.

\SideImage{O ilustrador Emanuel Oliveira (Arquivo pessoal)}{PNLD2023-007-03.png}

\section{Sobre o gênero}

O gênero deste livro é \textit{conto; crônica; novela}.
O que define um gênero narrativo é o fato de, não importa qual seja sua forma, \textit{contar uma história}.
As especificidades do \textit{como} esta história será contada caracteriza os tipos de gênero narrativo, que podem ser: conto, crônica, novela, epopeia, romance ou fábula. 

Toda narrativa possui, necessariamente, um narrador, uma personagem, um enredo, um tempo e um espaço. O narrador, ou narradora, pode ser onisciente, literalmente \textit{que tudo sabe}, observador ou personagem --- categorias que não são exclusivas. O discurso elaborado por este narrador pode ser direto, indireto ou indireto livre --- ou seja, ele pode aparecer mais diretamente ou mais indiretamente; no último caso, sua voz se mistura à das personagens da história.

Sobre o enredo das narrativas curtas sabemos se diz que:

\begin{quote}
``comumente são simples, se passam em um espaço único, em um curto período de tempo e apresentam poucas personagens. Os temas giram em torno de episódios do mundo infantil ou de episódios envolvendo animais. As ilustrações ocupam quase toda a página e auxiliam 
a criança a identificar, ma narrativa, as características externas das personagens, as ações vividas por elas e os espaços onde ocorrem as cenas. A linguagem é simples, sem muitos elos frasais. A história se constrói, quase sempre, por meio de diálogos. A presença do narrador é bastante pequena.''\footnote{“Narrativas infantis”, de Luiza Vilma Pires Vale. In \textsc{saraiva}, J. A. (Org.) \textit{Literatura e alfabetização: do plano do choro ao plano da ação}. Porto Alegre: Artmed, 2001.}  
\end{quote}

O narrador \textbf{não é necessariamente} a voz do autor. É errada a afirmação de que o autor fala através do narrador de uma história. É bastante comum, há algum tempo na história literária, sobretudo desde os pré-modernistas, que o narrador represente justamente o contrário do que pensa o autor. Neste caso, utiliza-se elementos como a \textbf{ironia} para sugerir que o autor \textit{não é confiável}.

Já as personagens variam quanto a sua \textbf{profundidade}. Há personagens planas, ou personagens-tipo, e personagens redondas, ou complexas. Personagens planas são facilmente repetíveis pois se amparam em lugares-comuns da cultura, como o vilão, o herói, a vítima, o palhaço, tudo isso com marcações de gênero e espécie --- o herói tradicionalmente é um homem, a vítima, uma mulher, e o vilão, uma figura que se afasta da humanidade por alguma razão, às vezes sobrenatural. Personagens redondos, por outro lado, estão mais próximos das \textit{pessoas reais}. Uma personagem complexa pode ser, em um dado momento da narrativa, vilã, e em outro, heroína. É importante notar como as visões de mundo influenciam na caracterização das personagens de uma história.

O tempo de uma narrativa pode ser cronológico ou psicológico. No tempo cronológico, o enredo segue a ordem ``normal'' dos acontecimentos, aquela marcada pelo relógio e pelo calendário. Os acontecimentos vêm um após o outro e se delimita muito bem \textit{passado}, \textit{presente} e \textit{futuro}. Já no tempo psicológico, segue-se uma ordem \textit{subjetiva} dos acontecimentos, e portanto, \textit{não linear}, já que a influência emocional e psíquica afeta a racionalidade do tempo cronológico. 

O espaço, por fim, é o lugar onde se passa a narrativa. Dependendo do caso, ele pode funcionar mais como um plano de fundo, sem muita interferência no enredo, ou mais ativamente, aproximando-se das características das personagens e influenciando no desenrolar da trama. 

\Image{O que define um gênero narrativo é o fato de, não importa qual seja sua forma, contar uma história. (Dorothe/Px Here; Domínio público)}{PNLD2023-007-07.png}

O último aspecto de um gênero narrativo que podemos abordar é sua \textit{extensão}. Dentre os elementos que distinguem um subgênero de outro é o tamanho da história: uma crônica e um conto são \textit{necessariamente} curtos, ao passo que uma epopeia e um romance, são longos. Uma novela está no ponto intermediário entre um romance e um conto. Ainda poderíamos falar dos registros de cada subgênero: a epopeia é originalmente um subgênero \textit{oral}, versificado, e metrificado, já o romance é tradicionalmente \textit{escrito} em prosa.  Desde meados do século \textsc{xviii}, no entanto, o estabelecimento dos gêneros e subgêneros narrativos tornam-se cada vez menos rígido, com as características cada vez mais fluidas e intercomunicativas.

Como o presente livro se trata de uma narrativa \textit{curta}, finalizamos com as palavras de Luiza Vilma Pires a respeito do
subgênero:

\begin{quote}
``sob o nome de narrativa curta, estão situadas obras que apresentam uma trama um pouco mais complexa, que ocorre em diversos espaços e em uma temporalidade que pode ser de vários dias, semanas ou meses. Entretanto a função das ilustrações continua as mesmas, são complementares à história e contribuem para sua compreensão. Os temas relacionam-se a vivência infantis (brincadeiras, passeios, pequenas aventuras), a aspectos ligados à interioridade das personagens (busca de identidade, insegurança,  
medos) ou a relações interpessoais (desentendimentos familiares, entre amigos, solidariedade).''\footnote{“Narrativas infantis”, de Luiza Vilma Pires Vale. In \textsc{saraiva}, J. A. (Org.) \textit{Literatura e alfabetização: do plano do choro ao plano da ação}. Porto Alegre: Artmed, 2001.} 
\end{quote}

\section{Atividades}

\subsection{Pré-leitura}

\BNCC{EF04GE11}
% Identificar: conservação e ação humanas nas paisagens naturais;
\BNCC{EF05GE10}
% Identificar: formas de poluição dos cursos de água e dos oceanos;
\BNCC{EF05GE11}
% Identificar: problemas ambientais que ocorrem no entorno do aluno; 
\BNCC{EF05CI05}
% Produzir: ``propostas coletivas para um consumo mais consciente''; ``reciclagem''  
\BNCC{EF05LP17}
%Produzir: roteiro de reportagem; 

\paragraph{Tema} A poluição do meio ambiente.

\paragraph{Conteúdo} Estudo sobre a poluição da natureza e produção de reportagem digital sobre formas de prevenir o despejo irregular de lixo.

\paragraph{Objetivo} Aproximar os estudantes da discussão sobre a defesa do meio ambiente e incentivar que eles identifiquem problemas ambientais de seu entorno. 

\Image{O plástico é responsável por 80\% do lixo nos oceanos. (Tkremmel/Pixabay; Domínio público)}{PNLD2023-007-08.png}

\paragraph{Justificativa} Para introduzir o livro \textit{O dia em que o mundo parou}, é importante abordar o tema das consequências das ações humanas nas paisagens naturais, em especial o plástico que vai para os oceanos. Junto dos professores de Geografia e Ciências, será possível aprofundar essa questão de forma que envolva os estudantes a se mobilizarem para defender a natureza em seu entorno.

\paragraph{Metodologia} Para iniciar a atividade, deverá ser apresentado um panorama sobre as principais causas da poluição na natureza. Como é um tema amplo, poderá ter um enfoque na poluição dos oceanos, que é um dos tópicos centrais de \textit{O dia em que o mundo parou}. Sugere-se que seja exibido em sala de aula o vídeo \textit{Microplásticos e a poluição nos oceanos}, do canal \textit{Minuto da Terra}. Pode ser acessado gratuitamente no \href{https://youtu.be/adc0cOqE4qs}{Youtube}. 

Em seguida, proponha aos alunos que se dividam em grupos pequenos. Cada grupo realizará uma pesquisa sobre um caso próximo a escola em que alguma área tenha sido afetada por ações dos seres humanos, em especial o despejo irregular de lixo. A pesquisa poderá ser feita na internet ou em meios impressos, como jornais e revistas. Deverão registrar as características geográficas do local, quais aspectos naturais foram prejudicados e se existem iniciativas para tentar reverter esse cenário.

\Image{A reciclagem ajuda a conservar recursos naturais. (OpenClipart-Vectors/Pixabay; Domínio público)}{PNLD2023-007-09.png} 

Os grupos vão redigir um roteiro para produzir uma reportagem digital sobre o assunto. A reportagem poderá ser feita por escrito ou em formato de vídeo. A ideia é que tenha um caráter informativo e que hajam propostas de prevenção e cuidado com o meio ambiente. Quando finalizarem, poderão inserir a reportagem no site da escola, em um blog criado pelo grupo ou um perfil de rede social destinado ao projeto.

\paragraph{Tempo estimado} Quatro aulas de 50 minutos.

\subsection{Leitura}

\BNCC{EF15LP16} 
%Ler: narrativas, contos, crônicas; +grupo, +professor e +sozinho; Mundo imaginário;
\BNCC{EF15LP11} 
%Identificar: +Falar: conversação; ``roda de conversa'', ``discussão'';
\BNCC{EF35LP21}
% Ler: +sozinho; Gênero; preferências por gêneros; Opinião;
\BNCC{EF35LP26}
% Ler: +sozinho; narrativa com personagem; Compreensão de texto
\BNCC{EF05LP27}
% Produzir; redação; pronomes anafóricos; relações de sentido; Gramática 

\paragraph{Tema} O enredo de \textit{O dia em que o mundo parou}.

\paragraph{Conteúdo} Leitura individual da obra, discussão acerca dos elementos principais da narrativa e redação de texto argumentativo.

\paragraph{Objetivo} Incentivar que os estudantes identifiquem os diferentes momentos da narrativa e expressem sua opinião sobre as questões trazidas pela obra.

\Image{Ilustração do livro, página 11}{PNLD2023-007-04.png}

\paragraph{Justificativa} \textit{O dia em que o mundo parou} é uma narrativa curta, que poderá ser um exercício interessante para uma leitura individual. É uma obra que deixa diversas questões em aberto, que serão discutidas em sala de aula, proporcionando um aprofundamento de sua compreensão textual.

\paragraph{Metodologia} Peça aos alunos que leiam \textit{O dia em que o mundo parou} silenciosamente. É importante proporcionar um ambiente tranquilo para que os estudantes consigam se concentrar. Acompanhe a sala de aula, auxiliando no que for necessário. Dedique um tempo da aula para essa leitura.

\Image{Ilustração do livro, página 21}{PNLD2023-007-05.png}

Após a leitura individual, será o momento de trabalhar coletivamente alguns trechos mais marcantes da obra. Sugere-se que o professor leia em voz alta as seguintes passagens:

\begin{quote}
— E você acha que tá acontecendo o que, se não é o fim do mundo, pai? – Clara perguntou \ldots{}\footnote{Página 31 do livro.}\end{quote}

\begin{quote}
Especulava-se sobre o fim do mundo, magia negra, profecia maldita e tudo que pudesse imaginar, mas ninguém podia explicar o que estava acontecendo.\footnote{Página 32 do livro.}\end{quote}

\begin{quote}
Nós, como seres humanos, temos a mania de pensar que somos donos da natureza, e não que somos parte dela. Não pensamos em nossas ações do dia a dia, que o equilíbrio de nossas vidas depende do funcionamento perfeito de todas as suas funções, dos dias, das horas, do lugar que o sol nasce, do lugar que o sol se põe. E o que nós fazemos? Vivemos sem nos importar com as consequências.\footnote{Página 36 do livro.}\end{quote}

A partir desses trechos, será feita uma conversa livre, em que os estudantes poderão expressar quais sensações tiveram ao longo da leitura. Peça que os alunos identifiquem quem são os personagens principais, qual o conflito da narrativa e como se desenvolve.

\Image{Ilustração do livro, página 25}{PNLD2023-007-06.png} 

Em seguida, cada aluno escreverá um texto argumentativo, em que explicarão o que acreditam que aconteceu para o dia parar. Deverão abordar os motivos por trás do sol ter nascido do lado oposto e defender o que poderá ter acontecido no dia seguinte. Como se trata de um tema fantasioso, é interessante incentivar os estudantes a realizar uma escrita criativa a partir de relações de causa e efeito. O texto dos alunos deverá ser avaliado e entregue na próxima aula com comentários e correções gramaticais.

\paragraph{Tempo estimado} Duas aulas de 50 minutos.

\subsection{Pós-leitura}

\BNCC{EF15LP14}
% Identificar: ``fale com suas próprias palavras'', quadrinhos, tirinhas;
\BNCC{EF15AR04}
% Produzir: desenho, pintura, colagem, quadrinhos, dobradura, escultura, modelagem...; 
\BNCC{EF15AR13}
%Ler +Apreciar: ``escutar músicas'' de gêneros diferentes reconhecendo a função; 
\BNCC{EF15AR23}
%Identificar: relação entre diferentes/diversas linguagens artísticas;

\paragraph{Tema} \textit{O dia em que o mundo parou} em quadrinhos.

\paragraph{Conteúdo} Confecção de uma história em quadrinhos a partir do livro e da música \textit{O dia em que a Terra parou}, composta por Raul Seixas e Cláudio Roberto.


\paragraph{Objetivo} Incentivar a expressão artística dos alunos através da relação entre diferentes linguagens artísticas.
\SideImage{O cantor Raul Seixas, em 1972. (Arquivo Nacional; Domínio público)}{PNLD2023-007-10.png} 

\paragraph{Justificativa} O livro \textit{O dia em que o mundo parou} aborda temas atuais, como a preocupação com o futuro do planeta e desastres ambientais. É possível relacionar o texto com a canção \textit{O dia em que a Terra parou} de Raul Seixas, que também trata de um dia em que tudo parou. Além disso, essas duas obras artísticas podem ser associadas ao contexto da pandemia da \textsc{covid}-19 e a vivência da quarentena. Através de uma atividade artística, será possível encorajar os estudantes a retratarem sua visão sobre um dia de pausa no planeta.

\paragraph{Metodologia} Apresente em sala de aula a música \textit{O dia em que a Terra parou}, de Raul Seixas, que está disponível no \href{https://youtu.be/SqQfySakoK0}{Youtube}. Em seguida, leia em voz alta a parte inicial da canção:

\begin{verse}
Essa noite eu tive um sonho\\
de sonhador\\
Maluco que sou, eu sonhei\\
Com o dia em que a Terra parou\\
com o dia em que a Terra parou

Foi assim\\
No dia em que todas as pessoas\\
Do planeta inteiro\\
Resolveram que ninguém ia sair de casa\\
Como que se fosse combinado em todo\\
o planeta\\
Naquele dia, ninguém saiu de casa, ninguém ninguém

\ldots

No dia em que a Terra parou
\end{verse}

Proponha uma conversa em sala de aula em que os estudantes poderão falar sobre as relações possíveis entre esta canção, o livro \textit{O dia em que a Terra parou} e a experiência de estar quarentenado, em decorrência da pandemia do coronavírus. Estimule que os alunos relatem vivências pessoais e expressem o que compreendem sobre um ``mundo pausado''.

Em seguida, os alunos produzirão uma história em quadrinhos sobre um dia em que a Terra parou. A criação livre deve ser incentivada, com a invenção de personagens e cenários. Será interessante revisitar a reportagem feita na atividade de pré-leitura e o texto argumentativo da atividade de leitura para buscar inspiração. 

As histórias em quadrinhos poderão ser feitas individualmente ou em grupo. Disponibilize os materiais artísticos da escola ou peça que os alunos tragam de casa. Poderão ser utilizadas diferentes técnicas artísticas, como o desenho, a pintura ou a colagem. Acompanhe a produção dos alunos com o auxílio do professor de Artes. Ao final da atividade, as histórias em quadrinhos poderão ser exibidas em um mural da escola.

\paragraph{Tempo estimado} Quatro aulas de 50 minutos.

\section{Sugestão de referências complementares}

\subsection{Filmes}

\subsection{Audiovisual}

\begin{itemize}

\item \textit{Microplásticos e a poluição nos oceanos}. Vídeo do canal \textit{Minuto da Terra}. 

A animação apresenta de forma didática as consequências do despejo de plástico nos oceanos, explicando em especial a existência dos microplásticos. Pode ser acessado no \href{https://youtu.be/adc0cOqE4qs}{Youtube}. 

\item \textit{O menino e o mundo}. Animação dirigida por Alê Abreu, 2013.

O filme é uma animação emocionante, que conta a história de Cuca, um menino que vive em um mundo distante, numa pequena aldeia no interior de seu mítico país. Sofrendo com a falta do pai, que parte em busca de trabalho na desconhecida capital, Cuca deixa sua aldeia e sai mundo afora a procura dele. Durante sua jornada, Cuca descobre uma sociedade marcada pela pobreza, exploração de trabalhadores e falta de perspectivas.

\item \textit{Wall-e}. Dirigido por Andrew Stanton, 2008.

A história segue um robô chamado WALL·E, criado no ano de 2100 para limpar a Terra coberta por lixo. Ele se apaixona por um outro robô, chamado EVA, que tem a missão de encontrar pelo menos uma planta na superfície do planeta Terra. Poderá ser um material complementar para a atividade de pré-leitura.

\end{itemize}

\subsection{Museus}

\begin{itemize}

\item \href{https://museucatavento.org.br/}{Catavento – Museu de Ciências}

Localizado no centro da cidade de São Paulo, o museu é dividido em quatro seções: universo, vida, engenho e sociedade. É um museu educativo e interativo.

Endereço: Av.\,Mercúrio, s/n--Parque \,Dom Pedro II, Centro, São Paulo--SP

\item {Museu do Amanhã}

Localizado no centro histórico do Rio de Janeiro, é um museu de ciências que foi inaugurado em 2015. Possui diversos ambientes audiovisuais e tecnologias interativas com o intuito de imaginar futuros possíveis.

Endereço: Praça Mauá, 1--Centro, Rio de Janeiro--RJ.

\end{itemize}

\section{Bibliografia comentada}

\begin{itemize}

\item \textsc{amoroso}, Marta; \textsc{lima}, Ana Gabriela Morim, \textsc{oliveira}, Joana Cabral. \textit{Vozes vegetais}. São Paulo: Ubu, 2020.

A obra trata da diversidade, resistências e histórias da floresta, propondo uma nova relação dos seres humanos com as plantas.

\item \textsc{foer}, Jonathan Safran. \textit{Nós somos o clima}. São Paulo: Rocco, 2021.

O livro aborda aspectos básicos da sustentabilidade em uma linguagem acessível.

\item \textsc{krenak}, Ailton. \textit{Ideias para adiar o fim do mundo}. São Paulo:
Companhia das Letras, 2019.

O líder indígena critica a ideia de humanidade como algo separado da natureza e recusa a ideia do humano como superior aos demais seres.

\item \textsc{latour}, Bruno. \textit{Diante de Gaia --- Oito conferências sobre a natureza no Antropoceno}. São Paulo: Ubu, 2021.

A obra reúne oito conferências de Bruno Latour, antropólogo e filósofo, acerca da catástrofe ecológica contemporânea.

\end{itemize}

\end{document}
