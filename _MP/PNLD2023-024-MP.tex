% \Image{Capa do livro (; )}{PNLD2022-001-01.png}
% \Image{Ilustração do livro (Acorde/Manuella Silveira; Acorde)}{PNLD2022-001-04.png}
% \Image{Ilustração do livro (Acorde/Manuella Silveira; Acorde)}{PNLD2022-001-05.png}
% \Image{Ilustração do livro (Acorde/Manuella Silveira; Acorde)}{PNLD2022-001-06.png}

\documentclass[11pt]{extarticle}
\usepackage{manualdoprofessor}
\usepackage{fichatecnica}
\usepackage{lipsum,media9}
\usepackage[justification=raggedright]{caption}
\usepackage[one]{bncc}
\usepackage[ayllon]{../edlab}
\usepackage{marginnote}
\usepackage{pdfpages}

\newcommand{\AutorLivro}{Richard Burton}
\newcommand{\TituloLivro}{O cavalo de ébano}
\newcommand{\Tema}{Diversão e aventura}
\newcommand{\Genero}{Lendas; mitos; fábula}
%\newcommand{\imagemCapa}{./images/PNLD2022-001-01.jpeg}
\newcommand{\issnppub}{XXX-XX-XXXXX-XX-X}
\newcommand{\issnepub}{XXX-XX-XXXXX-XX-X}
% \newcommand{\fichacatalografica}{PNLD0001-00.png}
\newcommand{\colaborador}{Paulo Pompermaier}

\begin{document}

\title{\TituloLivro}
\author{\AutorLivro}
\def\authornotes{\colaborador}

\date{}
\maketitle

\tableofcontents

\section{Carta ao professor}

Caro professor,

Este material tem a intenção de contribuir para que você desenvolva um trabalho aprofundado com a obra \textit{O cavalo de ébano} em sala de aula.
Você encontrará informações sobre o autor, sobre o gênero e também 
algumas propostas de trabalho para a sala de aula que você poderá explorar livremente, 
da forma que considerar mais apropriada para os seus estudantes.

O autor do livro, Richard Francis Burton (1821-1890), foi um dos maiores
exploradores do século \textsc{xix}, além de profundo conhecedor das culturas orientais e africanas. Durante suas perambulações pelo Oriente Médio, coletou as narrativas que depois publicaria nos dezesseis volumes das histórias das \textit{Mil e uma noites}, obra da qual foi extraída a narrativa de \textit{O cavalo de ébano}. Tal sua importância para a literatura que sua tradução para o inglês das \textit{Mil e uma noites} era consultada e apreciada pelo grande escritor e intelectual argentino Jorge Luis Borges.

Em \textit{O cavalo de ébano}, acompanhamos a história do filho do grande rei persa Sabur, o príncipe Kamar al-Akmar, em sua aventura para conseguir se casar com a princesa de um reino distante. Com o cavalo de ébano, artífice de um mago que permite que seu cavaleiro possa voar pelos ares, o príncipe conhece a princesa pela qual se apaixona, mas também passa por muitas peripécias. O desejo de vingança do mago, aprisionado após intentar contra a família do rei, e o amor que a princesa desperta em qualquer um que a contemple são alguns dos elementos que vão dificultar a jornada de Kamar al-Akmar.

Além da narrativa saborosa, que prende a atenção do leitor em meio a tantas aventuras, o livro é interessante, pois apresenta ao estudante um dos maiores clássicos da história da literatura, as \textit{Mil e uma noites}. Assim, pode-se não apenas explorar o enredo e a estrutura narrativa da história, bem como abordar características da lenda e do mito, além, claro, de expandir o conhecimento dos estudantes sobre os povos árabes, seus traços típicos, características sociais e culturais. No mundo globalizado no qual vivemos, saber reconhecer e respeitar as diferenças torna-se competência crucial para uma boa convivência e uma formação sólida.

Ao longo do manual, todos esses aspectos serão explorados e relacionados a sugestões de atividades. Com isso, pretendemos oferecer algumas ideias e inspirações para um trabalho que pode ser desenvolvido tanto a curto, quanto a médio e longo prazo. Sinta-se à vontade para personalizar a aula e torná-la sua, aplicando seus conhecimentos, sua 
personalidade e aproveite para fortalecer seu vínculo com a turma.
Boa aula!

\section{Sobre o livro}
Certa noite, para celebrar o Ano Novo e o Equinócio de Outono, o maior rei persa da história, Sabur, recebe em seu palácio três magos que o presenteiam com objetos encantados: um homem de ouro com trompete que faz um inimigo ou invasor morrer diante de sua imagem; um relógio formato por uma bandeja de prata e 25 figuras de pavões de ouro; um cavalo de ébano que, através de um mecanismo de alavancas, consegue voar e levar seu cavaleiro a qualquer parte do mundo.

Encantado sobretudo com o cavalo mágico, o rei promete a mão de sua filha ao mago. Este, no entanto, era um homem demasiado feio e velho, o que deixa a princesa muito triste e preocupada com seu futuro. Seu irmão, o príncipe Kamar al-Akmar, tenta convencer o pai a desfazer o casamento arranjado, mas é enganado pelo velho mago e lançado com o cavalo de ébano ao céu, quase chegando ao sol.

Quando finalmente consegue entender o mecanismo do cavalo, o príncipe volta para a terra e, na jornada de regresso ao seu reino, conhece a bela princesa de outro sultanato persa, por quem se apaixona. Perseguido pelo rei --- que pensava que, para subir ao quarto da princesa, Kamar deveria ser um demônio ---, o jovem volta às terras de seu pai Sabur, onde conta sua aventura e pede que libertem o mago em agradecimento. Retorna ao distante sultanado e foge com a princesa, com quem pretendia se casar.

No palácio de seu pai, deixa a princesa e o cavalo de madeira em um jardim, para preparar uma suntuosa recepção, quando o feiticeiro percebe a princesa e foge com ela por meio do cavalo. Ambos são capturados por um rei distante, que prende o mago, recolhe o cavalo ao seus tesouros e planeja se casar com a jovem moça. Kamar al-Akmar, percebendo-se logrado, anda incessantemente por todas as terras na procura de sua amada. Perto da Grécia tem notícias da princesa e, infiltrando-se no castelo, consegue enganar o rei e fugir com a jovem amada de volta para as terras de Sabur.

O rei fica muito feliz com a volta do filho, Kamar casa-se finalmente com a jovem princesa e ambos vivem muitos anos de paz e alegria. Prevenido, no entanto, Sabur destrói o cavalo de ébano, para que não volte a provocar novas desarmonias e confusões em sua família.

\reversemarginpar
\marginparwidth=5cm

\SideImage{Retrato do autor de finais da década de 1890. (CC BY-NC 2.0)}{PNLD2023-024-02.jpg}


\section{Sobre os autores}

\paragraph{O autor} Richard Francis Burton (1821-1890) foi um dos maiores
exploradores do século \textsc{xix}, além de profundo conhecedor das culturas orientais e africanas. Consta que Burton, que chegou a obter a patente de Capitão do Exército inglês, falava 29 idiomas. Em 1842 partiu para a Índia e, em Bombaim, estudou hindustani e persa. Em 1853 foi para o Cairo e percorreu os lugares sagrados da religião islâmica, experiência
que resultou no livro \textit{Peregrinações a Medina e Meca} (1855).
Fez diversas expedições pela África sob o patrocínio da Real Sociedade Geográfica inglesa em busca das nascentes do rio Nilo. Descobriu o Lago Tanganica, entre as repúblicas da
Tanzânia, Congo, Burundi e Zâmbia. Começou a publicar em 1855 sua tradução das \textsc{Mil e uma noites} em dezesseis volumes. Recebeu o título de Sir, outorgado pela Rainha Vitória,
em 1866. Em 1881, publicou um comentário sobre \textsc{Os lusíadas}, de Luís de Camões. Conheceu o Brasil e morou em Santos, no litoral do estado de São Paulo. Também percorreu
Minas Gerais, em viagem que registrou no livro \textsc{Explorações nos planaltos do Brasil} (1869).

\paragraph{A tradutora} Marta Chiarelli de Miranda nasceu no Rio de Janeiro, mas
atualmente reside em Florianópolis. É graduada em Comunicação Visual pela \textsc{PUC} do Rio de Janeiro e antes de trabalhar como revisora e tradutora atuou como desenhista gráfica e arte-finalista na redação do \textit{Jornal do Brasil}, no Rio de Janeiro, e na University Press da Carolina do Norte (\textsc{eua}). Como revisora de tradução realizou diversos trabalhos junto a editoras de São Paulo e do Rio de Janeiro, nos mais diversos gêneros literários, desde o início dos anos 1990. No campo da tradução tem livros publicados tanto na área de história, como de ficção e literatura infantojuvenil.

\paragraph{A ilustradora} Andréa Corbani nasceu em São Paulo em 1970. Sempre
quis ser desenhista. Na escola gostava de física e queria fazer astronomia. No vestibular escolheu física porque queria descobrir como o mundo funciona. Um dia, na biblioteca
da faculdade, viu um anúncio: “Descubra com quantos traços se faz um artista!” Ali percebeu que era com o desenho que descobriria o mundo. Foi desenhar e aprender gravura – uma técnica por meio da qual você grava na madeira ou no metal e reproduz a imagem no papel. Ilustrou seu primeiro livro em 2001. Gostou tanto de brincar com as imagens e o texto que não parou mais de ilustrar livros.

\Image{A lenda e o mito são narrativas fantasiosas transmitida pela tradição oral através dos tempos (The ponta cabeça; CC-BY-SA-4.0)}{PNLD2023-024-07.png}


\section{Sobre o gênero}

%55 caracteres
\paragraph{O gênero} O gênero deste livro é \textit{Lendas; mitos; fábula}. 

A lenda e o mito são narrativas fantasiosas transmitidas pela tradição oral através dos tempos. De caráter fantástico, as lendas e os mitos combinam fatos reais e históricos com fatos que não têm comprovação de acontecimento, a não ser pela palavra dos que sobraram para contar a história. As lendas e mitos de uma sociedade são fundamentais para que entendamos quem são essas pessoas e no que acreditam, bem como suas tradições. Uma lenda é verdadeira até que se prove o contrário. Com exemplos bem definidos em todos os países do mundo, as lendas e os mitos de um povo geralmente fornecem explicações plausíveis, e até certo ponto aceitáveis, para coisas que não têm explicações científicas comprovadas, como acontecimentos misteriosos ou sobrenaturais.

A fábula é uma narrativa curta em que os personagens principais geralmente são seres personificados. Esses seres apresentam características humanas, tais como a fala e traços de personalidade. Essas personagens podem ser também objetos animados ou deuses. Em cada história há uma lição de moral: uma mensagem de cunho educativo que busca conscientizar o leitor. A fábula tem estreita relação com o gênero conto, mas se diferencia pela centralidade dos personagens animais e pelo intuito de concluir a história com um ensinamento. É uma história que pode ser contada em prosa ou em versos. 

Sobre a origem da fábula, Douglas Tufano afirma que:

\begin{quote}
A fábula teria nascido provavelmente na Ásia Menor e daí teria passado pelas ilhas gregas, chegando ao continente helênico. Há registros sobre fábulas egípcias e hindus, mas sua criação é atribuída à Grécia, pois é onde a fábula passa a ser considerada como um tipo específico de criatividade dentro da teoria literária. 

Na Grécia, os primeiros exemplos de fábula datam do século \textsc{viii} a.C. Isso nos mostra, é claro, que Esopo não foi o inventor do gênero, mas sim o mais conhecido fabulista na Antiguidade como autor e narrador dessas pequenas histórias.\footnote{\textsc{tufano}, Douglas. \textit{Esopo: Fábulas completas}. São Paulo: Moderna, 2015.}
\end{quote}

Esopo foi um autor da Grécia Antiga a quem são atribuídas algumas das mais famosas fábulas, como \textit{A raposa e o cacho de uvas} e \textit{A galinha de ovos de ouro}. Diversas  histórias suas foram recontadas por La Fontaine, que é também um dos mais clássicos fabulistas do Ocidente.

No caso de \textit{O cavalo de ébano}, tem-se o caso clássico de uma lenda. A narrativa faz parte do conhecido ciclo de histórias das \textit{Mil e uma noites}. Segundo a tradição oriental, o rei Xariar, ao ser traído por uma de suas esposas, resolveu mandar matar as mulheres que desposava após a noite de núpcias.
Para evitar o mesmo destino, a rainha Xerazade conta ao rei, noite após noite, histórias e mais histórias, sempre deixando o desfecho para a noite seguinte, de
modo a garantir que Xariar poupe sua vida para saber o final.

\textit{As mil e noites}, livro que reúne as histórias contadas por
Xerazade a seu rei, já eram conhecidas na cultura árabe no século \textsc{ix}, formadas por narrativas de origem persa, sânscrita e grega. As histórias foram conservadas em manuscritos
dos séculos \textsc{xiii} a \textsc{xv}. A primeira tradução para um idioma
europeu foi feita por Antoine Galand para o francês e publicada entre os anos de 1704 e 1717. O manuscrito utilizado por Galan continha 282 noites.

O tradutor passou a incorporar histórias de outras fontes, como a de ``Simbad, o Marujo'',
ausente de todas as tradições manuscritas, além de ``Aladim
e a lâmpada maravilhosa'' e ``Ali Babá e os quarenta ladrões'',
que teriam sido ouvidas de um contador de histórias sírio.

A partir da tradução de Galan, \textit{As mil e uma noites} tornaram-se conhecidas em todo o mundo como um grande patrimônio de cultura, costumes e histórias tradicionais.

\Image{Desenho de Marie-Éléonore Godefroid, de 1842, retratando o rei Xariar e Xerazade. (CC BY-NC 2.0)}{PNLD2023-024-03.jpg}

\section{Atividades}

\subsection{Pré-leitura}
\BNCC{EF15AR03}
\BNCC{EF05ER02}
\BNCC{EF05GE02}

\paragraph{Tema} Lendas e mitos, suas características estéticas e culturais.

\paragraph{Conteúdo} Aproximação do gênero lendas e mitos, para o reconhecimento das características desse gênero literário, ampliando a compreensão e o repertório literário dos estudantes.

\paragraph{Objetivo} Proporcionar aos estudantes a aproximação com a literatura árabe, reconhecendo a cultura, a diversidade e as influências que compõem lendas e mitos árabes.

\paragraph{Justificativa} O gênero das lendas e mitos tem fortes raízes na oralidade e se constitui na união de eventos reais com elementos fantasiosos. As lendas e os mitos possuem forte relação com tradições, modos, costumes e a cultura em geral de uma sociedade e transmitem valores quistos àqueles que as criam --- a partir da mistura de eventos e personagens possivelmente reais e outros fantasiosos. Por essa razão, entre outras, este gênero possui uma gama ampla de informações que permite aos leitores conhecer a cultura de determinada sociedade, além de proporcionar o deleite de uma história repleta de fantasia. 

\paragraph{Metodologia} Para esta primeira aproximação, é importante explorar as características gerais do gênero literário de lendas e mitos, ressaltando os aspectos que constituem este gênero como único, orientando em relação às dúvidas, mas também estimulando perguntas e hipóteses prévias, tais como:

\begin{itemize}
\item Quais seriam as características das lendas e mitos?
\item Alguém conhece a história do Aladin?
\item E a história do marujo Simbad?
\item Se sim, a cultura e os costumes retratados nessas histórias são como os nossos?
\item O que têm de diferente? 
\item E de semelhante?
\end{itemize}

Após esta primeira aproximação, fale sobre as lendas árabes e suas características, citando o \textit{Livro das mil e uma noites}, seu mito de origem e algumas das histórias mais representativas desse compêndio de narrativas. Apresente ilustrações e fotos que retratam a cultura árabe, comparando-as com a realidade brasileira. É importante ressaltar a riqueza da diversidade cultural e social do mundo, falando sobre os fluxos migratórios e a configuração globalizada do mundo no século \textsc{xxi}. Pode-se projetar imagens que mostram o café da manhã nessas culturas, bem como as vestimentas e práticas religiosas, ou outros aspectos culturais que o docente acreditar pertinentes para a discussão na turma.

\paragraph{Tempo estimado} Duas aulas de 50 minutos.

\Image{Desenho de Léon Carré (1878-1942) para ilustrar a edição francesa de \textit{As mil e uma noites} de 1926. Ao alto, observa-se o príncipe voando no cavalo alado. (CC BY-NC 2.0)}{PNLD2023-024-04.jpg}


\subsection{Leitura}
\BNCC{EF04GE01}
\BNCC{EF05LP12}
\BNCC{EF05HI01}

\paragraph{Tema} A cultura árabe e sua influência na produção literária.

\paragraph{Conteúdo} Leitura dialogada de \textit{O cavalo de ébano} e discussão a respeito da cultura árabe retratada na lenda.

\paragraph{Objetivo} Aprofundar a compreensão sobre o gênero literário das lendas e mitos e proporcionar a reflexão sobre diferentes costumes, tradições e valores de uma sociedade e da tradição oral da qual vêm as lendas e mitos.

\paragraph{Justificativa} A leitura compartilhada proporciona integração da turma, permitindo que a experiência estética coletiva amplie as possibilidades de exploração da história ao aproximar distintas percepções. Além disso, a leitura em voz alta remonta à tradição oral da contação de histórias, que é a raiz de muitos gêneros literários, além de trabalhar a qualidade da leitura dos estudantes. Aliado a isso, a discussão em grupo sobre as características culturais das histórias árabes contribui para a valorização da diversidade, bem como o reconhecimento de distinções culturais de nível continental.

\paragraph{Metodologia} Após a introdução na atividade de pré-leitura da cultura árabe, realize uma leitura compartilhada em que diferentes alunos assumem a leitura conforme a mudança de capítulo. Após a leitura, peça que os alunos compartilhem suas impressões sobre aspectos da história que retratam a cultura árabe.
O professor pode motivar o debate com algumas perguntas como:

\begin{itemize}
\item Quais palavras vocês estranharam?
\item Vamos pesquisar seu significado?
\item Quem pode citar um aspecto cultural da narrativa que é diferente do que conhecemos aqui no Brasil?
\item Os casamentos ocorrem a partir de arranjos familiares, como na história?
\item Se não, quais são as diferenças que percebem?
\item Quais aspectos das ilustrações chamaram a atenção?
\item Como são as vestimentas e as características arquitetônicas retratadas nos desenhos?
\item Quais as diferenças em relação às vestimentas e construções que percebemos no Brasil?
\end{itemize}

Após a discussão, divida os alunos em grupos de cinco e solicite que pesquisem e elaborem um glossário com as palavras e expressões mais diferentes que encontraram no texto e tragam para compartilhar com a turma toda. 

\paragraph{Tempo estimado} Duas aulas de 50 minutos.


\Image{Desenho de Léon Carré (1878-1942) para ilustrar a edição francesa de \textit{As mil e uma noites} de 1926. Nota-se o aspecto fantástico das personagens. (CC BY-NC 2.0)}{PNLD2023-024-05.jpg}


\subsection{Pós-leitura}
\BNCC{EF35LP25}
\BNCC{EF04GE01}
\BNCC{EF04LP21}


\paragraph{Tema} Redação: o contemporâneo nas lendas e mitos.

\paragraph{Conteúdo} Produção de lenda ou mito que retrata os valores pessoais e coletivos da turma, aliados às criações fantásticas.

\paragraph{Objetivo} O objetivo é que os estudantes possam se apropriar do conhecimento sobre lendas e mitos, produzindo eles mesmos seu próprio texto, considerando o contexto social e cultural em que estão inseridos.

\paragraph{Justificativa} A apropriação de valores culturais relacionados a uma sociedade se relaciona com a compreensão de valores locais, comparando diferentes contextos sociais e sua relação com a produção cultural, incluindo a literária. Compreende-se que ressaltar as características de grupos sociais distintos contribui para o entendimento da diversidade enquanto expressão dos povos que partilham suas crenças, modos de vida, influências estéticas e particularidades das diversas regiões.

Na cultura árabe, as relações familiares e de formação de pares estão relacionadas à herança de valores que são partilhados por grupos étnicos e religiosos,  como na história do \textit{Cavalo de ébano}, em que o casamento com a princesa está condicionado às qualidades que o rei, seu pai, julga pertinentes aos valores já estabelecidos em seu reino. Desta maneira, é necessário que o rei permita a união, a depender de sua avaliação sobre o pretendente.

No final da história, o reino do príncipe e o da princesa se tornam aliados, a julgar pela ligação entre eles, fortalecida pelo oferecimento de presentes valiosos e cartas gentis. Considera-se que esses pactos são resultado dos valores alimentados pela sociedade dos reinos que encontram na cerimônia a expressão de sua cultura e costumes.

A escrita de uma lenda ou de um conto que retrata a sociedade na qual os estudantes vivem é um ensaio que pode contribuir para a reflexão sobre os valores de distintas sociedades, ao mesmo tempo em que ressalta a sua própria.

\paragraph{Metodologia} Inicialmente, o docente deverá retomar os aspectos mais gerais das discussões anteriores e também sobre a história lida do \textit{Cavalo de ébano}, bem como os costumes mais próprios que são retratados no texto. Em seguida, irá introduzir um debate sobre a sociedade em que vivemos, considerando as características locais, ampliando para as nacionais, ressaltando semelhanças e dissonâncias que podem aparecer no grupo. 

Por fim, os estudantes deverão produzir uma lenda ou mito em que essas características sejam ressaltadas em seus personagens e sua narrativa, respeitando o estilo e a estética desse gênero literário.

Alguns exemplos de onde os alunos podem partir para suas redações são:

\begin{itemize}
\item Trabalhar com a redação de um fato autobiográfico do aluno, usando algo visto ou vivido pelo estudante que ele acredite caracterizar sua cultura;

\item Recriar alguma das aventuras do \textit{Cavalo de ébano} vistas em aula, pensando, por exemplo, em diferentes desfechos, ou em diferentes ambientes (o bairro ou a cidade do aluno, por exemplo);

\item Partir de algum acontecimento de interesse social, divulgado em \textsc{tv}, rádio, mídia impressa e digital, para criar uma narrativa fantástica que retrate determinado aspecto de sua sociedade.
\end{itemize}

Os estudantes podem contar com o auxílio docente para tirar dúvidas e compartilhar ideias que possam contribuir para sua criação. Após a redação, o professor pode fazer uma roda para que os alunos leiam suas histórias e compartilhem na turma.

\paragraph{Tempo estimado} Duas aulas de 50 minutos.


\end{document}

