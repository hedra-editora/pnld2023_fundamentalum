\documentclass[11pt]{extarticle}
\usepackage{manualdoprofessor}
\usepackage{fichatecnica}
\usepackage{lipsum,media9}
\usepackage[justification=raggedright]{caption}
\usepackage[one]{bncc}
\usepackage[araucaria]{../edlab}
\usepackage{marginnote}
\usepackage{pdfpages}

\newcommand{\AutorLivro}{Fê}
\newcommand{\TituloLivro}{O reizinho que só falava sim}
\newcommand{\Genero}{Conto; crônica; novela}
%\newcommand{\imagemCapa}{./images/PNLD2022-001-01.jpeg}
\newcommand{\issnppub}{XXX-XX-XXXXX-XX-X}
\newcommand{\issnepub}{XXX-XX-XXXXX-XX-X}
% \newcommand{\fichacatalografica}{PNLD0001-00.png}
\newcommand{\colaborador}{Renier Silva}

\begin{document}

\title{\TituloLivro}
\author{\AutorLivro}
\def\authornotes{\colaborador}

\date{}
\maketitle

\tableofcontents

\section{Carta ao professor}

Caros professores e professoras,

esperamos, com este material,
auxiliá-los no trabalho com o \textbf{Ensino Fundamental \textsc{i}} em 
sala de aula. \textit{O reizinho que só falava sim}, de Fê, é um livro singular
por vários motivos e possibilita atividades didáticas interessantíssimas,
como vocês acompanharão no decorrer do manual. Trata-se da estreia do autor, produzido depois de uma extensa carreira ilustrando livros de outros escritores.

Os adultos perceberão, ao ler este livro, que não é somente para as crianças que ele 
tem um grande valor. Baseado em experiências pessoais de Fê, o livro versa sobre a capacidade de tomar as rédeas da própria vida e saber
impor os próprios limites, sem no entanto deixar-se cair numa rigidez restritiva absoluta -- habilidade muito cara à vida adulta. Na infância e adolescência, no entanto,
reconhecer e impor limites é ainda mais importante para se evitar abusos de diversos
aspectos. Por isso, consideramos o trabalho com este livro de grande valia 
para o desenvolvimento destes indivíduos. 

Começamos o manual com uma atividade de imaginação conjunta que trabalha
habilidades de socialização articulando os limites entre o Eu e o Outro:
assim como na brincadeira de imaginar desenhos nas nuvens, esta relação
deve ser favorável para ambas as partes e ser uma via de mão dupla,
com todos participando ativamente de maneira intercalada.
Fala e escuta são competências fortemente trabalhadas aqui. 

Então, passamos para uma atividade na qual a gramática se mostra mais presente. 
Vamos entender como funcionam as concessões e negações da vida real 
no universo da análise sintática. Trata-se dos advérbios de negação, afirmação e dúvida. 
Neste ponto, propomos uma abordagem que favoreça o sentido da discussão trazida
pela obra literária para a apresentação dos conceitos gramaticais.

Por fim, propomos uma atividade de criação artística, tanto textual quanto visual, 
na qual alunos e alunas sejam convidados a contar, sob seus próprios
pontos de vista, enquanto narradores-personagens, um fato de suas vidas. 

Esperamos, professores, que este material sirva como guia 
para seu trabalho em sala de aula. Já contamos, no entanto, com as adaptações
que surgirão organicamente na recepção do mesmo por vocês, que possuem 
trajetórias e escolhas didáticas específicas, bem como no contato com os 
alunos, que tanto têm a oferecer para o enriquecimento da experiência didática.

Boa aula!


\section{Sobre o livro}

\textit{O reizinho que só falava sim} é o primeiro livro feito inteiramente pelo autor, Fê.
Foi criado num momento importante de sua vida marcado por perdas quando ele 
passou a questionar sua posição em relação ao mundo. O livro conta a história de uma criança chamada Guilherme
que vive no reino de seu pai, o rei George. Um dia Gui ocupará o lugar de seu pai e,
para isso, deve aprender como se portar como tal. 

A principal dificuldade que o reizinho Guilherme encontra é que ele não sabe dizer \textit{não},
nem \textit{talvez}. Sua única resposta aos pedidos dos outros -- sobretudo de seus colegas
que vêm brincar com seus numerosos brinquedos, que só o filho de um rei poderia ter -- 
é \textit{sim}. Isto lhe aborrece pois nem sempre ele está à vontade para dar o que é seu. 

A história começa a mudar quando o corpo do reizinho sofre grandes mudanças:
todos os \textit{nãos} e \textit{talvez} que ele não disse o fazem inchar
cada vez mais. Caso não ocorra nenhuma mudança em seu comportamento, 
ele vai inchar tanto a ponto de explodir, aí sim, jorrando essas palavras
para todos os lados. 

Com este perigo eminente e com a ajuda de um amigo, o pássaro azul,
o reizinho Gui aprende a responder de outras formas aos pedidos
do mundo e sua vida se torna bem mais interessante e mais leve.

A história do \textit{reizinho que só falava sim} é ilustrada do começo ao fim,
o que deve tornar a leitura ainda mais divertida para os estudantes. 


\section{Sobre o autor}

Nascido em Santos, no litoral de São Paulo, Fernando Luiz, ou \textbf{Fê}, seu
nome artístico, é um escritor e ilustrador que há algumas décadas trabalha nesta área.
Mora desde 2005 em Londres mas sempre vem ao Brasil para uma temporada de divulgação de 
seu trabalho. Para ele, a criatividade é maior expressão de comunicação com o mundo, e não
há lugar mais favorável para ela do que a literatura infantojuvenil. 

Seus materiais de trabalho mais básicos são um \textit{tablet} e uma caneta digital.
A partir de um software chamado \textit{Painter}, ele simula digitalmente várias
técnicas da pintura tradicional, como a aquarela, o pastel, o carvão, o óleo, o lápis de cera,
dentre outras, que garantem um estilo único e versátil às suas ilustrações, cada vez mais interessantes. 

\Image{Escritor e ilustrador Fê. (Arquivo pessoal).}{PNLD2023-038-04.jpg}

Fê formou-se em Arquitetura na Universidade de Arquitetura e Urbanismo e Santos, 
depois, em Comunicação Visual, e fez pós graduação em tecnologia gráfica na Faculdade de
Arquitetura e Urbanismo da Universidade de São Paulo (\textsc{fau--usp}).
Antes disso, ainda garoto, por indicação da professora de artes de sua escola, 
foi matriculado numa escola de desenho e pintura, a Società Italiana di Beneficenza,
onde teve seu primeiro contato mais especializado com as artes. 

\textit{O reizinho que só falava sim} é seu primeiro livro completamente autoral,
depois de anos ilustrando vários livros escritos por outros autores. 
Ele surge num momento delicado da vida do autor. Depois de uma série de perdas
de pessoas queridas, Fê passou a questionar sua forma de se relacionar com 
as pessoas e o mundo, em geral tendenciosa para o ``sim'' como resposta a tudo e
todos. Daí surgiu a inspiração para contar a história do livro. 

Depois de sua estreia, diversos outros livros autorais foram publicados,
como \textit{No mundo do faz de conta}, \textit{Brinconto}, \textit{Ki-som-será?}, na coleção Criantiva
da Editora Paulinas, \textit{ACabe...}, pela mesma editora, \textit{A pinta fujona},
\textit{A menina que engoliu o mundo}, pela Editora Iluminuras; pela Editora 
Palavras, lançou \textit{\textsc{ooobbaaaa}!}, e pela Editora \textsc{sei},
\textit{Meu canto e o seu encanto}. Sua publicação mais recente é \textit{Quantas rodas tem uma bicicleta?},
pela Editora Casa do Lobo, em coautoria com Eliandro Rocha. 

Fê também já ilustrou mais de quarenta títulos, dentre os quais estão \textit{Guerra dos bichos}, da Editora Brinque-Book, \textit{As meias dos Flamingos} e \textit{Os três tesouros}, da Editora Larousse,
\textit{Ana e Ana}, da Editora \textsc{dcl}, \textit{Ensinei meu gato a falar francês}, pelo qual recebeu o Prêmio Açorianos de ilustração em 2006, e \textit{O homem que escrevia ao contrário}, pela Editora Paulinas, \textit{Contos de Perrault}, \textit{Conversa de passarinhos}, \textit{DiaNoite, Haikais para crianças}, \textit{Estação dos bichos},
\textit{Meu cavalinho vermelho}, \textit{Se o menino tem asas}, \textit{O gato do mato e o cachorro do morro},
\textit{A borboleta chique}, enfim... A lista continuaria ainda por muito tempo. 

Atualmente, Fê trabalha como ilustrador das crônicas de José Simão no jornal \textit{Folha de S.\,Paulo}.


\section{Sobre o gênero}

\paragraph{O gênero} O gênero deste livro é a \textit{conto; crônica; novela}. 

\paragraph{Descrição} O que define um gênero narrativo é o fato de, não importa
qual seja sua forma, eles \textit{contarem uma história}.
As especificidades do \textit{como} esta história será contada é que
qualificaram os tipos de gênero narrativo, que podem ser: conto, crônica, novela,
epopeia, romance ou fábula. 

Toda narrativa possui, necessariamente, um narrador, uma personagem, um enredo,
um tempo e um espaço. O narrador, ou narradora, pode ser onisciente, literalmente
\textit{que tudo sabe}, observador ou personagem --- categorias que não são excludentes.
O discurso elaborado por este narrador ou narradora pode ser direto, indireto ou indireto livre 
--- ou seja, ele ou ela pode aparecer mais diretamente ou mais indiretamente; no último caso,
sua voz se mistura à das personagens da história.

O narrador \textbf{não é necessariamente} a voz do autor. É errada a afirmação
de que o autor fala através do narrador de uma história. É bastante comum,
há algum tempo na história literária, sobretudo desde os pré-modernistas, que 
o narrador represente justamente o contrário do que pensa o autor. Neste caso, 
utilizam-se elementos como a \textbf{ironia} para sugerir que o autor \textit{não é confiável}.

Já as persponagens variam quanto a sua \textbf{profundidade}. Há personagens planas, ou
personagens-tipo, e personagens redondas, ou complexas. Personagens planas
são facilmente repetíveis pois se amparam em lugares-comuns da cultura, como
o vilão, o herói, a vítima, o palhaço, tudo isso com marcações de gênero e espécie ---
o herói tradicionalmente é um homem, a vítima, uma mulher, e o vilão, uma figura que 
se afasta da humanidade por alguma razão, às vezes sobrenatural. 
Personagens redondos, por outro lado, estão mais próximos das \textit{pessoas reais}.
Uma personagem complexa pode ser, em um dado momento da narrativa, vilã, e em 
outro, heroína. É importante notar como as visões de mundo, um traço cultural e 
portanto relativo, influenciam na caracterização das personagens, planas 
ou redondas, de uma história.

O tempo de uma narrativa pode ser cronológico ou psicológico.
No tempo cronológico, o enredo segue a ordem ``normal'' dos acontecimentos,
aquela marcada pelo relógio e pelo calendário. Os acontecimentos vêm um após o 
outro, e \textit{passado}, \textit{presente} e \textit{futuro} são muito bem delimitados.
Já no tempo psicológico, segue-se uma ordem \textit{subjetiva} dos acontecimentos, 
e portanto, \textit{não linear}, já que a influência emocional e psíquica 
da subjetividade afeta a racionalidade do tempo cronológico. 

O espaço, por fim, é o lugar onde se passa a narrativa. Dependendo do caso, 
ele pode funcionar mais como um plano de fundo, sem muita interferência
no enredo, ou mais ativamente, aproximando-se das características das personagens
e influenciando no desenrolar da trama. 

O último aspecto de um gênero narrativo que podemos abordar é sua 
\textit{extensão}. Dentre os elementos que distinguem um subgênero 
de outro é o tamanho da história: uma crônica e um conto são \textit{necessariamente}
curtos, ao passo que uma epopeia e um romance são longos. Uma novela
está no ponto intermediário entre um romance e um conto.
Ainda poderíamos falar dos registros de cada subgênero: 
a epopeia é originalmente um subgênero \textit{oral}, versificado, e metrificado,
já o romance é tradicionalmente \textit{escrito} em prosa. 
Desde meados do século \textsc{xviii}, no entanto, o estabelecimento
dos gêneros e subgêneros narrativos tornam-se cada vez menos rígido,
com as características cada vez mais fluidas e intercomunicativas.

Como o presente livro é uma narrativa \textit{curta},
finalizamos com as palavras de Luiza Vilma Pires a respeito do
subgênero:

\begin{quote}
sob o nome de narrativa curta, estão situadas obras que apresentam uma trama 
um pouco mais complexa, que ocorre em diversos espaços e em uma temporalidade 
que pode ser de vários dias, semanas ou meses. Entretanto a função das ilustrações 
continua as mesmas, são complementares à história e contribuem para sua compreensão. 
Os temas relacionam-se a vivência infantis (brincadeiras, passeios, pequenas aventuras), 
a aspectos ligados à interioridade das personagens (busca de identidade, insegurança, 
medos) ou a relações interpessoais (desentendimentos familiares, entre amigos, solidariedade).\footnote{“Narrativas infantis”, de Luiza Vilma Pires Vale. In \textsc{saraiva}, J. A. (Org.) \textit{Literatura e alfabetização: do plano do choro ao plano da ação}. Porto Alegre: Artmed, 2001.} 
\end{quote}

\section{Atividades}

\subsection{Pré-leitura}

\BNCC{EF03AR00}
\BNCC{EF03AR04}
\BNCC{EF15AR24}

\subsubsection{Atividade 1}

\paragraph{Tema} Brincando de olhar para o céu!

\paragraph{Conteúdo} Brincar de imaginar coletivamente desenhos formados pelas nuvens no céu.

\paragraph{Justificativa} Olhar para o céu e imaginar desenhos nas nuvens é uma brincadeira muito 
simples que trabalha faculdades importantes no desenvolvimento do ser humano. Por exercitar a capacidade 
imaginativa dos indivíduos, esta brincadeira está ligada à habilidade de solucionar problemas
na vida quotidiana por meio da busca de soluções. Além disso, por se tratar de uma atividade ao ar livre,
em contato com a natureza, evidencia para as crianças a pluralidade de experiências que 
os elementos naturais podem oferecer quando se é dada a atenção adequada. 
Já no que diz respeito à \textbf{relação do Eu com o Outro}, a atividade de elaborar uma imaginação conjunta
propicia o exercício da coletividade no grupo.

\paragraph{Metodologia} Numa área externa da escola, peça que os alunos e alunas
se deitem no chão em grupos de mais ou menos três pessoas. 
É preciso que seja \textbf{um dia de sol com nuvens}.
Então, deixe os grupos livres para imaginarem o que quiserem.
Não deixe de lembrar-lhes que a imaginação pode ser feita de modo
individual mas também coletivamente. 

Ainda na área externa, dê folhas em branco e instrumentos de artes 
como lápis de cor, tinta e giz de cera colorido às crianças e 
peça que ilustrem alguns dos desenhos que viram. 
Ainda que a atividade de imaginar os desenhos nas nuvens tenha sido
feita de modo coletivo, é importante que, neste momento, o registro seja individual. 
Os desenhos serão usados mais à frente em outra atividade. 

%\Image{O príncipe Gui e seus amigos brincando de imaginar desenhos nas nuvens. (Retirado do livro.)}{PNLD2023-038-02.jpg}

\paragraph{Tempo estimado} Duas aulas de cinquenta minutos.


\subsection{Leitura}

\BNCC{EF35LP01}
\BNCC{EF35LP05}

\subsubsection{Atividade 1}

\paragraph{Tema} Lendo em voz alta com a turma!

\paragraph{Conteúdo} Leitura e escuta compartilhada e autônoma.

\paragraph{Justificativa} A leitura em voz alta, ainda que evitada nas séries mais elevadas
do Ensino Médio, tem sua importância nos anos iniciais e finais do \textbf{Ensino Fundamental}.
Ela funciona como um marcador da qualidade da leitura dos alunos e alunas, além
de ser um elemento socializador, ao fazer com que estudantes que geralmente não participam
das aulas falem.

\paragraph{Metodologia} O professor ou professora pode começar lendo o primeiro parágrafo do texto,
e a partir do segundo, passar para os alunos. Eventuais correções de pronúncia devem
ser feitas, conforme achar necessário. 

Durante a leitura, faça algumas perguntas de verificação de leitura, como:

\begin{itemize}
\item Por que o reizinho Gui está ficando inchado?
\item Qual o problema de sempre dizer \textit{sim}?
\item Qual foi a solução encontrada para não ficar só entre o \textit{sim} e o \textit{não}?
\end{itemize}

Finalize a leitura pedindo que eles \textbf{escrevam uma frase} que resuma o que 
eles aprenderam com a história. Então, que leiam em voz alta para o restante da turma. 

\paragraph{Tempo estimado} Duas aulas de cinquenta minutos.

\subsubsection{Atividade 2}

\BNCC{EF06LP06}

\paragraph{Tema} Entendendo a importância das escolhas com a gramática.

\paragraph{Conteúdo} Análise sintática: os advérbios \textit{sim}, \textit{não} e \textit{talvez}.

\paragraph{Justificativa} A abordagem dos aspectos linguísticos e semióticos 
pela perspectiva enunciativo-discursiva é feita pela leitura dos efeitos de 
sentido produzidos pelas práticas de linguagem nos diferentes campos de atuação 
por meio dos diversos gêneros textuais, neste caso, uma narrativa curta.
A este respeito, a \textsc{bncc} diz que:

``Os conhecimentos sobre a língua, as demais semioses e a norma-padrão não devem ser tomados como uma lista de conteúdos dissociados das práticas de linguagem, mas como propiciadores de reflexão a respeito do funcionamento da língua no contexto dessas práticas. A seleção de habilidades na \textsc{bncc} está relacionada com aqueles conhecimentos fundamentais para que o estudante possa apropriar-se do sistema linguístico que organiza o português brasileiro.''\footnote{\textsc{bncc} -- Língua portuguesa no Ensino Fundamental. Cap.\,4.1.1.2, p.\, 137 -- dezembro de 2017.} 


\paragraph{Metodologia} Para introduzir a noção de \textbf{advérbio}, escreva na lousa a
seguinte definição: ``Advérbios são palavras invariáveis que determinam o sentido de um verbo, adjetivo ou outro 
advérbio.''

Siga com a seguinte reflexão: dado que \textbf{verbos} expressam uma ação, estado ou fenômeno da natureza, 
podemos entender, então, que os advérbios \textbf{mudam a forma como as coisas acontecem}.
\textit{Sim}, \textit{não} e \textit{talvez} são advérbios, cada um de uma subclasse:

\begin{itemize}
	\item Advérbios de afirmação;
	\item Advérbios de negação;
	\item Advérbios de dúvida.
\end{itemize}

Nem sempre na vida temos certeza, nem positiva nem negativamente sobre alguma situação,
e para isso existem os advérbios de dúvida. No entanto, estes são apenas os mais conhecidos
de cada subclasse. Dentre os advérbios de \textbf{afirmação}, temos, além de \textit{sim}:

\begin{itemize}
	\item Realmente;
	\item Certamente;
	\item Perfeitamente;
	\item Deveras.
\end{itemize}

Dentre os de \textbf{negação}, além de \textit{não}:

\begin{itemize}
	\item Nem;
	\item Nunca (que combina \textbf{negação} e \textbf{tempo});
	\item Jamais (que também combina essas duas circunstâncias);
	\item Tampouco.
\end{itemize}

E, por fim, dentre os de \textbf{dúvida}, além de \textit{talvez}:

\begin{itemize}
	\item Provavelmente;
	\item Supostamente.
\end{itemize}

%\Image{O reizinho Gui descobre as possibilidades além do \textit{sim}. (Retirado do livro).}{PNLD2023-0038-01.jpg}

Agora que a turma conhece algumas subclasses de advérbios, o professor ou professora
pode propor alguns exercícios de sua escolha para praticar o conhecimento recém adquirido,
seja com a leitura de trechos do próprio livro, seja com frases aleatórias para a prática
dos advérbios!


\paragraph{Tempo estimado} Duas aulas de cinquenta minutos.


\subsection{Pós-leitura}

\subsubsection{Atividade 1}

\paragraph{Tema} Assumindo o protagonismo! 

\paragraph{Conteúdo} Criação de uma curta narrativa autobiográfica ilustrada.

\paragraph{Justificativa} O livro \textit{O reizinho que só dizia sim}
trata de um tema muito importante para os jovens estudantes do Ensino Fundamental,
e para todas as pessoas num geral: a capacidade de decidir sobre a própria
vida e ter \textbf{autonomia sobre si mesmo e sobre seu corpo}. 
Por meio da história do reizinho Gui, 
eles devem ter percebido como o ato de se posicionar perante os acontecimentos
da vida com maleabilidade e respeito aos próprios limites é de grande importância.
Ao fazer isso, o indivíduo passa a tomar um lugar de protagonismo sobre sua própria história.
Por isso, alunos e alunas devem ser incentivados a \textit{contar uma história}
verídica de suas vidas, de modo que, com o auxílio das ferramentas artísticas
da literatura e da ilustração, se sintam à vontade para exercitar a prática saudável de 
comunicar-se com o seu entorno. 

\Image{É essencial para um bom desenvolvimento do indivíduo que as crianças aprendam a falar sobre si mesmas. (Licença Creative Commons).}{PNLD2023-038-03.jpg}

\paragraph{Metodologia} Comece a aula pedindo que a turma retome os desenhos que fizeram
na \textbf{Atividade de pré-leitura}, quando ilustraram os
desenhos que viram nas nuvens. 
Agora, devem \textbf{contar uma história} em primeira pessoa de algo que aconteceu com eles.  
Eles e elas devem necessariamente ocupar a posição de narrador-personagem.
Para isso, devem construir os elementos que compõem uma narrativa: espaço, tempo, personagens, enredo, e,
neste caso, as ilustrações. 

O trabalho deve ser feito individualmente. As crianças podem usar os desenhos do livro
para se inspirar e mesmo os dos colegas. Devem, no entanto, manter-se fiéis 
ao que querem contar, a história que \textbf{só elas sabem}. 

Para as ilustrações, o professor ou professor pode sugerir que trabalhem com \textbf{aquarela}.
Trata-se de uma técnica simples de pintura, além de se aproximar esteticamente
da primeira atividade de sensibilização que alunos e alunas tiveram na brincadeira
de imaginar desenhos nas nuvens. 

O resultado do trabalho pode ser exposto para a turma e em um \textit{blog} criado
pela turma, para que as famílias, amigos e a comunidade escolar e do entorno possam 
ler as histórias de todos e todas. 

\paragraph{Tempo estimado} Quatro aulas de cinquenta minutos.


\subsubsection{Atividade 2}

\BNCC{EF15AR24}
\BNCC{EF03LP13}

\paragraph{Tema} Oficina de contação de histórias. 

\paragraph{Conteúdo} Organização de uma roda de contação de histórias
com protagonismo dos alunos e alunas.

\paragraph{Justificativa} A contação de histórias trabalha competências
que dizem respeito ao campo linguístico e sociocomunicativo. 
Ao contar uma história, o indivíduo se apropria do lugar do narrador e
atuar como cocriador da mesma. Além do conteúdo propriamente dito 
da história, toda a estrutura linguística e gramatical, a sintaxe 
e o vocabulário presentes no texto serão trabalhados. 
O grande diferencial desta atividade é que seu objetivo não está 
exclusivamente no exercícios destas capacidades linguísticas, 
mas sim ligados a elas e à capacidade de \textbf{apresentar oralmente}
um texto a um público. Fala e corpo estão totalmente ligados nesta atividade. 

\paragraph{Metodologia} Nesta aula, a turma deverá apresentar as histórias
que compuseram na última \textbf{Atividade}. Além da partilha escrita e visual,
é importante que os e as jovens possam exercitar o ato de \textbf{contar oralmente}
suas próprias histórias. 

Diferente da última atividade, na qual trabalharam sozinhos, agora eles podem 
solicitar a participação de colegas para realizar uma encenação de suas histórias. 
É interessante que os donos e donas das histórias sejam narradores-personagens 
nas cenas, ou seja, que eles tenham a voz principal e dialoguem com o público --- o restante da turma. 

Apresente à turma, para lhes inspirar na criação das cenas, algumas interpretações
de grupos de teatro infantojuvenil que deixamos na seção de \textbf{Sugestões de referências complementares}.


\paragraph{Tempo estimado} Duas aulas de cinquenta minutos.


\section{Sugestões de referências complementares}

\paragraph{Músicas, vídeos e filmes}

\begin{itemize}
\item \textit{O Rei Leão}. Roger Allers, Rob Minkoff. \textsc{eua}, 1994.

Simba, filho do rei Musafa, é seu sucessor na linhagem real. Para assumir tal posto,
precisa estar preparado. O filme mostra seu processo de amadurecimento da personagem
até chegar ao trono.

\end{itemize}


\paragraph{Livros e artigos}

\begin{itemize}
	
	\item \textsc{lima}, Romeu R.\,de. \textit{OcÊ QuÉ SabÊ?}. Clube de Autores, 2010. 

\item Companhia Arte e Manhas. \textit{Os três porquinhos}, \textit{O sítio do picapau amarelo em: o circo de cavalinhos},
\textit{Páscoa em apuros}. Todos disponíveis em: \url{https://www.youtube.com/channel/UCn0kXgFQr4r91FycGnAidqA}. Último acesso em 11 de janeiro de 2022.

\item \textsc{alberti}, Verena. ``Literatura e autobiografia: a questão do sujeito na narrativa''. \textit{Estudos Históricos}, Rio de Janeiro, v.\,4, n.\,7, 1991.

\item \textsc{freire}, Paulo. \textit{A importância do ato de ler em três artigos que se completam}. São Paulo: Autores Associados/Cortez, 1989.

\item \textsc{braga}, Marília da Costa; \textsc{bezerra}, Adriano Alves. ``A literatura fantástica como incentivo à leitura''. \textit{Anais \textsc{v enlije}}. Agosto de 2014.

\item \textsc{gama-khalil}, Marisa Martins. ``A literatura fantástica: gênero ou modo?''. \textit{Terra roxa e outras terras. Revista de Estudos Literários. v.\,26}. Dezembro de 2013. 

\item \textsc{laplatine}, F.; \textsc{trindade}, L.\,S. \textit{O que é imaginário}. São Paulo: Brasiliense, 2003. 

\item \textsc{leão}, J.\,O. ``A literatura fantástica e a formação de leitores no século \textsc{xxi}.'' \textit{Revista Húmus}. Setembro a dezembro de 2011. 

\item \textsc{todorov}, Tzvetan. \textit{Introdução à literatura fantástica}. São Paulo: Perspectiva, 2004.

\end{itemize}

\section{Bibliografia comentada}

\subsection{Livros}

\begin{itemize}

	\item \textsc{albrecht}, Tatiana D'Ornellas. \textit{Atividades lúdicas no Ensino Fundamental}. Universidade Católica Dom Bosco, \textsc{ms}, 2009. Disponível em: \url{https://site.ucdb.br/public/md-dissertacoes/8072-atividades-ludicas-no-ensino-fundamental-uma-intervencao-pedagogica.pdf}. Último acesso em 24 de dezembro de 2021.

	As atividades lúdicas, quando bem aplicadas e no momento oportuno, trazem
grandes benefícios. Contudo, a grande maioria das escolas não utiliza esse instrumento. Por
que será que existe essa resistência por parte das escolas e dos professores? Por que não
adequar o lúdico ao cotidiano escolar de maneira prática, educativa e ao mesmo tempo
divertida?

\item \textsc{brasil}. Ministério da Educação. \textit{Base Nacional Comum Curricular}. Brasília, 2018.

Consultar a \textsc{bncc} é essencial para criar atividades para a turma. Além de especificar 
quais habilidades precisam ser desenvolvidas em cada ano, é fonte de informações sobre 
o processo de aprendizagem infantil. 

 \item \textsc{bandoch}, Adriana Rodrigues Vieira. \textit{A inserção do teatro nas séries iniciais do Ensino Fundamental}.
 Universidade Tecnológica Federal do Paraná, 2012. Disponível em: \url{http://repositorio.utfpr.edu.br/jspui/bitstream/1/20738/2/MD_EDUMTE_II_2012_03.pdf}. Último acesso em 24 de dezembro de 2021.

	O teatro no Ensino Fundamental é uma das formas de se trabalhar o conhecimento,
pois nele há a possibilidade do ser humano em se integrar, vivenciar, expressar e
criar situações, condições para novas aprendizagens éticas, sociais, culturais,
históricas.


\item \textsc{van der linden}, Sophie. \textit{Para ler o livro ilustrado}. São Paulo: Cosac Naify, 2011.

Livro sobre as particularidades do livro ilustrado, que apresenta as diferenças entre o livro ilustrado e o livro com ilustração. 
\end{itemize}


\end{document}