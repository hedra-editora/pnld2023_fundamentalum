\documentclass[11pt]{extarticle}
\usepackage{manualdoprofessor}
\usepackage{fichatecnica}
\usepackage{lipsum,media9}
\usepackage[justification=raggedright]{caption}
\usepackage[one]{bncc}
\usepackage[araucaria]{../edlab}
\usepackage{marginnote}
\usepackage{pdfpages}

\newcommand{\AutorLivro}{Lalau}
\newcommand{\TituloLivro}{Sobrevoos}
%\newcommand{\Tema}{}
\newcommand{\Genero}{Poesia; poema; trava-línguas; parlendas; adivinhas; provérbios; quadrinhas e congêneres}
%\newcommand{\imagemCapa}{./images/PNLD2022-001-01.jpeg}
\newcommand{\issnppub}{XXX-XX-XXXXX-XX-X}
\newcommand{\issnepub}{XXX-XX-XXXXX-XX-X}
% \newcommand{\fichacatalografica}{PNLD0001-00.png}
\newcommand{\colaborador}{Ana Lancman}
\begin{document}

\title{\TituloLivro}
\author{\AutorLivro}
\def\authornotes{\colaborador}

\date{}
\maketitle

%\begin{abstract}\addcontentsline{toc}{section}{Carta ao professor}
%\pagebreak

\tableofcontents

\begin{abstract}

O presente manual tem como objetivo oferecer uma orientação ao professor sobre a obra \textit{Sobrevoos}. A partir deste manual, os professores poderão incentivar a prática da leitura aos estudantes e proporcionar um conteúdo enriquecedor. Apresentamos aqui sugestões de atividades a serem realizadas antes, durante e após a leitura do livro, com propostas que buscam introduzir os gêneros literários e aprofundar as discussões trazidas pelas obras. Você encontrará informações sobre o autor, sobre o gênero e sobre os temas trabalhados ao longo do livro. Ao fim do manual, você encontrará também sugestões de livros, artigos e sites selecionados para enriquecer a sua experiência de leitura e, consequentemente, a de seus estudantes.

O livro \textit{Sobrevoos} apresenta onze aves brasileiras que contam suas histórias através da poesia. São espécies de diversas regiões brasileiras que narram em primeira pessoa sobre práticas de seu dia a dia e características dos habitats em que vivem. O texto foi escrito pela dupla Lalau e Laurabeatriz, autores que trabalham juntos desde 1994 criando versos, histórias, imagens
e sonhos para as crianças. Neste livro, eles criaram uma composição de poemas e ilustrações que homenageiam o dom de voar. 

A partir desta obra, será possível realizar um estudo interdisciplinar em sala de aula. Por ser um livro que parte da linguagem poética para apresentar características científicas das aves, as atividades contemplam \textsc{bncc}'s de diversas áreas, como Ciências, Geografia, Artes e Língua Portuguesa. Esperamos que as atividades sugeridas e o material indicado sejam proveitosos em sala de aula! 

\end{abstract}

\section{Sobre o livro}

\paragraph{O livro} \textit{Sobrevoos} é uma espécie de catálogo de pássaros, escrito em um formato poético e acompanhado de ilustrações. As aves se apresentam e contam sobre a paisagem ao seu redor, falam sobre o meio ambiente e a biodiversidade brasileira, além de sua relação com outros animais e seres da natureza. Ao final da obra há uma descrição científica sobre cada uma dessas espécies. 

\section{Sobre os autores}

\SideImage{O autor Lalau (Arquivo pessoal)}{PNLD2023-006-02.png}

\paragraph{A autora} Laurabeatriz, nome artístico de Laura Beatriz de Oliveira Leite de Almeida, nasceu no Rio de Janeiro em 1949, mas adotou São Paulo com sua cidade. Artista plástica apaixonada pela natureza, trabalha com desenhos, xilogravuras e pinturas, com as quais já participou de diversas exposições. Foi redatora publicitária e crítica de cinema do jornal \textit{Folha da Tarde}, além de ilustradora de outros jornais e revistas.

\paragraph{O autor} Lalau, nome artístico de Lázaro Simões Neto, nasceu em São Paulo, no bairro do Cambuci, em 1954. Formado em Comunicação Social, trabalhou com criação publicitária e projetos literários. Desde 1994 escreve poemas para crianças, incentivado pelo grande poeta José Paulo Paes.

\Image{A ilustradora Laurabeatriz (Arquivo pessoal)}{PNLD2023-006-03.png}

\paragraph{Publicações} Além de \textit{Os números}, Lalau e Laurabeatriz já publicaram mais de 60 livros para crianças juntos. Desde 1994, quando publicaram seu primeiro livro juntos, \textit{Bem-te-vi e outras poesias}, não pararam mais de produzir, com Lalau escrevendo versos e Laurabeatriz ilustrando as histórias. Uma característica de todas as suas obras é a exploração da fauna e da flora brasileiras. No início dos anos 2000, publicaram a série ``Brasileirinhos'', com cinco volumes explorando a vida de ``brasileirinhos'' de diferentes regiões do país. 

\paragraph{Currículo} Lalau e Laurabeatriz já ganharam diversos prêmios por seus livros infantis. Em 1995, seu livro \textit{Fora da gaiola e outras poesias} recebeu o Prêmio da Fundação Nacional do Livro Infantil e Juvenil (\textsc{fnlij}) de Melhor Livro de Poesia. Em 1997, \textit{Uma cor, duas cores, todas elas} recebeu o Título Altamente Recomendável pela \textsc{fnlij}. A mesma instituição concedeu-lhes, em 1998, o prêmio de Melhor Livro Informativo por \textit{Histórias da preta}. Inaugurando a coleção Bicho-Poema, o livro \textit{Boniteza Silvestre} foi considerado um dos 30 melhores títulos infantis publicados em 2007 pela revista \textit{Crescer}. 

\section{Sobre o gênero}

\paragraph{O gênero} O gênero deste livro é \textit{poesia}. 

\SideImage{O gênero poético incentiva a curiosidade e a imaginação. (LACMA/Remedios Varo; CC BY-NC 2.0)}{PNLD2023-006-07.png}

\paragraph{Descrição} Um dos meios mais expressivos de comunicação e inovação da linguagem, a poesia é uma das mais antigas formas de arte literária, anterior até mesmo à escrita, pois existe desde a tradição oral. Ela combina palavras, significados, sonoridades, ritmos e, muitas vezes, também imagens para permitir uma experiência estética. A linguagem poética é condensada e emotiva e busca trabalhar a língua de forma que o leitor experimente as palavras de uma forma nova. Na maior parte das vezes, a poesia é dividida em versos que, juntos, são chamados de estrofes. O ponto de vista do autor e sua visão pessoal do mundo estão muito presentes nesse tipo de texto e, justamente por essa particularidade, a experiência da leitura de uma poesia é extremamente individual e subjetiva.

\paragraph{Interação} Esse gênero é um grande aliado na formação do leitor. O olhar da criança para o mundo é, em essência, um olhar poético, calcado na curiosidade pelo mundo. A poesia é a forma perfeita de valorizar esse olhar e incentivar que a criança brinque com as palavras, observe os sons e experimente novos ritmos. Por sua liberdade e criatividade, a poesia tem potencial para estabelecer um diálogo único com os pequenos leitores. A presença de fantasias, imagens, repetição e símbolos permite uma maior identificação, pois a criança ainda está construindo seu mundo interior e experimenta a vida de forma diferente do adulto. 

\paragraph{Competências} 
O caráter polissêmico do texto poético pode e deve ser explorado no ambiente escolar, assim como a dimensão lúdica da linguagem e as suas possibilidades. A própria estrutura do poema já produz aprendizado: ela seduz e desafia o leitor, apresenta ritmos, efeitos sonoros e, ao mesmo tempo, apresenta novas vivências, oferecendo possibilidades para a criança simbolizar suas próprias experiências. Cada dupla de páginas do livro \textit{Sobrevoos} apresenta composições de versos e ilustrações. Assim, a leitura da poesia se faz em paralelo com a observação de uma ilustração que sugere caminhos de sentido e interpretação à criança. A leitura do poema, realizada pelo educador, aumenta o repertório do aluno, incentiva o desenvolvimento do vocabulário e da fluidez do discurso. A associação entre a aquisição da linguagem e a poesia, ademais, permite explorar múltiplas competências ao mesmo tempo, pois relaciona os princípios linguísticos à linguagem poética, introduzindo o aluno no universo lúdico e artístico da poesia.

\section{Atividades}

\subsection{Pré-leitura}

\BNCC{EF15LP11} 
%Identificar: +Falar: conversação; "roda de conversa", "discussão";
\BNCC{EF02CI04}
%Produzir +Descrever: “características de plantas e animais cotidianos”
\BNCC{EF03CI06}
%Identificar +Comparar: “animais, com base em características externas”
\BNCC{EF03CI04}
%Identificar: “características sobre o modo de vida dos animais”

\paragraph{Tema} As aves e seus habitats.

\paragraph{Conteúdo} Estudo interdisciplinar com professores de Ciências acerca das aves apresentadas nos livros e os habitats em que vivem.

\paragraph{Objetivo} Introduzir aos alunos um conhecimento científico acerca das espécies que serão abordadas na obra.

\paragraph{Justificativa} \textit{Sobrevoos} é uma obra que consegue tratar de características científicas das aves através de uma linguagem poética. Algumas das aves apresentadas no livro podem ser conhecidas do cotidiano dos estudantes, a depender da região em que vivem. Estudar essas espécies em sala de aula tornará a leitura da obra mais envolvente.

\paragraph{Metodologia} Primeiramente, trabalhe com os estudantes o conceito de \textbf{ornitologia}. Informações sobre este termo podem ser encontradas no \href{https://www.infoescola.com/biologia/ornitologia/}{site da Infoescola}. Para introduzir o estudo dos pássaros, sugere-se algumas questões a serem feitas em sala de aula: 

\begin{itemize}

\item Vocês sabem o que faz um ornitólogo?

\item Quais espécies de pássaros vocês conhecem?

\item Por que é importante estudar os pássaros?

\item Vocês sabiam que o Brasil é o terceiro país com a maior diversidade de aves por área?

\item Devemos prender os pássaros em gaiolas?

\end{itemize}

Em seguida, divida a turma em grupos pequenos. Cada grupo será responsável por pesquisar sobre uma das aves presentes no livro: acauã, quero-quero, beija-flor, tuiuiú, harpia, tico-tico, gaivota, bem-te-vi, viuvinha, coruja e martim-pescador. A pesquisa poderá ser feita na internet ou através livros de Biologia. Na bibliografia comentada há sugestões de guias especializados no assunto. Caso tenham fácil acesso a internet, uma ótima plataforma a ser usada é o \href{https://www.wikiaves.com.br/}{WikiAves}. É um portal brasileiro com grande quantidade de informação sobre aves, em que é possível até escutar a gravação de seus sons.

\Image{Tuiuiú à beira de um lago no Parque Nacional do Pantanal Matogrossense. (Maíra Mendonça da Rocha; CC-BY-SA-3.0)}{PNLD2023-006-08.png}

Os grupos poderão montar um painel informativo sobre a ave pesquisada, com fotos, mapas que apresentem sua localização no Brasil e no mundo, junto das principais características de sua espécie. Alguns tópicos importantes que devem constar no painel são:

\begin{itemize}

\item Nome científico da espécie, ordem e família.

\item Habitat natural.

\item Características físicas.

\item Regiões onde pode ser encontrada.

\item Alimentação.

\item Interação com outras espécies.

\end{itemize}

\Image{Martim-pescador-grande avistado no vale do Ribeira, próximo à cidade de Cananeia. (Dario Sanches; CC-BY-SA-2.0)}{PNLD2023-006-09.png}

Por fim, cada grupo apresentará o seu painel para o resto da sala, compartilhando o conteúdo pesquisado.

\paragraph{Tempo estimado} Duas aulas de 50 minutos.

\subsection{Leitura}

\BNCC{EF12LP07}
% Identificar: rimas, aliterações, assonâncias;
\BNCC{EF12LP19}
% Identificar: Poemas, rimas; Associações;
\BNCC{EF15LP04}
%Identificar: "interpretação de imagem", recursos gráfico-visuais, multissemióticos;

\paragraph{Tema} O gênero \textbf{poesia} em \textit{Sobrevoos}.

\paragraph{Conteúdo} Análise das rimas dos poemas e sua forma literária.

% \Image{Ilustração do livro, página 11}{PNLD2023-006-04.png}

\paragraph{Objetivo} Aprofundar o conhecimento dos estudantes acerca da poesia e praticar a leitura em voz alta.

\paragraph{Justificativa} Ao propor que os estudantes leiam os poemas em voz alta e identifiquem seus artifícios estilísticos, com auxílio do professor, será possível estudar o gênero de forma mais eficaz. Além disso, ao estudar os versos detalhadamente, a leitura da obra poderá tornar-se mais atenta e convidativa.

% \Image{Ilustração do livro, página 21}{PNLD2023-006-05.png}

\paragraph{Metodologia} Proponha uma primeira leitura conjunta em sala de aula. Cada aluno poderá ler uma estrofe, com ajuda e orientação do professor. Essa primeira leitura poderá ser mais livre, com pausas para contemplar as ilustrações e permitir que os estudantes se envolvam pelos poemas. 

% \Image{Ilustração do livro, página 25}{PNLD2023-006-06.png}

Em seguida, recite novamente os poemas para a turma e peça que os alunos manuseiem o livro, a partir das seguintes propostas:

\begin{itemize}

\item Os estudantes deverão assinalar as rimas encontradas.

\item Deverão ser trabalhadas as diferentes sonoridades dessas rimas.

\item Classifique junto da turma os tipos de rima encontrados.

\item Discorra acerca da estruturação dos versos a partir da observação dos alunos.

\item Peça que os estudantes anotem as palavras que não conhecem na obra e busquem seus significados em um dicionário. 

\item Os alunos podem criar um vocabulário do livro, uma ``coleção de palavras novas''. 

\item A partir desse momento de aproximação do texto, o professor poderá apresentar um conteúdo específico do gênero poético. 

\end{itemize}

\paragraph{Tempo estimado} Duas aulas de 50 minutos.


\subsection{Pós-leitura}

\BNCC{EF15LP05}
% Produzir: planejar "partes do texto", para quem escreve, tema, "adequação";
\BNCC{EF15AR04} 
%Produzir: desenho, pintura, colagem, quadrinhos, dobradura, escultura, modelagem...
\BNCC{EF02LP07}
% Produzir +Escrever: palavras, frases, textos curtos; “Redação”;
\BNCC{EF02CI04}
% Produzir +Descrever: “características de plantas e animais cotidianos”
\BNCC{EF03CI05}
% Produzir +Descrever: “alterações que ocorrem desde o nascimento em animais”

\paragraph{Tema} Poema animal.

\paragraph{Conteúdo} Redação de um poema ilustrado sobre um animal a escolha do aluno.

\paragraph{Objetivo} Incentivar os estudantes a exercitar a criatividade através da linguagem poética e das artes visuais.

\paragraph{Justificativa} Nesta atividade, será possível aprofundar o conteúdo das atividades de pré-leitura e leitura, ao relacionar a linguagem poética aos conhecimentos científicos em relação ao animal de sua preferência. 

\paragraph{Metodologia} Proponha que cada aluno escolha um animal. Poderá ser um animal doméstico ou silvestre. Deverão realizar uma pesquisa acerca do animal escolhido, na qual o foco deverá ser o espaço geográfico em que está inserido e suas características principais. O professor de Ciências poderá auxiliar nesse processo. 

\Image{As crianças deverão fazer um poema ilustrado a partir de um animal de sua escolha. (Andrew Krizhanovsky; CC-0)}{PNLD2023-006-10.png}

A partir dos dados coletados, os estudantes deverão produzir um poema ilustrado sobre esse animal. Poderão voltar ao livro \textit{Sobrevoos} para buscar inspiração. A ideia é que seja praticada a escrita poética a partir de informações científicas, como é feito na obra. Algumas questões poderão ser sugeridas aos alunos:

\begin{itemize}

\item Se você fosse esse animal, o que gostaria de contar?

\item O que ele mais gosta do lugar em que vive?

\item Ele se dá bem com outros animais?

\item Como é sua relação com as árvores? E com a terra? E com a água?

\item O que o animal mais gosta de fazer em seu dia a dia?

\item Qual seu alimento preferido?

\item Ele gosta da cidade ou prefere o campo?

\end{itemize}

Cada uma dessas perguntas poderá inspirar versos dos poemas. Caso seja de interesse dos alunos, poderão treinar a criação de rimas. O professor poderá orientar nessa parte, a partir dos estudos realizados na atividade de leitura. O professor de Artes poderá acompanhar essa atividade também, para sugerir ideias e técnicas nas ilustrações dos poemas. Os retratos dos animais poderão ser feitos com os materiais disponíveis na escola. Podem ser desenhos, pinturas, colagens e o que mais a imaginação permitir.

\paragraph{Tempo estimado} Duas aulas de 50 minutos.

\section{Sugestão de referências complementares}

\subsection{Audiovisual}

\begin{itemize}

\item \textit{Acauã}. Canção de Zé Dantas, interpretada por Luiz Gonzaga.

Com ritmo de baião, a canção homenageia o pássaro acauã e sua importância para o bioma do sertão nordestino. Pode ser escutada gratuitamente no \href{https://youtu.be/NGVW49KZVgg}{Youtube}.

\item \textit{O menino e o mundo}. Animação dirigida por Alê Abreu, 2013.

O filme é uma animação emocionante, que conta a história de Cuca, um menino que vive em um mundo distante, numa pequena aldeia no interior de seu mítico país. Sofrendo com a falta do pai, que parte em busca de trabalho na desconhecida capital, Cuca deixa sua aldeia e sai mundo afora a procura dele. Durante sua jornada, Cuca descobre uma sociedade marcada pela pobreza, exploração de trabalhadores e falta de perspectivas.

\item \textit{Profissão ornitólogo}.

Vídeo detalhado sobre a profissão de quem estuda aves. Também pode ser um material complementar para a atividade de pré-leitura. Pode ser acessado no \href{https://youtu.be/2swKMwcnYTM}{Youtube}.

\item \textit{Tito e os pássaros}. Animação dirigida por Gustavo Steinberg, Gabriel Bitar e André Catoto Dias, 2019.

O filme conta a história de Tito, um menino que adora ajudar seu pai cientista Rufus com suas invenções. Seu pai inventa uma máquina que pode entender o canto dos pássaros.

\end{itemize}

\subsection{Museus}

\begin{itemize}

\item \href{https://museucatavento.org.br/}{Catavento – Museu de Ciências}

Localizado no centro da cidade de São Paulo, o museu é dividido em quatro seções: universo, vida, engenho e sociedade. É um museu educativo e interativo.

Endereço: Av Mercúrio, s/n - Parque Dom Pedro II, Centro, São Paulo - SP

\item \href{https://www.museudalinguaportuguesa.org.br/}{Museu da Língua Portuguesa}

Localizado no centro histórico da cidade de São Paulo, o museu foi reformado e reaberto em 2021 e busca valorizar a diversidade da língua portuguesa. O museu é interativo, com uso de tecnologia nas exposições temporárias e permanentes.

\item Museu de Zoologia do Memorial do Cerrado

O maior museu de ornitologia do mundo ficava no Brasil, em Goiânia. Foi fundado em 1968 e possuía diversas peças pré-históricas. Foi fechado em 2018 e parte do seu acervo foi transferido para o Museu de Zoologia da PUC Goiás.

Endereço: Av. Engler, s/n - Jardim Mariliza, Goiânia - GO, Brasil.

\item \href{https://mz.usp.br/}{Museu de Zoologia da Universidade de São Paulo}

Localizado no bairro do Ipiranga, em São Paulo, o museu conta com exposições temporárias e permanentes, a partir de pesquisas nos eixos de biodiversidade e sustentabilidade.

\end{itemize}

\section{Bibliografia comentada}

\begin{itemize}

\item \textsc{brasil}. Ministério da Educação. \textit{Base Nacional Comum Curricular}. Brasília, 2018.

Consultar a \textsc{BNCC} é essencial para criar atividades para a turma. Além de especificar 
quais habilidades precisam ser desenvolvidas em cada ano, é fonte de informações sobre 
o processo de aprendizagem infantil. 

\item \textsc{frisch}, Christian Dalgas e Johan Dalgas Frisch. Aves Brasileiras e Plantas que as Atraem. São Paulo: Dalgas, 2005.

Uma das mais completas obras sobre ornitologia do Brasil. 

\item \textsc{maciel}, Ricardo. \href{http://www.funed.mg.gov.br/wp-content/uploads/2018/10/GUIA-DE-AVES-FUNED-Vers\%C3\%A3o-Net_final.pdf}{Guia de aves}. Fundação Ezequiel Dias.

Guia fundamental de catalogação de aves brasileiras.

\end{itemize}

\end{document}
