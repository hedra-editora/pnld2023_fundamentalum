% \Image{Capa do livro (; )}{PNLD2022-001-01.png}
% \Image{Ilustração do livro (Acorde/Manuella Silveira; Acorde)}{PNLD2022-001-04.png}
% \Image{Ilustração do livro (Acorde/Manuella Silveira; Acorde)}{PNLD2022-001-05.png}
% \Image{Ilustração do livro (Acorde/Manuella Silveira; Acorde)}{PNLD2022-001-06.png}

\documentclass[11pt]{extarticle}
\usepackage{manualdoprofessor}
\usepackage{fichatecnica}
\usepackage{lipsum,media9}
\usepackage[justification=raggedright]{caption}
\usepackage[one]{bncc}
\usepackage[ayllon]{../edlab}
\usepackage{marginnote}
\usepackage{pdfpages}

\newcommand{\AutorLivro}{Klévisson Viana}
\newcommand{\TituloLivro}{Artimanhas de Pedro Malazartes}
\newcommand{\Tema}{Diversão e aventura}
\newcommand{\Genero}{Cordel em quadrinho}
%\newcommand{\imagemCapa}{./images/PNLD2022-001-01.jpeg}
\newcommand{\issnppub}{XXX-XX-XXXXX-XX-X}
\newcommand{\issnepub}{XXX-XX-XXXXX-XX-X}
% \newcommand{\fichacatalografica}{PNLD0001-00.png}
\newcommand{\colaborador}{Paulo Pompermaier}

\begin{document}

\title{\TituloLivro}
\author{\AutorLivro}
\def\authornotes{\colaborador}

\date{}
\maketitle

\tableofcontents

\begin{abstract}
Este material tem a intenção de contribuir para que você consiga desenvolver um trabalho aprofundado com a obra \textit{Artimanhas de Pedro Malazartes} em sala de aula.
Você encontrará informações sobre o autor, sobre o gênero e também 
algumas propostas de trabalho para a sala de aula que você poderá explorar livremente, 
da forma que considerar mais apropriada para os seus estudantes.

O autor desse cordel em quadrinho, Klévisson Viana, é um dos principais cordelistas contemporâneos.
Nascido em Quixeramobim, interior do Ceará, desde pequeno o autor entrou em contato com o universo do cordel.
Além de escrever e publicar diversos poemas de cordel, Klévisson é um nome importante do universo cordelista pelo importante papel de divulgação que exerce: ilustra, diagrama e edita diversas obras de cordelistas do nordeste. 
Após fundar a editora Tupynanquim, em 1995, já publicou e divulgou centenas de cordéis pelo Brasil.

Nessa obra, percebemos a mastreia com a qual seu autor transita entre diferentes universos e gêneros, do tradicional ao pop. Sua personagem, Pedro Malazartes, é uma figura típica dos causos e contos orais do nordeste. De origem ibérica, a personagem é caracterizada por uma inteligência aguda e a malícia no trato com as pessoas, que o faz sempre sair na vantagem mesmo diante das situações mais adversas. As aventuras de Malazartes aqui, no entanto, são contadas através de uma história em quadrinho que utiliza os traços típicos da xilogravura: aliam-se, assim, os elementos tradicionais da narrativa cordelista (o traço da gravura em maneira, a clássica história de Pedro Malazartes) aos elementos artísticos e culturais contemporâneos, como a própria foram da história em quadrinho. Uma ótima oportunidade, portanto, para não apenas fruir as deliciosas aventuras de Pedro Malazartes, mas também explorar a forma do cordel e sua constante interação com novas formas de expressão artística, configurando um dos mais ricos patrimônios culturais do Brasil.

Ao longo do manual, todos esses aspectos serão explorados e relacionados a sugestões de atividades. Com isso, objetiva-se oferecer algumas ideias e inspirações para um trabalho que pode ser desenvolvido tanto a curto, quanto a médio e longo prazo. Sinta-se à vontade para personalizar a aula e torná-la sua, aplicando seus conhecimentos, sua 
personalidade e aproveite para fortalecer seu vínculo com a turma.
Boa aula!
\end{abstract}

\SideImage{Típica capa de um cordel feita a partir da gravura em madeira, a xilogravura (CC-BY-SA-4.0)}{PNLD2023-004-06.jpg}

\section{Sobre o livro}
O livro começa com um leitor de cordel, imitando uma espécie de rapsodo, contando a história do anti-herói, Pedro Malazartes. Fala primeiro da forma do cordel e seus temas comuns, histórias com ``sextilhas bem rimadas, / de amarelos sabidões / de Malazartes, Camões / Grilo e outros camaradas''. Cita então a origem ibérica de Pedro Malazartes, sua vinda para o Brasil com a colonização e os estudos pelos quais essa personagem passou através das obras dos folcloristas Silvio Romero e Câmara Cascudo.

Assim, pode introduzir a história propriamente: narra a infância pobre do menino que, desde pequeno, gostava de vadear e aprontar com os moradores da vila, safando-se das encrencas com esperteza e engenho.
Depois da morte do pai e das agruras da pobreza, deixa a terra e a casa de sua família para o irmão, trabalhador, e parte como viajante sem destino.
Sua primeira aventura dá-se em uma pequena vila na qual uma mulher tinha dificuldades durante o parto. Passando-se por benzedeiro, Pedro finge fazer mandingas e patuás que acabam ajudando a criança a nascer com saúde. Passa pouco tempo na cidadezinha, aclamado por seus moradores, até ser descoberto pela população e ter que fugir.

Segue vagando sem rumo. Encontra alguns urubus em volta de uma carcaça e, com uma brincadeira de criança, consegue caçar e capturar uma das aves. Percebe à frente uma casa de gente rica e pede algo para comer. A empregada da casa escorraça Pedro que, percebendo alguma coisa, esconde-se no telhado da casa. Espiando para dentro, depara com um lindo banquete que a dona da casa havia preparado para seu amante, enquanto o fazendeiro seu marido estava viajando. 

O marido, inesperadamente, retorna antes de viagem; o amante foge, a esposa esconde o banquete e recebe o fazendeiro com um prato de comida simples.
Pedro percebe sua oportunidade: bate novamente à porta da casa, pede comida e, atendido pelo marido, é recebido com humildade e generosidade. Esconde o urubu embaixo da mesa e o faz passar por uma ave encantada que adivinha o que vai acontecer. Pisa no bicho, que grita, e diz interpretar seu grito, dizendo que a ave está anunciando as delícias que a mulher preparou para o marido. Tudo que Pedrou havia visto de cima do telhado ele finge que o urubu está adivinhando, comunicando com antecedência o que a mulher estava prestes a trazer.
Assim, Pedro Malazartes come do bem e do melhor. De barriga cheia, consegue ainda vender o urubu para o dono da casa.
Volta para encontrar o irmão e, com tal montante em dinheiro, tornam-se ricos empresários.

\reversemarginpar
\marginparwidth=5cm


\section{Sobre o autor}


%532 caracteres
\paragraph{O autor}
Klévisson Viana nasceu em 3 de novembro de 1972 no interior do Ceará, na cidade de Quixeramobim. Além de escrever poesia em cordel, ele desenha, diagrama, imprime e comercializa suas obras, tendo um papel ativo na publicação e divulgação da obra de diversos cordelistas. Em 1980, aos oito anos, Klévisson mudou-se com a família para Canindé, tendo estudado apenas até o primeiro ano do segundo grau.

Canindé, a essa época, era uma espécie de centros da devoção popular da fé católica, congregando crentes, artistas sacros e vendedores de imagens católicas não
apenas do Ceará, mas do Nordeste inteiro. De família camponesa humilde, Klévisson ingressou nesse grande mercado de Canindé aos nove anos, vendendo bombons, artigos religiosos e velas nas datas e ocasiões especiais, como no Dia dos Finados.
Ali se tornou amigo dos diversos cordelistas que vinham a Canindé vender suas obras para os romeiros.

Apesar disso, sua relação com a arte e a literatura de cordel já vinha de sua família, como nos relembra José Nêumanne:

\begin{quote}
Sua avó paterna era uma pessoa esclarecida, dada a leituras, e havia inoculado no pai dele um gosto especial pelos romances de aventura, humor, conhecimento e fé, contidos nos folhetos comprados em feiras. A família era possuidora de um acervo razoável desses folhetos, que o pai costumava ler para os filhos quando voltava do trabalho duro --- de sol a sol, como se diz por lá --- de semear e colher no semiárido.\footnote{``Introdução''. Em: \textit{Biblioteca de Cordel: Klévisson Viana}. São Paulo: Hedra, 2007, p. 11.}
\end{quote}

Aos 14 anos, Klévisson ingressou de vez no universo da arte, fazendo ilustrações para os jornais de bairro de Canindé. Nessa mesma época, promovia salões de humor e eventos culturais na cidade. Na década de 1990, mudou-se para a capital, Fortaleza, onde passou a integrar diversos eventos e manifestações de artistas e humoristas cearenses.

Continuou, ainda assim, a colaborar na imprensa. Entre 1990 e 1995, foi ilustrador do jornal \textit{O povo}. Junto ao jornalista Tarcísio matos, foi editor da página \textit{Muro baixo}, na qual publicava charges, cartuns, caricaturas e piadas. Outro jornal de Fortaleza, a
\textit{Tribuna do Ceará}, também recebeu diversas colaborações textuais e desenhos seus.

Do formato clássico do cordel, logo passou a experimentar outras formas artísticas. Através da influência do irmão, Arievaldo Viana, descobriu seu talento como quadrinista, que o levaria a ganhar o prêmio nacional \textsc{hq} Mix, o mais importante do país no que se refere às histórias em quadrinhos, em 1998. Na ocasião, foi premiado por sua obra \textit{Lampião\ldots{} Era o cavalo do tempo atrás da besta da vida}, que reconta a vida e aventura do famoso cangaceiro em formato de quadrinho.

A primeira edição da obra foi publicada por uma então desconhecida gráfica e editora cearense, a Tupynanquim Aldeia, Mídia \& Tal, fundada por Klévisson e alguns amigos. Através da Tupynanquim, o artista passou a editar e publicar seus próprios folhetos de cordel, além de publicar e divulgar a obra de outros cordelistas. Desde sua fundação, em 1995, sua editora publicou mais de cinquenta poetas do Brasil inteiro.


\paragraph{A obra de Klévisson Viana}
Multiartista, agitador cultural, editor, Klévisson Viana também imprimiu essa heterogeneidade à sua produção poética. Em suas pesquisas na memória sertaneja do cangaço, o artista se aprofundou  nas tradições da poesia popular unindo-as a experimentações em diferentes gêneros artísticos. 
Tendo em vista a diversidade de sua produção, o pesquisador José Nêumanne propõe uma divisão de sua obra em alguns eixos temáticos:

\noindent\textsc{o repórter}: Na tradição do cordel, muito verso é criado de improviso em cima de acontecimentos da realidade. Nos rincões do país, com a difícil circulação de jornais, a falta de acesso ao rádio e à televisão, os sertanejos tinham o costume de se informar sobre os fatos cotidianos através dos cordéis e romances adquiridos nas feiras semanais de sua cidade.
No caso de Klévisson, sua produção cordelística não ignorou essa típica matriz de produção poética. Assim, entre seus diversos títulos, pode-se relembrar um cordel sobre a guerra entre os Estados Unidos e o Iraque; sobre a notícia de jumentos vendidos a R\$ 1,00; sobre 
os atentados terroristas que demoliram as Torres Gêmeas em Nova York e parte do Pentágono, em Washington, em 11 de setembro de 2001; sobre a morte de Celso Daniel, prefeito de Santo André e coordenador do programa de governo da campanha de Lula (\textsc{pt}) à Presidência da República; além da própria vitória eleitoral de Lula em novembro de 2002, também tematizado em sua obra.

\noindent\textsc{o romancista}: Da mesma forma que não chegavam jornais ao interior nordestino, também eram escassos os livros, com exceção das igrejas e casas paroquiais.
Nesse cenário, abundavam os cordéis fantásticos, que abordavam narrativas clássicas, temas mitológicos e medievais, típicos dos romances. 
Klévisson, como anota José Nêumanne, não ficou de fora dessa produção também:

\begin{quote}
O jovem cearense mostra-se um fabulador à altura de clássicos como \textit{O pavão mysteriozo} ou \textit{O país de São Saruê}, no recriar as viagens de aventuras de Hércules e outros heróis mitológicos, a partir de padrões fixados na literatura medieval europeia e reproduzida nos folhetos de cordel de antigamente. Merecem ser destacados dois títulos: \textit{O boi dos chifres de ouro ou O vaqueiro das três virtudes} e \textit{O príncipe do Oriente e o pássaro misterioso}.\footnote{\textit{Ibidem}, p. 21.}
\end{quote}

\noindent\textsc{o quengo}: Na definição sertaneja, ``quengo'' refere-se aos nordestinos que conseguem fugir da desgraça com ginja, malemolência e graça. Talvez sua figura mais conhecida seja João Grilo, eternizado pela peça de Ariano Suassuna, \textit{O auto da Compadecida}, e o posterior filme homônimo. Em Klévisson, podem-se encontrar a clássica figura do quengo em cordéis como \textit{Viagem ao país de São Cornélio} e \textit{O rapaz que namorou com a velha dos papangus pensando que era a Carla Perez}. 


\noindent\textsc{o historiador}: Klévisson também abordou muito a história e a biografia em suas obras. Além do \textsc{hq} já mencionado sobre Lampião, com a típica temática do cangaço, o autor também versou sobre personalidades de outras artes, como o cinema, no caso do cordel \textit{Charlie Chaplin, o Carlitos: do Big Bem à Coluna da Hora}.

\noindent\textsc{o inventor de episódios}: Outra característica da literatura de cordel é a reprodução de desafios de repentistas. Talvez o cordel que siga essa temática que mais obteve sucesso foi o folheto versando sobre a peleja entre o Cego Aderaldo e Zé Pretinho. Klévisson também frequenta o gênero, com cordéis como \textit{A insustentável
peleja de Zé Maria de Fortaleza com Calixtão de Guerra}, \textit{A grande peleja virtual de Klévisson Viana com Rouxinol do Rinaré} e \textit{A grande peleja de Beneval com José Mota Pinheiro}.

\noindent\textsc{o adaptador}: Por fim vale lembrar o frequente hábito entre os cordelistas de adaptar históricas clássicas de romances, de lendas medievais europeias ou da Grécia Antiga, e inclusive dos próprios colegas cordelistas. Entre sua vasta obra, podemos citar a adaptação que Klévisson fez de Homero, em \textit{Helena de Troia e o cavalo misterioso}, de histórias da carochinha com \textit{A história de João e o pé de feijão} e do cordelista Zé Pacheco, escrevendo uma espécie de continuidade para seu clássico \textit{A festa dos cachorros} com \textit{O divórcio da cachorra}.

\paragraph{A autora}
Arlene Holanda nasceu em Limoeiro do Norte, no Ceará. Conviveu durante sua infância com o universo mágico do cordel. Entre suas história preferidas, estão a do Pavão e de
Juvenal. A curiosidade e o gosto por histórias me fizeram-na escolher o curso de História. Especializou-se também em Artes Visuais. Escreve em variados gêneros e estilos literários. Tem cerca de 50 livros publicados, entre literatura (adulto, infantil e juvenil), didáticos e obras complementares.


\paragraph{O ilustrador}
Maércio Siqueira nasceu em Santana do Cariri, Ceará, em 21 de novembro de 1977.
Mora em Crato, Ceará, desde os cinco anos. No Curso de Letras, em 1999, conheceu o
maravilhoso mundo da literatura de cordel, e passou a escrever alguns
folhetos, vindo a ser membro da Academia dos Cordelistas do Crato.
Nessa mesma época aprendeu a fazer xilogravura, uma importante arte
plástica nordestina e universal. Teve a oportunidade, com seu trabalho plástico, de ilustrar muitas capas de cordel e vários livros.


\Image{Disposição dos folhetos em uma típica feira de cordéis (Diego Dacal; CC-BY-SA-4.0)}{PNLD2023-004-07.jpg}

\section{Sobre o gênero}

%55 caracteres
\paragraph{O gênero} O gênero deste livro é \textit{cordel em quadrinho}. 



Alguns estudiosos defendem que o termo \textit{cordel} venha de Portugal, onde os \textit{folhetos} eram vendidos em feiras pendurados em barbantes, em cordões que se chamavam cordéis. Já para outros, o cordel era assim chamado porque as brochuras eram encadernados com barbantes. No Brasil, porém, não se falava em cordel. Somente a partir dos anos 1960, com a persistência dos pesquisadores europeus pelo nome, os poetas passaram a ser chamados de cordelistas. Para o público mais popular, no entanto, ele continua sendo chamado de \textit{romance} e \textit{folheto}.

Chama-se cordel as histórias curtas em versos rimados de personagens lendárias impressas em cadernos, geralmente artesanais, com ilustrações feitas sob a técnica da 
xilogravura, e comercializadas originalmente em feiras livres do Nordeste do Brasil. 
Sua origem remete às cantigas portuguesas medievais trazidas pelos colonos.
O cordel não tem nem um limite nem uma receita pronta. É o verso da 
nossa tradição popular brasileira. Hoje, o gênero sente-se à vontade para falar de qualquer 
assunto, abordar qualquer temática, refletir o mundo do nosso tempo:

\begin{quote}
Não há um só grande acontecimento local, nacional, ou mesmo mundial que não tenha sido tratado pela literatura de cordel. O folheto mostra a realidade, mais do que os grandes meios de comunicação, porque não é atrelado a coisa alguma. É independente e é a opinião do autor. Não tem interesse em grupos econômicos, nem tem patrocinadores. Por isso, critica e aborda, como nenhum outro meio. 
Sendo honesto em suas abordagens, é natural que o cordel se sinta ameaçado --- da mesma forma que  a televisão e o rádio ameaçaram o jornal impresso.\footnote{``Klévisson Viana -- Cordel para os intelectuais e folheto para o povo.'' Entrevista para o jornal \textit{A nova democracia}, ano \textsc{i}, nº 8, abril de 2003.}
\end{quote}

Dizem ainda os especialistas, que a poesia de cordel é uma poesia escrita para
ser lida, enquanto o repente ou o desafio é a poesia feita oralmente, que mais tarde pode
ser registrada por escrito. Essa divisão é muito esquemática. Por exemplo, o
cordel, mesmo sendo escrito e impresso para ser lido, costumava ser lido em
volz alta e desfrutado por outros ouvintes além do leitor. A poesia popular,
praticada principalmente no Nordeste do Brasil, tem muita influência da
linguagem oral, aproveita muito da língua coloquial praticada nas ruas e na
comunicação cotidiana. 

Naturalmente, portanto, pode"-se considerar a poesia narrativa do cordel uma
forma de poesia mais compartilhada e desfrutada coletivamente, o que dá também
uma grande ressonância social. Muitos dos temas do cordel são originários das
tradições populares e eruditas da Europa medieval e moderna. Outros temas são
retirados de tradições orientais, das novelas de cavalaria medievais e das narrativas
bíblicas. Ao lado destes temas mais literários, encontram"-se os temas locais,
quase sempre narrados na forma de crônicas de coisas realmente acontecidas.

Os grandes poemas de cordel são perfeitamente metrificados e rimados. A métrica
e a rima são recursos que favorecem a memorização e tradicionalmente se costuma
dizer que são resquícios de uma cultura oral, na qual toda a tradição e
sabedoria são sabidas de cor.  


\paragraph{O sertão geográfico e cultural}

O sertão tem mitos culturais próprios. Contemporaneamente, o sertão evoca
principalmente o sofrimento resignado daqueles que padecem a falta de chuva e
de boas safras na lavoura. Evoca a experiência histórica de uma região
empobrecida, embora tenha sido geradora de riquezas, como o cacau e cana de
açúcar, ambos bens muito valiosos. 

O sertão formou também o seu imaginário por meio de grandes personalidades e
uma pujante expressão artística. Além do cordel, o sertão viu nascer ritmos tão
importantes quanto o forró e o baião. Produziu artistas tão expressivos quanto
Luiz Gonzaga, grande cantor da vida do sertanejo em canções como “Asa branca”.
Um escultor como Mestre Vitalino criou toda uma tradição de representação da
vida e dos hábitos sertanejos em miniaturas de barro. A gravura popular, que
sempre acompanha os folhetos de cordel, também floresceu em diversos pontos e
ficou mais famosa em Juazeiro do Norte, no Ceará, e em Caruaru, no estado de
Pernambuco. 

Dentre os grande mitos do sertão, está certamente o do cangaço com seu líder
histórico, mas também mítico, Virgulino Ferreira, o Lampião. Até hoje as
opiniões se dividem: para alguns foi uma grande homem, para outros um bandido
impiedoso. 

Uma figura muito presente na cultura nordestina é o Padre Cícero Romão,
considerado beato pela Igreja Católica. Consta que teria feito milagres e
dedicado sua vida aos pobres. 

\paragraph{Variação linguística}

A linguística moderna usa o termo “idioleto” para marcar grupos distintos no
interior de uma língua. Um idioleto pode ser a fala peculiar de uma região, de
um grupo étnico ou de uma dada profissão. 

Uma das grandes forças da poesia popular do Nordeste se origina em sua forma
muito própria de falar, com um ritmo muito diferente dos falares do sul, e
também muito diferentes entre si, pois percebe"-se a diferença entre os falares
de um baiano, um cearense e um pernambucano, por exemplo.

Além desse aspecto rítmico, quase sempre também há palavras peculiares a certas
regiões.

\paragraph{O quadrinho}

O cordel aqui em questão carrega influências de outro gênero literário, também
verbo-visual: a história em quadrinhos, ou \textsc{hq}. A \textsc{hq} é um 
gênero que trabalha ao mesmo tempo a linguagem verbal e a visual, portanto trata-se de 
uma \textit{narrativa gráfica}. Não há uma hierarquia entre o texto e a ilustração: nem 
o texto é mera legenda da imagem, nem a imagem mera ilustração do texto; são dois elementos 
de uma mesma obra, que deve ser lida como um todo.

Ambas as formas literárias exercitam a imaginação e a criatividade das crianças e dos jovens quando bem utilizadas. Podem servir de reforço à leitura e constituem uma linguagem altamente dinâmica.  São linguagens que, ainda que de uma origem longínqua, são adequadas à nossa era devido à fluidez, à intensidade e sobretudo à abertura à inovação que lhes constitui.

\paragraph{A xilogravura}

Tanto o cordel quanto as histórias em quadrinhos têm algo em comum:
a presença de imagens. Ao se pensar em cordel, logo se pensa em \textit{xilogravura}, 
mas a xilogravura não surgiu com a literatura de cordel. Ela começou a fazer
parte dos folhetos a partir da década de 1950. Tradicionalmente, trata-se
de uma matriz de madeira que imita um clichê de chumbo. O clichê em si 
já é uma imitação da xilogravura, \textit{uma técnica milenar dos egípcios
e chineses}: recorta-se uma figura em relevo sobre uma madeira. A figura 
em relevo imprime, como um carimbo sobre um papel em branco, e as partes
cortadas são os sulcos onde a tinta não aparece. 

A xilogravura entrou na vida da literatura de cordel como uma alternativa 
ao poeta sem recursos para ilustrar a capa de um folheto. Ela entrou indiretamente na 
estrutura do folheto e o público não se identificou de imediato. Hoje, se por um lado 
o público intelectual que gosta de folheto, o estudioso ou o turista que compra o 
folheto como uma curiosidade, prefere a capa com a xilogravura, o público mais 
tradicional prefere a capa com desenhos, fotografias. 
Os autores e editores tentam sempre agradar a todos, trabalhando tanto com a xilogravura 
como com desenhos, figuras, etc.

\section{Atividades}

\subsection{Pré-leitura}

\BNCC{EF04GE01}
\BNCC{EF15AR03}


\paragraph{Tema} As origens de Pedro Malazartes.

\paragraph{Conteúdo} Leitura do conto \href{https://pt.wikisource.org/wiki/Contos_Populares_do_Brazil/Uma_de_Pedro_Malas-Artes}{``Uma das de Pedro Malas-Artes''}. coletada pelo folclorista Sílvio Romero no Sergipe.

\paragraph{Objetivo} Contextualizar os alunos na produção da literatura de cordel e mostrar sua importância cultural para o país; mostrar a relação do cordel, tanto formalmente como tematicamente, com a tradição trovadora ibérica; e habilitar os alunos a ter uma leitura crítica e contextualizada das narrativas apresentadas na poesia de cordel.

\paragraph{Justificativa} Como bem demonstrou o escritor paraibano Ariano Suassuna com sua pesquisa armorial, a cultura nordestina é fortemente influenciada por mitos, lendas e folclores medievais europeus. Com a colonização, muitos elementos formadores do caldo cultural medieval foram importados para a colônia portuguesa e, como eram transmitidos oralmente, aqui se continuou essa tradição. Cavaleiros, criaturas mágicas, feiticeiros e bobos visionário, típicos das lendas medievais, abundam também nas narrativas dos cordéis, misturados às crenças, personagens e tradições locais.
\SideImage{O escritor e folclorista brasileiro Sílvio Romero (1851-1914). (CC-BY-SA-4.0)}{PNLD2023-004-05.jpg}

Esse é precisamente o caso da história de Pedro Malazartes: a personagem que inspirou o cordel em quadrinho de Klévisson Viana é uma figura típica das lendas populares da Península Ibérica. Diversas variações sobre suas aventuras podem ser ouvidas das bocas dos contadores nordestinos, sempre mantendo, com a história ibérica, o traço distintivo da personalidade de Pedro: uma personagem que, diante das maiores adversidades, consegue se sair bem com o intelecto aguçado e invencionices que permitem-lhe driblar as adversidades do meio e de sua condição social.

A partir da história de Sílvio Romero, coloca-se uma oportunidade para o aluno interagir com as diferentes linguagens e matrizes que constituem o fazer artístico, além de estimular a apreensão dos elementos constituintes de um enredo literário, como as características e traços pessoais de uma personagem.


\paragraph{Metodologia} O professor pode fazer uma primeira leitura para a classe da história do Pedro Malas-Artes de Sílvio Romero. É uma narrativa curta que, disponível online, pode ser projetada na sala ou impressa e distribuída, para que os estudantes acompanhem a leitura.

A partir dessa primeira apresentação do texto, o professor pode chamar a atenção para os elementos do enredo: há um rei e uma princesa; Malas-Artes anda por um vilarejo rural, povoado por camponeses e pequenos agricultores; os bens obtidos por Pedro ao longo da narrativa fazem parte do ambiente rural (botijas de azeite, galinhas, perus, carneiros etc.).
O professor pode situar esses elementos e relacioná-los ao ambiente medieval ibérico no qual a história foi gestada: apesar da história ser coletada por Sílvio Romero no Sergipe, seus traços evocam claramente o universo medieval peninsular, com seus reis, princesas e súditos temerosos do poder real.

Em seguida, chama-se a atenção para as características de Pedro: como maliciosamente explora a ingenuidade dos camponeses para conseguir trocar seus bens por coisas mais preciosas, até conseguir enganar ao próprio rei.
O professor pode perguntar aos alunos o que acharam da história e da personalidade de Pedro.
Podem explorar, por exemplo, a questão da desigualdade e do ambiente desfavorável que faz com que o anti-herói use seu intelecto para trapacear e se sair bem: afinal, Pedro não engana os camponeses gratuitamente, mas utiliza-se de artimanhas para ganhar os favores do rei em uma terra e uma época em que prevalecia a miséria da maioria em detrimento do bem-estar da família real.

A partir dessa exposição, o professor pode explicar como a história medieval ibérica veio parar no nordeste brasileiro com a colonização, e como tais narrativas ibéricas foram fundamentais para a constituição da literatura de cordel.
Alguns pontos sobre os quais pode-se debater com os alunos, após a leitura do conto de Sílvio Romero, sua contextualização ibérica e sua relação com a cultura nordestina, são:

\begin{itemize}
\item A contextualização da literatura de cordel e sua importância cultural para o país;

\item A relação do contexto nordestino com a forma do cordel;

\item O panorama da vida rural no interior do país apresentado pelo cordel;

\item A relação do cordel, tanto formalmente como tematicamente, com a tradição trovadora ibérica;

\item A leitura crítica e contextualizada das narrativas apresentadas na poesia de cordel.
\end{itemize}

Ler a história, falar sobre a origem ibérica, a influência da culturap popular medieval, perguntar quem conhece a história.
Depois da leitura: quem conhece personagens com as mesmas características? O que chamou a atenção?

\paragraph{Tempo estimado} Duas aulas de 50 minutos.

\includepdf[nup=2x3, 				% grid
			%offset=-15mm -5mm, 	% posição
			scale=.8, 				% tamanho da página
            delta=4mm 4mm, 			
            frame,
            pages={9-14}]{./pdfs/\jobname_MIOLO.pdf}

\subsection{Leitura}

\subsubsection{Atividade 1}

\BNCC{EF15LP19}
\BNCC{EF04LP07}
\BNCC{EF15AR03}

\paragraph{Tema} Leitura do cordel em quadrinho \textit{Artimanhas de Pedro Malazartes}.

\paragraph{Conteúdo} Exercícios de leitura compartilhada do quadrinho em cordel.


\paragraph{Objetivo} Através de uma leitura conjunta, objetiva-se despertar o interesse do aluno pela história e aproximá-lo de algumas características: a forma como as estrofes são encadeadas em rimas; os traços típicos da linguagem do cordel; a estrutura do enredo.

\paragraph{Justificativa} De forma geral, as pessoas que estão se iniciando no universo da leitura têm mais dificuldade em ler poesia do que prosa. Isso porque a poesia, acredita-se, tem um texto mais elíptico, cifrado, formal e, portanto, de assimilação mais difícil do que o texto em prosa. A partir de uma leitura acompanhada entre o professor e os alunos, facilita-se a apreensão da narrativa poética. Além disso, através de uma leitura dialogada, o professor pode explorar pontos do enredo e a própria forma de apresentação da poesia, estruturada como uma moderna história em quadrinhos.


\paragraph{Metodologia} A ideia da atividade é fazer uma leitura alternada do professor e dos alunos. O professor pode fazer uma primeira leitura integral, para apresentar a narrativa e elucidar possíveis dúvidas, como algumas palavras que podem apresentar maior dificuldade de compreensão e elementos do enredo.

Em seguida, pode-se fazer propriamente a leitura alternada, em que o professor lê o primeiro balão de fala ou narrativa do quadrinho e, em sequência, solicita a um aluno que leia o balão seguinte, até que todos tenham lido ou que a história se encerre.
Explore a percepção dos traços típicos da linguagem do cordel: elementos de ritmo e rima, vocabulários característicos, o uso de palavras específicas para obedecer ao ritmo e à melodia. Fale sobre a relação entre essa estrutura rimada e o contexto dos cordelistas, que normalmente decoravam as histórias e, assim, utilizavam as rimas para auxiliar a memória e a fixação do texto da mente.

Por fim, pode-se explorar mais detidamente a estrutura do enredo. Fale sobre os diferentes elementos que o compõem:

\begin{enumerate}
\item O uso de um leitor de cordel para introduzir a história;

\item A introdução da personagem;

\item A narração de sua infância;

\item O contexto da seca no nordeste e das rotas de imigração, pelas quais Pedro atravessa;

\item A primeira aventura pela qual passa;

\item A segunda aventura;

\item As características de Pedro para que se saia bem nas duas aventuras;

\item O retorno para casa;

\item O desfecho favorável para os irmãos.
\end{enumerate}

A partir desses elementos, os alunos começam a tomar contato e familiaridade com a estrutura de um texto ficcional, e como diferentes situações e personagens são invocadas para gerar o movimento da história e tramar a narrativa. Incentive que os alunos interajam com a obra, apelando para seus gostos e impressões.
Pode-se fazer perguntas como:

\begin{itemize}
\item De qual personagem vocês mais gostaram? Por que?

\item Qual das aventuras vocês acharam mais interessante?

\item Qual a cena mais emocionante da história?

\item O que vocês acharam de Pedro Malazartes?

\item Qual foi a moral da história?

\end{itemize}

Dessa forma, estimula-se a apreensão dos alunos do enredo e da estrutura do poema em cordel.

\paragraph{Tempo estimado} Duas aulas de 50 minutos.


\subsubsection{Atividade 2}

\BNCC{EF35LP29}
\BNCC{EF15AR03}
\BNCC{EF15AR25}

\paragraph{Tema} As relações de \textit{Artimanhas de Pedro Malazartes} e do conto de Sílvio Romero.

\paragraph{Conteúdo} Após a leitura da obra de Klévisson Viana e de Sílvio Romero, chegou o momento de comparar os dois enredos.

\paragraph{Objetivo} Aprofundar a compreensão dos estudantes sobre as diferentes matrizes culturais envolvidas na criação artística e sobre a estrutura da narrativa, comparando dois enredos distintos com a mesma personagem.

\paragraph{Justificativa} A narrativa de cordel, como se expôs até aqui, tem forte relação com a cultura ibérica medieval. A partir do trabalho em sala com duas narrativas paralelas --- uma contemporânea, criada por um cordelista, e outra mais antiga, datada por alguns estudiosos como Câmara Cascudo de meados do século \textsc{xvi}, e coletada por Sílvio Romero em princípios do século \textsc{xx} ---, o aluno terá oportunidade de aprofundar sua compreensão da história, percebendo semelhanças e diferenças na construção de dois enredos distintos, além de perceber nitidamente a influência de diferentes matrizes culturais (não só a europeia, como a indígena e a africana) na constituição da cultura nacional.


\paragraph{Metodologia} Agora que os alunos estão familiarizados com a narrativa de Klévisson Viana e uma de suas possíveis fontes primárias (o conto coletado por Sílvio Romero), pode-se explorar os elementos em comum das duas narrativas.
Alguns pontos para iniciar essa comparação são:

\begin{itemize}
\item O ambiente no qual as narrativas se passam;

\item As personagens que aparecem;

\item As diferentes artimanhas boladas pelos dois Pedros;

\item O desfecho das histórias;

\item As características centrais da personalidade de Pedro Malas-Artes e Pedro Malazartes.
\end{itemize}

A partir das observações dos estudantes, o professor pode aprofundar os pontos de contato e também instigá-los a pensar nas diferenças que notaram entre os dois enredos.
Um ponto de interesse a ser explorado é a personalidade dos dois personagens.
Pode-se falar sobre suas ações e como os alunos as percebem; de qual personagem gostam mais; qual acham que se deu melhor; etc.

Explore a característica do ``bom malandro'', típica da cultura popular e traço marcante de Malas-Artes e Malazartes: como ambos os personagens estão em situações desfavorecidas (seja na pobreza camponesa do reinado ibérico ou da seca nordestina brasileira) e como, com graça e gênio, conseguem contornar essas situações, sem nenhum pudor em utilizar-se do que dispõem à mão, sejam galinhas ou a ingenuidade camponesa.

Fale sobre a estrutura dos dois enredos que, através das artimanhas de Pedro, apresentam uma sucessão e um encadeamento narrativo semelhantes. Nos dois casos, o personagem parte de um estágio inicial de pobreza e, com suas artimanhas, vai progressivamente superando desafios e ganhando mais bens, até encerrar a narrativa em um estado confortável e venturoso. Na narrativa de Sílvio Romero, tal encadeamento é evidente: a cada parada que faz, Pedro Malas-Artes consegue enganar seu hospedeiro e ganhar um bem mais valioso do que o que deixara ali, concretizando, ao final, a troca de botijas de azeite por lindas mulheres e obtendo, assim, as graças do rei. O Pedro Malazartes de Klévisson procede de forma semelhante: após uma primeira sorte passando-se por santo parteiro, vislumbra uma oportunidade ainda melhor, através da qual vai progressivamente se dando melhor: de um urubu, consegue um peru assado, depois um leitão e todo um banquete, até conseguir vender o próprio urubu por muito dinheiro e acabar rico.

Explore esses elementos do enredo, ressaltando sempre as semelhanças e diferenças e relacionando-as à economia narrativa. A partir do caso acima citado, por exemplo, os alunos podem observar melhor as características em comum das duas personagens e como, em ambas as histórias, coloca-se uma estrutura de funcionamento parecida: o estágio inicial de miséria, o gênio malandro que apresenta linhas de fuga e superação, a situação progressivamente vantajosa para o personagem e, por fim, seu desfecho em situação muito melhor do que a inicial.

\paragraph{Tempo estimado} Uma aula de 50 minutos.

\subsection{Pós-leitura}
\BNCC{EF05LP26}
\BNCC{EF05LP27}
\BNCC{EF15AR01}
\BNCC{EF04LP07}
\BNCC{EF05LP19}

\paragraph{Tema} As novas artimanhas de Pedro Malazartes. 

\paragraph{Conteúdo} Redigir uma pequena aventura do seu próprio Malazartes, tendo em vista as características da personagem apreendidas nas atividades anteriores.


\paragraph{Objetivo} Incentivar a produção textual dos estudantes em relação com alguns conhecimentos linguísticos e gramaticais que envolvem a escrita de um texto, tais como: regras sintáticas de concordância nominal e verbal, pontuação, coesão pronominal, articuladores de relações de sentido e concordância entre artigo, substantivo e adjetivo.

\paragraph{Justificativa} A partir da apreensão poética do cordel, passa-se ao momento de aplicar alguns conhecimentos trabalhados ao longo das aulas na produção de um cordel do próprio estudante. Assim, pode-se relacionar o universo artístico dos cordéis, com suas características como os versos rimados, com a produção escrita e algumas de suas características esperadas em aluno do 4º e 5º ano. Essa é uma forma de desenvolver o conhecimento gramatical e linguístico de forma afetiva e artística, pois relaciona os conhecimentos aprendidos sobre o cordel com as regras gramaticais e as próprias experiências dos estudantes, que podem servir de base para a criação de sua pequena história em cordel.


\paragraph{Metodologia} Já mais familiarizados com as características do cordel, os alunos devem passar então à produção de um texto nesse formato. Como os estudantes viram, ao longo das aulas, duas narrativas que falavam sobre Pedro Malazartes e as características desse malandro popular, a ideia é desenvolver uma narrativa curta que envolva Pedro.
É aconselhável incentivar os estudantes a explorar as características da personagem e das narrativas estudadas em sala de aula.

O aluno pode partir de diferentes lugares para se inspirar na redação de sua nova aventura de Pedro Malazartes. 
Algumas sugestões:

\begin{itemize}
\item Trabalhar com a redação de um fato autobiográfico do aluno, usando algo visto ou vivido pelo estudante como mote para seu Pedro Malazartes;

\item Recriar alguma das histórias vistas em aula, pensando, por exemplo, em diferentes desfechos, ou em diferentes ambientes (o bairro ou a cidade do aluno, por exemplo);

\item Partir de algum acontecimento de interesse social, divulgado em \textsc{tv}, rádio, mídia impressa e digital, para criar uma narrativa em cima de uma fato real, como era tradição entre os poetas cordelistas.
\end{itemize}

O tema da redação não precisa ser fechado, mas é interessante que os alunos se norteiem pelas características de Pedro exploradas em sala tentando pensar em novas possibilidades e variações a partir dessa figura tão típica e característica da cultura nordestina.
O próprio quadrinho em cordel de Klévisson, vale ressaltar com os estudantes, é uma espécie de variação em cima da narrativa clássica apresentada por Sílvio Romero e outros folcloristas, mas repetida desde muito tempo nas regiões do nordeste.

Após a redação, pode-se facilmente montar um cordel com uma folha sulfite a4. 
Pode-se, por exemplo, dobrar a folha em quatro e cortar as bordas, colando ou grampeando alguma das extremidades para montar uma caderneta de quatro páginas.
Os alunos podem pintar e desenhar a primeira folha, criando uma capa para os seus cordéis.

Com o material produzido, é possível realizar uma grande
feira de cordéis na escola. Junto ao professor de artes, sugere-se que
os alunos preparem uma decoração adequada, valorizando as cores e os
traços presentes nesse tipo de material tão distinto. Além da exposição
dos cordéis produzidos, sugere-se também a promoção de encenações de
trechos de obras lidas, ou produzidas pelos alunos, a declamação de
poemas, competição de repentes. Também se estimula a aproximação deste
gênero com outras produções artísticas mundiais. Nada impede que, em uma
dessas encenações ou leituras dramáticas, os alunos busquem trechos
de peças, filmes ou obras da literatura mundial para encená-los, valendo-se
da estética e do vocabulário típicos dos cordéis.

\paragraph{Tempo estimado} Duas aulas de 50 minutos.

\section{Sugestões de referências complementares}


\begin{itemize}
\item \textsc{diegues júnior}, Daniel. \textit{Literatura popular em verso}. Estudos. Belo Horizonte: Itatiaia, 1986. 

\item \textsc{marco}, Haurélio. \textit{Breve história da literatura de cordel}. São Paulo: Claridade, 2010.

\item \textsc{tavares}, Braulio. \textit{Contando histórias em versos. Poesia e romanceiro popular no Brasil}. São Paulo: 34, 2005.

\item \textsc{tavares}, Braulio. \textit{Os martelos de trupizupe}. Natal: Edições Engenho de Arte, 2004.
\end{itemize}

\section{Bibliografia comentada}

\subsection{Livros}

\begin{itemize}
\item \textsc{brasil}. Ministério da Educação. Base Nacional Comum Curricular. Brasília, 2018.

Consultar a \textsc{bncc} é essencial para criar atividades para a turma. Além de especificar 
quais habilidades precisam ser desenvolvidas em cada ano, é fonte de informações sobre 
o processo de aprendizagem infantil. 


\item \textsc{van der linden}, Sophie. Para ler o livro ilustrado. São Paulo: Cosac Naify, 2011.

Livro sobre as particularidades do livro ilustrado, que apresenta as diferenças entre o livro ilustrado e o livro com ilustração. 
\end{itemize}

\subsection{Filmes}

\begin{itemize}
\item \textit{O auto da compadecida}. Direção de Guel Arraes, Brasil, 2000.

Baseado na peça homônima de Ariano Suassuna, o filme conta as aventuras de João Grilo e Chicó. Na personagem de João Grilo, percebemos a figura típica do herói malandro, que consegue sair das piores confusões com artimanhas astutas e engenhosas. A personagem aproxima-se do Pedro Malazares, podendo ser explorada em sala em suas semelhanças com o anti-herói de Klévisson Viana.
\end{itemize}

\end{document}