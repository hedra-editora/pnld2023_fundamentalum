\documentclass[11pt]{extarticle}
\usepackage{manualdoprofessor}
\usepackage{fichatecnica}
\usepackage{lipsum,media9}
\usepackage[justification=raggedright]{caption}
\usepackage[one]{BNCC}
\usepackage[araucaria]{../edlab}
\usepackage{marginnote}
\usepackage{pdfpages}

\newcommand{\AutorLivro}{Kate Woodard}
\newcommand{\TituloLivro}{Pedrinho petrificado}
%\newcommand{\Tema}{}
\newcommand{\Genero}{Poesia; poema; trava-línguas; parlendas; adivinhas; provérbios; quadrinhas e congêneres}
%\newcommand{\imagemCapa}{./images/PNLD2022-001-01.jpeg}
\newcommand{\issnppub}{XXX-XX-XXXXX-XX-X}
\newcommand{\issnepub}{XXX-XX-XXXXX-XX-X}
% \newcommand{\fichacatalografica}{PNLD0001-00.png}
\newcommand{\colaborador}{Ana Lancman}
\begin{document}

\title{\TituloLivro}
\author{\AutorLivro}
\def\authornotes{\colaborador}

\date{}
\maketitle

%\begin{abstract}\addcontentsline{toc}{section}{Carta ao professor}
%\pagebreak

\tableofcontents

\begin{abstract}
Caro professor,\medskip

O presente manual tem como objetivo oferecer uma orientação sobre a obra \textit{Pedrinho petrificado}. A partir deste manual, os professores poderão incentivar a prática da leitura aos estudantes e proporcionar um conteúdo enriquecedor. Apresentamos aqui sugestões de atividades a serem realizadas antes, durante e após a leitura do livro, com propostas que buscam introduzir os gêneros literários e aprofundar as discussões trazidas pelas obras. Você encontrará informações sobre o autor, sobre o gênero e sobre os temas trabalhados ao longo do livro. Ao fim do manual, você encontrará também sugestões de livros, artigos e sites selecionados para enriquecer a sua experiência de leitura e, consequentemente, a de seus estudantes.

\textit{Pedrinho petrificado} é uma obra atual, que aborda temas como o `medo' e a `reclusão' com uma linguagem divertida e lúdica. Ao longo da narrativa acompanhamos a jornada de Pedrinho, um menino que precisa ganhar coragem para encarar o medo do desconhecido. Com autoria de Kate Woodard, o livro possui um tom sensível e ao mesmo tempo humorístico. As ilustrações de Sara Sanchez são cativantes ao retratar cada momento na trajetória do personagem. 

Este livro pode ser um disparador interessante para que cada estudante possa reconhecer em si mesmo experiências similares. Ao nos permitirmos ver o que há por detrás da porta, é possível superar medos e rirmos de nós mesmos, mas também perceber como a prática pode magoar e até mesmo traumatizar colegas, caracterizando inclusive \textit{bulling}. 

Esperamos que as atividades sugeridas e o material indicado sejam proveitosos em sala de aula! 

\end{abstract}

\section{Sobre o livro}

\paragraph{O livro} \textit{Pedrinho petrificado} foi escrito por Kate Woodard e ilustrado por Sara Sanchez. É um livro de poesia ilustrado que oferece uma narrativa cativante. É uma obra que trata de temas difíceis, como medo, solidão e ansiedade, de uma forma bem-humorada. Com ilustrações coloridas e cheias de detalhes, somos convidados a conhecer o universo interno de Pedrinho e sua jornada de superação de medos.

\paragraph{Descrição} \textit{Pedrinho petrificado} apresenta Pedrinho, um garoto considerado dócil por seus vizinhos. Dentro de sua casa, descobrimos que ele se assusta com tudo: tem medo de barulhos, de tomar banho e não quer sair de baixo da cama. Um dia, ouve um barulho na porta que o deixa aterrorizado. Ele tenta esperar o barulho passar, em vão. Quando consegue coragem para abrir a porta, descobre que é um galho de uma árvore batendo. Pedrinho começa a gargalhar e fica aliviado.

\section{Sobre as autoras}

\SideImage{Foto da autora (Arquivo pessoal)}{PNLD2023-002-02.png}

\paragraph{A autora} Kate Woodard escreve livros infantis desde 2017 e \textit{Pedrinho petrificado} foi o seu primeiro livro publicado. Ela nasceu em uma região de montanhas nos Estados Unidos, em que não havia muito o que fazer. Com suas amigas e parentes, começou a fazer vídeos de humor e escrever histórias. Este foi o início de sua descoberta como escritora. Formada em teatro, especializou-se em dramaturgia e em seguida estudou jornalismo. Kate começou a escrever livros infantis quando percebeu que passava horas em livrarias e bibliotecas na seção de livros infanto-juvenis, o que a inspirou a criar seus próprios mundos e personagens.

\paragraph{A ilustradora} Sara Sanchez é de Madrid, Espanha. Ela é ilustradora e designer. Foi responsável pela ilustração de todos os livros de Kate Woodard. Com a técnica de ilustração digital, ela oferece uma experiência visual com entretenimento, humor e profundidade de significados. No caso de \textit{Pedrinho petrificado}, as ilustrações possuem cores vivas, texturas delicadas e detalhes que dão harmonia para as páginas. O trabalho de composição entre texto e imagem torna a narrativa mais especial e convidativa.

\SideImage{Foto da ilustradora (Arquivo pessoal)}{PNLD2023-002-03.png}

\section{Sobre o gênero}

\paragraph{O gênero} O gênero deste livro é \textit{poesia}. 

\paragraph{Descrição} Um dos meios mais expressivos de comunicação e inovação da linguagem, a poesia é uma das mais antigas formas de arte literária, anterior até mesmo à escrita, pois existe desde a tradição oral. Ela combina palavras, significados, sonoridades, ritmos e, muitas vezes, também imagens para permitir uma experiência estética. A linguagem poética é condensada e emotiva e busca trabalhar a língua de forma que o leitor experimente as palavras de uma forma nova. Na maior parte das vezes, a poesia é dividida em versos que, juntos, são chamados de estrofes. O ponto de vista do autor e sua visão pessoal do mundo estão muito presentes nesse tipo de texto e, justamente por essa particularidade, a experiência da leitura de uma poesia é extremamente individual e subjetiva.
\SideImage{O gênero poético incentiva a curiosidade e a imaginação. (LACMA/Remedios Varo; CC BY-NC 2.0)}{PNLD2023-001-07.png}

\paragraph{Interação} Esse gênero é um grande aliado na formação do leitor. O olhar da criança para o mundo é, em essência, um olhar poético, calcado na curiosidade pelo mundo. A poesia é a forma perfeita de valorizar esse olhar e incentivar que a criança brinque com as palavras, observe os sons e experimente novos ritmos. Por sua liberdade e criatividade, a poesia tem potencial para estabelecer um diálogo único com os pequenos leitores. A presença de fantasias, imagens, repetição e símbolos permite uma maior identificação, pois a criança ainda está construindo seu mundo interior e experimenta a vida de forma diferente do adulto. 


\paragraph{Competências} O caráter polissêmico do texto poético pode e deve ser explorado no ambiente escolar, assim como a dimensão lúdica da linguagem e as suas possibilidades. A própria estrutura do poema já produz aprendizado: ela seduz e desafia o leitor, apresenta ritmos, efeitos sonoros e, ao mesmo tempo, apresenta novas vivências, oferecendo possibilidades para a criança simbolizar suas próprias experiências. Cada dupla de páginas do livro \textit{Pedrinho petrificado} apresenta composições de versos e ilustrações. Assim, a leitura da poesia se faz em paralelo com a observação de uma ilustração que sugere caminhos de sentido e interpretação à criança. A leitura do poema, realizada pelo educador, aumenta o repertório do aluno, incentiva o desenvolvimento do vocabulário e da fluidez do discurso. A associação entre a aquisição da linguagem e a poesia, ademais, permite explorar múltiplas competências ao mesmo tempo, pois relaciona os princípios linguísticos à linguagem poética, introduzindo o aluno no universo lúdico e artístico da poesia.

\section{Atividades}

\subsection{Pré-leitura}

\BNCC{EF01ER05} 
%Identificar +Reconhecer: ``acolher sentimentos, lembranças, memórias e saberes''; ``religião''
\BNCC{EF01HI01} 
%Produzir +Registrar: ``crescimento, lembranças particulares e/ou familiares''
\BNCC{EF02LP13}
%Produzir: bilhetes e cartas; ``Vida cotidiana''; 

\paragraph{Tema} Uma carta ao ``eu'' do início da pandemia;

\paragraph{Conteúdo} Exibição de vídeo sobre medos e desejos das crianças em relação à pandemia, debate sobre o assunto e redação de carta.

\paragraph{Objetivo} Acolher sentimentos e lembranças dos alunos sobre um momento difícil da história brasileira e oferecer um espaço para compartilhamento de medos, para que tenham uma percepção de suas superações ao longo do tempo.

\paragraph{Justificativa} A obra \textit{Pedrinho petrificado} trata de temas sensíveis ao apresentar um garoto com muitos medos e ansiedades. Essa atividade poderá ser um momento de troca sobre esses sentimentos. Também é abordado no livro a aflição de sair de casa e lidar com o mundo exterior --- sensações comuns na volta às aulas presenciais pós-pandemia. Dessa forma, é importante abordar essa questão com os alunos para que se sintam acolhidos e consigam reconhecer as mudanças em suas trajetórias.

\paragraph{Metodologia} Reúna os estudantes em roda e proponha uma conversa sobre nossos medos e a experiência com a pandemia da \textsc{covid}-19. Aborde a questão do medo de sair de casa e incentive os estudantes a relatarem como se sentiram no início da pandemia. Pergunte sobre o momento em que começaram a frequentar a escola presencialmente e quais foram as diferenças entre esses dois períodos.

\Image{Converse com os alunos sobre as diferenças do início da pandemia e da volta as aulas presenciais. (Igor Santos/Secom; CC BY-NC 2.0)}{PNLD2023-002-10.png}

Em seguida, sugere-se que seja apresentado o vídeo \textit{O mundo pós-pandemia para as crianças}, disponível gratuitamente no \href{https://youtu.be/clP9tvFbqyw}{Youtube}. É um vídeo de 3 minutos, em que diversas crianças relatam como foi sua experiência durante a quarentena e o que desejam para um mundo pós-pandemia. O vídeo apresenta ilustrações que acompanham as falas das crianças. Apresente o vídeo mais uma vez e peça que os estudantes, enquanto assistem, registrem quais falas se identificaram e quais não concordaram. 

\Image{Cada aluno vai escrever uma carta para seu ``eu'' do passado (Anneke Wolf ; CC BY-SA 2.5)}{PNLD2023-002-08.png}

A partir da conversa inicial e da exibição do vídeo, cada aluno vai escrever uma carta para seu ``eu'' do passado. A proposta é que contem para o seu ``eu'' do início da pandemia o que aconteceria a partir de então e como seria a volta para a escola. A ideia é que eles reconheçam a importância da superação dos medos ao enfrentar momentos desafiadores. A carta pode acompanhar desenhos, colagens e fotos. Acompanhe os alunos durante a redação do texto e aproveite para abordar o tema das lembranças particulares e familiares no reconhecimento de sua própria trajetória.

\paragraph{Tempo estimado} Duas aulas de 50 minutos.

\subsection{Leitura}

\BNCC{EF15LP16} 
%Ler: narrativas, contos, crônicas; +grupo, +professor e +sozinho; Mundo imaginário;
\BNCC{EF15LP04}
%Identificar: ``interpretação de imagem'', recursos gráfico-visuais, multissemióticos;
\BNCC{EF15LP11} 
%Identificar: +Falar: conversação; ``roda de conversa'', ``discussão'';

\paragraph{Tema} O enredo de \textit{Pedrinho petrificado}.

\paragraph{Conteúdo} Leitura conjunta em sala de aula, conversa sobre a compreensão de texto e sobre os recursos gráficos-visuais presentes no livro.

\paragraph{Objetivo} Incentivar os estudantes a identificar os diferentes momentos da narrativa e ampliar seu entendimento sobre as questões trazidas pela obra.

\paragraph{Justificativa} A \textit{Pedrinho petrificado} possui diversos detalhes narrativos e visuais que, ao analisados em conjunto, contribuem para uma maior compreensão da obra. Serão discutidos coletivamente os elementos da história pra que seja elaborada uma conclusão ao final da atividade.


\pagebreak
\paragraph{Metodologia} 

\begin{itemize}

\item Primeiramente, proponha uma leitura conjunta em sala de aula. Com auxílio do professor, os alunos que se voluntariarem deverão ler algumas páginas do livro e passar para o próximo voluntário. Como o principal tema do livro é o medo, pode ser interessante indagar para os estudantes ao longo da leitura sobre as possíveis ameaças que podem assustar Pedrinho.

\item Ao longo da leitura, é interessante abordar com os alunos as diferenças entre as disposições das frases em cada página. Quais os efeitos das linhas tortas e curvas? Por que algumas palavras aparecem maiores do que outras? Incentive que os estudantes façam suposições sobre esses recursos gráfico-visuais e seus efeitos narrativos.

\Image{Ilustração do livro, página 13}{PNLD2023-002-04.png}

\item Em seguida, peça para os alunos identificarem os diferentes momentos da história e resumirem as ideias principais do livro. A proposta é trabalhar a compreensão de leitura. Incentive que a conversa seja desenvolvida em grupo, para que todos possam participar. É interessante que cada estudante traga um elemento da narrativa para construírem um sentido comum coletivamente.

\item Converse com os alunos sobre o final do livro, a partir do seguinte trecho:

\begin{quote} {Por que Pedrinho estava rindo? 

Por que tamanha comoção?}
\end{quote}

\item Registrem coletivamente as respostas que surgirem para essas questões em uma cartolina e faça um mural com as hipóteses dos alunos.

\end{itemize}

\Image{Ilustração do livro, página 26}{PNLD2023-002-05.png}

\Image{Ilustração do livro, página 28}{PNLD2023-002-06.png}

\paragraph{Tempo estimado} Uma aula de 50 minutos.

\subsection{Pós-leitura}

\BNCC{EF15AR04} 
%Produzir: desenho, pintura, colagem, quadrinhos, dobradura, escultura, modelagem...;
\BNCC{EF02HI02}
%Produzir +Descrever: ``práticas e papéis sociais em diferentes comunidades
\BNCC{EF03GE02}
%Identificar:  marcas de contribuição cultural e econômica de grupos de diferentes origens;

\paragraph{Tema} Conhecer a vizinhança.

\paragraph{Conteúdo} Redação de um perfil imaginário de um vizinho, a partir de fatos reais. Conversa sobre diferentes costumes de cada bairro e também sobre possíveis preconceitos existentes entre moradores.

\paragraph{Objetivo} Aproximar os estudantes de um senso de comunidade no local em que vivem e incentivar um olhar imaginativo.

\paragraph{Justificativa} Será interessante utilizar elementos debatidos no livro \textit{Pedrinho petrificado} para propor aos alunos uma maior percepção sobre sua vizinhança e as características de seu bairro. Como a obra também trata do medo do desconhecido, essa atividade também proporcionará um debate sobre os preconceitos que podem estar presentes entre moradores do bairro.

\paragraph{Metodologia} Proponha para os estudantes que observem, nos arredores de sua casa, as características e peculiaridades de seus vizinhos. Podem caminhar, junto de seus familiares, pela rua em que vivem e reconhecer diferentes perfis de moradores do seu bairro. Outra opção pode ser realizar uma breve entrevista com as pessoas que moram sua casa e perguntar sobre os vizinhos que conhecem e registrar suas histórias.

\Image{Os alunos podem caminhar, junto de seus familiares, pela rua em que vivem e conhecer a vizinhança. (GLandovsky; CC-BY-SA-4.0)}{PNLD2023-002-09.png}

Em seguida, os alunos vão escolher um vizinho para escrever um perfil em sala de aula. Sugira que façam um retrato desenhado de seu vizinho. A partir de fatos conhecidos, podem imaginar e registrar seus gostos, suas manias e sua rotina. Acompanhe os estudantes ao longo desse registro. Alguns temas que podem servir de inspiração para os alunos podem ser:

\begin{itemize} 

\item O seu vizinho vive sozinho ou com outras pessoas?

\item O que ele costuma fazer no seu dia a dia?

\item Acorda cedo, dorme tarde? Como é a sua rotina?

\item Gosta de cozinhar? Qual sua comida preferida?

\item Convive com os outros vizinhos ou gosta de ficar mais fechado em casa como Pedrinho?

\item É parecido com Pedrinho ou passa mais tempo na rua?

\item Como é a sua casa por dentro? Parece com a sua?

\item Tem bichinhos de estimação? Como eles são?

\end{itemize}

Reúna os alunos em roda e peça que exibam os perfis criados no centro da sala. Peça para os estudantes apontarem quais as semelhanças e diferenças que surgiram entre os perfis dos vizinhos. É importante indicar que, mesmo que tenham feito os perfis a partir de observações próprias e informações recolhidas, parte dos registros foi criada e imaginada pelos alunos e não exprime a realidade.

A partir dessa reflexão, poderá ser abordado o tema do preconceito que pode surgir entre moradores do bairro, seja de cor, de idade, de gênero, origem ou condição econômica. A questão do respeito às diferenças deve ser ressaltado, para que seja incentivada uma visão comunitária de sua vizinhança.

\paragraph{Tempo estimado} Três aulas de 50 minutos.

\section{Sugestão de referências complementares}

\subsection{Filmes}

\begin{itemize}

\item \textit{Coraline e o mundo secreto}. Dirigido por Henry Selick, 2009.

A animação conta a história de Coraline, uma menina que se sente entediada em sua nova casa. De repente, ela encontra uma porta secreta que é um portal para uma versão melhorada de sua própria vida. Rapidamente ela percebe que esse mundo aparentemente perfeito torna-se assustador.

\item \textit{Divertidamente}. Dirigido por Pete Docter, 2015.

O filme se passa na mente de uma menina, Riley Andersen, onde cinco emoções — Alegria, Tristeza, Medo, Raiva e Nojinho — tentam conduzir sua vida quando ela se muda com seus pais para uma nova cidade. 

\item \textit{Monstros S.A.}. Dirigido por Pete Docter, 2001.

A história se passa em Monstrópolis, uma cidade habitada por monstros, que é alimentada pela energia dos gritos de crianças humanas. Os monstros trabalham na fábrica da \textit{Monsters, Inc.}, em que precisam assustar as crianças e colher seus gritos, através de portas que ativam portais para os armários dos quartos delas. 

\item \textit{O menino e o mundo}. Dirigido por Alê Abreu, 2013.

O filme é uma animação emocionante, que conta a história de Cuca, um menino que vive em um mundo distante, numa pequena aldeia no interior de seu mítico país. Sofrendo com a falta do pai, que parte em busca de trabalho na desconhecida capital, Cuca deixa sua aldeia e sai mundo afora a procura dele. Durante sua jornada, Cuca descobre uma sociedade marcada pela pobreza, exploração de trabalhadores e falta de perspectivas.

\end{itemize}

\subsection{Livros}

\begin{itemize}

\item \textit{Chapeuzinho amarelo}, Chico Buarque. Autêntica, 2017.

Ilustrado por Ziraldo, este livro é um clássico reinventado, em que Chapeuzinho precisa superar o medo do lobo e conseguir se divertir com outras crianças.

\item \textit{Vizinho, vizinha}, Roger Mello. Companhia das Letras, 2002.

O livro conta a história de dois vizinhos que vivem no mesmo prédio em uma grande metrópole.

\item \textit{Quem tem medo do quê?}, Ruth Rocha. Salamandra, 2012.

Neste livro, Ruth Rocha lista diversos medos existentes entre crianças através de uma linguagem humorística.

\end{itemize}

\subsection{Músicas}

\begin{itemize}

\item \textit{Medo da chuva}, de Raul Seixas. Disponível no \href{https://www.youtube.com/watch?v=SQAozIai-IA&ab_channel=marciofernandes}{Youtube}. 

Nesta canção, parceria entre Raul Seixas e Paulo Coelho, o medo da chuva é uma metáfora para a impermanência da vida e a vontade de caminhar de acordo com o fluxo natural das coisas.

\item \textit{Sem medo}, de Mahmundi. Disponível no \href{https://www.youtube.com/watch?v=poHe_QYAZ4U&ab_channel=MahmundiVEVO}{Youtube}. 

A canção de Mahmundi traz, através de uma letra cheia de poesia, imagens que relacionam o vento, o sol, o tempo e a chuva para encararmos os aprendizados da vida sem medo.

\end{itemize}

\subsection{Bibliografia comentada}

\begin{itemize}
\item \textsc{brasil}. Ministério da Educação. Base Nacional Comum Curricular. Brasília, 2018.

Consultar a \textsc{BNCC} é essencial para criar atividades para a turma. Além de especificar 
quais habilidades precisam ser desenvolvidas em cada ano, é fonte de informações sobre 
o processo de aprendizagem infantil. 
 
\item \textsc{brasil}. Ministério da Educação. Secretaria de Alfabetização. PNA Política Nacional de Alfabetização/Secretaria 
de Alfabetização. Brasília: \textsc{mec, sealf}, 2019.

Um guia fundamental para trabalhar a alfabetização de estudantes, que ressalta a importância da Literacia e da Numeracia. 

\item \textsc{van der linden}, Sophie. Para ler o livro ilustrado. São Paulo: Cosac Naify, 2011.

Livro sobre as particularidades do livro ilustrado, que apresenta as diferenças entre o livro ilustrado e o livro com ilustração. 
\end{itemize}

\end{document}