\documentclass[11pt]{extarticle}
\usepackage{manualdoprofessor}
\usepackage{fichatecnica}
\usepackage{lipsum,media9}
\usepackage[justification=raggedright]{caption}
\usepackage[one]{bncc}
\usepackage[araucaria]{../edlab}
\usepackage{marginnote}
\usepackage{pdfpages}

\newcommand{\AutorLivro}{Eliane Camargo}
\newcommand{\TituloLivro}{A mulher que virou tatu}
%\newcommand{\Tema}{}
\newcommand{\Genero}{Lendas; mitos; fábula}
%\newcommand{\imagemCapa}{./images/PNLD2022-001-01.jpeg}
\newcommand{\issnppub}{XXX-XX-XXXXX-XX-X}
\newcommand{\issnepub}{XXX-XX-XXXXX-XX-X}
% \newcommand{\fichacatalografica}{PNLD0001-00.png}
\newcommand{\colaborador}{Ana Lancman}
\begin{document}

\title{\TituloLivro}
\author{\AutorLivro}
\def\authornotes{\colaborador}

\date{}
\maketitle

%\begin{abstract}\addcontentsline{toc}{section}{Carta ao professor}
%\pagebreak

\tableofcontents

\pagebreak

\begin{abstract}

O presente manual tem como objetivo oferecer uma orientação ao professor sobre a obra \textit{A mulher que virou tatu}. A partir deste manual, os professores poderão incentivar a prática da leitura aos estudantes e proporcionar um conteúdo enriquecedor. Apresentamos aqui sugestões de atividades a serem realizadas antes, durante e após a leitura do livro, com propostas que buscam introduzir os gêneros literários e aprofundar as discussões trazidas pelas obras. Você encontrará informações sobre o autor, sobre o gênero e sobre os temas trabalhados ao longo do livro. Ao fim do manual, você encontrará também sugestões de livros, artigos e sites selecionados para enriquecer a sua experiência de leitura e, consequentemente, a de seus estudantes.

Essa obra, organizada pela linguista Eliane Camargo, é uma narrativa tradicional indígena, originária da etnia Caxinauá. Este grupo constitui a mais numerosa população indígena do Acre. \textit{A mulher que virou tatu} apresenta a criação do mito da domesticação da batata doce pelo tatu. O livro conta a história de uma velha que não consegue comer milho seco por ser desdentada. Ela deseja comer milho verde, alimento que conseguiria comer por ser macio. Porém, sua família lhe dá batata doce, uma vez que é mais fácil de produzir. A velha insiste que quer comer o milho verde e sua filha lhe permite. Logo a senhora é criticada em sua aldeia por ter comido toda a reserva de milho. Triste pela situação, a velha parte de sua casa decidida a virar tatu. 

A partir desta obra, será possível aproximar os estudantes da culturas indígenas dos Caxinauá. Com a presença de professores de História e Geografia, as atividades proporcionarão um conteúdo interdisciplinar e de extrema importância para a celebração da diversidade no Brasil. Esperamos que as atividades sugeridas e o material indicado sejam proveitosos em sala de aula! 

\end{abstract}

\section{Sobre o livro} 

\textit{A mulher que virou tatu} foi organizado pela linguista Eliane Camargo. Trata-se de uma pequena seleção de um grande grupo de narrativas do povo Caxinauá. 
A narrativa escolhida acompanha ilustrações feitas por Anita Ekman, que usou materiais como o carimbo para registrar grafismos indígenas nas páginas. Além da narrativa propriamente dita, o livro contém esclarecimentos necessários ao leitor, com explicações sobre a origem da obra e informações sobre os Caxinauá.

\Image{Aldeia caxinauá no Acre (Agência de Notícias do Acre; CC BY 2.5)}{PNLD2023-010-08.png}

Os quase oito mil Caxinauá fazem parte da família linguística pano, composta por cerca de trinta grupos. Seus territórios estão na fronteira entre o Brasil e o Peru. No Brasil, eles vivem em doze terras indígenas e, no Peru, eles ocupam todo o rio Curanja e uma parte do rio Purus --- da cidade de Puerto Esperanza até a embocadura do rio Curanja. 

A escrita dessa narrativa oral começou, no entanto, a ser colhida e fixada muito. O historiador (1853--1927) trabalhou com dois jovens Caxinauás no início do século \textsc{xx}, que trabalharam com ele no registro de sua língua e de seu modo de vida. E este encontro deu origem ao livro \textit{Hantxa Huni Kuin --- A língua dos Caxinauás do rio Ibuaçu, afluente do Muru}, publicado em 1914 e que deu início ao estudo mais aprofundado das culturas Caxinauá. 

\section{Sobre a organizadora e a ilustradora}

\paragraph{A organizadora} Eliane Camargo fez a graduação e o mestrado na Universidade de Paris \textsc{iii}. Seu doutorado foi concluído em 1986 na Universidade de Paris \textsc{iv}, em Linguística hispânica. O pós-doutorado foi realizado no Departamento de Antropologia Social da Universidade de São Paulo (\textsc{usp}). Eliane é membro do Centro de Estudos de Línguas Indígenas da América (\textsc{celia}) e do Laboratório de Etnologia e Sociologia Comparada. Entre 2006 e 2009, coordenou o projeto de ``Documentação da língua e cultura caxinauá'' junto do antropólogo Philippe Erikson.

\SideImage{A organizadora Eliane Camargo (Arquivo pessoal}{PNLD2023-010-02.png}

\paragraph{A ilustradora} Anita Ekman é artista visual e performer. Ela pesquisa artes ameríndias e afro-brasileiras e trabalhou na coleção ``Grandes mestres de arte popular ibero-americana''. Ilustrou diversos livros da coleção Mundo Indígena, de qual \textit{A mulher que virou tatu} faz parte. Atualmente está desenvolvendo um projeto chamado DescolonizARTE.

\Image{Família caxinauá (Vihh; CC BY-NC-ND 2.0)}{PNLD2023-010-10.png}

\paragraph{O gênero} O gênero deste livro é \textit{Lendas; mitos; fábula}. 

\paragraph{Descrição} A lenda e o mito são narrativas com elementos de fantasia transmitidas pela tradição oral através dos tempos. De caráter fantástico, as lendas e os mitos combinam fatos reais e históricos com fatos que não têm comprovação de acontecimento, a não ser pela palavra dos que sobraram para contar a história. As lendas e mitos de uma sociedade são fundamentais para que entendamos quem são essas pessoas e no que acreditam, bem como suas tradições. Uma lenda é verdadeira até que se prove o contrário. Com exemplos bem definidos em todos os países do mundo, as lendas e os mitos de um povo geralmente fornecem explicações plausíveis, e até certo ponto aceitáveis, para coisas que não têm explicações científicas comprovadas, como acontecimentos misteriosos ou sobrenaturais.

\Image{A lenda e o mito são narrativas com elementos de fantasia transmitidas pela tradição oral através dos tempos. (The ponta cabeça; CC-BY-SA-4.0)}{PNLD2023-010-07.png}

A fábula é uma narrativa curta em que os personagens principais geralmente são animais personificados. Estes animais apresentam características humanas, tais como a fala e traços de personalidade. Essas personagens podem ser também objetos animados ou deuses. Em cada história há uma ``lição de moral'': uma mensagem de cunho educativo que busca conscientizar o leitor. A fábula veio do conto, mas se diferencia pela centralidade dos personagens animais e pelo intuito de concluir a história com um ensinamento. É uma história que pode ser contada em prosa ou em versos. 

\section{Atividades}

\subsection{Pré-leitura}

\BNCC{EF04GE06}
%Identificar: territórios étnico-culturais existentes no Brasil;
\BNCC{EF04HI02}
%Identificar +Observar: “significado dos marcos da história da humanidade”; “nomadismo, agricultura, etc”
\BNCC{EF05HI08}
% Identificar +Relacionar: ``formas de marcação da passagem do tempo em distintas sociedades''
\BNCC{EF05GE02}
%Identificar: diferenças e desigualdades étnico-culturais;

\paragraph{Tema} A etnia Caxinauá.

\paragraph{Conteúdo} Pesquisa sobre os Caxinauá e apresentação em sala de aula.

\paragraph{Objetivo} Introduzir aos estudantes elementos essenciais sobre as culturas Caxinauá.

\paragraph{Justificativa} Como se trata de uma narrativa indígena Caxinauá, será importante que os alunos tenham um conhecimento básico sobre essa etnia para uma maior compreensão da obra. Além disso, essa pesquisa aproximará os estudantes do combate ao preconceito e defesa da diversidade no país.

\Image{Região em que vivem os caxinauá, na bacia amazônica em torno do Rio Purus. (Kmusser; CC BY-SA 3.0)}{PNLD2023-010-09.png}

\paragraph{Metodologia} Divida a turma em grupos pequenos. Cada grupo será responsável por pesquisar um tema referente à etnia Caxinauá. Os grupos poderão montar um painel informativo sobre sua pesquisa, que deverá conter informações básicas sobre os Caxinauá e os dados encontrados sobre o tema principal. O \href{https://pib.socioambiental.org/pt/Povo:Huni_Kuin_(Kaxinaw\%C3\%A1)}{site dos \textit{Povos indígenas no Brasil}}, organizado pelo Instituto Socio-ambiental, é uma fonte confiável para a pesquisa. Cada painel poderá conter fotos e mapas que apresentem sua localização no Brasil e no mundo. Alguns dos tópicos sugeridos para os grupos são:

\begin{itemize}

\item Alimentação e práticas de plantio.

\item Organização social.

\item Histórico do encontro dos Caxinauá com não-indígenas.

\item Arte e artesanato Caxinauá.

\item Rituais.

\end{itemize}

\Image{Cinta caxinauá com 840 dentes de macaco em exibição no Museu Estadunidense de História Natural, em Nova Iorque, nos Estados Unidos (Daderot ; CC-0)}{PNLD2023-010-06.png}

O professor deverá acompanhar os grupos e poderá convidar os professores de História e Geografia para a atividade. Por fim, cada grupo apresentará o seu painel para o resto da sala, compartilhando o conteúdo pesquisado.

\paragraph{Tempo estimado} Duas aulas de 50 minutos.

\subsection{Leitura}

\BNCC{EF15LP16} 
%Ler: narrativas, contos, crônicas; +grupo, +professor e +sozinho; Mundo imaginário;
\BNCC{EF15LP04}
%Identificar: ``interpretação de imagem'', recursos gráfico-visuais, multissemióticos;
\BNCC{EF15LP11} 
%Identificar: +Falar: conversação; ``roda de conversa'', ``discussão'';
\BNCC{EF05ER05}
%Identificar: “tradição oral nas culturas indígenas, afro-brasileiras, ciganas”; “religião”

\paragraph{Tema} O enredo de \textit{A mulher que virou tatu}.

\paragraph{Conteúdo} Leitura conjunta de \textit{A mulher que virou tatu} e conversa acerca dos elementos fundamentais da narrativa.

\paragraph{Objetivo} Identificar os aspectos centrais do enredo e incentivar que os alunos expressem verbalmente o que absorveram da leitura.

\Image{Ilustração do livro, página 5}{PNLD2023-010-03.png}

\paragraph{Justificativa} Ao realizar uma leitura acompanhada, será possível trabalhar a compreensão do texto de forma detalhada e discutir com profundidade os temas que aparecem ao longo da narrativa. Serão discutidos coletivamente os elementos da história pra que seja elaborada uma conclusão ao final da atividade.

\paragraph{Metodologia} Faça uma leitura coletiva em sala de aula. Chame alunos voluntários para lerem uma página e passarem para o próximo. É possível que haja uma pausa para contemplar as ilustrações, que são recheadas de detalhes a serem observados. Sugere-se que, ao ser finalizada essa primeira leitura, seja realizada uma nova leitura, dessa vez individual. Para esse segundo momento, o professor pode propor aos estudantes que anotem quais passagens foram mais marcantes para eles. Também peça que prestem atenção em algumas questões, como as personagens que aparecem e quais os distintos momentos e espaços da narrativa.

\Image{Ilustração do livro, página 7}{PNLD2023-010-04.png}

Em uma roda, proponha que os alunos contem como entenderam a história. Poderá ser feita uma conversa livre, em que os estudantes poderão expressar quais sensações tiveram ao longo da leitura. 

Sugestões de questões a serem feitas para fomentar o debate:

\begin{itemize}

\item O que sabemos sobre a velha logo de início?

\item Qual o principal alimento de sua família?

\item Qual a diferença para a velha entre comer o milho seco e o milho verde?

\item O que podemos entender acerca do trabalho no milharal para essa família?

\item O que pode significar a metáfora de ``cavar um buraco'', quando a velha quer virar tatu?

\item Quais são as características do tatu que fazem a mulher querer ser um?

\item Quais as mensagens que o livro traz?

\end{itemize}

\Image{Ilustração do livro, página 27}{PNLD2023-010-05.png}

\paragraph{Tempo estimado} Duas aulas de 50 minutos.

\subsection{Pós-leitura}

\BNCC{EF04LP13}
% Identificar: formatação em textos instrucionais;
\BNCC{EF04GE01}
% Identificar: histórias familiares da comunidade com elementos indígenas, afro-brasileiras
\BNCC{EF05CI08}
% Produzir: ``ardápio equilibrado com base nos grupos alimentares'' 

\paragraph{Tema} A influência indígena na cozinha brasileira.

\paragraph{Conteúdo} Escolha de uma receita familiar e pesquisa acerca das heranças indígenas nos alimentos.

\paragraph{Objetivo} Reconhecer elementos indígenas nas histórias familiares através dos alimentos do cotidiano e valorizar um cardápio equilibrado a partir de ingredientes provenientes da natureza.

\paragraph{Justificativa} Um dos temas do livro são os alimentos básicos das culturas Caxinauá, como a batata doce e o milho. Nessa atividade, será possível relacionar alimentos do cotidiano dos estudantes com suas origens históricas e redigir um texto instrucional em formato de receita.

\paragraph{Metodologia} Primeiramente, proponha aos alunos que peçam para seus familiares receitas típicas que gostam de cozinhar. Pode ser interessante que investiguem a origem dessa receita na família: como chegou até sua casa, se foi passada de geração em geração. Os estudantes deverão coletar o passo a passo dessa receita e redigir em um formato instrucional.

Com as receitas em mãos, em sala de aula, proponha uma conversa com os alunos a respeito. Eles poderão compartilhar o que descobriram acerca da receita, se esta traz consigo uma experiência familiar marcante. Pode ser interessante também abordar a questão da herança familiar que é passada através das receitas.

Em seguida, será realizada uma pesquisa sobre a origem dos ingredientes e dos métodos culinários presentes na receita. A pesquisa poderá ser realizada na internet e em livros disponíveis na escola sobre o assunto. Os alunos poderão investigar se os ingredientes utilizados e seus métodos de preparo têm origens indígenas. Também podem ser exploradas as regiões em que esses alimentos são produzidos e como são transportados pelo país.

Por fim, sugere-se que seja consultado coletivamente o \href{https://bvsms.saude.gov.br/bvs/publicacoes/guia_alimentar_populacao_brasileira_2ed.pdf}{\textit{Guia alimentar para a população brasileira}}, organizado pelo Ministério da Saúde. Os alunos poderão averiguar em quais categorias se encaixam os ingredientes utilizados na receita. Também é importante analisar se os alimentos escolhidos se encaixam nos parâmetros do guia para uma alimentação saudável e equilibrada.

\paragraph{Tempo estimado} Duas aulas de 50 minutos.

\section{Sugestões de referências complementares.}

\subsection{Filmes}

\begin{itemize}

\item \textit{O abraço da Serpente}. Direção: Ciro Guerra. (Colômbia, 2016).

Situado durante o período da Febra da Borracha, em 1909 e 1940, o filme retrata dois momentos da história de Karamakate, um xamã amazônico e último sobrevivente de seu povo, e sua experiência de contato com dois cientistas, o alemão Theodor Koch-Grünberg e o americano Richard Evans Schultes, que vão até ele em busca da yakruna, uma planta sagrada. 

\item \textit{A febre}. Direção: Maya Da-Rin (Brasil, 2019).

Em Manaus, no Amazonas, o filme \emph{A febre} acompanha a personagem Justino, um indígena que há bastante tempo vive na cidade, trabalhando como segurança no porto local. A filha de Justino, Vanessa, trabalha em um posto de saúde e consegue ingressar na Faculdade de Medicina da Universidade de Brasília. A trama gira em torno do impasse de Vanessa entre seguir seu sonho e abandonar seu pai, além do surgimento de uma febra estranha que aparece no local, paralelamente a uma série de ataques a animais.

\item \textit{Chuva é cantoria na aldeia dos mortos}. Direção: Renée Nader. (Brasil, 2019).

O filme conta a história de Ihjãc, um jovem da etnia Krahô, que mora na aldeia Pedra Branca, no Tocantins. Após a morte de seu pai, recusando a ideia de se tornar um xamã, ele busca refúgio na cidade. Distante de seu povo e da sua cultura, com uma fluidez 
entre momentos ficcionais e documentais, Ihjãc passa a enfrentar as dificuldades de ser um indígena habitando os espaços das cidades no Brasil contemporâneo.

\item \textit{Ex-Pajé}. Direção: Luiz Bolognesi. (Brasil, 2018).

O filme conta a história do povo Paiter Suruí, da terra indígena Sete de Setembro, em Rondônia. O protagonista, Perpera, tinha 20 anos quando seu povo teve o primeiro contato com não-indígenas, em 1969. Com o contato, veio também um pastor evangélico que condenava o xamanismo, e Perpera, que era o xamã de seu povo, se vê obrigado a abandonar sua ancestralidade.

\item \textit{A História dos Caxinauás Por Eles Mesmos}. (Brasil, 2014).

Vídeo organizado pelo SESC em que Eliane Camargo, organizadora de \textit{A mulher que virou tatu}, conta como se aproximou dos Caxinauá. Disponível no \href{https://youtu.be/80Wip9XFtMU}{Youtube}.

\end{itemize}

\subsection{\emph{Sites}}

\begin{itemize}

\item \textit{Comissão Pró-Índio}

No site da ONG, é possível conhecer os trabalhos realizados junto de comunidades indígenas e quilombolas, assim como um extenso banco de dados sobre os processos de homologação de suas terras. Acesse em \url{https://cpisp.org.br/}.
\item \textsc{funai}

O \href{http://www.gov.br/funai}{site} da Fundação Nacional do Índio é um importante banco de dados sobre povos e terras indígenas. Também informa sobre meio ambiente e direitos sociais.

\item Museu do Índio

O \href{http://www.museudoindio.gov.br/}{museu} possui um acervo com milhares de peças, além de biblioteca, galeria de arte e espaços agradáveis para receber os visitantes. Fica no bairro do Botafogo, no Rio de Janeiro.

\item Instituto Socioambiental -- \textsc{isa}

O \href{https://www.socioambiental.org/pt-br}{site} do Instituto Socioambiental -– \textsc{isa} é outro importante banco de dados 
sobre povos e terras indígenas, fornecendo monitoramento e proposições de alternativas às políticas públicas, a fim de se assegurar a defesa dos direitos socioambientais. 

\end{itemize}

\section{Bibliografia comentada}

\begin{itemize}

\item \textsc{carvalho}, Bernardo. \textit{Nove noites}. São Paulo: Companhia de Bolso, 2006.

Fruto de profunda pesquisa, o livro narra a história do antropólogo americano Buell Quain, que se matou em 1939, aos 27 anos, enquanto tentava voltar para a civilização, vindo de uma aldeia indígena no interior do Brasil.

\item \textsc{cascudo}, Luís da Câmara. \textit{História da Alimentação no Brasil}. São Paulo: Global, 2011.

O livro apresenta um estudo completo sobre as origens da comida brasileira, a partir da sociologia da alimentação.

\item \textsc{chatwin}, Bruce. \textit{O rastro dos cantos}. São Paulo: Companhia das Letras, 1996.

O escritor vai atrás do rastro dos cantos, ligado aos mitos de aborígenes da Austrália Central sobre seres lendários que atravessaram o continente no tempo da criação, cantando o que viam e dando existência ao mundo através do canto.

\item \textsc{costa}, da. \textit{Os Melhores Contos da América Latina}. Rio de Janeiro: Agir, 2008.

O livro é uma grande antologia de contos da América Latina, reunindo textos de todos os países da região, das mais diversas épocas e escolas literárias.

\item \textsc{eliade}, Mircea. \textit{Cosmos e história: o mito do eterno retorno}. São Paulo: Mercúryo, 2004.

Este trabalho fundador da história das religiões aborda as expressões e atividades de uma grande variedade de culturas religiosas arcaicas e ``primitivas''.

\item \textsc{fernandes}, Caloca. \textit{Viagem Gastronômica Através do Brasil}. São Paulo: Senac, 2001.

Importante documento que retrata a história da cozinha brasileira a partir de suas origens indígenas, africanas e portuguesas.

\item \textsc{kopenawa}, Davi; \textsc{albert}, Bruce. \textit{A queda do céu: palavras de um
xamã Yanomami}. São Paulo: Companhia das Letras, 2015.

Xamã e porta-voz dos Yanomami, o autor dá seu testemunho autobiográfico neste volume, que é também um manifesto xamânico e libelo contra a destruição da Amazônia.

\item \textsc{krenak}, Ailton. \textit{Ideias para adiar o fim do mundo}. São Paulo:
Companhia das Letras, 2019.

O líder indígena critica a ideia de humanidade como algo separado da natureza e recusa a ideia do humano como superior aos demais seres.

\item \textsc{lévi-strauss}, Claude. \textit{Tristes trópicos}. São Paulo: Companhia das
Letras, 1996.

Com um texto que se posiciona entre o ensaio e a narrativa de viagem, o renomado antropólogo desloca parâmetros consagrados, questionando viajantes e cientistas.

\item \textsc{munduruku}, Daniel. \textit{Contos indígenas brasileiros}. São Paulo:
Global Editora, 2004.

Os oito contos selecionados pelo autor, a partir de um critério linguístico, retratam através de seus mitos a caminhada de povos
indígenas de norte a sul do Brasil.

\item \textsc{storto}, Luciana. \textit{Línguas indígenas: tradição, universais e
diversidade}. Campinas: Mercado de Letras, 2019.

A autora traça um painel das línguas indígenas atualmente faladas no Brasil, unindo reflexões da Antropologia e da Linguística.

\item \textsc{werá}, Kaká. \textit{A terra dos mil povos: história indígena do Brasil}.
São Paulo: Editora Peirópolis, 2020.

Nesta obra, diversos antropólogos se debruçam sobre a questão de quem eram e como pensavam os primeiros habitantes do Brasil.

\end{itemize}

\end{document}
