\documentclass[11pt]{extarticle}
\usepackage{manualdoprofessor}
\usepackage{fichatecnica}
\usepackage{lipsum,media9}
\usepackage[justification=raggedright]{caption}
\usepackage[one]{bncc}
\usepackage[acorde]{../edlab}
\usepackage{marginnote}
\usepackage{pdfpages}

\newcommand{\AutorLivro}{Klévisson Viana}
\newcommand{\TituloLivro}{Meu baú de cordeis}
\newcommand{\Genero}{Cordel}
%\newcommand{\imagemCapa}{./pdf/capa.jpg}
\newcommand{\issnppub}{XXX-XX-XXXXX-XX-X}
\newcommand{\issnepub}{XXX-XX-XXXXX-XX-X}
% \newcommand{\fichacatalografica}{PNLD0001-00.png}
\newcommand{\colaborador}{Renier Silva}

\begin{document}

\title{\TituloLivro}
\author{\AutorLivro}
\def\authornotes{\colaborador}

\date{}
\maketitle

%\begin{abstract}\addcontentsline{toc}{section}{Carta ao professor}
%\pagebreak

\tableofcontents


\begin{abstract}

Caros professores,\medskip

Gostaríamos de apresentar a obra \textit{Meu baú de cordeis}, do autor, editor,
tipógrafo cearense Klévisson Vianna, autor conhecido pelo manejo dos gêneros verbo-visuais
do cordel e das histórias em quadrinhos.

Neste livro, você encontrará uma gama de possibilidades para trabalhar em sala de 
aula. O primeiro poema do livro, o ``\textsc{abc} do cordel'', apresenta, 
em forma de acróstico,\footnote{Composição poética em que cada verso principia por uma das letras da palavra que lhe serve de tema; Tipo de texto em que as primeiras letras de cada linha ou parágrafo formam verticalmente uma ou mais palavras.} as principais características do gênero cordelístico,
além de seus principais autores, as diferenças em relação a outros ``gêneros irmãos'',
como o repente, dentre outras informações, sempre guiadas por uma métrica e ritmo
definidos, características do gênero mas impecavelmente executados pelo poeta. 
Destacamos uma atividade de leitura inicial que contemple este poema, afim de
usar didaticamente o texto do próprio autor. 

Antes de iniciar a leitura dos poemas em si, que deverá ser oral e em conjunto,
para fazer jus ao gênero, propomos que os alunos sejam sensibilizados 
ou que retomem um conhecimento prévio das expressões da cultura oral,
que em uns deve aparecer de formas diferentes em decorrência de suas origens 
familiares --- para aqueles que provêm de regiões mais rurais,
o tipo de oralidade deve se aproximar mais da canção tradicional,
já para os que provêm das áreas urbanas, outras formas de linguagem
oral pode ser mais predominante, como as expressões do \textit{hip hop},
por exemplo. 

Por fim, propomos que, após a leitura do livro em sala de aula, 
o professor ou a professora instigue os alunos a \textbf{se apropriar da linguagem},
tanto visual, com as ilustrações, quanto verbal, com os versos em si. 
Ao fim do trabalho com o livro, esperamos que os alunos percebam a proximidade
destes gêneros verbo-visuais da vida quotidiana e a possibilidade de 
serem utilizados não só pelos grandes poetas consagrados, mas também por eles!

Esperamos, professor ou professora, que este material sirva como um guia 
para seu trabalho em sala de aula. Já contamos, no entanto, com as adaptações
que surgirão organicamente na recepção do mesmo por vocês, que possuem 
trajetórias e escolhas didáticas específicas, bem como no contato com os 
alunos, que tanto têm a oferecer para o enriquecimento da experiência didática. 

Boa aula!
\end{abstract}

\section{Sobre o livro}

%550 caracteres
\paragraph{O livro} \textit{Meu baú de cordéis} é uma reunião de poemas escritos por Klévisson Viana. 
Nesta coletânea poética, Klévisson mostra um leque variado de poemas na 
tradição cordelística, desde biografias até pequenas anedotas engraçadas, desde 
reflexões sobre a vida até fantasias sobre mundos maravilhosos. Klévisson revisita 
o “País de São Saruê” criado por Manuel Camilo dos Santos;\footnote{
Manuel Camilo dos Santos (Guarabira, 9 de junho de 1905 — Rio de Janeiro, 9 de abril de 1987) foi um escritor, poeta popular, violeiro, repentista, horoscopista, comerciante e compositor brasileiro.
Atuou como ambulante, marceneiro e, em 1942, fundou, em Guarabira, uma pequena tipografia de nome ``Tipografia e Folhetaria Santos'', transferida mais tarde para a cidade de Campina Grande em 1957, que imprimia  folhetos no gênero literatura de cordel. \url{https://pt.wikipedia.org/wiki/Manuel_Camilo_dos_Santos}. }
reconta histórias impagáveis 
de como Lampião tirou um espinho de um pé alheio ou um cavalo que bebeu cerveja num 
balcão; descreve de maneira emocionada a chegada no céu de seu irmão, o grande e 
saudoso Arievaldo Vianna; conta aventuras fantásticas que vão desde o sertão nordestino 
às cidades mexicanas.

\section{Sobre o autor}

\paragraph{Klévisson Viana}

Klévisson Viana (Antônio Clévisson Viana Lima) é escritor, cordelista, roteirista, cartunista, xilogravador, editor e presidente da \textsc{aestrofe} --- Associação de Escritores, Trovadores e Folheteiros do Estado do Ceará. É também membro da \textsc{ablc} -- Academia Brasileira de Literatura de Cordel (\textsc{rj}). Coordena o projeto editorial da Tupynanquim Editora, onde já publicou cerca de mil obras de quase uma centena de autores. 

Como autor, Klévisson Viana publicou mais de 30 livros e quase 200 folhetos de Literatura de Cordel. Seus trabalhos fluíram pelos quadrinhos, pela televisão e por adaptações para o teatro. Destaca-se o folheto \textit{A quenga e o delegado}, transformado em episódio da série Brava Gente da Rede Globo. Tem trabalhos publicados em diversas editoras nacionais e internacionais como Chandeigne --- Paris (\textsc{fr}), Editora Leya --- Lisboa (\textsc{pt}), Editora Hedra --- São Paulo (\textsc{br}), Nova Alexandria --- São Paulo (\textsc{br}), Editora Demócrito Rocha --- Ceará (\textsc{br}), Editora Amarilys --- São Paulo (\textsc{br}), Edelbra --- Porto Alegre (\textsc{br}), Nova Alexandria --- São Paulo (\textsc{br}), dentre outras. Tem outras obras publicadas em antologias na Turquia, Israel, Bélgica, Itália e Holanda. 

Dentre sua extensa obra podemos encontrar os livros \textit{Sertão menino}, de 2008, \textit{Abecedário dos bichos}, de 2013, \textit{O Guarani em cordel}, de 2013, e \textit{Miolo da rapadura}, de 2017; os álbuns em quadrinhos \textit{O mundo do Cajulino}, de 1993, \textit{Lampião... Era o cavalo do tempo atrás da besta da vida}, de 1999, \textit{Admirável riso novo}, de 2004, e \textit{O cangaceiro do futuro e o jumento espacial}, de 2017; e os folhetos de cordel \textit{A chegada de Ariano Suassuna no céu}, \textit{Carta de um jumento a Jô Soares}, \textit{Cinco anos do São Paulo capital do Nordeste}, \textit{A triste partida de Patativa do Assaré}, \textit{O cordelista na França}, e \textit{Seu Lunga --- o homem mais zangado do mundo, volumes \textsc{i}, \textsc{ii} e \textsc{iii}}.

Seu currículo consta de diversos prêmios importantes. Foi vencedor seis vezes consecutivas do \textsc{pnbe} --- Programa Nacional da Biblioteca Escolar (\textsc{mec}), três vezes do Troféu \textsc{hq} Mix, uma vez do \textsc{pnaic} --- Programa Nacional de Alfabetização na Idade Certa (\textsc{mec}) e "Prêmio Jabuti de Literatura" concedido anualmente pela Câmara Brasileira do Livro (\textsc{cbl}), dentre outros. 

Klévisson Viana coordena eventos culturais, ministra palestras, oficinas e recitais em todo o Brasil e já levou sua arte a países como França, Portugal, México, Cabo Verde e Costa Rica. 

\SideImage{Edição francesa de um livro em cordel de Klévisson Viana.}{PNLD2023_022_04.jpg}

\section{Sobre o gênero ``cordel''}

Alguns estudiosos defendem que o termo \textit{cordel} venha de Portugal, onde os \textit{folhetos} 
eram vendidos em feiras pendurados em barbantes, em cordões que se chamavam cordéis. Já para
outros, o cordel era assim chamado porque as brochuras eram encadernados com barbantes. 
No Brasil, porém, não se falava em cordel. Somente a partir dos anos 1960, com a persistência 
dos pesquisadores europeus pelo nome, os poetas passaram a ser chamados de 
cordelistas. Para o público mais popular, no entanto, ele continua sendo chamado de \textit{romance} e 
\textit{folheto}.

Chama-se cordel as histórias curtas em versos rimados de personagens 
lendárias impressas em cadernos, geralmente artesanais, com ilustrações feitas sob a técnica da 
xilogravura, e comercializadas originalmente em feiras livres do Nordeste do Brasil. 
Sua origem remete às cantigas portuguesas medievais trazidas pelos colonos.
O cordel não tem nem um limite nem uma receita pronta. É o verso da 
nossa tradição popular brasileira. Hoje, o gênero sente-se à vontade para falar de qualquer 
assunto, abordar qualquer temática e reflete sobre o mundo do nosso tempo. Como diz o próprio
autor: 

\begin{quote}
``Não há um só grande acontecimento local, nacional, ou mesmo mundial que não tenha sido tratado 
pela literatura de cordel. O folheto mostra a realidade, mais do que os grandes meios de comunicação, 
porque não é atrelado a coisa alguma. É independente e é a opinião do autor. Não tem interesse em 
grupos econômicos, nem tem patrocinadores. Por isso, critica e aborda, como nenhum outro meio. 
Sendo honesto em suas abordagens, é natural que o cordel se sinta ameaçado --- da mesma forma que 
a televisão e o rádio ameaçaram o jornal impresso.''\footnote{``Klévisson Viana --- Cordel para os intelectuais e folheto para o povo.'' Entrevista para o jornal \textit{A nova democracia, ano \textsc{i}, nº 8, abril de 2003.}}
\end{quote}

O cordel aqui em questão carrega influências de outro gênero literário, também
verbo-visual: a história em quadrinhos, ou \textsc{hq}. A \textsc{hq} é um 
gênero que trabalha ao mesmo tempo a linguagem verbal e a visual, portanto trata-se de 
uma \textbf{narrativa gráfica}. Não há uma hierarquia entre o texto e a ilustração: nem 
o texto é mera legenda da imagem, nem a imagem, mera ilustração do texto; são dois elementos 
de uma mesma obra, que deve ser lida como um todo.

Ambas as formas literárias exercitam a imaginação e a criatividade das crianças e dos jovens 
quando bem utilizadas. Podem servir de reforço à leitura e constituem uma linguagem altamente dinâmica. 
São linguagens que, ainda que de uma origem longínqua, são adequadas à nossa era devido à fluidez, 
à intensidade e sobretudo à abertura à inovação que lhes constitui.

\paragraph{A xilogravura}

Tanto o cordel quanto as histórias em quadrinhos têm algo em comum:
a presença de imagens. Ao se pensar em cordel, logo se pensa em \textit{xilogravura}, 
mas a xilogravura não surgiu com a literatura de cordel. Ela começou a fazer
parte dos folhetos a partir da década de 1950. Tradicionalmente, trata-se
de uma matriz de madeira que imita um clichê de chumbo. O clichê em si 
já é uma imitação da xilogravura, \textbf{uma técnica milenar dos egípcios
e chineses}: recorta-se uma figura em relevo sobre uma madeira. A figura 
em relevo imprime, como um carimbo sobre um papel em branco, e as partes
cortadas são os sulcos onde a tinta não aparece. 

A xilogravura entrou na vida da literatura de cordel como uma alternativa 
ao poeta sem recursos para ilustrar a capa de um folheto. Ela passou a compor 
a estrutura do folheto, embora o público não tenha se identificado de imediato. Hoje, se por um lado 
o público intelectual que gosta de folheto, o estudioso ou o turista que compra o 
folheto como uma curiosidade, prefere a capa com a xilogravura, o público mais 
tradicional prefere a capa com desenhos, fotografias. 
Os autores e editores tentam sempre agradar a todos, trabalhando tanto com a xilogravura 
como com desenhos, figuras, etc.

\section{Atividades}

\subsection{Pré-leitura}

\subsubsection{Atividade 1}

\paragraph{Tema} Ouvindo o cordel.

\paragraph{Conteúdo} Audição de gravações autênticas de cordel, repente e canções
nordestinas.

\paragraph{Justificativa} Visto que um dos principais elementos que constituem o gênero do cordel é a 
\textbf{oralidade}, sugerimos ao professor ou à professora que inicie o 
trabalho com uma introdução a este tema com os alunos. 
Ainda que o cordel, folheto, ou romance --- outras nomes que ele recebe --- 
não \textit{precise} ser \textit{performado}, é comum que isso aconteça
em eventos como as feiras de cordeis, ou, mais tradicionamente, 
nas feiras abertas das cidades do interior onde eles eram comercializados.
Nesta ocasião, o cordelista, responsável pela escrita, ilustração e venda
do folheto, também cuidava da etapa de reprodução do trabalho.

Há outro gênero, que podemos chamar de \textit{irmão do cordel},
onde a \textit{performance} não é facultativa, mas o seu meio de 
divulgação e mesmo de produção. 
O \textbf{repente}, popular nas mesmas regiões que o cordel, é 
baseado no improviso cantado alternado por dois cantores. 
Não se trata, como no cordel, necessariamente de uma narrativa;
diversos assuntos podem surgem numa toada de repente, sendo as mais comuns
as invectivas entre os dois violeiros. 
Um elemento que, no entanto, aproxima os dois gêneros, e justifica esta atividade,
é o fato de que, bem como no cordel, todos os versos do repente são
obrigatoriamente \textbf{rimados}, o que lhe garante uma forte musicalidade. 

\paragraph{Metodologia} O professor ou a professora deve executar algumas faixas 
do antológico álbum \href{https://www.youtube.com/watch?v=wS6jzcZcc6U}{Nordeste: Cordel, 
Repente e Canção}, de 1975, que
reune exemplares da poesia oral nordestina em forma de repente e cordel. 
Durante a audição, pergunte aos alunos se algo lhes soa familiar:

\begin{itemize}
\item Vocês já ouviram alguma vez algo como isso?
\item Se sim, em que ocasião?
\item Quais são os instrumentos que eles usam?
\end{itemize}

\paragraph{Tempo estimado} Duas aulas de cinquenta minutos.



\subsubsection{Atividade 2} 

\paragraph{Tema} Tradição oral como meio de transmissão de valores.

\paragraph{Conteúdo} Aula expositiva acerca do tema acima exposto. 

\paragraph{Justificativa} Há culturas no Brasil e no mundo onde os principais valores fundadores
da sociedade não são transmitidos por um livro sagrado, como a Bíblia, para os cristãos, 
o Torá, para os judeus, ou o Alcorão, para os muçulmanos. Povos indígenas brasileiros 
como os Guarani guardam e trasmitem uma sabedoria milenar por meio de cantos
sagrados chamados \textit{guahu}. Para os povos da tradição de Ifá, no Oeste africano,
e seus descendentes no Brasil, os \textit{itãs} são histórias sagradas que são consultadas
por meio de um sacerdote para solucionar problemas quotidianos da comunidade. 
Esses \textit{itãs} também são, tradicionalmente, passados de sacerdote a sacerdote 
por via oral. Também no Oeste africano, há a figura do \textit{griot}, cantador e contador de histórias
em praças públicas. 

\SideImage{``Homero cego guiado pelo gênio da Poesia''. (Museu Metropolitano de Arte, Nova Iorque. (CC-BY-2.0)}{PNLD2023_03_01.jpg}

Durante a Idade Média europeia, os \textit{trovadores} representavam esta cultura.
Eram eles que reproduziam, publicamente, versos criados por eles ou por outros, 
acompanhados em geral de um instrumento de cordas. O cordel e o repente nordestino
são descendentes diretos desta tradição, instaurada no país com a colonização ibérica
a partir do século \textsc{xv}. É imprescindível que os alunos 
percebam que a tradição oral é um elemento comum nas diferentes expressões
culturais do mundo.


\paragraph{Metodologia} Embasado nas informações expostas acima e nos
artigos e vídeos das \textbf{Sugestões de referências complementares},
o professor ou professora pode realizar uma aula expositiva para os
alunos acerca do tema. Escreva na lousa os principais tópicos,
que os alunos devem copiar em seus cadernos em forma de fichamento. 
Não deixe de mostrar as imagens presentes aqui no manual.
\SideImage{\textit{Griot} da antiga África Ocidental. (CC-BY-2.0)}{PNLD2023_03_02.jpg}

\Image{Representação de uma dupla de repentistas. (CC-BY-2.0)}{PNLD2023_03_03.jpg}

\paragraph{Tempo estimado} Duas aulas de cinquenta minutos.

\subsection{Leitura}

\subsubsection{Atividade 1} 

\paragraph{Tema} Características do cordel.

\paragraph{Conteúdo} Esquematização das informações acerca do cordel 
apresentadas no poema de abertura do livro. 

\paragraph{Metodologia} A partir da leitura do ``\textsc{abc} do cordel'', o professor ou a professora pode pedir
aos alunos uma esquimatização das informações apresentadas, como:
definição do cordel, apresentação do narrador, grandes nomes do cordel,
outros nomes para o gênero, e quantas outras informações o professor ou professor
julgar importante. 

\paragraph{Tempo estimado} Uma aula de cinquenta minutos.

\BNCC{EF68AR33}
\BNCC{EF69AR34}
\BNCC{EF69LP44}

\Image{Escultura em madeira de Padre Cícero, líder religioso ícone da cultura popular nordestina, citado ao lado de Lampião, em ``\textsc{abc} do cordel''.(CC-BY-2.0)}{PNLD2023_03_04.jpg}

\pagebreak

\subsubsection{Atividade 2}


\paragraph{Tema} Lendo e cantando o cordel. 

\BNCC{EF35LP27}

\paragraph{Conteúdo} Interpretação dos poemas de cordel em quadrinhos com um
acompanhamento musical.

\paragraph{Justificativa} O cordel é um gênero textual que vem diretamente da
tradição oral. Portanto, para sua apreciação é imprescindível 
a presença de elementos de performance musical, além, é claro, da leitura
em voz alta, que pode transformar-se em canto.

\paragraph{Metodologia} Para a leitura em sala de aula, o professor ou a professora pode solicitar o
professor de Música da escola, caso haja um, ou de Artes, para que a leitura
seja acompanhada por um instrumento. Idealmente, precisaria-se de uma viola nordestina 
e um pandeiro. No entanto, estes instrumentos podem ser adaptados. 


\paragraph{Tempo estimado} Quatro aulas de cinquenta minutos.


\Image{Pandeiro: tradicional instrumento utilizado nas \textit{performances} de cordel e repente.(CC-BY-2.0)}{PNLD2023_03_05.jpg}

\Image{Uma lata pode funcionar como um pandeiro improvisado.(CC-BY-2.0)}{PNLD2023_03_06.jpg}


\subsubsection{Atividade 3}

\BNCC{EF35LP23}
\paragraph{Tema} A métrica na poesia.

\paragraph{Conteúdo} Estudo das principais formas métricas de um poema.

\paragraph{Justificativa} A métrica acompanha toda expressão poética,
em versos ou oral. O seu estudo é imprescindível para a melhor fruição
estética dos poemas, bem como para o exercício poético em si.

\paragraph{Metodologia} O professor ou professora pode aproveitar o ensejo do trabalho com o cordel
e apresentar, caso ainda não seja um conteúdo dado, ou revisar as diferentes
métricas que um poema pode ter. 

A redondilha menor, com cinco sílabas poéticas, e a redondilha maior, com sete,
são os mais comuns na poesia popular. Já os decassílabos, com dez, e os alexandrinos,
com doze sílabas poéticas, estão presentes nos clássicos como as epopeias de Homero
e de Camões. Os versos livres, aqueles que não apresentam uma padronização métrica,
ganharam evidência no Ocidente com as escolas modernistas do começo do século \textsc{xx}.

Escolha um cordel do livro e peça que os alunos indiquem a metrificação dos versos. 
Este procedimento se chama \textit{escansão}.

\paragraph{Tempo estimado} Duas aulas de cinquenta minutos.


\subsubsection{Atividade 4}

\BNCC{EF35LP29}

\paragraph{Tema} Lendo nas entrelinhas do poema.

\paragraph{Conteúdo} Discussão em sala de aula a partir do poema ``A metade da vida''.

\paragraph{Justificativa} A fim de inferir a presença de valores sociais, 
culturais e de diferentes pontos de vista sobre o mundo em textos literários,
é importante que os alunos reconheçam que as obras possibilitam 
que se estabeleça múltiplos olhares sobre identidades, sociedades e culturas, 
sempre a partir da autoria e do contexto sócio-histórico da produção.

\paragraph{Metodologia} Acerca da história ``A metade da vida'', faça as seguintes perguntas
à turma:

\begin{itemize}
\item Qual o principal tema tratado neste cordel? 
\item Qual a ironia na fala final do velho barqueiro?
\end{itemize}

Caso necessário, explique do que se trata a figura de linguagem \textit{ironia} e 
aproveite o exemplo do cordel para ilustrá-la.

Depois, acerca das ``Bravuras de dois vaqueiros e o lobisomem fantasma'', 
pergunte aos alunos \textbf{quais elementos de outras histórias conhecidas 
no imaginário popular podemos encontrar dentro deste cordel?} Caso eles 
tenham dificuldade, chame a atenção a lugares-comuns  como a \textbf{donzela presa 
num castelo à espera de um cavaleiro, com quem se casará no final}, \textbf{um cavaleiro 
que derrota um monstro/dragão}, \textbf{o ouro, no final da aventura, como recompensa pela salvação da donzela 
e da morte do monstro} etc.

\paragraph{Tempo estimado} Duas aulas de cinquenta minutos.


\subsection{Pós-leitura}

\subsubsection{Atividade 1} 

\paragraph{Tema} Oficina de xilogravura.

\paragraph{Conteúdo} Oferecer aparatos acessíveis ao quotidiano 
das salas de aula do Brasil para a produção de ilustrações
seguindo a base da xilogravura. 

\paragraph{Justificativa} O termo \textit{xilogravura} significa, literalmente, ``gravura feita em madeira''. 
Isto porque os gravuristas especializados nesta técnica entalham, com o auxílio de 
um \textbf{formão}, desenhos e escrituras sobre uma placa de madeira. 
Criam, assim, um molde que, coberto de tinta, em geral preta, será usado para a reprodução 
do desenho ou escritura repetidas vezes. 
Este é o processo dos folhetos de cordel que têm as tradicionais ilustrações feitas com
a técnica da xilogravura. 

\Image{Exemplo de xilogravura.(CC-BY-2.0)}{PNLD2023_03_07.jpg}

\paragraph{Metodologia} Como para o contexto da sala de aula este procedimento pode não ser tão
acessível por conta dos materiais --- formão, goiva, rolo, placa de linólio... ---,
propomos que o professor ou professora, com o auxílio do professora ou professora
de Artes, desenvolva uma atividade de ilustração utilizando objetos mais 
próximos, como o \textbf{isopor} presente em embalagens, que pode ser reutilizado.
Neste caso, não se tratará de uma xilogravura mas de uma \textit{isogravura}.
Sugerimos \href{https://www.youtube.com/watch?v=8sq9Qq-wrls}{um vídeo didático}
acerca da técnica. O professor ou
professora deve lembrar aos alunos que os resultados da produção serão utilizados
na atividade seguinte, que é a criação de um cordel inteiro. 


\paragraph{Tempo estimado} Duas aulas de cinquenta minutos.


\subsubsection{Atividade 2}


\paragraph{Tema} Oficina de escrita de cordel.

\BNCC{EF35LP25}
\BNCC{EF15AR04}
\BNCC{EF15AR05}

\paragraph{Conteúdo} Adaptação para a estrutura do cordel
de uma história conhecida.

\paragraph{Justificativa} O cordel é antes de tudo uma \textbf{forma} 
de contar uma história. Em sua origem, graças à forte presença
da oralidade e da musicalidade em sua composição e a consequente
facilidade em sua circulação, muitas histórias importantes passaram
a ser documentadas desta forma. Em certo momento, nos interiores 
do país, os cordeis tornaram-se mais importantes mesmo que a imprensa
oficial enquanto meio de divulgação de notícias para a população. 
Visto isso, é imprescindível que, após a experiência com a leitura dos
cordeis deste livro os alunos e alunas experimentem agora a \textbf{produção}
dos mesmos, sobretudo no que diz respeito à característica de adaptação 
de uma história já conhecida à sua forma, como se percebe em algumas
partes da obra.

\paragraph{Metodologia} 
Seguindo o exemplo do cordel ``Bravuras de dois vaqueiros e o lobisomem 
fantasma'', que utiliza elementos comuns ao universo das lendas, os alunos 
podem fazer uma adaptação para o cordel de uma história de sua escolha.
Releia o poema, chamando a atenção aos \textbf{elementos de outras histórias conhecidas 
no imaginário popular que podemos encontrar dentro deste cordel}. Caso eles 
tenham dificuldade, chame a atenção a lugares-comuns  como a \textbf{donzela presa 
num castelo à espera de um cavaleiro, com quem se casará no final}, \textbf{um cavaleiro 
que derrota um monstro/dragão}, \textbf{o ouro, no final da aventura, como recompensa pela salvação da donzela 
e da morte do monstro} etc.

Para o exercício autoral, dê sugestões de narrativas conhecidas como as 
dos super-heróis do cinema, as das lendas do folclore nacional e 
estrangeiro, ou mesmo eventos quotidianos de alcance nacional: 
um evento esportivo marcante, uma manchete do noticiário etc.

\paragraph{Tempo estimado} Quatro aulas de cinquenta minutos.


\subsubsection{Atividade 3}

\BNCC{EF15AR06}

\paragraph{Tema} Feira de cordéis.

\paragraph{Conteúdo} Organização de uma feira de cordéis seguindo o modelo tradicional.

\paragraph{Justificativa} Após o trabalho individual dos alunos ou dos grupos,
é importante que haja uma interação entre a turma onde os envolvidos
deverão compartilhar impressões e apreciações acerca das obras.

\paragraph{Metodologia} Após a realização dos cordéis em quadrinhos,
 o professor ou a professora deve animar a realização de uma feira de cordéis 
  seguindo os moldes tradicionais: os livretos pendurados em varais.
 Neste caso, a exposição pode ser feita em uma área comum da escola para o acesso de outros alunos
 de outras turmas. Outra opção é expor o resultado na Festa Junina da escola, para que não só a comunidade
 escolar, mas também os familiares e vizinhos possam usufruir seus trabalhos.


\paragraph{Tempo estimado} Duas aulas de 50 minutos.
 
\Image{Klévisson Viana performando numa feira de cordéis na França.(Arquivo do site do autor.)}{PNLD2023_022_05.jpg}



\section{Sugestões de referências complementares}

\subsection{Músicas} 

\begin{itemize}
	\item ``Perseguição'', de Sérgio Ricardo. 

Trilha sonora do filme \textit{Deus e o Diabo na Terra do Sol}, de Glauber Rocha. 
Um trecho da música é citado por uma das crianças que brincam de cangaço enquanto os mais velhos contam a história
da última batalha de Lampião.

	\item Álbum musical \href{https://www.youtube.com/watch?v=wS6jzcZcc6U}{Nordeste: Cordel, Repente, Canção}. 
\end{itemize}

\subsection{Filme} 
\begin{itemize}

\item \href{https://www.youtube.com/watch?v=xFOZxwBcUmo}{Nordeste: Cordel, Repente, Canção}.

\end{itemize}

\subsection{Artigos}

\begin{itemize}
\item \href{educacaopublica.cecierj.edu.br/artigos/19/1/multiculturalismo-e-suas-implicaes-na-educao}{``Multiculturalismo e suas aplicações na educação''}.

	A atualidade educacional é um espelho da ausência de modelos, de referenciais que antes balizavam a sociedade brasileira. Em educação, vivenciar o multiculturalismo e a inserção das tecnologias vem se transformando em desafio à prática pedagógica. O currículo escolar representa um grande esforço para trabalhar com a diversidade cultural, a mensagem gerada pela indústria cultural e a aquisição de conhecimentos e informações. Este texto apresenta uma problematização relacionada à temática do currículo escolar a partir do recorte cultural e social.

	\item ``Textos orais e textura oral'', ``Literatura oral e oralidade escrita'', ``A literatura africana e a questão da língua'', ``Estilo oral'', ``A palavra na sabedoria banto'', ``O significado da literatura em culturas orais''. Ver \href{letras.ufmg.br/padrao_cms/documentos/eventos/vivavoz/A\%20tradi\%C3\%A7\%C3\%A3o\%20oral_diagramado_16jun2016.pdf}{Site Letras da Universidade de Minas Gerais}. 

	Nesse tipo de comunicação, o suporte da transmissão de experiência
de \textsc{a} a \textsc{b} é a fala. No plano individual, a comunicação oral se elabora
a partir das limitações impostas pela presença do interlocutor. A pronúncia, suporte material, 
resultará de um equilíbrio constante, a ser assegurado, entre uma interlocução cuidada --- 
exigida pelo esforço de compreensão, no nível do ouvinte --- uma elocução relaxada,
determinada pela lei do menor esforço.

\item \href{pepsic.bvsalud.org/scielo.php?script=sci_arttext&pid=S1415-69542013000100004}{``A arte de contar histórias e o conto de tradição oral em práticas educativas''}. 

Este trabalho é fruto da pesquisa que se iniciou com o estudo de especialização em literatura infanto-juvenil ``sobre as histórias que se contam para as crianças'' na Universidade de São Paulo--\textsc{usp}. O conto da tradição de transmissão oral é a forma primitiva da arte de dizer. A tradição perpetuou essas narrativas como uma forma de ensinamentos transmitidos oralmente. Sua idade perde-se na poeira dos tempos, como diziam os poetas, e seu \textit{lócus nascendi} ninguém sabe, ninguém viu.

\end{itemize}


\end{document}