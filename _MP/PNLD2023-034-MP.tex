\documentclass[11pt]{extarticle}
\usepackage{manualdoprofessor}
\usepackage{fichatecnica}
\usepackage{lipsum,media9}
\usepackage[justification=raggedright]{caption}
\usepackage[one]{bncc}
\usepackage[acorde]{../edlab}
\usepackage{marginnote}
\usepackage{pdfpages}
\usepackage[printwatermark]{xwatermark}
% \newwatermark[pagex=2]{\includegraphics[scale=3.3]{watermarks/test-a.png}}	% página específica
% %\newwatermark[oddpages]{\includegraphics{watermarks/test-a.png}}			% páginas ímpars
% %\newwatermark[evenpages]{\includegraphics{watermarks/test-a.png}}			% págimas pares
% \newwatermark[allpages]{\includegraphics[scale=3.3]{watermarks/test-b.png}}
% \pagecolor{cyan!0!magenta!10!yellow!28!black!28!}

\newcommand{\AutorLivro}{Arthur Nestrovski}
\newcommand{\TituloLivro}{Os doze trabalhos de Hércules}
\newcommand{\Genero}{Lendas, mitos, fábula}
%\newcommand{\imagemCapa}{./images/PNLD0001-01.png}
\newcommand{\issnppub}{978-65-99441-24-0}
\newcommand{\issnepub}{978-65-99441-27-1}
% \newcommand{\fichacatalografica}{PNLD0001-00.png}
\newcommand{\colaborador}{Gabriela Karam}

\begin{document}

\title{\TituloLivro}
\author{\AutorLivro}
\def\authornotes{\colaborador}

\date{}
\maketitle

%\begin{abstract}\addcontentsline{toc}{section}{Carta ao professor}
%\pagebreak

\tableofcontents


\begin{abstract}

Caro professor,
Este material tem a intenção de contribuir para que você consiga desenvolver um trabalho aprofundado com a obra \textit{Os doze trabalhos de Hércules} em sala de aula. Você encontrará informações sobre o autor, sobre o gênero e também algumas propostas de trabalho para a sala de aula que você poderá explorar livremente, da forma que considerar mais apropriada para os seus estudantes. A principal razão que leva uma pessoa a estudar os mitos são os frutos simbólicos que têm para entender mais profundamente o ser humano. O fato é que, no mundo racional da sociedade contemporânea, nos afastamos cada vez mais dos sonhos e não entendemos bem o papel dos mitos, dos heróis e das cerimônias rituais, estas, reguladoras das práticas sociais. Perdemos o sentido das metáforas. Não sabemos desvelá-las e o ponto de vista cartesiano nos faz acreditar que essas histórias são primitivas, não nos dizem respeito. O autor Joseph Campbell, uma das maiores autoridades mundiais nesse assunto, em seus estudos, adquirido em anos de pesquisas sobre os mitos de inúmeras sociedades, permitiram que Campbell formulasse ideias originais sobre a similaridade entre os povos em suas relações com o cosmos. Tais noções nos convidam a buscar uma nova forma de interpretar a nossa gênese. Em nossa diversidade, ele ressalta que somos um único povo – o povo terrestre – pois do alto, do espaço sideral, a Terra é vista sem fronteiras políticas, sem barreiras que impeçam a nossa inter-relação. Em \textit{Os doze trabalhos de Hércules}, o autor Arthur Nestrovski conta, de forma leve e despretensiosa, um mito grego muito conhecido e ainda tem o apoio de ilustrações de Zansky que complementam essa aventura de Héracles, um semideus filho de Zeus e de Alcmena, filha de um rei mortal. Portanto, Herácles é um dos filhos que Zeus teve fora do casamento com Hera. Héracles se tornou o mais forte e poderoso de todos os mortais, uma espécie de super-herói mitológico, e os romanos o chamavam de Hércules, nome como ficou mais conhecido com o passar dos tempos. No mito original, o semideus Hércules teve de cumprir 12 tarefas como penitência por ter, sob o efeito de um feitiço, matado a mulher e os filhos. Na obra, dirigida ao público infantil, Hera, mulher de Zeus, imbuída de ciúmes, consegue transformar seu menino humano predileto, Euristeu, em rei, e obriga Hércules a se submeter a ele caso queira garantir a imortalidade. Foi por conta disso que Hércules teve de realizar os famosos doze trabalhos, ordenados pelo rei Euristeu. Ao longo do manual, todos esses aspectos serão explorados e relacionados a sugestões de atividades. Com isso, objetiva-se oferecer algumas ideias e inspirações para um trabalho que pode ser desenvolvido tanto a curto, quanto a médio e longo prazo. Sintase à vontade para personalizar a aula e torná-la sua, aplicando seus conhecimentos, sua personalidade e aproveite para fortalecer seu vínculo com a turma. Boa aula! 

\end{abstract}

\section{Sobre o livro}
\Image{Os deuses gregos (Pinterest; Domínio público)}{PNLD2023-034-02.png}

\section{Sobre o autor}

\section{Sobre o gênero}

\section{Proposta de Atividades}
\subsection{Pré Leitura}

\BNCC{EF15LP02}% Estabelecer expectativas em relação ao texto que vai ler (pressuposições antecipadoras dos sentidos, da forma e da função social do texto), apoiando-se em seus conhecimentos prévios sobre as condições de produção e recepção desse texto, o gênero, o suporte e o universo temático, bem como sobre saliências textuais, recursos gráficos, imagens, dados da própria obra (índice, prefácio etc.), confirmando antecipações e inferências realizadas antes e durante a leitura de textos, checando a adequação das hipóteses realizadas.}
\BNCC{EF03LP26}% Identificar e reproduzir, em relatórios de observação e pesquisa, a formatação e diagramação específica desses gêneros (passos ou listas de itens, tabelas, ilustrações, gráficos, resumo dos resultados), inclusive em suas versões orais.
\BNCC{EF04HI02}% Identificar mudanças e permanências ao longo do tempo, discutindo os sentidos dos grandes marcos da história da humanidade (nomadismo, desenvolvimento da agricultura e do pastoreio, criação da indústria etc.)
\BNCC {EF04ER04}% Identificar as diversas formas de expressão da espiritualidade (orações, cultos, gestos, cantos, dança, meditação) nas diferentes tradições religiosas.

\paragraph{Tema} Reflexões iniciais sobre mitologia. 

\paragraph{Conteúdo} Mitologia Grega.

\paragraph{Objetivo} Introduzir os estudantes no universo da Mitologia Grega. 

\paragraph{Justificativa} O mundo da tecnologia e da produtividade afasta do cotidiano, cada vez mais, o universo das lendas e dos mitos. Estudar esse universo é, primeiramente, uma forma de imergir numa outra forma de conhecimento e estimular a imaginação dos estudantes. Conhecer a mitologia é entender-lhe a função social e a religiosidade, analisando-as à luz do mundo contemporâneo.  

\paragraph{Metodologia}

Antes de mergulhar no universo de \textit{Os Doze Trabalhos de Hércules}, o educador pode estimular a turma a explorar, em outras fontes de informação, a mitologia de uma forma geral. Para isso, incentive a pesquisa, em livros e sites de internet, sobre mitologia. A indicação inicial para promover uma problematização do gênero da obra em questão é o vídeo \textit{A mitologia grega é adequada para crianças?}, do canal \textit{Noites Gregas}, disponível em \url{https://youtu.be/9qlhKoUj_dw}, no qual o professor Moreno problematiza a mitologia grega para crianças. É um ponto de vista que nos faz pensar sobre o assunto e pode gerar inúmeros debates com as crianças. O professor Moreno desenvolve um pensamento a respeito das modificações que são feitas nos mitos para que o público infanto-juvenil possa também ter um primeiro contato com essas histórias. Proponha que as crianças se sentem em forma de círculo com suas carteiras para que aconteça um debate acerca do que esperam das histórias mitológicas, auxiliando-os a pensarem nos pontos interessantes que esse tipo de obra pode trazer para a vida deles, e quais as visões que eles têm a respeito dos mitos. Depois do debate, o professor pode sugerir que todos façam uma pesquisa em conjunto sobre alguns pontos:

\begin{itemize}
\item O que é a mitologia grega e o que ela significa para o ser humano mesmo depois de tanto tempo?
\item Quais as ligações entre os mitos e o mundo contemporâneo?
\item Quais eram as características históricas do mundo grego?
\item Onde está a Grécia no mapa e que características a levaram para a criação de mitos?
\item Qual era a relação entre a mitologia grega e a religiosidade desse povo?
\item Quais as diferenças entre o politeísmo, encontrado na Grécia Antiga, e o monoteísmo? %PNLD2023-034-04
\end{itemize}

\Image{A mitologia grega (Pinterest; Domínio público)}{PNLD2023-034-03.png}

\Image{O minotauro (Pinterest; Domínio público)}{PNLD2023-034-04.png}

Para auxiliar nas reflexões que os itens acima propõe, o professor pode trazer uma série de repertório. Sugerimos iniciar pelo vídeo que conta o mito do Minotauro (disponível no Youtube em \url{https://youtu.be/20bVidCkrdw}) e depois o da Medusa (disponível no Youtube em: \url{https://youtu.be/6EG5fEECKYY}). Logo depois, para inteirá-los ainda mais no universo da mitologia, é interessante fazer com que a turma conheça mais os deuses e deusas gregos. Sugerimos a leitura e o debate sobre essas personagens mitológicas, para que as crianças consigam entender a importância e a grandeza dessas figuras imortais e poderosas e saibam distingui-las de modo bastante consciente do Deus do mundo religioso. Logo em seguida, pode ser mostrada a animação cantada \textit{Os deuses do olimpo}, do canal \textit{Um pouco sobre quase tudo}, disponível no Youtube em: \url{https://youtu.be/DJ-5cxJVI64}, que conta a história dos deuses do olimpo, entre eles Zeus - pai da nossa personagem Hércules. 

\Image{O professor pode trazer uma série de repertório (Educa Mais Brasil; Domínio público)}{PNLD2023-034-05.png}

Ao final da exibição dos vídeos e da pesquisa em si, realize junto com a turma um relatório sobre as informações que colherem, baseado nos itens norteadores, levando em conta as características principais dos mitos gregos, suas funções sociais, a presença da religiosidade e a conexão de tudo isso com o mundo contemporâneo. Essas e outras questões podem ser debatidas em sala de aula, assim como o significado de matáfora e como ela é usada no contexto mitológico, sempre dando liberdade para que eles expressem possíveis dúvidas ou questionamentos.

\Image{Realize junto com a turma um relatório sobre as informações que colherem (Educa Mais Brasil; Domínio público)}{PNLD2023-034-06.png}

\paragraph{Tempo Estimado} Quatro aulas de 50 minutos.

\subsection{Leitura}

\BNCC{EF35LP21}% Ler e compreender, de forma autônoma, textos literários de diferentes gêneros e extensões, inclusive aqueles sem ilustrações, estabelecendo preferências por gêneros, temas, autores.
\BNCC{EF35LP03}% Identificar a ideia central do texto, demonstrando compreensão global.
\BNCC{EF15LP16}% Ler e compreender, em colaboração com os colegas e com a ajuda do professor e, mais tarde, de maneira autônoma, textos narrativos de maior porte como contos (populares, de fadas, acumulativos, de assombração etc.) e crônicas.

\paragraph{Tema} Leitura e compreensão de \textit{Os Doze Trabalhos de Hércules}.

\paragraph{Conteúdo} Estrutura de de \textit{Os Doze Trabalhos de Hércules}.  

\paragraph{Objetivo} Ler e compreender o texto de \textit{Os Doze Trabalhos de Hércules} e seus pressupostos, especialmente os conhecimentos a respeito da mitologia, estudados na Atividade de Pré-Leitura. Promover o gosto pelos livros e pela leitura, estimular o desenvolvimento da linguagem e da criatividade.   

\paragraph{Justificativa} O professor tem grande influência na formação de leitores, especialmente quando se trata de mitos e lendas, para ampliar o repertório cultural dos alunos. É preciso fazer a mediação entre a obra, sua linguagem, suas estruturas, seus pressupostos e os estudantes, de preferência estabelecendo uma relação fundamentada no prazer, na identificação e na liberdade de interpretação. Eis o nosso desafio: ler com os alunos, apresentando as passagens decisivas de um texto, explicando por que elas chamam a atenção e  ouvindo as impressões dos estudantes a respeito de tudo isso. 

Após a pesquisa sobre mitologia e um primeiro contato com Hércules, nosso herói semideus, podemos dividir a turma em grupos os quais ficarão responsáveis, cada um, por um dos trabalhos de Hércules. Cada coletivo fica responsável por ler a sua parte - escolhida pelo professor - e preparar uma narrativa para contá-la para toda a turma. Sugerimos para que o educador inicie os trabalhos lendo as primeiras páginas do livro, que explica o nascimento do nosso herói e como ele foi submetido a esses doze trabalhos. Logo em seguida, se pode organizar a classe para que cada grupo se prepare para a sua narração. Quando a leitura acabar, em círculo, estimule os estudantes a debaterem sobre qual narração elas acharam mais interessante e os porquês. A partir daí se pode levantar um debate sobre a oralidade das histórias e a importância de passá-las adiante de modo que todas as pessoas entendam e possam repassá-las à sua maneira. Ainda em roda convoque a turma para trazerem histórias que eles já tenham ouvido dos seus pais, avós, tias e tios, e que acreditem que possam ser classificados como mitos. Sugerimos que o professor leve essas narrativas para discussão sobre elementos que foram vistos na obra, relacionando as histórias trazidas com as metáforas presentes em \textit{Os doze trabalhos de Hércules}, sejam atos heroicos, deuses e semideuses poderosos. 

\Image{Realize junto com a turma um relatório sobre as informações que colherem (Educa Mais Brasil; Domínio público)}{PNLD2023-034-07.png}

\paragraph{Tempo Estimado} Quatro aulas de 50 minutos.

\subsection{Pós-leitura}

\BNCC{EF35LP25}% Criar narrativas ficcionais, com certa autonomia, utilizando detalhes descritivos, sequências de eventos e imagens apropriadas para sustentar o sentido do texto, e marcadores de tempo, espaço e de fala de personagens.
\BNCC{EF05HI08}% Identificar formas de marcação da passagem do tempo em distintas sociedades, incluindo os povos indígenas originários e os povos africanos.}
\BNCC{EF05HI09}% Comparar pontos de vista sobre temas que impactam a vida cotidiana no tempo presente, por meio do acesso a diferentes fontes, incluindo orais.}

\paragraph{Tema} Mitologia. 

\paragraph{Conteúdo} Comparação entre mitologia grega e indígena.

\paragraph{Objetivo} Comparar diferentes culturas. Desenvolver exercícios de escrita.

\paragraph{Justificativa} Enquanto as atividades de leitura, compreensão e análise caracterizam-se pelas primeiras aproximações do texto, seguidas de atividades de descrição de suas características, a prática de redação abre espaço para que os estudantes criem suas próprias narrativas. Além disso, a comparação entre mitos gregos e indígenas serve para a compreensão da diferença entre as culturas. 

Após a leitura, é interessante fazer um comparativo entre as expectativas que foram relatadas no primeiro momento da atividade com o que foi lido e aprendido na obra. Proponha um debate para que os estudantes articulem suas opiniões a respeito. Depois de uma breve discussão, iniciar uma pesquisa sobre quais outras mitologias existem para além do mundo grego: observem se entre as histórias narradas na atividade anterior, alguma está embutida nas nossas culturas originárias, sejam elas africanas ou indígenas. Se não, promover então uma pesquisa das nossas próprias mitologias que nos trouxeram, mesmo sem saber, e através de metáforas, até aqui. Para isso, a matéria \textit{Quais são os principais deuses da mitologia indígena brasileira}, da \textit{Revista super interessante}, disponível em \url{https://super.abril.com.br/mundo-estranho/quais-sao-os-principais-deuses-da-mitologia-indigena-brasileira/}, pode nortear e enriquecer a discussão. O artigo traz alguns dos deuses da mitologia indígena, falando sobre suas características e os mitos que fundam essas figuras, bem como a noção da passagem do tempo para esses povos. Auxilie os estudantes a debaterem sobre as diferenças entre a mitologia grega e a indígena, buscando relacionar a ligação desses mitos com o tempo presente. Após a imersão, sugira que os alunos, divididos em novos grupos, usem muita criatividade e redijam em conjunto um mito com base em tudo o que foi pesquisado e na leitura. Promova uma busca sobre o que mais poderia ser contado para que essa história cause interessante para quem ouve, para que ela possa passar de geração a geração e até ser transformada em livro. Estimule a criação de um livro dos mitos que foram criados pelos grupos, levando em consideração os comentários feitos sobre a narrativa, pedindo, ainda, que haja uma imagem que represente esse mito. Ela pode ser feita usando diferentes materiais, como lápis de cor, giz de cera, tinta guache e o que mais a escola disponibilizar. Sugerimos que todos os mitos, criados por cada grupo, gerem um livro que ficará na biblioteca da escola para que possam ser lidos e passados adiante assim como a mitologia grega que se mantém imortal grande parte das culturas, tanto ocidentais como orientais.

\paragraph{Tempo Estimado} Quatro aulas de 50 minutos.

\Image{Após a imersão, sugira que os alunos, divididos em novos grupos, usem muita criatividade e redijam em conjunto um mito com base em tudo o que foi pesquisado e na leitura (Mitologia; Domínio público)}{PNLD2023-034-08.png}

\Image{Realize junto com a turma um relatório sobre as informações que colherem (Abril; Domínio público)}{PNLD2023-034-09.png}

\section{Sugestões de referências complementares}

\subsection{Livros} 

\begin{itemize}
\item \textsc{campbell}, Joseph. \textit{O poder do mito}. São Paulo: Cultrix, 1990.

Livro proveniente de uma série de conversas mantidas entre Joseph Campbell e o jornalista Bill Moyers a respeito de mitologia.
%PNLD2023-034-10
\end{itemize}

\section{Bibliografia comentada}
\subsection{Livros}

\begin{itemize}
\item \textsc{brasil}. Ministério da Educação. Base Nacional Comum Curricular. Brasília, 2018.

Consultar a \textsc{bncc} é essencial para criar atividades para a turma. Além de especificar quais habilidades precisam ser desenvolvidas em cada ano, é fonte de informações sobre o processo de aprendizagem infantil. 

\item \textsc{lispector}, Clarice. Todos os contos. São Paulo: Rocco, 2016.

Trata-se de uma obra ficcional lançada em 2016 que reúne os contos escritos por Clarice Lispector. 
 
\item \textsc{grimm}, Jacob e Wilhelm. Contos maravilhosos infantis e domésticos. São Paulo: Editora 34, 2018.

Inúmeros contos fantásticos que foram publicados inicialmente em 1812. Contos clássicos que originaram diferentes histórias conhecidas no mundo ocidental.

\item \textsc{kimmel}, Eric. \textit{Mitos gregos}. São Paulo: WMF Martins Fontes, 2013.

Livro em que o autor se preocupa em aproximar os heróis gregos do jovem leitor de hoje.

\item \textsc{munduruku}, Daniel. \textit{Contos indígenas brasileiros}. São Paulo: Global Editora, 2004.

Livro que traz contos dos povos originários e apresenta a diversidade cultural e linguística no Brasil.

\item \textsc{coelho}, Nelly Novaes. Literatura infantil, teoria, análise, didática. 1ª ed. São Paulo: Moderna, 2000.

Livro que fala sobre os espaços da literatura infantil na contemporaneidade e a importância de as crianças estarem ligadas ao seu imaginário pela via literária.

\end{itemize}

\subsection{\textit{Sites}}

\begin{itemize}
\item Artigo "A Importância da Leitura dos Contos de Fadas na Educação Infantil", por Ana Maria da Silva. Disponível em: \url{https://siteantigo.portaleducacao.com.br/conteudo/artigos/educacao/a-importancia-da-leitura-dos-contos-de-fadas-na-educacao-infantil/30151}. 
Acesso em 20 dez. de 2021.

No artigo, a autora fala sobre a importância da construção do imaginário pela via da literatura para as crianças, trazendo elementos que analisam o mundo pós-moderno e os espaços que a literatura infantil, principalmente os contos, devem ter.

\item Artigo "Literatura infantil: A contribuição dos contos de fadas para a construção do imaginário infantil" Disponível em: \url{http://docs.uninove.br/arte/fac/publicacoes/pdf/v3-n1-2012/Francy.pdf}. Acesso em 20 dez. de 2021

\item Artigo "Mitologia para crianças", do blog "Filosofia animada". Disponível em: \url{https://danielmcarlos.wordpress.com/2014/02/09/mitologia-para-criancas/}. Acesso em 23 dez. de 2021. 

Artigo que se propõe a mostrar, de forma lúdica, parte da mitologia grega para o público infantil.

\end{itemize}

\subsection{\textit{Filmes}}

\begin{itemize}
\item \textit{Hércules}. Dirigido por Ron Clements e John Musker, 1997.

Meio homem, meio deus, o semideus Hércules anda atormentado pela terra. Depois de doze árduos trabalhos realizados e a perda de sua família, ele conhece seis assassinos impiedosos e une-se ao grupo em busca de trabalho, até que o rei da Trácia convida-o para treinar o seu exército.

\item \item \textit{Fúria de titãs}. Dirigido por Louis Leterrier, 2010.

Perseu descobre que é o filho mortal de Zeus, mas recusa-se a aceitar tal condição. Ele encontra-se entre a batalha dos deuses e sem ajuda para salvar a cidade e sua família da vingança de seu tio Hades, o deus do submundo. Sem nada a perder, Perseu lidera um grupo de guerreiros em uma perigosa missão para prevenir que Hades cause uma devastação na Terra.

\item \textit{Percy Jackson e o Ladrão de Raios}. Dirigido por Chris Columbus, 2010.

A vida do adolescente Percy Jackson, que está sempre pronto para entrar em uma confusão, torna-se bem mais complicada quando ele descobre que é filho do deus grego Poseidon. Em um campo de treinamento para filhos das divindades, Percy aprende a tirar proveito de seus poderes divinos e se preparea para a maior aventura de sua vida.

\end{itemize}
\end{document} 
