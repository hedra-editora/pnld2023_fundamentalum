\documentclass[11pt]{extarticle}
\usepackage{manualdoprofessor}
\usepackage{fichatecnica}
\usepackage{lipsum,media9}
\usepackage[justification=raggedright]{caption}
\usepackage[one]{bncc}
\usepackage[acorde]{../edlab}
\usepackage{marginnote}
\usepackage{pdfpages}
\usepackage[printwatermark]{xwatermark}
% \newwatermark[pagex=2]{\includegraphics[scale=3.3]{watermarks/test-a.png}}	% página específica
% %\newwatermark[oddpages]{\includegraphics{watermarks/test-a.png}}			% páginas ímpars
% %\newwatermark[evenpages]{\includegraphics{watermarks/test-a.png}}			% págimas pares
% \newwatermark[allpages]{\includegraphics[scale=3.3]{watermarks/test-b.png}}

% \pagecolor{cyan!0!magenta!10!yellow!28!black!28!}

\newcommand{\AutorLivro}{Márcio Araújo}
\newcommand{\TituloLivro}{Figurinha Carimbada}
\newcommand{\Genero}{Conto}
%\newcommand{\imagemCapa}{./images/PNLD0001-01.png}
\newcommand{\issnppub}{978-65-99441-24-0}
\newcommand{\issnepub}{978-65-99441-27-1}
% \newcommand{\fichacatalografica}{PNLD0001-00.png}
\newcommand{\colaborador}{Gabriela Karam}

\begin{document}

\title{\TituloLivro}
\author{\AutorLivro}
\def\authornotes{\colaborador}

\date{}
\maketitle

%\begin{abstract}\addcontentsline{toc}{section}{Carta ao professor}
%\pagebreak

\tableofcontents


\begin{abstract}

Este material tem a intenção de contribuir para que vocês desenvolvam um trabalho aprofundado com a obra \textit{Figurinha Carimabada} em sala de aula.
Vocês encontrarão informações sobre o autor, sobre o gênero e também 
algumas propostas de trabalho que poderão ser exploradas livremente, 
da forma que vocês considerarem mais apropriada para os seus estudantes.

Contamos com vocês para mergulharmos juntos nas histórias de seis meninos que compartilham mais ou menos as mesmas idades dos estudantes os quais vocês acompanham e que provavelmente passam pelas mesmas angústias, medos e dificuldades emocionais na relação com os colegas e o mundo ao redor. 

Não é à toa que Márcio Araújo, o autor, escolhe a frase ``viver dói'', de Clarice Lispector, uma das maiores escritoras dos últimos tempos, como epígrafe do livro. Para contar as experiências de seis meninos na fase da pré-adolescência, Araújo divide o livro em seis capítulos independentes narrados em primeira pessoa, que podem ser considerados contos e cujos títulos são os nomes dos protagonistas de cada história. As histórias de Cauê, Guilherme, Luiz Felipe, Marquinhos, João e Ícaro às vezes se cruzam, mas se assemelham sobretudo pelas angústias vividas pelos meninos. Cada uma delas carrega emoções muito profundas que certamente ajudarão em uma identificação com momentos em que as crianças estão passando ou mesmo já passaram. Um livro para ser lido e relido para não esquecermos nunca a criança que fomos e ainda somos.

As histórias são contadas com a atmosfera de diário, algo que pode servir de inspiração para que as crianças passem a escrever mais sobre suas experiências e aprendam a externalizar sentimentos que não são verbalizados nem colocados no papel. Além disso, situações que permeiam a vida das crianças são colocadas em pauta, como o termo ``esquisito'', corriqueiramente usado para depreciar pessoas que não estão dentro de alguns padrões e acabam sendo vítimas de \textit{bullying}. No livro, o termo é utilizado por algumas das personagens para falarem de si mesmas e de seus amigos, desmistificando o termo e mostrando que estar fora dos padrões pode não ser um problema. 

Os meninos relatam suas fragilidades e sensações na escola e na vida familiar. Essas trajetórias são fundamentais para mostrar aos estudantes que cada pessoa tem sua subjetividade. A leitura do livro permite compreender que as pessoas têm suas próprias questões e sentimentos para lidar e que se abrir para fazer amizades é um passo muito importante para estabelecer relações com o mundo.

A literatura tem um poder muito forte de mostrar às pessoas que existem muitas formas de pensar a vida, muitas histórias diferentes e diversidade em inúmeros sentidos. Em \textit{Figurinha carimbada}, esse papel fundamental da literatura é ponto central, já que as crianças entram em contato com subjetividades que possivelmente nunca vieram à tona em suas vidas. Além disso, a obra promove uma infinidade de atividades das quais a turma poderá desfrutar, além de crescer muito e se divertir. 

Ao longo do manual, todos esses aspectos serão explorados e relacionados a sugestões de atividades. Com isso, apresentamos a seguir ideias e inspirações para um trabalho que pode ser desenvolvido tanto a curto, quanto a médio e longo prazo. Sinta-se à vontade para personalizar a aula e torná-la sua, aplicando seus conhecimentos, sua 
personalidade e aproveite para fortalecer seu vínculo com a turma.
Boa aula!
\end{abstract}

\section{Sobre o livro}

Os contos presentes em \textit{Figurinha carimbada} mostram a vida de seis meninos que narram, em primeira pessoa, suas angústias, sonhos e emoções. De maneira aprofundada, a obra traz à tona as questões infantis que as crianças que não se enquadram nos padrões fixos da sociedade sofrem. 

\SideImage{O livro começa contando a história de Cauê (Falarcao; Domínio público)}{PNLD2023-021-03.png}

\section{Sobre o autor}

Márcio Araújo é um ator, autor, roteirista e diretor brasileiro. Entre seus trabalhos estão textos para mais de 500 episódios do \textit{Cocoricó} da TV Cultura. Participa do grupo de teatro Pocilgas e Cia. Escreveu e dirigiu o Musical \textit{Nara} sobre a vida da cantora Nara Leão, com Fernanda Couto. O musical ganhou o Prêmio Contigo!

\Image{Márcio Araújo é um ator, autor, roteirista e diretor brasileiro (Márcio Araújo; Domínio público)}{PNLD2023-021-04.png}

\section{Sobre o gênero}

\paragraph{O gênero} O gênero deste livro é \textit{conto; crônica; novela}.

Cada um dos capítulos de \textit{Figurinha Carimbada} constitui um conto independente, a respeito das experiências de seus narradores-protagonistas. Algumas personagens participam de mais de um conto. De maneira geral, todos os textos versam a respeito da mesma temática, que confere ao livro sua feição de conjunto.     

\paragraph{Descrição} O que define um gênero narrativo é o fato de, não importa qual seja sua forma, \textit{contar uma história}.
As especificidades do \textit{como} esta história será contada caracteriza os tipos de gênero narrativo, que podem ser: conto, crônica, novela, epopeia, romance ou fábula. 

Toda narrativa possui, necessariamente, um narrador, uma personagem, um enredo, um tempo e um espaço. O narrador, ou narradora, pode ser onisciente, literalmente \textit{que tudo sabe}, observador ou personagem --- categorias que não são exclusivas. O discurso elaborado por este narrador pode ser direto, indireto ou indireto livre --- ou seja, ele pode aparecer mais diretamente ou mais indiretamente; no último caso, sua voz se mistura à das personagens da história.

Sobre o enredo das narrativas curtas sabemos se diz que:

\begin{quote}
comumente são simples, se passam em um espaço único, em um curto período de tempo e apresentam poucas personagens. Os temas giram em torno de episódios do mundo infantil ou de episódios envolvendo animais. As ilustrações ocupam quase toda a página e auxiliam 
a criança a identificar, ma narrativa, as características externas das personagens, as ações vividas por elas e os espaços onde ocorrem as cenas. A linguagem é simples, sem muitos elos frasais. A história se constrói, quase sempre, por meio de diálogos. A presença do narrador é bastante pequena.\footnote{“Narrativas infantis”, de Luiza Vilma Pires Vale. In \textsc{saraiva}, J. A. (Org.) \textit{Literatura e alfabetização: do plano do choro ao plano da ação}. Porto Alegre: Artmed, 2001.}  
\end{quote}

O narrador \textbf{não é necessariamente} a voz do autor. É errada a afirmação de que o autor fala através do narrador de uma história. É bastante comum, há algum tempo na história literária, sobretudo desde os pré-modernistas, que o narrador represente justamente o contrário do que pensa o autor. Neste caso, utiliza-se elementos como a \textbf{ironia} para sugerir que o autor \textit{não é confiável}.

Já as personagens variam quanto a sua \textbf{profundidade}. Há personagens planas, ou personagens-tipo, e personagens redondas, ou complexas. Personagens planas são facilmente repetíveis pois se amparam em lugares-comuns da cultura, como o vilão, o herói, a vítima, o palhaço, tudo isso com marcações de gênero e espécie --- o herói tradicionalmente é um homem, a vítima, uma mulher, e o vilão, uma figura que se afasta da humanidade por alguma razão, às vezes sobrenatural. Personagens redondos, por outro lado, estão mais próximos das \textit{pessoas reais}. Uma personagem complexa pode ser, em um dado momento da narrativa, vilã, e em outro, heroína. É importante notar como as visões de mundo influenciam na caracterização das personagens de uma história.

O tempo de uma narrativa pode ser cronológico ou psicológico. No tempo cronológico, o enredo segue a ordem ``normal'' dos acontecimentos, aquela marcada pelo relógio e pelo calendário. Os acontecimentos vêm um após o outro e se delimita muito bem \textit{passado}, \textit{presente} e \textit{futuro}. Já no tempo psicológico, segue-se uma ordem \textit{subjetiva} dos acontecimentos, e portanto, \textit{não linear}, já que a influência emocional e psíquica afeta a racionalidade do tempo cronológico. 

O espaço, por fim, é o lugar onde se passa a narrativa. Dependendo do caso, ele pode funcionar mais como um plano de fundo, sem muita interferência no enredo, ou mais ativamente, aproximando-se das características das personagens e influenciando no desenrolar da trama. 

\Image{O que define um gênero narrativo é o fato de, não importa qual seja sua forma, contar uma história. (Dorothe/Px Here; Domínio público)}{PNLD2023-007-07.png}

O último aspecto de um gênero narrativo que podemos abordar é sua \textit{extensão}. Dentre os elementos que distinguem um subgênero de outro é o tamanho da história: uma crônica e um conto são \textit{necessariamente} curtos, ao passo que uma epopeia e um romance, são longos. Uma novela está no ponto intermediário entre um romance e um conto. Ainda poderíamos falar dos registros de cada subgênero: a epopeia é originalmente um subgênero \textit{oral}, versificado, e metrificado, já o romance é tradicionalmente \textit{escrito} em prosa.  Desde meados do século \textsc{xviii}, no entanto, o estabelecimento dos gêneros e subgêneros narrativos tornam-se cada vez menos rígido, com as características cada vez mais fluidas e intercomunicativas.

Como o presente livro se trata de uma narrativa \textit{curta}, finalizamos com as palavras de Luiza Vilma Pires a respeito do
subgênero:

\begin{quote}
sob o nome de narrativa curta, estão situadas obras que apresentam uma trama um pouco mais complexa, que ocorre em diversos espaços e em uma temporalidade que pode ser de vários dias, semanas ou meses. Entretanto a função das ilustrações continua as mesmas, são complementares à história e contribuem para sua compreensão. Os temas relacionam-se a vivência infantis (brincadeiras, passeios, pequenas aventuras), a aspectos ligados à interioridade das personagens (busca de identidade, insegurança,  
medos) ou a relações interpessoais (desentendimentos familiares, entre amigos, solidariedade).\footnote{“Narrativas infantis”, de Luiza Vilma Pires Vale. In \textsc{saraiva}, J. A. (Org.) \textit{Literatura e alfabetização: do plano do choro ao plano da ação}. Porto Alegre: Artmed, 2001.} 
\end{quote}

\section{Proposta de Atividades}
\subsection{Pré-Leitura}

A seguir você encontrará a descrição de uma aula-modelo como exemplo prático para a exploração do livro com os estudantes. Esta seção apresentará orientações sobre como organizar a sala de aula para recebê-los, exercitar a interação entre eles e prepará-los para que o momento da leitura tenha mais riqueza de conteúdo. Antes que a leitura comece, é interessante aquecermos a turma e deixá-la com a sensibilidade mais inspirada. Dessa forma, os alunos terão mais recursos para absorver a obra.

\BNCC{EF35LP07}% Utilizar, ao produzir um texto, conhecimentos linguísticos e gramaticais, tais como ortografia, regras básicas de concordância nominal e verbal, pontuação (ponto final, ponto de exclamação, ponto de interrogação, vírgulas em enumerações) e pontuação do discurso direto, quando for o caso.
\BNCC{EF03LP07} %Identificar a função na leitura e usar na escrita ponto final, ponto de interrogação, ponto de exclamação e, em diálogos(discurso direto), dois-pontos e travessão.
\BNCC{EF05LP26}% Utilizar, ao produzir o texto, conhecimentos linguísticos e gramaticais: regras sintáticas de concordância nominal e verbal, convenções de escrita de citações, pontuação (ponto final, dois-pontos, vírgulas em enumerações) e regras ortográficas.
\BNCC{EF15AR01}%Identificar e apreciar formas distintas das artes visuais tradicionais e contemporâneas, cultivando a percepção, o imaginário, a capacidade de simbolizar e o repertório imagético.}
\BNCC{EF15LP18}%Relacionar texto com ilustrações e outros recursos gráficos.}

\paragraph{Tema} Redação. 

\paragraph{Conteúdo} Descrição subjetiva. 

\paragraph{Objetivo} Estimular o exercício criativo da redação por meio de uma descrição de um objeto. Preparar o aluno para os temas explorados em \textit{Figurinha carimbada}. 

\paragraph{Justificativa} A redação é uma forma de expressão da própria identidade e da manutenção da autoestima elevada.  

\paragraph{Metodologia} Peça que cada estudante traga um objeto que o represente, alguma coisa que goste muito ou que tenha participação ativa na sua vida. Na próxima aula, já com os objetos, estimule que cada um conte uma história sobre esse objeto e a importância dele em sua vida. Em seguida, instrua os alunos a escrever, em uma página, o que acabaram de contar, colocando os seus nomes no final. Peça a eles que reparem em sua escrita, provocando-os a estarem atentos às concordâncias nominal e verbal e à pontuação adequada para seus textos. Segue exemplo abaixo:

\textit{Eu trouxe um chaveiro de filtro de barro porque, quando mudaram para filtro de barro na minha casa, fiquei muito triste, achava o gosto da água muito esquisito. Minha mãe dizia que com o tempo esse gosto ia passar, mas eu não acreditava. Até que passou. E hoje, lá de casa eu sou a pessoa que mais gosta da água do filtro de barro, porque ela é fresquinha! Então um dia a gente tava na praia e meu pai viu um moço vendendo esse chaveiro e comprou pra mim.}

Depois que os alunos tiverem escrito a própria história, sugira que eles façam um desenho que a represente. Para o desenho, estimule-os a trabalhar com os materiais disponíveis na escola: lápis, tinta, papéis, diferentes texturas. Depois disso, guarde todos os desenhos juntos. Coloque cinco objetos deles na frente da sala e convide cinco alunos a se aproximarem deles. Entregue a eles os textos de cada objeto correspondente e peça para que leiam como se fossem seus. Inclusive, ao invés de ler o nome que está escrito, falar o próprio nome da criança que está lendo para trazer intimidade na leitura do texto de outra pessoa. A dinâmica tem potencial para fazer com que a turma experimente ouvir um pouco a respeito da história de seus colegas e para que possam externalizar intimidades e trocar com seus colegas. 

\paragraph{Tempo Estimado} Quatro aulas de 50 minutos.

\Image{Depois que cada aluno tiver sua história escrita, sugira que eles façam um desenho que represente essa história (Rede estadual da Bahia; Domínio público)}{PNLD2023-021-05.png}

\subsection{Leitura}

\BNCC{EF03LP12}%Ler e compreender, com autonomia, cartas pessoais e diários, com expressão de sentimentos e opiniões, dentre outros gêneros do campo da vida cotidiana, de acordo com as convenções do gênero carta e considerando a situação comunicativa e o tema/assunto do texto.
\BNCC{EF15LP03}%Localizar informações explícitas em textos.
\BNCC{EF15LP09}% Expressar-se em situações de intercâmbio oral com clareza, preocupando-se em ser compreendido pelo interlocutor e usando a palavra com tom de voz audível, boa articulação e ritmo adequado.
\BNCC{EF15LP10} %Escutar, com atenção, falas de professores e colegas, formulando perguntas pertinentes ao tema e solicitando esclarecimentos sempre que necessário.

\paragraph{Tema} Leitura e compreensão de \textit{Figurinha carimbada}.

\paragraph{Conteúdo} Estrutura de \textit{Figurinha carimbada}: enredo, personagens, tempo, espaço e foco narrativo.  

\paragraph{Objetivo} Ler e compreender os contos de \textit{Figurinha carimbada} e sua estrutura. Promover o gosto pelos livros e pela leitura, estimular o desenvolvimento da linguagem e da criatividade.   

\paragraph{Justificativa} O professor tem grande influência na formação de leitores. É preciso fazer a mediação entre a obra, sua linguagem, suas estruturas, seus pressupostos e os estudantes, de preferência estabelecendo uma relação fundamentada no prazer, na identificação e na liberdade de interpretação. Eis o nosso desafio: ler com os alunos, apresentando as passagens decisivas de um texto, explicando por que elas chamam a atenção e  ouvindo as impressões dos estudantes a respeito de tudo isso. 
\SideImage{Fale sobre os diferentes momentos que o compõem (André Bets; Domínio público)}{PNLD2023-027-07.png}

A escolha da narrativa feita pelo autor Márcio Araújo foi a divisão de cada capítulo pelo nome do narrador-protagonista, que conta a própria história. Cada narrativa retrata de forma sensível o momento da vida pelo qual o narrador está passando. Para realizar essa leitura de forma agradável e que não gere um desconforto para quem está lendo, sugere-se que as carteiras estejam da mesma forma como se posicionam numa sala de aula. O educador pode escolher seis alunas e alunos para lerem do seu próprio lugar a sua parte e sugerir que os outras fechem os olhos para absorver mais a história. Após a leitura, peça que cada pessoa escreva o nome de uma personagem de que mais gostou e divida a turma em grupos pelos nomes escolhidos. Depois de cada capítulo, peça que os estudantes desenhem num papel um objeto que julguem que representa a personagem pela história contada. Findada a leitura, faça uma exposição de cada grupo, mostrando e colando o desenho na frente da sala e pedindo que eles expliquem o motivo da escolha desse objeto com clareza e falem sobre o que acham que a personagem sentiria pelo objeto. Deixe que os alunos usem a imaginação e a criatividade, abrindo a possibilidade, inclusive, de revelarem os motivos pelos quais se identificam com as personagens. 

Através dessa leitura conjunta, pretende-se despertar o interesse do aluno pela identificação com a obra e as crianças que aparecem abrindo seus sentimentos no livro. No final de cada apresentação, deixe que a turma se envolva fazendo perguntas para os colegas, observações sobre as escolhas das crianças e comentando a atividade, sempre relacionando à experiência da leitura. 

\Image{A escolha da narrativa feita pelo autor Márcio Araújo foi a divisão de cada capítulo pelo nome do protagonista (Coc; Domínio público)}{PNLD2023-021-06.png}

Por fim, pode-se explorar mais detidamente os elementos do enredo. Fale sobre os diferentes momentos que o compõem:

\begin{enumerate}

\item A introdução das personagens;
\item Os sentimentos presentes em cada vida que se apresenta;
\item A sensação de exclusão;
\item A importância da socialização com amigos;
\item As fragilidades que uma criança pode enfrentar ao longo de sua vida. 

\end{enumerate}

\paragraph{Tempo Estimado} Quatro aulas de 50 minutos.


\subsection{Pós-leitura}

\BNCC{EF35LP07}% Utilizar, ao produzir um texto, conhecimentos linguísticos e gramaticais, tais como ortografia, regras básicas de concordância nominal e verbal, pontuação (ponto final, ponto de exclamação, ponto de interrogação, vírgulas em enumerações) e pontuação do discurso direto, quando for o caso.
\BNCC{EF04LP05}% Identificar a função na leitura e usar, adequadamente, na escrita ponto final, de interrogação, de exclamação, dois-pontos e travessão em diálogos (discurso direto), vírgula em enumerações e em separação de vocativo e de aposto.}
\BNCC{EF04LP06}% Identificar em textos e usar na produção textual a concordância entre substantivo ou pronome pessoal e verbo (concordância verbal).}
\BNCC{EF03LP13} %Planejar e produzir cartas pessoais e diários, com expressão de sentimentos e opiniões, dentre outros gêneros do campo da vida cotidiana, de acordo com as convenções dos gêneros carta e diário e considerando a situação comunicativa e o tema/assunto do texto.

\paragraph{Tema} Carta. 

\paragraph{Conteúdo} Redação de carta.

\paragraph{Objetivo} Verificar a compreensão da leitura do livro, de forma criativa. 

\paragraph{Justificativa} Enquanto as atividades de leitura, compreensão e análise caracterizam-se pelas primeiras aproximações do texto, seguidas de atividades de descrição de suas características, a prática de redação abre espaço para que os estudantes criem suas próprias narrativas. 

Após a leitura, é interessante promover uma conversa sobre todas as histórias, pedindo que os estudantes identifiquem quais personagens participam de mais de um capítulo e qual é a relação que há entre elas. Peça que os alunos tragam informações detalhadas para verificar o entendimento do livro como um todo. 

Depois dessa discussão, proponha a redação de uma carta para alguma das personagens, contando de sua vida e seus anseios do momento. Incentive os alunos a trazerem elementos que foram vistos no texto para incluir, na carta, a subjetividade da personagem escolhida também. Ao longo da escrita dos alunos, circule entre as mesas, sanando possíveis dúvidas e sugerindo ajustes aos desvios à norma-padrão, explicando os motivos e, quando houver recorrência de um desvio específico, vá até a lousa e retome o conteúdo para a turma. Dessa forma, o educador poderá mapear as dificuldades das crianças em relação às questões gramaticais. 

Concluída a redação dos textos, peça que eles sejam afixados nas paredes da sala para que os alunos possam ler as cartas dos colegas. Deixe-os curtir o momento de leitura e troca entre eles, auxiliando-os a serem gentis em relação ao trabalho dos colegas.

\Image{Após a leitura, é interessante promover uma conversa sobre todas as histórias no geral (Jrmcoaching; Domínio público)}{PNLD2023-027-08.png}

Ainda depois da atividade, sugere-se que o educador proponha que os estudantes relatem as dificuldades que tiveram ao longo da escrita e peça que eles contem, utilizando a terceira pessoa do singular, a história da personagem para quem redigiram a carta, mas do ponto de vista de alguém que possivelmente possa conhecer aquela personagem. 

Estimule-os a, um por vez, se posicionarem na frente da sala e darem início à atividade, criando personagens, ou mesmo utilizando alguma que já exista nas narrativas, para que contem a história com mais propriedade. Por exemplo:

\textit{Meu nome é Maria e eu sou vizinha do Cauê. Eu conheço ele desde criança e estudo na mesma escola que ele. Na hora do recreio, eu até chamo ele pra brincar, mas ele não vem. Ele costuma ficar sozinho, se bem que ultimamente ele tem passado os recreios com um outro menino que eu não conheço...}


Ao final, o professor pode promover um último debate que o livro traz. É interessante questionar os estudantes sobre os motivos pelos quais determinadas pessoas têm mais dificuldade de se entrosar com um grupo e como este grupo pode ajudar uma pessoa a ficar mais próxima, sem deixar ninguém de lado em um ambiente. O interessante, aqui, é que seja iniciada uma conversa sobre o \textit{bullying} e demais violências de ordem social que acabam acontecendo no período da infância e da adolescência. 
\SideImage{Estimule-os a, um por vez, se posicionarem na frente da sala e darem início à atividade (Planneta Educação; Domínio público)}{PNLD2023-027-09.png}

\paragraph{Tempo Estimado} Quatro aulas de 50 minutos.


\section{Sugestões de referências complementares}

\subsection{Livros} 

\begin{itemize}
\item \textsc{andersen}, Hans Christian. \textit{O patinho feio}. São Paulo: Estrela, 2018.

Um conto clássico da literatura infantil universal. Foi escrito no século XIX e conta a história de um patinho diferente que sofre por ter fama de esquisito. 

\item \textsc{munduruku}, Daniel. \textit{Contos indígenas brasileiros}. São Paulo: Global Editora, 2004.

Livro que traz contos dos povos originários e apresenta a diversidade cultural e linguística no Brasil.
\end{itemize}

\section{Bibliografia comentada}
\subsection{Livros}

\begin{itemize}
\item \textsc{brasil}. Ministério da Educação. \textit{Base Nacional Comum Curricular}. Brasília, 2018.

Consultar a \textsc{bncc} é essencial para criar atividades para a turma. Além de especificar quais habilidades precisam ser desenvolvidas em cada ano, é fonte de informações sobre o processo de aprendizagem infantil. 
\SideImage{Ao final, o professor pode promover um último debate que o livro traz (Istockphoto; Domínio público)}{PNLD2023-027-10.png}

\item \textsc{lispector}, Clarice. \textit{Todos os contos}. São Paulo: Rocco, 2016.

Trata-se de uma obra ficcional lançada em 2016 que reúne os contos escritos por Clarice Lispector. 
 
\item \textsc{grimm}, Jacob e Wilhelm. \textit{Contos maravilhosos infantis e domésticos}. São Paulo: Editora 34, 2018.

Inúmeros contos fantásticos que foram publicados inicialmente em 1812. Contos clássicos que originaram diferentes histórias conhecidas no mundo ocidental.

\item \textsc{coelho}, Nelly Novaes. \textit{Literatura infantil, teoria, análise, didática}. 1ª ed. São Paulo: Moderna, 2000.

Livro que fala sobre os espaços da literatura infantil na contemporaneidade e a importância de as crianças estarem ligadas ao seu imaginário pela via literária.

\end{itemize}

\subsection{\textit{Sites}}

\begin{itemize}
\item Artigo \href{https://siteantigo.portaleducacao.com.br/conteudo/artigos/educacao/a-importancia-da-leitura-dos-contos-de-fadas-na-educacao-infantil/30151}{``A Importância da Leitura dos Contos de Fadas na Educação Infantil''}, por Ana Maria da Silva. 

No artigo, a autora discute a importância da construção do imaginário por meio da literatura para as crianças, trazendo elementos que analisam o mundo pós-moderno e os espaços que a literatura infantil, principalmente os contos, devem ter.

\item Artigo \href{http://docs.uninove.br/arte/fac/publicacoes/pdf/v3-n1-2012/Francy.pdf}{``Literatura infantil: A contribuição dos contos de fadas para a construção do imaginário infantil''}. 

\item Vídeo \href{https://www.youtube.com/watch?v=kqUU9f6dX8E}{“Marcelino Freire -- Encontros literários”}, da página \textit{Canal Arte 1}. 

Vídeo que mostra o autor Marcelino Freire falando sobre sua produção de contos e a importância deles para a inserção no universo literário.
\end{itemize}

\subsection{\textit{Filmes}}

\begin{itemize}
\item \textit{O menino e o mundo}. Dirigido por Alê Abreu, 2014.

Um menino mora com os pais em uma pequena cidade do campo. Diante da falta de trabalho, um dia, ele vê o pai partindo para a cidade grande. Os dias que se seguem são tristes e de memórias confusas para o garoto. Até que ele faz as malas, pega o trem e vai descobrir o novo mundo em que seu pai mora. Para a sua surpresa, a criança encontra uma sociedade marcada pela pobreza, exploração de trabalhadores e falta de perspectivas.

\item \textit{Diário de um banana}. Dirigido por Thor Freudenthal e David Bowers, 2010. 

A escola não é uma experiência agradável para o quase adolescente Greg Heffley, mas sim um campo minado que ele precisa enfrentar. Para isso, ele trama para ganhar o reconhecimento que acha que merece, mas tudo dá errado. Greg escreve um diário sobre as suas desventuras, pensamentos e opiniões para divulgar no dia em que não precisar mais se submeter a absurdos.

\item \textit{Divertidamente}. Dirigido por Pete Docter, 2015.

O filme gira ao redor de Riley, nascida em Minnesota, e as emoções que estão dentro da sua cabeça (e da cabeça de todas as pessoas), e eles são: Alegria, Tristeza, Nojo (aversão), Medo e Raiva. As emoções vivem na Sede, como é chamada a mente consciente de Riley, onde eles influenciam nas ações e nas memórias de Riley através de um painel de controle.

\end{itemize}
\end{document} 