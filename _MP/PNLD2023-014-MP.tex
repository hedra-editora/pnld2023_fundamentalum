\documentclass[11pt]{extarticle}
\usepackage{manualdoprofessor}
\usepackage{fichatecnica}
\usepackage{lipsum,media9}
\usepackage[justification=raggedright]{caption}
\usepackage[one]{bncc}
\usepackage[maisemelhores]{../edlab}
\usepackage{marginnote}
\usepackage{pdfpages}

\newcommand{\AutorLivro}{Machado de Assis}
\newcommand{\TituloLivro}{Ideias de canário}
%\newcommand{\Tema}{}
\newcommand{\Genero}{Conto; crônica; novela}
%\newcommand{\imagemCapa}{./images/PNLD2022-001-01.jpeg}
\newcommand{\issnppub}{XXX-XX-XXXXX-XX-X}
\newcommand{\issnepub}{XXX-XX-XXXXX-XX-X}
% \newcommand{\fichacatalografica}{PNLD0001-00.png}
\newcommand{\colaborador}{Ana Lancman}
\begin{document}

\title{\TituloLivro}
\author{\AutorLivro}
\def\authornotes{\colaborador}

\date{}
\maketitle

%\begin{abstract}\addcontentsline{toc}{section}{Carta ao professor}
%\pagebreak

\tableofcontents

\section{Carta ao professor}

O presente manual tem como objetivo oferecer uma orientação ao professor sobre a obra \textit{Ideias de canário}. A partir deste manual, os professores poderão incentivar a prática da leitura aos estudantes e proporcionar um conteúdo enriquecedor. Apresentamos aqui sugestões de atividades a serem realizadas antes, durante e após a leitura do livro, com propostas que buscam introduzir os gêneros literários e aprofundar as discussões trazidas pelas obras. Você encontrará informações sobre o autor, sobre o gênero e sobre os temas trabalhados ao longo do livro. Ao fim do manual, você encontrará também sugestões de livros, artigos e sites selecionados para enriquecer a sua experiência de leitura e, consequentemente, a de seus estudantes.

O conto \textit{Ideias de canário}, de Machado de Assis, revela em uma narrativa breve a sagacidade de um dos mais importantes autores brasileiros. Com traços de fábula, a história apresenta um canário, preso em uma gaiola, que é encontrado por um estudioso de pássaros em um brechó. Ao descobrir que o canário fala, o homem fica encantado e o compra. Em cada momento da história, o canário narra sua perspectiva de mundo, medida pela realidade que o cerca. O conto está repleto de elementos que incitam reflexões sobre ampliar seus horizontes e sentir-se livre.

A partir dessa obra, será possível trabalhar \textsc{bncc}'s de diversas áreas, desde Artes, Geografia e História, além de Língua Portuguesa. É um texto que abre margem para um debate sobre a relação dos humanos com a natureza e a importância da reciclagem de objetos. Nas atividades serão exercitadas as habilidades artísticas manuais e encenações teatrais. Esperamos que as atividades sugeridas e o material indicado sejam proveitosos em sala de aula! 

\section{Sobre o livro}

\paragraph{O livro} \textit{Ideias de canário}, de Machado de Assis, é um conto publicado originalmente em 1899. Nessa versão adaptada para um livro ilustrado, a narrativa ganha uma vivacidade que a aproxima dos leitores de hoje em dia. O texto usa um irônico e bem-humorado para tratar de temas existenciais, como a sensação de liberdade e visões de mundo.

\paragraph{Descrição} O conto apresenta Macedo, que é um ornitólogo, um homem dedicado a estudar pássaros. Ele encontra em um brechó um canário falante. O ornitólogo fica fascinado pelo pássaro e decide comprá-lo. Em seguida, compra uma gaiola maior e a coloca na varanda de sua casa. Passa a estudar sobre o canário de forma quase obsessiva. Os dois conversam por horas a fio e a cada mudança na vida do canário, este relata uma visão de mundo diferente da anterior. Percebemos que, conforme o cenário muda, o pássaro percebe o mundo de acordo com os elementos presentes a seu redor e "esquece" do resto. O ápice da situação é quando o canário foge e é encontrado em uma árvore. Quando Macedo o pergunta se ele gostaria de voltar para sua vasta gaiola, o pássaro afirma que o mundo é um infinito espaço azul, com o sol por cima --- e nada mais.

\section{Sobre os autores}

\paragraph{O autor} Joaquim Maria Machado de Assis nasceu em 1839 no Rio de Janeiro. É frequentemente considerado o maior escritor
brasileiro de todos os tempos. Ele escreveu em diversos gêneros literários: poesia, romance, crônica, dramaturgia, conto e folhetim. Trabalhou como jornalista no \textit{Diário do Rio de Janeiro} de 1860 a 1867 e colaborou para o \textit{Jornal das Famílias} sob o uso de pseudônimos. Dentre seus romances mais conhecidos, estão \textit{Memórias póstumas de Brás Cubas} (1881), \textit{Quincas Borba} (1891) e \textit{Dom Casmurro} (1899). Foi um dos fundadores da Academia Brasileira de Letras em 1897 e também o seu primeiro presidente.

\paragraph{A ilustradora} Luana Geiger nasceu em São Paulo, em 1974. Formou-se em Arquitetura pela Universidade de São Paulo (\textsc{usp}) e em pedagogia e artes visuais pelo Claretiano com especialização em Mídias Interativas pelo Senac. Tem diversos livros publicados. Em 1999, recebeu prêmio de melhor trabalho de artes plásticas no Projeto Nascente 9 do Centro Universitário Maria Antônia. Ela conta que desde a primeira vez que leu \textit{Ideias de canário} já imaginou as ilustrações. A escolha em colorir apenas uma parte de cada página do livro veio do intuito de ressaltar as diferentes percepções da realidade, tal como Machado de Assis explora no conto.

\SideImage{Retrato de Machado de Assis (Marc Ferrez; Domínio público)}{PNLD2023-014-02.png}

% \SideImage{Foto da ilustradora (Arquivo pessoal)}{PNLD2023-014-03.png}

\section{Sobre o gênero}

\paragraph{O gênero} O gênero deste livro é \textit{conto; crônica; novela}.

\paragraph{Descrição} O que define um gênero narrativo é o fato de, não importa qual seja sua forma, \textit{contar uma história}.
As especificidades do \textit{como} esta história será contada caracteriza os tipos de gênero narrativo, que podem ser: conto, crônica, novela, epopeia, romance ou fábula. 

Toda narrativa possui, necessariamente, um narrador, uma personagem, um enredo, um tempo e um espaço. O narrador, ou narradora, pode ser onisciente, literalmente \textit{que tudo sabe}, observador ou personagem --- categorias que não são exclusivas. O discurso elaborado por este narrador pode ser direto, indireto ou indireto livre --- ou seja, ele pode aparecer mais diretamente ou mais indiretamente; no último caso, sua voz se mistura à das personagens da história.

Sobre o enredo das narrativas curtas sabemos que

\begin{quote}
comumente são simples, se passam em um espaço único, em um curto período de tempo e apresentam poucas personagens. Os temas giram em torno de episódios do mundo infantil ou de episódios envolvendo animais. As ilustrações ocupam quase toda a página e auxiliam 
a criança a identificar, ma narrativa, as características externas das personagens, as ações vividas por elas e os espaços onde ocorrem as cenas. A linguagem é simples, sem muitos elos frasais. A história se constrói, quase sempre, por meio de diálogos. A presença do narrador é bastante pequena.\footnote{“Narrativas infantis”, de Luiza Vilma Pires Vale. In \textsc{saraiva}, J. A. (Org.) \textit{Literatura e alfabetização: do plano do choro ao plano da ação}. Porto Alegre: Artmed, 2001.}  
\end{quote}

O narrador \textbf{não é necessariamente} a voz do autor. É errada a afirmação de que o autor fala através do narrador de uma história. É bastante comum, há algum tempo na história literária, sobretudo desde os pré-modernistas, que o narrador represente justamente o contrário do que pensa o autor. Neste caso, utiliza-se elementos como a \textbf{ironia} para sugerir que o autor \textit{não é confiável}.

Já as personagens variam quanto a sua \textbf{profundidade}. Há personagens planas, ou personagens-tipo, e personagens redondas, ou complexas. Personagens planas são facilmente repetíveis pois se amparam em lugares-comuns da cultura, como o vilão, o herói, a vítima, o palhaço, tudo isso com marcações de gênero e espécie --- o herói tradicionalmente é um homem, a vítima, uma mulher, e o vilão, uma figura que se afasta da humanidade por alguma razão, às vezes sobrenatural. Personagens redondos, por outro lado, estão mais próximos das \textit{pessoas reais}. Uma personagem complexa pode ser, em um dado momento da narrativa, vilã, e em outro, heroína. É importante notar como as visões de mundo influenciam na caracterização das personagens de uma história.

O tempo de uma narrativa pode ser cronológico ou psicológico. No tempo cronológico, o enredo segue a ordem ``normal'' dos acontecimentos, aquela marcada pelo relógio e pelo calendário. Os acontecimentos vêm um após o outro e se delimita muito bem \textit{passado}, \textit{presente} e \textit{futuro}. Já no tempo psicológico, segue-se uma ordem \textit{subjetiva} dos acontecimentos, e portanto, \textit{não linear}, já que a influência emocional e psíquica afeta a racionalidade do tempo cronológico. 

O espaço, por fim, é o lugar onde se passa a narrativa. Dependendo do caso, ele pode funcionar mais como um plano de fundo, sem muita interferência no enredo, ou mais ativamente, aproximando-se das características das personagens e influenciando no desenrolar da trama. 

\Image{O gênero conto é uma história curta, normalmente centrada em apenas um núcleo narrativo. (Dorothe/Px Here; Domínio público)}{PNLD2023-014-07.png}

O último aspecto de um gênero narrativo que podemos abordar é sua \textit{extensão}. Dentre os elementos que distinguem um subgênero de outro é o tamanho da história: uma crônica e um conto são \textit{necessariamente} curtos, ao passo que uma epopeia e um romance, são longos. Uma novela está no ponto intermediário entre um romance e um conto. Ainda poderíamos falar dos registros de cada subgênero: a epopeia é originalmente um subgênero \textit{oral}, versificado, e metrificado, já o romance é tradicionalmente \textit{escrito} em prosa.  Desde meados do século \textsc{xviii}, no entanto, o estabelecimento dos gêneros e subgêneros narrativos tornam-se cada vez menos rígido, com as características cada vez mais fluidas e intercomunicativas.

O presente livro se trata de um \textbf{conto}, que é uma história curta, com poucos personagens e centrada em apenas um núcleo narrativo. Geralmente, o conto tem elementos lúdicos e utiliza expressões metafóricas, com o intuito de retratar o cotidiano de maneira poética e subjetiva. O texto também possui características da fábula. A fábula veio do conto, mas se diferencia pela centralidade de personagens animais e pelo intuito de concluir a história com um ensinamento moral.

\subsection{Pré-leitura}

\BNCC{EF02GE04}
%Identificar +Diferenciar: “hábitos e relações com a natureza em diferentes lugares”
\BNCC{EF02HI04}
%Identificar +Perceber: “significado de objetos e documentos pessoais como memória”
\BNCC{EF02HI05}
%Identificar +Perceber: “função e significado de objetos e documentos pessoais”
\BNCC{EF02HI09}
%Identificar: “preservação ou descarte de objetos/documentos pessoais”
\BNCC{EF03GE08}
%Identificar: lixo doméstico e consumo excessivo;

\paragraph{Tema} Conceitos-chave de \textit{Ideias de canário}.

\paragraph{Conteúdo} Conversa em sala de aula sobre a profissão de ornitólogo, sobre a origem dos brechós e a importância da obra de Machado de Assis.

\paragraph{Objetivo} Introduzir conceitos importantes para a compreensão da obra e apresentar a vida de Machado de Assis.

\paragraph{Justificativa} O conto \textit{Ideias de canário} foi publicado originalmente em 1889 e tem uma linguagem que pode ser desafiadora. No texto estão presentes alguns termos que não são comumente usados hoje em dia e cujo significado é importante para compreender a narrativa. Portanto, nesta atividade, estes conceitos deverão ser trabalhados, assim como os temas de debate que os envolvem. Além disso, é essencial que os estudantes conheçam o valor da obra de Machado de Assis. A relevância dos temas tratados pelo autor e seu estilo literário marcante são elementos que podem incentivar a leitura do conto. 

\paragraph{Metodologia} Primeiramente, trabalhe com os estudantes o conceito de \textbf{ornitologia}. Informações sobre este termo podem ser encontradas no \href{https://www.infoescola.com/biologia/ornitologia/}{site da Infoescola}. É possível convidar o professor de Ciências da Natureza para participar da atividade e contribuir com o conteúdo específico da área.

\Image{Um ornitólogo é quem estuda pássaros. (Kardoslavik; CC-BY-SA-4.0)}{PNLD2023-014-08.png}

Sugestões de perguntas a serem feitas para os alunos:

\begin{itemize}

\item Vocês sabem o que faz um ornitólogo?

\item Quais espécies de pássaros vocês conhecem?

\item Por que é importante estudar os pássaros?

\item Vocês sabiam que o Brasil é o terceiro país com a maior diversidade de aves por área?

\item Devemos prender os pássaros em gaiolas?

\end{itemize}

Em seguida, será abordado o conceito de \textbf{belchior}, que significa "brechó". O termo tem origem no século \textsc{xix}, devido a Belchior que era um mascate --- comerciante que percorre as ruas e vende produtos para as pessoas, muitas vezes em seu domicílio. Ele se tornou conhecido no Rio de Janeiro por vender roupas e objetos de segunda mão. Com o tempo, o nome se transformou em "brechó". Um dos registros históricos de uso desse termo foi o conto de Machado de Assis. Apresente este breve histórico para os alunos.

\Image{O brechó é um espaço que pode contribuir para o consumo consciente e reutilização de roupas e objetos. (MAKY.OREL; CC-BY-SA-4.1)}{PNLD2023-014-09.png}

Sugestões de questões para a sala de aula:

\begin{itemize}

\item Vocês sabem o que é um brechó? Já frequentaram?

\item Qual é a importância de ter uma loja que venda objetos usados atualmente?

\item Você se sente apegado a suas coisas?

\item O que você costuma fazer com os objetos que não usa mais?

\item O brechó pode ajudar no processo de reciclagem?

\end{itemize}

Finalmente, faça uma apresentação de Machado de Assis. Explique aos estudantes qual é a sua importância na literatura brasileira. Introduza uma breve biografia de Machado e apresente também suas obras mais importantes. Pode ser interessante recuperar as mudanças na valorização do autor e como a representação de sua figura mudou ao longo do tempo. 

Sugere-se que seja exibido o episódio \textit{Machado de Assis}, que faz parte da série Pequenos Ilustres. É um desenho animado curto que apresenta de forma divertida a vida de Machado. Pode ser acessado gratuitamente no \href{https://youtu.be/ahafgLr7M0I}{Youtube}.

\paragraph{Tempo estimado} Três aulas de 50 minutos.

\subsection{Leitura}

\BNCC{EF01LP26}
%Identificar: personagens, enredo, tempo e espaço.
\BNCC{EF15LP16} 
%Ler: narrativas, contos, crônicas; +grupo, +professor e +sozinho; Mundo imaginário;
\BNCC{EF15LP04}
%Identificar: "interpretação de imagem", recursos gráfico-visuais, multissemióticos;
\BNCC{EF15LP11} 
%Identificar: +Falar: conversação; "roda de conversa", "discussão";
\BNCC{EF15AR04}
%Produzir: desenho, pintura, colagem, quadrinhos, dobradura, escultura, modelagem...; 

\paragraph{Tema} O enredo de \textit{Ideias de canário}

\paragraph{Conteúdo} Leitura em sala de aula e debate sobre as principais questões trazidas pelo conto.

\paragraph{Objetivo} Incentivar os estudantes a identificar os diferentes momentos da narrativa e ampliar seu entendimento sobre as questões trazidas pela obra.

\paragraph{Justificativa} A leitura de \textit{Ideias de canário} pode trazer alguns desafios no entendimento dos alunos acerca da história. Uma leitura conjunta em sala de aula, seguida de debate acerca da narrativa, será essencial para aprofundar a compreensão do conto.

\paragraph{Metodologia} Nessa atividade, será realizada a leitura integral do conto em sala de aula, com a intermediação do professor. Primeiramente, leia o conto de forma conjunta, convidando alunos que se voluntariem para ler algumas páginas. É possível que haja uma pausa nas ilustrações, que são recheadas de detalhes a serem observados. A leitura pode trazer alguns desafios na compreensão dos alunos e apresenta uma grande quantidade de informações por página. Sugere-se que, ao ser finalizada essa primeira leitura, seja realizada uma nova leitura individual. Para esse segundo momento de leitura, o professor pode propor aos estudantes que anotem quais termos são desconhecidos para eles. Peça também que prestem atenção em algumas questões, como as personagens que aparecem e quais os distintos momentos e espaços da narrativa.

A partir da nova leitura, faça perguntas aos alunos que fomentem o debate sobre o conto e ampliem a compreensão de leitura.
Sugestão de questões para a sala de aula:

\begin{itemize}

\item Como a história começa? 

\item Quem conta a história?

Releia o trecho inicial para apontar a diferenciação entre o narrador em primeira e terceira pessoa:

\begin{quote}Um homem dado a estudos de ornitologia, por nome Macedo, referiu a alguns amigos um caso tão extraordinário que ninguém lhe deu crédito.
Alguns chegam a supor que Macedo virou o juízo. Eis aqui o resumo da narração.
No princípio do mês passado, — disse ele, — indo por uma rua, sucedeu que um tílburi à disparada, quase me atirou ao chão. (...)\footnote{Página 6 de \textit{Ideias de canário}}\end{quote} 

\item O que existe nesse conto que pode ser considerado inusitado? 

\item Quais os elementos que podemos apreender da conversa de Macedo com o canário?

\item Quais os símbolos trazidos pela figura do pássaro?

\end{itemize}

\Image{Ilustração do livro, página 12}{PNLD2023-014-04.png}

Em seguida, realize uma discussão com os estudantes sobre as falas do canário. Deverão ser abordadas suas diferentes perspectivas de mundo ao longo da narrativa. Ao final, o que representa a dúvida do canário? Relembre aos alunos o fato de que, depois de tomar contato com a amplitude do azul do céu, o canário chega a duvidar de que houvesse brechós. Proponha que cada aluno registre em um desenho que simbolize sua própria interpretação de qual a mensagem que o conto transmite. Cada estudante apresentará seu desenho para o restante da sala. Poderá ser feita uma conversa livre sobre os elementos que mais apareceram e os que foram únicos em cada desenho.

\Image{Ilustração do livro, página 38 (; )}{PNLD2023-014-05.png}

Nesse momento, o professor poderá trazer algumas questões centrais para o debate, como:

\begin{itemize}

\item Vocês reconhecem o tom com o que o pássaro se comunica no conto? 

\item É possível interpretar que o canário é arrogante? 

\item A postura do pássaro pode estar relacionada à restrição de perspectiva a que ele é submetido?

\item Por que o canário esquece da vida anterior que tinha com a mudança de cenário?

\item Como pode ser interpretada a vida fora da gaiola?

\item Como a forma em que percebemos os elementos ao nosso redor muda a nossa visão de mundo?

\end{itemize}

Por fim, retome o seguinte trecho com os alunos:

\begin{quote}Também a linguagem sofreu algumas retificações, e certas conclusões, que me tinham parecido simples, vi que eram temerárias, Não podia ainda escrever a memória que havia de mandar ao Museu Nacional, ao Instituto Histórico e às universidades alemãs, não porque faltasse matéria, mas para acumular primeiro todas as observações e ratificá-las.\footnote{Página 24 de \textit{Ideias de canário}}\end{quote} 

\Image{Ilustração do livro, página 24}{PNLD2023-014-06.png}

Pergunte aos estudantes se sabem o que são o Museu Nacional e o Instituto Histórico e Geográfico Brasileiro. Eles existem até hoje? Faça uma breve apresentação dessas instituições e sua importância no contexto brasileiro.

\paragraph{Tempo estimado} Três aulas de 50 minutos.

\subsection{Pós-leitura}

\BNCC{EF02LP27}
%Produzir +Reescrever: textos narrativos "com suas próprias palavras
\BNCC{EF15LP05}
%Produzir: planejar "partes do texto", para quem escreve, tema, "adequação";
\BNCC{EF15AR20}
%Produzir: improvisações teatrais;
\BNCC{EF15AR21}
%Produzir: imitação e o faz de conta; encenar com música, imagem; 

\paragraph{Tema} Conto e fábula.

\paragraph{Conteúdo} Estudo sobre o gênero fábula, redação de uma cena e apresentação teatral.

\paragraph{Objetivo} Apresentar os traços de fábula presentes no conto e aprimorar a prática da escrita, além de incentivar a expressão artística da linguagem teatral.

\paragraph{Justificativa} Este conto tem um traço fundamental de fábula. As fábulas são composições literárias curtas em que animais aparecem como personagens, com qualidades humanas. Além disso, oferecem uma moral ao final do texto, uma mensagem com caráter educativo. A ideia é que os estudantes possam reconhecer estes traços no conto \textit{Ideias de canário} e também aprenderem sobre o gênero. Ao reescrever a moral da história com suas próprias palavras, os estudantes poderão aprofundar seu conhecimento sobre a obra. Além disso, na convivência em grupo, poderão exercer o trabalho em equipe e praticar a expressão corporal por meio do teatro. 

\paragraph{Metodologia} Apresente aos alunos as características principais da fábula e suas intersecções com o gênero \textbf{conto}.

A partir dessa introdução, proponha aos estudantes que se dividam em grupos pequenos.

\begin{itemize}

\item Cada grupo deverá redigir uma moral da história segundo o que interpretaram no conto \textit{Ideias de canário}.

\item Em seguida, poderão adaptar essa moral para uma nova fábula. Esta fábula deve apresentar um animal como personagem, mas não poderá ser um pássaro. É um momento dos alunos usarem sua criatividade, a partir dos temas discutidos na atividade de pré-leitura e leitura.

\item A partir do texto dessa nova fábula, cada grupo montará uma breve encenação teatral. Poderão utilizar materiais artísticos disponíveis na escola ou trazer de casa para montar um cenário e figurino. Acompanhe os estudantes, com o auxílio do professor de Artes, e oriente no que for necessário. 

\item Cada grupo apresentará para o restante a sua encenação teatral. Peça que os estudantes colaborem coletivamente para que adaptar os espaços em sala de aula ou o local da escola escolhido para as apresentações. 

\end{itemize}

\Image{Os alunos vão se dividir em grupos e montar uma apresentação teatral. (Lisandrajorge; CC-BY-SA-3.0)}{PNLD2023-014-10.png}

Por fim, reúna todos os alunos para um debate sobre os temas abordados nas cenas e as mensagens finais escolhidas por cada grupo. É interessante recuperar as questões trazidas pelo conto de Machado de Assis e propor um debate relacionando as distintas formas literárias e suas múltiplas interpretações.

\paragraph{Tempo estimado} Quatro aulas de 50 minutos.

\section{Sugestão de referências complementares}

\subsection{Audiovisual}

\begin{itemize}

\item \textit{Brechó: Casa de Belchior}.

No vídeo, produzido pela Tocaya TV, é apresentada a história dos brechós e sua relação com o termo Belchior. Também apontam a relação com o conto \textit{Ideias de canário}, de Machado de Assis. Pode ser um material complementar para a atividade de pré-leitura. É possível assistir gratuitamente no \href{https://youtu.be/mGhzDNLkiV4}{Youtube}.

\item \textit{Profissão ornitólogo}.

Vídeo detalhado sobre a profissão de quem estuda aves. Também pode ser um material complementar para a atividade de pré-leitura. Pode ser acessado no \href{https://youtu.be/2swKMwcnYTM}{Youtube}.

\item \textit{Um canibal nos trópicos}. Dirigido por Edson Martins, 2019.

O documentário, realizado com apoio da Universidade Federal de Viçosa, apresenta a vida e obra de Machado de Assis.

\end{itemize}

\subsection{Links}

\begin{itemize}

\item \href{https://www.academia.org.br/academicos/machado-de-assis/biografia}{Biografia de Machado de Assis}

Artigo da Academia Brasileira de Letras (\textsc{abl}) sobre a vida e obra de Machado de Assis.

\item \href{https://artsandculture.google.com/partner/museu-nacional-ufrj}{Museu Nacional no Google Arts}

Visite o acervo do Museu Nacional da Universidade Federal do Rio de Janeiro (\textsc{ufrj}) através de um \textit{tour online} com visões em 360º e fotos de obras em alta qualidade.

\item \href{https://www.wikiaves.com.br/}{WikiAves}

Maior portal brasileiro de registros de aves, com fotos, nomes detalhados das espécies e mapas que marcam os locais em que foram encontrados. Os internautas podem escrever sobre sua experiência observando a ave. Também é possível escutar gravações dos sons das aves.

\end{itemize}

\subsection{Museus}

\begin{itemize}

\item Museu da Língua Portuguesa

Localizado no centro histórico da cidade de São Paulo, o museu foi reformado e reaberto em 2021 e busca valorizar a diversidade da língua portuguesa. O museu é interativo, com uso de tecnologia nas exposições temporárias e permanentes.

\item Museu de Zoologia do Memorial do Cerrado

O maior museu de ornitologia do mundo ficava no Brasil, em Goiânia. Foi fundado em 1968 e possuía diversas peças pré-históricas. Foi fechado em 2018 e parte do seu acervo foi transferido para o Museu de Zoologia da PUC Goiás.

Endereço: Av. Engler, s/n - Jardim Mariliza, Goiânia - GO, Brasil.

\end{itemize}

\section{Bibliografia comentada}

\begin{itemize}

\item \textsc{calvino}, Ítalo. \textit{Por que que ler o clássicos?}. São Paulo: Companhia das Letras, 2007.

Neste livro o pensador italiano Ítalo Calvino oferece diversas respostas para definir o que é um clássico e os motivos de continuarmos a lê-los.

\item \textsc{duarte}, Eduardo de Assis. \textit{Machado de Assis Afrodescendente: Antologia e Crítica}. Rio de Janeiro: Malê, 2020.

A obra abrange poemas, contos, crônicas, crítica textual de Machado de Assis. O livro traz também um conjunto de seis ensaios críticos sobre a obra machadiana, na perspectiva de sua ancestralidade negra.

\item \textsc{japiassu}, Ricardo. \textit{Metodologia do ensino de teatro}. Campinas: Papirus, 2003.

Obra voltada para o ensino de práticas teatrais de 1º a 4º ano do Ensino Fundamental.

\item \textsc{lucas}, Fábio. \textit{O núcleo e a periferia de Machado de Assis}. São Paulo: Amarilys, 2009.

Nesta obra estão reunidos ensaios do escritor, professor e crítico mineiro Fábio Lucas, que estuda a obra de Machado de Assis há quarenta anos.

\item \textsc{rodrigues}, Henrique. \textit{Machado de Assis menino}. Rio de Janeiro: Malê, 2021.

Livro infantil inspirado na biografia de Machado de Assis.

\item \textsc{tufano}, Douglas. \textit{Esopo --- Fábulas completas}. São Paulo: Moderna, 2015.
 
Essa reunião de fábulas apresenta uma introdução sobre o gênero e notas explicativas.

\end{itemize}

\end{document}
