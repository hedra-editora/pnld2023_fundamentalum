\documentclass[11pt]{extarticle}
\usepackage{manualdoprofessor}
\usepackage{fichatecnica}
\usepackage{lipsum,media9}
\usepackage[justification=raggedright]{caption}
\usepackage[one]{bncc}
\usepackage[araucaria]{../edlab}
\usepackage{marginnote}
\usepackage{pdfpages}

\newcommand{\AutorLivro}{Denise Natale e Tatiane Moreira Lima}
\newcommand{\TituloLivro}{O corpo é meu, ninguém põe a mão}
\newcommand{\Genero}{Conto; crônica; novela}
%\newcommand{\imagemCapa}{./images/PNLD2022-001-01.jpeg}
\newcommand{\issnppub}{XXX-XX-XXXXX-XX-X}
\newcommand{\issnepub}{XXX-XX-XXXXX-XX-X}
% \newcommand{\fichacatalografica}{PNLD0001-00.png}
\newcommand{\colaborador}{Renier Silva}

\begin{document}

\title{\TituloLivro}
\author{\AutorLivro}
\def\authornotes{\colaborador}

\date{}
\maketitle

\tableofcontents

\section{Carta ao professor}

Caros professores e professoras,

esperamos, com este material,
auxiliá-los no trabalho com o \textbf{Ensino Fundamental \textsc{ii}} em 
sala de aula. \textit{O corpo é meu, ninguém põe a mão}, de Denise Natale e Tatiane Moreira Lima, é um livro singular
por vários motivos e possibilita atividades didáticas interessantíssimas,
como vocês acompanharão no decorrer do manual.

Trata-se de um livro que trata de um tema muito delicado para toda a sociedade: o abuso infantil. 
Trabalhar com ele em sala de aula requer grande sensibilidade e atenção da parte dos adultos. 
Vocês conhecerão a história de uma gatinha chamada Estrela que é vítima de uma raposa, o Lupi Lantra,
que tem segundas intenções em sua interação com ela. Esta personagem usa de abuso em sua forma verbal
por meio de ameaças e seduções em relação à garota. 

Com o auxílio dos profissionais de sua escola, a gatinha Estrela consegue 
contar aos adultos o que estava acontecendo e estes tomam as iniciativas
devidas: Lupi Lantra é acusado e preso. 

Dada a importância do ato de falar sobre si mesmo, propomos neste manual diversas atividades que circundam
este tema. Queremos incentivar, nas crianças, a familiaridade com o exercício da 
narração de suas próprias histórias. Acreditamos no poder constitutivo do sujeito
que provém desta atividade, que capacita o indivíduo a garantir sua autonomia 
na sociedade.

Esperamos, professores, que este material sirva como um guia 
para seu trabalho em sala de aula. Já contamos, no entanto, com as adaptações
que surgirão organicamente na recepeção do mesmo por vocês, que possuem 
trajetórias e escolhas didáticas específicas, bem como no contato com os 
alunos, que tanto têm a oferecer para o enriquecimento da experiência didática.

Boa aula!


\section{Sobre o livro}

\textit{O corpo é meu, ninguém põe a mão} é um livro composto por três pessoas,
duas escritoras e uma ilustradora. Ele conta a história de Estrela, uma gatinha 
que gosta de praticar atividades físicas, principalmente o salto. Tudo corre bem
em sua vida de criança, até que uma raposa adulta, Lupi Lantra, aparece em sua casa e começa
a investir maliciosamente na gatinha. 

Lupi seduz Estrela com presentes e elogios e a convida para sua casa. Lá,
propõe atividades incomuns como tirar a roupa e tirar fotos da gatinha.
Ela logo percebe que há algo estranho, mas, sob ameaças da raposa, 
é forçada a guardar segredo. 

Esta situação afeta a saúde de Estrela, que passa a perder o apetite e não
prestar mais atenção nas aulas. Estas alterações ficam evidentes nas ilustrações 
da história, com a presença dos fantasmas da raposa ao redor de sua cabeça e
de seu corpo. 

Por fim, com a ajuda de sua professora e diretora da escola, a gatinha
Estrela consegue colocar para fora as suas inquietações que estavam
lhe consumindo por dentro. Os adultos tomam uma previdência junto à justiça,
a raposa Lupi Lantra é presa, e Estrela finalmente segue sua vida,
tornando-se uma grande e bem sucedida atleta, o que era seu sonho. 

\paragraph{Violência e abuso infantil} \textit{O corpo é meu, ninguém põe a mão} trabalha uma
temática muito importante no quotidiano das crianças, a violência e o abuso enfantis.
Infelizmente, histórias como a da gatinha Estrela e da raposa Lupi Lantra são
muito comuns no Brasil e no resto do mundo. Adultos criminosos, representados
na história pela raposa, se aproveitam da incocência e da falta de cuidado
das crianças para violentarem seus corpos e suas vidas. 
Em geral se utilizam de técnicas como a \textit{sedução}, primeiro, e então 
a \textit{ameaça}, que imepede a criança de se comunicar com outros adultos
a respeito da violência que está sofrendo. 

A cada dia, cerca de 100 crianças e adolescentes até 14 anos são abusados sexualmente, segundo o Ministério da Saúde e o Fórum Brasileiro de Segurança Pública. Esse crime acontece em todas as camadas sociais e em 90\% das vezes é praticado dentro de casa, seja por um familiar, um amigo ou conhecido. As vítimas preferenciais são meninas, 85\%, especialmente entre 6 e 10 anos. O abusador, em geral, é do sexo masculino.

Como vemos na história, é muito importante que a criança aprenda desde 
cedo que ela tem direito sobre seu próprio corpo, e que ninguém tem o 
direito de violá-lo. Para isso, assim que perceber alguma situação estranha 
com um adulto, deve comunicar a alguém de sua confiança, seja do ambiente
familiar ou do ambiente educacional. As delegacias de polícia e departamentos
específicos para casos de violência infantil existem para auxiliar a criança e 
seus familiares nestas situações. Números como o \textbf{Disque 100}, 
o \textbf{Disque 191} e o \textbf{Disque 181} estão 24 horas em alerta
para atender possíveis vítimas que precisem de ajuda e proteção. 
Além disso, o \textbf{Conselho Tutelar} e as \textbf{Varas da Infância e da Juventude},
presentes em todas as cidades do Brasil, podem ser acionadas nestes casos. 

\section{Sobre o autor}

\paragraph{As escritoras} \textbf{Denise Natale} é formada em Jornalismo pela Fundação
Armando Álvares Penteado (\textsc{faap}), trabalha com edição de livros na 
Editora Papagaio, e faz reportagens que ajudam na luta dos direitos de crianças e mulheres. 
Como editora, trabalhou na publicação de livros como \textit{Gatos}, de 2015, e
\textit{O reizinho que só falava sim}, de 2021.

Denise nunca teve gatos, apenas cachorros, e agora tem um chamado Tom. 

\textbf{Tatiane Moreira Lima} é formada em Direito pela Pontifícia Universidade Católica de Campinas 
(\textsc{puc} -- Campinas). Atualmente é titular da Vara de Violência Doméstica e Familiar contra a Mulher, 
em São Miguel Paulista, zona leste da cidade de São Paulo. Conhece, portanto, de perto
inúmeros casos de abuso quotidianamente. Já publicou trabalhos como \textit{Violência doméstica e familiar},
e \textit{O Poder Judiciário e a rede de garantias para os direitos das mulheres}, ambos no ano de 2016.

Tatiane tem dois gatos, Vida e Fofa. 

\paragraph{A ilustradora} \textbf{Veridiana Scarpelli} nasceu em 1978 em São Paulo, onde mora e trabalha.
É formada em Arquitetura e Urbanismo pela Faculdade de Arquitetura e Urbanismo da Universidade de São
Paulo (\textsc{fau--usp}). trabalhei com projetoS de móveis e objetos e deU várias voltas até entender, em 2007,
que na ilustração estava seu lugar. Desde então, vem trilhando um caminho que começou com a ilustração de revistas e jornais. Em 2012, lançou seu primeiro livro como autora, \textit{O Sonho de Vitório}, pela Editora Cosac Naify, 
e começou a gradativamente se aproximar deste mundo que tanto lhe agrada: a ilustração de livros. 
Foi três vezes finalista do prêmio Jabuti, em 2014, 2017 e 2018.

Dentre os diversos livros que ilustrou estão \textit{O sítio do picapau amarelo}, \textit{O casamento de Narizinho} e
\textit{As aventuras do príncipe}, da Editora \textsc{ftd}, \textit{O educador: um perfil de Paulo Freire} e \textit{A mercadoria mais preciosa}, da Editora Todavia, \textit{oba! férias!}, catálogo do \textsc{sesc sp}, \textit{Revista Serrote}, do Instituto Moreira Salles (\textsc{ims}), além de \textit{O corpo é meu, ninguém põe a mão}, 
da Editora Papagaio. 

Veridiana sempre teve gatos, e hoje tem a Margô.


\section{Sobre o gênero}

\paragraph{O gênero} O gênero deste livro é a \textit{conto; crônica; novela}. 

%596 caracteres
\paragraph{Descrição} O que define um gênero narrativo é o fato de, não importa
qual seja sua forma, eles \textit{contarem uma história}.
As especificidades do \textit{como} esta história será contada é que
qualificaram os tipos de gênero narrativo, que podem ser: conto, crônica, novela,
epopeia, romance ou fábula. 

Toda narrativa possui, necessariamente, um narrador, uma personagem, um enredo,
um tempo e um espaço. O narrador, ou narradora, pode ser onisciente, literalmente
\textit{que tudo sabe}, observador ou personagem --- categorias que não são autoexclusivas.
O discurso elaborado por este narrador ou narradora pode ser direto, indireto ou indireto livre 
--- ou seja, ele ou ela pode aparecer mais diretamente ou mais indiretamente; no último caso,
sua voz se mistura à das personagens da história.

O narrador \textbf{não é necessariamente} a voz do autor. É errada a afirmação
de que o autor fala através do narrador de uma história. É bastante comum,
há algum tempo na história literária, sobretudo desde os pré-modernistas, que 
o narrador represente justamente o contrário do que pensa o autor. Neste caso, 
utiliza-se elementos como a \textbf{ironia} para sugerir que o autor \textit{não é confiável}.

Já as persponagens variam quanto a sua \textbf{profundidade}. Há personagens planas, ou
personagens-tipo, e personagens redondas, ou complexas. Personagens planas
são facilmente repetíveis pois se amparam em lugares-comuns da cultura, como
o vilão, o herói, a vítima, o palhaço, tudo isso com marcações de gênero e espécie ---
o herói tradicionalmente é um homem, a vítima, uma mulher, e o vilão, uma figura que 
se afasta da humanidade por alguma razão, às vezes sobrenatural. 
Personagens redondos, por outro lado, estão mais próximos das \textit{pessoas reais}.
Uma personagem complexa pode ser, em um dado momento da narrativa, vilã, e em 
outro, heroina. É importante notar como as visões de mundo, um traço cultural e 
portanto relativo, influenciam na caracterização das personagens, planas 
ou redondas, de uma história.

O tempo de uma narrativa pode ser cronológico ou psicológico.
No tempo cronológico, o enredo segue a ordem ``normal'' dos acontecimentos,
aquela marcada pelo relógio e pelo calendário. Os acontecimentos vêm um após o 
outro e se delimita muito bem \textit{passado}, \textit{presente} e \textit{futuro}.
Já no tempo psicológico, segue-se uma ordem \textit{subjetiva} dos acontecimentos, 
e portanto, \textit{não linear}, já que a influência emocional e psíquica 
da subjetividade afeta a racionalidade do tempo cronológico. 

O espaço, por fim, é o lugar onde se passa a narrativa. Dependendo do caso, 
ele pode funcionar mais como um plano de fundo, sem muita interferência
no enredo, ou mais ativamente, aproximando-se das características das personagens
e influenciando no desenrolar da trama. 

O último aspecto de um gênero narrativo que podemos abordar é sua 
\textit{extensão}. Dentre os elementos que distinguem um subgênero 
de outro é o tamanho da história: uma crônica e um conto são \textit{necessariamente}
curtos, ao passo que uma epopeia e um romance, são longos. Uma novela
está no ponto intermediário entre um romance e um conto.
Ainda poderíamos falar dos registros de cada subgênero: 
a epopeia é originalmente um subgênero \textit{oral}, versificado, e metrificado,
já o romance é tradicionalmente \textit{escrito} em prosa. 
Desde meados do século \textsc{xviii}, no entanto, o estabelecimento
dos gêneros e subgêneros narrativos tornam-se cada vez menos rígido,
com as características cada vez mais fluidas e intercomunicativas.

Como o presente livro se trata de uma narrativa \textit{curta},
finalizamos com as palavras de Luiza Vilma Pires a respeito do
subgênero:

\begin{quote}
sob o nome de narrativa curta, estão situadas obras que apresentam uma trama 
um pouco mais complexa, que ocorre em diversos espaços e em uma temporalidade 
que pode ser de vários dias, semanas ou meses. Entretanto a função das ilustrações 
continua as mesmas, são complementares à história e contribuem para sua compreensão. 
Os temas relacionam-se a vivência infantis (brincadeiras, passeios, pequenas aventuras), 
a aspectos ligados à interioridade das personagens (busca de identidade, insegurança, 
medos) ou a relações interpessoais (desentendimentos familiares, entre amigos, solidariedade).\footnote{“Narrativas infantis”, de Luiza Vilma Pires Vale. In \textsc{saraiva}, J. A. (Org.) \textit{Literatura e alfabetização: do plano do choro ao plano da ação}. Porto Alegre: Artmed, 2001.} 
\end{quote}

\section{Atividades}

\subsection{Pré-leitura}

\BNCC{EF67LP38}
\BNCC{EF07CI08}

\subsubsection{Atividade 1}

\paragraph{Tema} Os bichos também falam!

\paragraph{Conteúdo} Onomatopeias e sons produzidos pelos animais.

\paragraph{Justificativa} A narrativa de \textit{O corpo é meu, ninguém põe a mão}
possui elementos que podem aproximá-la do que chamamos de \textbf{fábula}. 
Nestas histórias, encontramos personagens que, na forma de animais, produzem um discurso
propriamente humano, além de vincular uma lição de moral em seu desfecho. 
Para familiarizar os alunos e alunas com este universo da linguagem dos animais,
é importante que eles tenham alguma noção linguística de elementos como as
\textbf{onomatopeias}, processo de formação de palavras que origina
boa parte dos verbos e substantivos designados aos sons emitidos por tais animais. 

\paragraph{Metodologia} Comece a aula fazendo as seguintes perguntas:

\begin{itemize}
	\item Quem tem animal em casa?
	\item Qual o som que ele faz?
	\item E outros bichos, quais os sons que eles fazem?
\end{itemize}

Apresente, então, o conceito de \textbf{onomatopeia}, que são palavras
que buscam \textit{imitar um som}. Por exemplo, a onomatopeia \textit{fon fon}
imita o som de uma buzina, enquanto \textit{crack} imita o som de um prato quebrando,
e, \textit{bum!}, uma bomba explodindo.

Este processo de formação de palavras é bastante comum quando vamos nomear os
sons produzidos pelos animais. Como sua forma de comunicação é diferente da nossa,
não podemos extrair um significado preciso dos sons que eles emitem, então
nos resta reproduzir, em palavras, o que ouvimos. 

No livro que vamos ler, que conta a história de uma gatinha chamada Estrela, 
vemos o seguinte trecho:

\Image{Trecho da página onze. (Retirado do livro).}{PNLD2023-039-002.jpg}

O \textit{ronrom} da gatinha é uma onomatopeia. Outra palavra que podemos usar
para se referir ao som produzido pelpos gatos é o \textbf{miado}, substantivo,
mas também na forma de verbo: \textbf{miar}.

Apresente, então, uma lista de sons dos animais que estão presentes 
no livro:

\begin{itemize}
	\item Gato: miado, miar;
	\item Cachorro: latido, latir;
	\item Leão: rugido, rugir;
	\item Passarinho: chilreado, chilrear;
	\item Macaco: guinchar;
	\item Coelho: chiar.
\end{itemize}

Além destes, peça que os alunos e alunas façam uma pesquisa sobre os nomes dos sons
produzidos pelos animais na Internet, acompanhados de suas respectivas onomatopeias.
Também podem pesquisar no YouTube gravações autênticas de tais sons. 


\paragraph{Tempo estimado} Duas aulas de cinquenta minutos.


\subsection{Leitura}

\BNCC{EF67LP08}
\BNCC{EF69LP04}
\BNCC{EF69LP05}

\subsubsection{Atividade 1}

\paragraph{Tema} Aprendendo a ler imagens. 

\paragraph{Conteúdo} Interpretação das ilustrações em diálogo com o texto 
da narrativa.

\paragraph{Justificativa} Neste como em outros livros onde há presença
de textos verbais e não verbais, é importante que a leitura dos diferentes 
registros seja feita de forma dialógica e não excludente. É preciso
que se aprenda a identificar os elementos convergentes entre palavras e 
imagens para a criação do sentido que, isolado em só um dos registros, perderia
força. Para isso, propomos uma interpretãção mais aprofundada de um 
aspecto do livro: a representação da raposa Lupi Lantra numa sequência
de páginas. 

\paragraph{Metodologia} No decorrer da leitura, chame à atenção dos alos e alunas
à representação visual da gatinha Estrela após as investidas maliciosas da raposa Lupi Lantra.
Na cena abaixo, ela aparece distraída na sala de aula enquanto a professora lhe dirige a palavra.
Ao redor de sua cabeça e de seu torso, pequenas raposas fantasmas lhe assombram.

\Image{A gatinha Estrela é atormentada pelos fantasmas da raposa durante a aula. (Retirado do livro).}{PNLD2023-039-003.jpg}

Algumas páginas à frente, quando Estrela, com a ajuda da professora e do diretor da escola, 
consegue \textbf{se expressar} sobre o que estava lhe atormentando, 
a representação das raposas muda:

\Image{Estrela consegue colocar para fora o que estava lhe perturbando. (Retirado do livro).}{PNLD2023-039-004.jpg}
\Image{(Retirado do livro).}{PNLD2023-039-005.jpg}

Chame a atenção dos alunos e alunas para as diferenças de tamanho e dimensões, nas próprias páginas do livro.
Neste último caso, as raposas ocupam mais de duas páginas. Elas são enormes e não
são visíveis apenas para Estrela, que sofria em silêncio o abuso causado por elas.
Agora que ela conseguiu colocar para fora por meio de palavras o que sentia,
seus problemas tomam forma e podem ser vistos e acolhidos por pessoas
preparadas para lidar com eles, no caso, os profissionais da educação que
trabalham na escola.

Faça algumas perguntas aos alunos:

\begin{itemize}
	\item Vocês já falaram alguma coisa que lhes deixou aliviados?
	\item Como foi?
	\item Vocês conseguem desenhar?
\end{itemize}

Então, peça que a turma pegue materiais de artes como lápis de cor, tinta e giz de cera,
e ilustrem este momento em que, por meio da fala, conseguiram lidar
com um problema que estavam vivendo, que pode ser desde uma dor de barriga
até algo mais sério. O desenho será usado em uma atividade posterior.

\paragraph{Tempo estimado} Duas aulas de cinquenta minutos.

\subsubsection{Atividade 2}

\BNCC{EF06LP06}

\paragraph{Tema} Aprendendo gramática com a gatinha Estrela!

\paragraph{Conteúdo} Análise sintática: o gênero na concordância nominal dos adjetivos.

\paragraph{Justificativa} A abordagem dos aspectos linguísticos e semióticos 
pela perspectiva enunciativo-discursiva é feita pela leitura dos efeitos de 
sentido produzidos pelas práticas de linguagem nos diferentes campos de atuação 
por meio dos diversos gêneros textuais, neste caso, uma narrativa curta.
A este respeito, a \textsc{bncc} diz que:

``Os conhecimentos sobre a língua, as demais semioses e a norma-padrão não devem ser tomados como uma lista de conteúdos dissociados das práticas de linguagem, mas como propiciadores de reflexão a respeito do funcionamento da língua no contexto dessas práticas. A seleção de habilidades na \textsc{bncc} está relacionada com aqueles conhecimentos fundamentais para que o estudante possa apropriar-se do sistema linguístico que organiza o português brasileiro.''\footnote{\textsc{bncc} -- Língua portuguesa no Ensino Fundamental. Cap.\,4.1.1.2, p.\,137 -- dezembro de 2017.} 


\paragraph{Metodologia} O professor ou professora deve iniciar a aula explicando à turma a definição de 
\textbf{adjetivo} e as primeiras noções de \textbf{concordância nominal}.
Assim, devem indicar que adjetivos são palavras que atribuem qualidades a substantivos.
Neste processo, ocorre o que chamamos de concordância nominal: as duas palavras devem estar de acordo em relação ao 
\textbf{gênero} e ao \textbf{número}. 
Aqui, vamos prestar atenção ao primeiro caso. 

\Image{Trecho da página seis.(Retirado do livro).}{PNLD2023-0039-001.jpg}

No caso da frase ``seu jeito brincalhão'', sabemos, por conta da leitura das frases anteriores,
que o narrador está falando da gatinha Estrela. Neste caso, poderíamos
nos questionar \textit{por que o adjetivo ``brincalhão'' está no masculino,
já que o substantivo ``gatinha'' é feminino?}

Porque o substantivo ao qual \textit{brincalhão} se refere diretamente não é \textit{gatinha},
mas \textit{jeito}, que por sua vez está ligado pelo pronome possessivo \textit{seu} 
a \textit{gatinha}. Desenhe o esquema de análise sintático na lousa para que os alunos
e alunas possam acompanhar melhor a explicação.

Então, apresente outras formas de transmitir o mesmo enunciado:

\begin{itemize}
	\item A gatinha Estrela é brincalhona.
	\item Todo mundo se diverte com a gatinha Estrela. 
	\item O jeito da gatinha Estrela é brincalhão.
\end{itemize}

Mostre que no caso do primeiro exemplo o adjetivo está de acordo, no aspecto de gênero,
com o substantivo, que desta vez não é mais masculino, mas feminino. 

Dê exemplos de outros casos onde haverá dois substantivos na mesma frase com gêneros diferentes:

\begin{itemize}
	\item O menino e a menina são \textbf{bonitos} e \textbf{calmos}.
	\item Pedro e Maria são \textbf{amigos}.
\end{itemize}

Neste caso, como devem ter percebido, os adjetivos devem ficar no \textbf{masculino} e no \textbf{plural}.

Agora, para finalizar, peça que os alunos e alunas peguem os livros e selecionem
frases para escrever em seus cadernos com uma análise sintática a respeito 
dos gêneros dos adjetivos e dos substantivos, conforme o exemplo que o professor
ou professora deve ter feito na lousa. 

\paragraph{Tempo estimado} Duas aulas de cinquenta minutos.


\subsection{Pós-leitura}

\subsubsection{Atividade 1}

\paragraph{Tema} Aprendendo a criar a partir do que foi vivido. 

\paragraph{Conteúdo} Criação de uma história ilustrada tendo como 
ponto de partida a leitura do livro e da \textbf{Atividade de leitura 1}. 
Oficina de criação literária.

\paragraph{Justificativa} O livro \textit{O corpo é meu, ninguém põe a mão}
trata de um tema muito importante para os jovens estudantes do Ensino Fundamental \textsc{ii},
além das crianças em geral: o abuso infantil. Por meio da narrativa da gatinha Estrela, 
eles devem ter percebido como o ato de contar a própria história para os outros é 
importante e, em alguns casos, pode os libertar de situações violentas. 
Por isso, os alunos e alunas devem ser incentivados a \textit{contar uma história}
verídica de suas vidas, de modo que, com o auxílio das ferramentas artísticas
da literatura e da ilustração se sintam à vontade para exercitar a prática saudável de 
comunicar-se com o seu entorno. 

\paragraph{Metodologia} Comece a aula pedindo que a turma retome os desenhos que fizeram
na \textbf{Atividade de leitura 1}, quando foram pedidos para ilustrar um momento de suas vida que 
precisaram contar a alguém o que estavam sentindo. 
Agora, dando continuidade a esta situação ou elencando outra de maior importância para eles,
devem \textbf{contar a história} completa deste acontecimento. Para isso, devem 
construir os elementos que compõem uma narrativa: espaço, personagens, enredo, e,
no caso desta, as ilustrações. 

O trabalho deve ser feito individualmente. As crianças podem usar os desenhos do livro
para se inspirar e mesmo os dos colegas. Devem, no entanto, manter-se dignos 
ao que eles querem contar, a história que \textbf{só eles sabem}. 

O resultado do trabalho pode ser exposto para a turma e em um \textit{blog} criado
pela turma, para que as famílias, amigos e a comunidade escolar e do entorno possam 
ler as histórias de todos e todas. 

\paragraph{Tempo estimado} Quatro aulas de cinquenta minutos.


\subsubsection{Atividade 2}

\BNCC{EF15AR24}
\BNCC{EF03LP13}

\paragraph{Tema} Oficina de contação de histórias. 

\paragraph{Conteúdo} Organização de uma roda de contação de histórias
com protagonismo dos alunos e alunas.

\paragraph{Justificativa} A contação de histórias trabalha competências
que dizem respeito ao campo linguístico e sociocomunicativo. 
Ao contar uma história, o indivíduo se apropria do lugar do narrador e
atuar como cocriador da mesma. Além do conteúdo propriamente dito 
da história, toda a estrutura linguística e gramatical, a sintaxe 
e o vocabulário presentes no texto serão trabalhados. 
O grande diferencial desta atividade é que seu objetivo não está 
exclusivamente no exercícios destas capacidades linguísticas, 
mas sim ligados a elas e à capacidade de \textbf{apresentar oralmente}
um texto a um público. Fala e corpo estão totalmente ligados nesta atividade. 

\paragraph{Metodologia} Nesta aula, a turma deverá apresentar as histórias
que compuseram na última \textbf{Atividade}. Além da partilha escrita e visual,
é importante que os e as jovens possam exercitar o ato de \textbf{contar oralmente}
suas próprias histórias. 

Diferente da última atividade, onde trabalharam sozinhos, agora eles podem 
solicitar a participação de colegas para realizar uma encenação de suas histórias. 
É interessante que os donos e donas das histórias sejam narradores-personagens 
nas cenas, ou seja, que eles tenham a voz principal e dialoguem com o público --- o restante da turma. 

Apresente à turma, para lhes inspirar na criação das cenas, algumas interpretações
de grupos de teatro infantojuvenil que deixamos na seção de \textbf{Sugestões de referências complementares}.


\paragraph{Tempo estimado} Duas aulas de cinquenta minutos.

\section{Sugestões de referências complementares}

\paragraph{Livros e artigos}

\begin{itemize}
	
\item \textsc{lima}, Romeu R.\,de. \textit{OcÊ QuÉ SabÊ?}. Clube de Autores, 2010. 

\item Companhia Arte e Manhas. \textit{Os três porquinhos}, \textit{O sítio do picapau amarelo em: o circo de cavalinhos},
\textit{Páscoa em apuros}. Todos disponíveis em: \url{https://www.youtube.com/channel/UCn0kXgFQr4r91FycGnAidqA}. Último acesso em 11 de janeiro de 2022.

\item \textsc{alberti}, Verena. ``Literatura e autobiografia: a questão do sujeito na narrativa''. \textit{Estudos Históricos}, Rio de Janeiro, v.\,4, n.\,7, 1991.

\item \textsc{freire}, Paulo. \textit{A importância do ato de ler em três artigos que se completam}. São Paulo: Autores Associados/Cortez, 1989.

\end{itemize}

\section{Bibliografia comentada}

\subsection{Livros}

\begin{itemize}

	\item \textsc{albrecht}, Tatiana D'Ornellas. \textit{Atividades lúdicas no Ensino Fundamental}. Universidade Católica Dom Bosco, \textsc{ms}, 2009. Disponível em: \url{https://site.ucdb.br/public/md-dissertacoes/8072-atividades-ludicas-no-ensino-fundamental-uma-intervencao-pedagogica.pdf}. Último acesso em 24 de dezembro de 2021.

	As atividades lúdicas, quando bem aplicadas e no momento oportuno, trazem
grandes benefícios. Contudo, a grande maioria das escolas não utiliza esse instrumento. Por
que será que existe essa resistência por parte das escolas e dos professores? Por que não
adequar o lúdico ao cotidiano escolar de maneira prática, educativa e ao mesmo tempo
divertida?

\item \textsc{brasil}. Ministério da Educação. Base Nacional Comum Curricular. Brasília, 2018.

Consultar a \textsc{bncc} é essencial para criar atividades para a turma. Além de especificar 
quais habilidades precisam ser desenvolvidas em cada ano, é fonte de informações sobre 
o processo de aprendizagem infantil. 

 \item \textsc{bandoch}, Adriana Rodrigues Vieira. \textit{A inserção do teatro nas séries iniciais do Ensino Fundamental}.
 Universidade Tecnológica Federal do Paraná, 2012. Disponível em: \url{http://repositorio.utfpr.edu.br/jspui/bitstream/1/20738/2/MD_EDUMTE_II_2012_03.pdf}. Último acesso em 24 de dezembro de 2021.

	O teatro no Ensino Fundamental é uma das formas de se trabalhar o conhecimento,
pois nele há a possibilidade do ser humano em se integrar, vivenciar, expressar e
criar situações, condições para novas aprendizagens éticas, sociais, culturais,
históricas.

\item {oliveira}, M.\,E. de, \textsc{stolz}, Tânia. ``Teatro na escola: considerações a partir de Vigotsky''. \textit{Revista Educar}, n.\,36. \textsc{ufpr}. Curitiba, 2010. Disponível em: \url{https://www.scielo.br/j/er/a/hLkXfdZ65VDTfztn8ng75Bd/?format=pdf&lang=pt}. Último acesso em 11 de janeiro de 2022.

Este artigo fundamenta-se nas ideias de Vygotsky a respeito da importância
da interação social e da arte no desenvolvimento humano: o que pressupõe,
além da dimensão cognitiva, a afetividade. Discute a realização de ativi-
dades teatrais na escola como prática educativa motivadora da aprendiza-
gem, da interação social e da expressão individual dos sujeitos. 


\item \textsc{van der linden}, Sophie. Para ler o livro ilustrado. São Paulo: Cosac Naify, 2011.

Livro sobre as particularidades do livro ilustrado, que apresenta as diferenças entre o livro ilustrado e o livro com ilustração. 
\end{itemize}

\end{document}