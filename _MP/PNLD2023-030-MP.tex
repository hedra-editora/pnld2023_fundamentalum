\documentclass[11pt]{extarticle}
\usepackage{manualdoprofessor}
\usepackage{fichatecnica}
\usepackage{lipsum,media9}
\usepackage[justification=raggedright]{caption}
\usepackage[one]{bncc}
\usepackage[acorde]{../edlab}
\usepackage{marginnote}
\usepackage{pdfpages}
\usepackage[printwatermark]{xwatermark}
% \newwatermark[pagex=2]{\includegraphics[scale=3.3]{watermarks/test-a.png}}	% página específica
% %\newwatermark[oddpages]{\includegraphics{watermarks/test-a.png}}			% páginas ímpars
% %\newwatermark[evenpages]{\includegraphics{watermarks/test-a.png}}			% págimas pares
% \newwatermark[allpages]{\includegraphics[scale=3.3]{watermarks/test-b.png}}

% \pagecolor{cyan!0!magenta!10!yellow!28!black!28!}

\newcommand{\AutorLivro}{Rudolf Erich Raspe}
\newcommand{\TituloLivro}{As aventuras do Barão Munchausen}
\newcommand{\Genero}{Conto}
%\newcommand{\imagemCapa}{./images/PNLD0001-01.png}
\newcommand{\issnppub}{978-65-99441-24-0}
\newcommand{\issnepub}{978-65-99441-27-1}
% \newcommand{\fichacatalografica}{PNLD0001-00.png}
\newcommand{\colaborador}{Gabriela Karam}

\begin{document}

\title{\TituloLivro}
\author{\AutorLivro}
\def\authornotes{\colaborador}

\date{}
\maketitle

%\begin{abstract}\addcontentsline{toc}{section}{Carta ao professor}
%\pagebreak

\tableofcontents

\section{Carta ao professor}
Caras e caros educadores,

O presente manual tem como finalidade trazer apoio pedagógico para que se aprofunde, de forma lúdica e divertida, o estudo acerca do livro \textit{As aventuras do Barão Munchausen}. A obra é um mergulho em uma aventura grandiosa da personagem Barão de Munchausen e é contada em primeira pessoa, gerando uma aproximação muito interessante para o leitor. E como toda aventura de grande porte, essa história está recheada de fatos fantásticos que levam nossa imaginação a lugares onde a realidade crua, do jeito que conhecemos, não tem espaço. Como afirma Cecília Meireles num de seus ensaios intitulado \textit{Aspectos da literatura infantil}, existem textos que não foram escritos para crianças, mas que, por uma razão ou outra, acabaram caindo em suas mãos, e elas os apreciaram. É o caso de \textit{As aventuras do Barão de Munchausen}, que começou sendo uma sátira às fanfarronadas atribuídas a esse oficial quando contou suas proezas na Rússia, por onde andara a combater os turcos. As primeiras aventuras do Barão foram publicadas em forma de anedotas, entre 1781 e 1783 e \textit{Rudolph Erich Raspe} recontou e reuniu, em livro, essas narrativas da revista, ao lado de outras, também protagonizadas pelo mesmo Barão. 

Essas histórias acabaram sendo rotuladas como contos populares e foram vendidas nas ruas - algo comum à época com esse tipo de literatura. Em pouco tempo, contudo, o livro alcançou enorme sucesso, cativando diversos públicos. Penetrou nas bibliotecas infantis e circulou em quase todos os idiomas, nas mais diversas adaptações. Mas a pergunta que não quer calar é: o Barão de Munchausen existiu realmente? E a resposta é sim, ele realmente existiu! Seu nome de batismo era Karl Friedrich Hyeronymus e nasceu em 11 de maio de 1720, em Bodenwerden, na Alemanha. Foi pajem, depois promovido a tenente, serviu num regimento russo e, acredita-se, teria lutado em duas guerras turcas. Depois de doze anos de serviço militar, aposentou-se. Voltou, então, para sua terra natal, onde morreu em fevereiro de 1797. Mas não sem antes, é claro, entreter, nas recepções que oferecia, amigos e convidados com suas aventuras fenomenais e incríveis viagens a terras estrangeiras. 

\Image{O Barão de Munchausen (Fandom; Domínio público)}{PNLD2023-030-02.png}

Nos dias de hoje, as aventuras do Barão Munchausen, em alguns aspectos, podem ser questionadas e debatidas em sala de aula, levando em consideração especialmente a forma como os animais são colocados na narrativa e a grande exaltação à prática da caça, que atualmente é proibida, além de já existirem leis públicas de proteção aos animais. É de grande valia que o educador dê ênfase ao período histórico em que essa história é contada, tanto para não chocar os alunos, como para não exaltar determinadas atitudes do Barão, principalmente porque estamos lidando com cidadãos em formação e nosso desejo é que eles, cada vez mais, façam parte de uma sociedade justa e humanitária. No entanto, \textit{As Aventuras do Barão Munchausen}, contadas por ele mesmo e com muito humor - ou seja, grandiosas mentiras contada por um adulto - são o grande feito deste livro. E talvez seja exatamente por isso que o público infantil se identifique e se encante com essa narrativa repleta de fatos irreais que nos levam pouco a pouco para um mundo mágico e absurdo - no melhor dos sentidos. Convenhamos: uma boa mentira contada por um bom contador de histórias atrai qualquer pessoa de qualquer idade. Ao longo do manual, todos esses aspectos serão explorados e relacionados a sugestões de atividades. Com isso, objetiva-se oferecer algumas ideias e inspirações para um trabalho que pode ser desenvolvido tanto a curto, quanto a médio e longo prazo. Sinta-se à vontade para personalizar a aula e torná-la sua, aplicando seus conhecimentos, sua personalidade e aproveite para fortalecer seu vínculo com a turma. Boa aula!

\Image{Histórias grandiosas (Meisterdrucke; Domínio público)}{PNLD2023-030-03.png}


\section{Sobre o livro}

Sabe aquelas típicas histórias de pescador? Substitua as pescarias por caçadas, batalhas e viagens, e as imagine sendo contadas por uma pessoa muito exagerada. Assim você já tem uma breve ideia das peripécias narradas no livro \textit{As aventuras do Barão de Munchausen}. O Barão de Munchausen é um personagem muito popular na literatura, principalmente na Europa, e suas histórias foram reunidas num livro pela primeira vez em 1785, pelo alemão Rudolf Erich Raspe. Na obra, o próprio Barão narra para alguns convidados os acontecimentos fantásticos que viveu em suas viagens pelo mundo. Enquanto tenta escapar da morte, encontrar seus amigos e salvar a cidade do ataque dos turcos, o Barão relembra a aposta com o Califa, a visita à Lua, a dança com Afrodite, entre outras aventuras. Ele cavalga durante uma batalha em um cavalo cortado ao meio; consegue escapar do ataque simultâneo de um leão e um crocodilo; é lançado contra uma cidade sitiada montado em uma bala de canhão; e por aí vai. O texto é muito cômico pois, a todo momento, o Barão conta como brigou com alguém que duvidou de seus relatos. Ele também diz que não gosta que outras pessoas contem suas aventuras, pois elas acabam exagerando e comprometendo a veracidade dos casos… A primeira metade do livro é mais pitoresca e engraçada; a segunda é mais imaginativa, quase surreal, com o personagem visitando uma ilha feita de queijo, conhecendo os moradores da lua, e encontrando deuses mitológicos como Vulcano e Vênus. Mas por que As aventuras do Barão de Munchausen, o maior mentiroso de todos os tempos, talvez só não mais que os grandes pescadores do nosso litoral brasileiro, cativa tantas pessoas há muito tempo? Simplesmente porque essas aventuras são, em sua essência, de um dos maiores prazeres da humanidade: a arte de contar e ouvir boas histórias — sejam elas verdadeiras ou não. 

\Image{(Literatura Brasil; Domínio público)}{PNLD2023-030-04.png}

\section{Sobre o autor}

Um dos aspectos mais intrigantes a respeito do Barão de Munchausen é que, por mais incrível e fantástico que pareça, ele realmente não é uma ficção. \textit{Karl Friedrich Hieronymus von Münchhausen} nasceu em Bodenwerder, na Alemanha, em maio de 1720, numa família aristocrática muito respeitada na região. Tendo de provar o seu valor junto da aristocracia russa, o jovem Duque seguiu para São Petersburgo, onde se alistou às tropas militares russas. Em fevereiro de 1738, Münchhausen participou da guerra Russo-Austríaca contra o império Otomano (1736–1739). Em 1739 Münchahusen foi nomeado alferes do regimento pela Imperatriz Anna Ivanovna da Rússia e, em 1740, se tornou tenente. Apesar dessa vida rica de experiências que são fantasticamente bem narradas na obra, não foi ele próprio quem redigiu e publicou o livro. O responsável por transformar suas histórias em livro foi um bibliotecário e cientista chamado Rudolph Erich Raspe - o autor do livro em questão -, nascido em Hanover, em 1737. Era conhecido por sua versatilidade intelectual, pois escrevia constantemente sobre todo tipo de tema. Tendo se apropriado de objetos da importante coleção pela qual era responsável e que pertencia a seu patrão, teve que fugir para a Inglaterra. Em Londres, seu domínio em inglês aliado a seus vastos conhecimentos permitiram que ele publicasse livros sobre vários assuntos para ganhar a vida, mas estava sempre em apuros financeiros: chegou até a ser preso por não poder pagar o alfaiate. Essa difícil situação, somada ao fato de sua reputação péssima ter alcançado Raspe em Londres, fez com que ele se mudasse para a Cornualha, onde pôs em prática seus conhecimentos de mineralogia por meio do trabalho em uma mina. Foi lá que, sempre atrás de dinheiro, escreveu \textit{As Aventuras do Barão de Munchausen}, cujo protagonista provavelmente conheceu, frequentando-lhe as reuniões na fazenda, na qual pessoas de muito longe ouviam aquelas histórias fantásticas. Os primeiros três destes contos foram publicados em 1761. Mas nem todas as histórias reunidas no livro foram contadas na vida real por seu protagonista: Rudolf Erich Raspe criou novas anedotas que atribuiu ao Barão, e diferentes autores adicionaram outros casos nas edições posteriores do livro. Com o tempo, Munchausen tornou-se cada vez mais popular e foi traduzido para muitas línguas e ilustrado por diversos artistas.

\Image{Rudolph Erich(String Fixer; Domínio público)}{PNLD2023-030-05.png}

\section{Sobre o gênero}

O presente livro é um \textbf{conto}, que consiste em uma história curta, com poucos personagens e centrada em apenas um núcleo narrativo. Geralmente, o conto tem elementos lúdicos e utiliza expressões metafóricas, com o intuito de retratar o cotidiano de maneira poética e subjetiva. O texto também possui características da fábula. A fábula veio do conto, mas se diferencia pela centralidade de personagens animais e pelo intuito de concluir a história com um ensinamento moral. No caso de \textit{As aventuras do Barão de Munchausen}, os contos são caracterizados pela aventura. A aventura é tipicamente aplicada às obras em que o protagonista ou outros grandes personagens são constantemente colocados em situações perigosas. 

\Image{(Wikipedia; Domínio público)}{PNLD2023-030-06.png}

\section{Proposta de Atividades}
\subsection{Pré Leitura}

\BNCC{EF15LP15}% Reconhecer que os textos literários fazem parte do mundo do imaginário e apresentam uma dimensão lúdica, de encantamento, valorizando-os, em sua diversidade cultural, como patrimônio artístico da humanidade.
\BNCC{EF15LP10} %Escutar, com atenção, falas de professores e colegas, formulando perguntas pertinentes ao tema e solicitando esclarecimentos sempre que necessário.
\BNCC{EF35LP25}% Criar narrativas ficcionais, com certa autonomia, utilizando detalhes descritivos, sequências de eventos e imagens apropriadas para sustentar o sentido do texto, e marcadores de tempo, espaço e de fala de personagens.

\paragraph{Tema} O mundo das aventuras.  

\paragraph{Conteúdo} A aventura no universo literário. 

\paragraph{Objetivo} Preparar os alunos para a leitura de \textit{As aventuras do Barão Munchausen} por meio do estímulo da imaginação.

\paragraph{Justificativa} Estimular a imaginação e a criatividade de nossos estudantes é um meio de estimular a leitura. A atividade proposta a seguir promove, por meio do trabalho em grupo, o relato oral e a redação de aventuras que introduzirão o aluno no universo de \textit{As aventuras do Barão Munchausen}.    

\paragraph{Metodologia} Antes de mergulhar no universo de \textit{As aventuras do Barão Munchausen}, o educador pode estimular os alunos a explorarem o mundo das aventuras de uma forma geral. Para isso, antes de mais nada, é sugerido que os alunos sejam organizados em um grande círculo. Uma vez organizada a sala, comece pedindo-lhes que contem sobre alguma aventura que viveram. Logo em seguida, incentive uma pesquisa sobre livros e até joguinhos que envolvam alguma situação aventureira com que já tenham tido algum contato e estimule que narrem essas histórias, além de ouvir o que os colegas trouxerem. Depois de algum tempo no debate, sugere-se que o professor faça uma ponte para conectar o assunto referente às aventuras que eles já viveram ao tema que o livro traz. Assim, aconselha-se que se passe na sala a animação \textit{Lanterna Mágica - Episódio 37}, na qual a página no Youtube \textit{Lanterna Mágica} disponibiliza um conteúdo que fala sobre as tais aventuras do Barão de Munchausen (disponível em \url{https://www.youtube.com/watch?v=PvEGYs5SUto})

\Image{Imagem do canal \textit{Lanterna Mágica}(YouTube; Domínio público)}{PNLD2023-030-07.png}

No episódio indicado, uma menina tenta chamar seu irmão mais novo para largar o videogame e ligar o seu projetor, que ela chama de lanterna mágica, para assistir às aventuras do Barão. Enquanto estão assistindo, o irmão faz uma série de perguntas duvidando da veracidade das histórias e a irmã o provoca dizendo que o Barão não está mentindo, está inventando para que fique mais interessante. É interessante que a turma assista ao vídeo para refletir sobre a importância dos textos literários para a ampliação do nosso poder de imaginação. Em seguida, sugerimos que o educador divida a turma em grupos de até cinco estudantes e desenvolva uma atividade na qual eles serão instruídos a inventar uma aventura por onde passaram. Para ficar mais simples a realização da tarefa, escolha o local desta aventura. Por exemplo, uma praia, uma cachoeira, um parque de diversão, uma floresta e outros lugares que estimulem a imaginação dos alunos. Mostre alguns artifícios para que uma história se torne mais interessante 

\Image{Divida a turma em grupos de até cinco estudantes e desenvolva uma atividade na qual eles serão instruídos a inventar uma aventura por onde passaram (Fleiturinha; Domínio público)}{PNLD2023-030-08.png}

Recomenda-se mostrar o seguinte trecho do livro, que mostra a indignação do Barão quando alguém duvida da veracidade das suas histórias, provocando ainda mais a curiosidade naquele que ouve ou lê:

\begin{quote}

Ao contar suas aventuras, a maioria dos viajantes tem por costume dizer que viu muito mais do que realmente viu. Portanto, não é de espantar que leitores e ouvintes algumas vezes sintam-se inclinados a não acreditar em tudo que leem e ouvem. Mas se houver, entre os presentes a que tenho a honra de me dirigir, alguém tentado a pôr em dúvida a veracidade dos relatos que faço, ficarei profundamente magoado por essa falta de confiança, e vou sugerir-lhes que a melhor coisa a fazer é despedir-se antes que eu comece a relatar minhas aventuras marítimas, pois elas são muito mais maravilhosas, embora não menos autênticas.

\end{quote}

Depois que eles elaborarem a aventura em conjunto, coloque as cadeiras em uma grande roda na sala de aula e se certifique de que cada uma seja contada com a participação de todos os estudantes do grupo. Em seguida, incentive que essa narrativa seja escrita como se fosse um livro. Ofereça referências de livros infantis que tenham na biblioteca, mostrando as diversas formas de se narrar uma história e como a pintura ou o desenho fazem parte primordial na construção de imagens para um melhor mergulho no que está sendo contado. 

\Image{Depois que eles elaborarem a aventura em conjunto, coloque as cadeiras em uma grande roda na sala de aula e se certifique de que cada uma seja contada com a participação de todos os estudantes do grupo (Aventuras no conhecimento; Domínio público)}{PNLD2023-030-09.png}

\paragraph{Tempo Estimado} Quatro aulas de 50 minutos. 

\subsection{Leitura}

\BNCC{EF35LP12}% Recorrer ao dicionário para esclarecer dúvida sobre a escrita de palavras, especialmente no caso de palavras com relações irregulares fonema-grafema.
\BNCC{EF05LP01}% Grafar palavras utilizando regras de correspondência fonema-grafema regulares, contextuais e morfológicas e palavras de uso frequente com correspondências irregulares.}
\BNCC{EF35LP26}% Ler e compreender, com certa autonomia, narrativas ficcionais que apresentem cenários e personagens, observando os elementos da estrutura narrativa: enredo, tempo, espaço, personagens, narrador e a construção do discurso indireto e discurso direto.
\BNCC{EF15LP19}% Recontar oralmente, com e sem apoio de imagem, textos literários lidos pelo professor.

\paragraph{Tema} Leitura de \textit{As aventuras do Barão Munchausen}.  

\paragraph{Conteúdo} Compreensão e análise dos elementos fundamentais de \textit{As aventuras do Barão Munchausen}: personagens, enredo, espaço, tempo e foco narrativo.  

\paragraph{Objetivo} Aprofundamento na leitura do livro  

\paragraph{Justificativa} O professor tem grande influência na formação de leitores. Para despertar o gosto pela leitura, precisamos ser mediadores entre a obra, sua linguagem, suas estruturas e os estudantes, de preferência estabelecendo uma relação fundamentada no prazer, na identificação e na liberdade de interpretação. Eis o nosso desafio: ler com os alunos, apresentando as passagens decisivas de um texto, explicando por que elas chamam a atenção e  ouvindo as impressões dos estudantes a respeito de tudo isso. 

\paragraph{Metodologia} Após o trabalho de narrativa e escrita de uma aventura criada pelos próprios alunos, a leitura do livro está pronta para ser iniciada. Sugerimos que os estudantes ainda permaneçam em roda para que se olhem durante a leitura e o educador comece os trabalhos lendo as primeiras páginas, as quais explicam se este Barão existiu, a maneira como foram editadas essas histórias e em que tempo elas se passaram - uma importante contextualização. Logo em seguida, incentive-os a ler as aventuras por capítulos para que, ao final de cada um, haja uma abertura para questionamentos e debates, certificando, assim, que haja um melhor entendimento da história, já que a linguagem não é muito simples por ter diferenças no tempo: a linguagem contemporânea é muito diferente da linguagem do século XVIII. Além disso, algumas atitudes do barão - como a caça e a forma como os animais são tratados por ele - geram ruídos, visto que felizmente ultrapassamos algumas das crueldades que eram feitas com os animais na época. Oriente os alunos a pesquisarem no dicionário o sentido das palavras que geraram dúvidas, pedindo também que eles grifem em seus livros palavras de correspondência fonema-grafema regulares, contextuais e morfológicas, passando por cada um para sanar dúvidas - inclusive a respeito da pesquisa no dicionário. Feito isso, propõe-se que o professor sugira que ao menos três alunos recontem o que ouviram do capítulo que foi lido. Auxilie-os a se articularem com clareza, colocando suas ideias de forma clara e concisa.

\paragraph{Tempo Estimado} Duas aulas de 50 minutos.  

\subsection{Pós-leitura}

\BNCC{EF15LP07}% Editar a versão final do texto, em colaboração com os colegas e com a ajuda do professor, ilustrando, quando for o caso, em suporte adequado, manual ou digital.
\BNCC{EF15LP06}% Reler e revisar o texto produzido com a ajuda do professor e a colaboração dos colegas, para corrigi-lo e aprimorá-lo, fazendo cortes, acréscimos, reformulações, correções de ortografia e pontuação.
\BNCC{EF15AR04}% Experimentar diferentes formas de expressão artística (desenho, pintura, colagem, quadrinhos, dobradura, escultura, modelagem, instalação, vídeo, fotografia etc.), fazendo uso sustentável de materiais, instrumentos, recursos e técnicas convencionais e não convencionais.

\paragraph{Tema} Elementos da narrativa.

\paragraph{Conteúdo} Criação de narrativas em textos e jogos.  

\paragraph{Objetivo} Promover trabalho coletivo de criação de narrativas.  

\paragraph{Justificativa} A redação coletiva permite que os estudantes se aprofundem nos elementos da narrativa (personagens, enredo, tempo, espaço e foco narrativo), além de permitir o desenvolvimento de habilidades de convívio, igualmente estimulada nos exercícios de correção. A criação de um jogo de tabuleiro requer também a capacidade de formular elementos de coesão e de coerência na narrativa. 

\paragraph{Metodologia} Após a leitura, é interessante promover uma conversa sobre toda a história no geral e quais aventuras causaram mais interesse e os motivos dessas escolhas. Procure levá-los a questionar se o gosto por determinada história estava ligado à forma com que era contada ou com a aventura em si. Levante também a possibilidade de que a forma como os colegas recontaram a história pode ter suscitado maior interesse. Esses e outros questionamentos podem ser levantados para que se aprofunde o entendimento de uma boa narração de história. Depois que a conversa trouxer bons frutos, aconselha-se que seja iniciada uma nova atividade: divida a turma em grupos, distribua as histórias do barão entre cada grupo e peça que eles mudem o final das histórias, estimulando-os a trabalhar coletivamente, respeitando as ideias dos colegas e buscando chegar em um consenso para essa alteração no texto. Passe em cada grupo tendo o cuidado de fazer com que eles vejam a necessidade do trabalho colaborativo que esse tipo de atividade exige. Concluída a escrita, pegue os textos produzidos e redistribua entre os grupos, pedindo que eles corrijam os textos dos colegas, com base nas correções de ortografia e pontuação, além de dar liberdade que eles sugiram cortes ou acréscimos nos textos. Explique que, em um trabalho tão coletivo quanto este, é necessário ter um desapego em relação às suas propostas - mesmo que elas sejam muito relevantes. Trata-se de uma atividade que demanda dos alunos uma boa relação interpessoal, e cabe ao educador estar presente para que eles estejam abertos a abrir concessões em suas ideias. 

É importante ressaltar que a criatividade das crianças será muito aflorada, além de aprenderem a se relacionar de uma forma muito interessante. Outra atividade que poderá ser realizada após a leitura, é a escolha das situações que causaram mais interesse no geral da turma e incentivar a construção de um jogo de tabuleiro usando essas aventuras como pontos estratégicos do jogo em si. Como referência para a realização deste projeto, o vídeo \textit{Como fazer um jogo de tabuleiro}, disponível no Youtube em \url{https://www.youtube.com/watch?v=wd2K4M8GGhQ}. Após essa tarefa, o educador poderá incentivar a execução, desta vez nos grupos criados anteriormente, de um novo jogo de tabuleiro com a aventura que foi feita por este grupo na atividade da pré-leitura, usando todos os elementos e discussões que fizeram parte do desenvolvimento do jogo de tabuleiro das Aventuras do Barão de Munchausen. Incentive os estudantes a escreverem as regras que conduzem o jogo, para que as pessoas possam entender como funciona.
Promover ao final da atividade a organização, em caixas, de todos esses jogos criados, para que fiquem disponibilizados na biblioteca para qualquer estudante da escola. Lembre que os alunos podem usar diversos materiais artísticos para a realização da atividade: permita que aflorem sua criatividade. 

\Image{Permita que aflorem a criatividade (Fábrica de criatividade; Domínio público)}{PNLD2023-030-10.png}

\section{Sugestões de referências complementares}

\subsection{Livros} 

\begin{itemize}
\item \textsc{campbell}, Joseph. \textit{O poder do mito}. São Paulo: Cultrix, 1990.

Livro proveniente de uma série de conversas mantidas entre Joseph Campbell e o jornalista Bill Moyers a respeito de mitologia.

\item \textsc{brenman}, Ilan e \textsc{vilela}, Fernando. \textit{As narrativas preferidas de um contador de histórias}. São Paulo: Martins Fontes, 2016.

Na reunião dos contos desse livro, narrados por Ilan Brenman e ilustrados por Fernando Vilela, encontre a tradição narrativa de diferentes povos ao redor do mundo.

\end{itemize}

\section{Bibliografia comentada}
\subsection{Livros}

\begin{itemize}
\item \textsc{brasil}. Ministério da Educação. Base Nacional Comum Curricular. Brasília, 2018.

Consultar a \textsc{bncc} é essencial para criar atividades para a turma. Além de especificar quais habilidades precisam ser desenvolvidas em cada ano, é fonte de informações sobre o processo de aprendizagem infantil. 

\item \textsc{lispector}, Clarice. Todos os contos. São Paulo: Rocco, 2016.

Trata-se de uma obra ficcional lançada em 2016 que reúne os contos escritos por Clarice Lispector. 
 
\item \textsc{grimm}, Jacob e Wilhelm. Contos maravilhosos infantis e domésticos. São Paulo: Editora 34, 2018.

Inúmeros contos fantásticos que foram publicados inicialmente em 1812. Contos clássicos que originaram diferentes histórias conhecidas no mundo ocidental.

\item \textsc{bojunga}, Lygia. \textit{A bolsa amarela}. São Paulo: Casa Lygia Bojunga, 2013.

A bolsa amarela é o romance de uma menina que entra em conflito consigo mesma e com a família ao reprimir três grandes vontades que ela esconde numa bolsa amarela.

\item \textsc{munduruku}, Daniel. \textit{Contos indígenas brasileiros}. São Paulo: Global Editora, 2004.

Livro que traz contos dos povos originários e apresenta a diversidade cultural e linguística no Brasil.

\item \textsc{coelho}, Nelly Novaes. Literatura infantil, teoria, análise, didática. 1ª ed. São Paulo: Moderna, 2000.

Livro que fala sobre os espaços da literatura infantil na contemporaneidade e a importância de as crianças estarem ligadas ao seu imaginário pela via literária.

\end{itemize}

\subsection{\textit{Sites}}

\begin{itemize}
\item Artigo "A Importância da Leitura dos Contos de Fadas na Educação Infantil", por Ana Maria da Silva. Disponível em: \url{https://siteantigo.portaleducacao.com.br/conteudo/artigos/educacao/a-importancia-da-leitura-dos-contos-de-fadas-na-educacao-infantil/30151}. 
Acesso em 20 dez. de 2021.

No artigo, a autora fala sobre a importância da construção do imaginário pela via da literatura para as crianças, trazendo elementos que analisam o mundo pós-moderno e os espaços que a literatura infantil, principalmente os contos, devem ter.

\item Artigo "Literatura infantil: A contribuição dos contos de fadas para a construção do imaginário infantil" Disponível em: \url{http://docs.uninove.br/arte/fac/publicacoes/pdf/v3-n1-2012/Francy.pdf}. Acesso em 20 dez. de 2021

\item Artigo "Mitologia para crianças", do blog "Filosofia animada". Disponível em: \url{https://danielmcarlos.wordpress.com/2014/02/09/mitologia-para-criancas/}. Acesso em 23 dez. de 2021. 

Artigo que se propõe a mostrar, de forma lúdica, parte da mitologia grega para o público infantil.

\end{itemize}

\subsection{\textit{Filmes}}

\begin{itemize}
\item \textit{Peixe grande e suas histórias maravilhosas}. Dirigido por Tim Burton, 2003.

William tem uma relação tensa com seu pai, Edward Bloom, porque ele sempre contou histórias exageradas sobre sua vida. Mesmo no leito de morte, Edward ainda narra histórias fantásticas. Quando William, que é jornalista, decide investigar os contos de seu pai, ele começa a entender melhor Edward e sua mania de contar histórias.

\item \item \textit{Edward mãos de tesoura}. Dirigido por Tim Burton, 1991.

Peg Boggs é uma vendedora que acidentalmente descobre Edward, jovem que mora sozinho em um castelo no topo de uma montanha, criado por um inventor que morreu antes de dar mãos ao estranho ser, que possui apenas enormes lâminas no lugar delas. Isto o impede de poder se aproximar dos humanos, a não ser para criar revolucionários cortes de cabelos. No entanto, Edward é vítima da sua inocência e, se é amado por uns, é perseguido e usado por outros.

\item \textit{A viagem de chihiro}. Dirigido por Hayao Miyazaki, 2001.

Chihiro e seus pais estão se mudando para uma cidade diferente. A caminho da nova casa, o pai decide pegar um atalho. Eles se deparam com uma mesa repleta de comida, embora ninguém esteja por perto. Chihiro sente o perigo, mas seus pais começam a comer. Quando anoitece, eles se transformam em porcos. Agora, apenas Chihiro pode salvá-los.

\end{itemize}
\end{document} 


