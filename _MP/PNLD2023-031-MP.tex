\documentclass[11pt]{extarticle}
\usepackage{manualdoprofessor}
\usepackage{fichatecnica}
\usepackage{lipsum,media9}
\usepackage[justification=raggedright]{caption}
\usepackage[one]{bncc}
\usepackage[araucaria]{../edlab}
\usepackage{marginnote}
\usepackage{pdfpages}

\newcommand{\AutorLivro}{Arnaldo Antunes}
\newcommand{\TituloLivro}{As coisas}
\newcommand{\Tema}{Diversão e aventura}
\newcommand{\Genero}{Poesia}
%\newcommand{\imagemCapa}{./images/PNLD2022-001-01.jpeg}
\newcommand{\issnppub}{XXX-XX-XXXXX-XX-X}
\newcommand{\issnepub}{XXX-XX-XXXXX-XX-X}
% \newcommand{\fichacatalografica}{PNLD0001-00.png}
\newcommand{\colaborador}{Renier Silva}

\begin{document}

\title{\TituloLivro}
\author{\AutorLivro}
\def\authornotes{\colaborador}

\date{}
\maketitle

\tableofcontents

\section{Carta ao professor}

Caros professores e professoras,

esperamos, com este material,
auxiliá-los no trabalho com o \textbf{Ensino Fundamental \textsc{ii}} em 
sala de aula. \textit{As coisas}, de Arnaldo Antunes, é um livro singular
por vários motivos e possibilita atividades didáticas interessantíssimas,
como vocês acompanharão a seguir.

Os poemas deste livro abordam, por meio de jogos linguísticos,
figuras de linguagem e as ilustrações de sua filha pequena, Rosa Moreau
Antunes, a (re)descoberta do mundo e das coisas que o compõem. 
Nada passa desapercebido do olhar poético: desde as \textit{coisas} 
mais materiais, como o sol e a tromba de um elefante, mas sobretudo,
inclusive nas próprias coisas físicas, aquilo que há de abstrato nelas,
e que se mostram terreno fértil para a imaginação do poeta. 
A \textit{coisa} em si nunca é simplesmente o que a palavra diz.
O nome de uma coisa leva a outros lugares, e a outras coisas.
É da relação entre essas descobertas que nasce o primor estético
desta poética. 

Neste manual, vocês encontrarão atividades que procuram 
trabalhar o aprimoramento do olhar poético dos alunos e alunas. 
Partimos do pressuposto muito válido de que não são só os poetas 
consagrados, como Arnaldo Antunes, que possuem tal capacidade. 
As crianças, principalmente, como o próprio autor indica
com a presença de sua filha na feitura do livro, têm até 
uma maior liberdade na lida com o fazer poético. 
Talvez por estarem menos moldadas do que os adultos 
às convenções sociais que, na contramão do que este livro propõe,
não admitem polissemia nem criatividade entre as \textit{coisas}
e as \textit{palavras}. 

Acreditamos fortemente que o trabalho da competência linguística
e artística é essencial para a formação de um indivíduo saudável 
intelectual e socialmente. E é desta premissa que partimos para
a elaboração deste manual!

Esperamos, professores, que este material sirva como um guia 
para seu trabalho em sala de aula. Já contamos, no entanto, com as adaptações
que surgirão organicamente na recepção do mesmo por vocês, que possuem 
trajetórias e escolhas didáticas específicas, bem como no contato com os 
alunos, que tanto têm a oferecer para o enriquecimento da experiência didática.

Boa aula!


\section{Sobre o livro}

\textit{As coisas} é o terceiro livro de poemas de Arnaldo Antunes. 
São ao todo 42 poemas escritos pelo poeta e ilustrados por sua filha
pequena, Rosa Moreau Antunes. 

O poema inicial, ``Abertura'', serve como uma introdução ao que será
apresentado a seguir: um caráter sobretudo prosaico e narrativo.

\begin{verse}
Todos eles traziam\
sacolas, que pare-\\
ciam muito pesa-\\
das. Amarraram\\
bem seus cavalos e\\
um deles adiantou-\\
se em direção a\\
uma rocha e gritou:\\
``Abre-te, cérebro!''\\
\end{verse}

Neste poema, vemos uma referência intertextual à história de Ali Babá 
e os quarenta ladrões. Lá, Ali Babá observa a chegada, à cavalo, de
quarenta ladrões que trazem sacolas aparentemente muito pesadas. 
Param à frente de uma rocha e, para que possam adentrar na caverna do tesouro, 
devem falar as palavras mágicas: "Abre-te, sésamo!". Destas palavras
reverbera a abertura da porta secreta. A parte inicial do poema, narrativa, 
poderia mesmo ser retirada da história. O último verso, porém, causa 
estranheza e prepara o leitor para os poemas que encontrará no livro.

A ordem dada pelo poeta, na última linha, é: "Abre-te, \textbf{cérebro}". Isto serve como
uma indicação metafórica de que o cérebro se encontra fechado como uma rocha, e para
receber, ou ler, poesia necessitaria ser, ou estar, aberto e receptivo.
As ``sacolas, que pareciam muito pesadas'' podem ser entendidas, metaforicamente, como os 
poemas do próprio livro, que, no caso de o ``cérebro'' estar aberto, poderão ser descarregados e
acumulados com o restante do tesouro que se encontra guardado --- o conhecimento. 

É uma característica de Arnaldo Antunes realizar estes jogos linguísticos. 
Assim como no
poema anterior, o ludismo encontrado em ``Tudo'', na página 24, é como se fosse uma
descoberta infantil a respeito da noção da palavra homônima ao seu título:

\begin{verse}
Todas as coisas\\
do mundo não\\
cabem numa\\
ideia. Mas tu-\\
do cabe numa\\
palavra, nesta\\
palavra tudo.\\
\end{verse}

\textit{Tudo} é uma palavra que pode abarcar a totalidade das coisas ou seres. Por mais
que se queira, é praticamente impossível abarcar ``todas as coisas do mundo...numa idéia'',
mas esse vocábulo pequeno pode ser tão abrangente e genérico como o vocábulo \textit{coisa},
normalmente utilizado para indicar qualquer objeto inanimado, e por vezes animais e pessoas. 
O jogo poético reside no fato de uma ``palavra'' ter a capacidade de conter ``todas
as coisas do mundo...tudo''. É como se fosse uma descoberta, por parte de uma criança, do
poder que as palavras possuem. Esta é, aliás, uma característica constante do livro. Os
poemas funcionam como uma descoberta de mundo e das \textit{coisas} que nos cercam, como
se fossem ditas a partir do ponto de vista de uma criança, com uma linguagem simples,
direta e objetiva.


\reversemarginpar
\marginparwidth=5cm

%\marginnote{\includegraphics[width=\marginparwidth]{./images/PNLD2022-001-02.png}\\
%A autora Camila Werner (Arquivo pessoal)}


\section{Sobre o autor}


%532 caracteres
\paragraph{O autor}

\Image{O multiartista Arnaldo Antunes.(Foto de Jefferson Rodrigues. CC BY 2.0)}{PNLD2023-031-01.jpg}

Nascido em 2 de setembro de 1960 na cidade de São Paulo, Arnaldo Antunes é um multiartista: 
poeta, compositor, cantor popular e artista visual.
Gosta de fazer brincadeira e dar risada. Gosta de crianças e de cachorros.
Gosta de brincar com as palavras.

Nas palavras do pesquisador na área de literatura contemporânea Nielson Ribeiro Modro, 
``Arnaldo Antunes não é apenas mais um entre os muitos poetas contemporâneos''.
Seu visual alternativo e seu próprio nome sempre chamaram atenção, mas foi como poeta 
e músico que Antunes possui várias facetas artísticas que desenvolve
há mais de uma década e devido a isto, há muito, já tem seu nome inscrito junto a um
restrito grupo de artistas com destaque a nível nacional. O público que acompanha sua
produção artística varia desde jovens roqueiros até velhos poetas, como os concretos Décio
Pignatari e Haroldo de Campos. Isto se deve ao fato de seu trabalho ser desenvolvido em
áreas distintas, porém que possuem uma íntima ligação: poesia, música e vídeo.

\begin{quote}
Percebe-se que Antunes
utiliza, em todos os seus livros, recursos oriundos de tendências literárias distintas para
construir seus poemas, principalmente advindos do Concretismo e Poesia Marginal. O uso
intencional do espaço em branco, o aproveitamento icônico, o jogo com as palavras, o
ludismo, a ingenuidade construída, a utilização de \textit{ready mades}, a originalidade, a síntese e
a objetividade podem ser apontados como características suas. Antunes consegue reunir
várias possibilidades poéticas distintas em sua obra de lorma a traçar um caminho próprio;
aproveita as possibilidades existentes, mescla-as e dá-lhes características próprias e
peculiares. \footnote{\textsc{modro}, Nielson R., \textit{A obra poética de Arnaldo Antunes.} Universidade Federal do Paraná, 1996.}
\end{quote}


Entre os livros de poesia publicados, estão \emph{Psia}, \emph{Tudos}, 
\emph{As coisas}, \emph{2 ou + corpos no mesmo espaço}, \emph{Palavra desordem}, 
\emph{ET, Eu, Tu}, \emph{N.d.a.} e \emph{Agora aqui ninguém precisa de si}, 
e livros de ensaios como \emph{40 escritos} e \emph{Outros 40}.  
Fez exposições de poesia visual em caligrafias, objetosa, vídeos, colagens e instalações. 
Como músico, lançou discos como \emph{Nome}, \emph{Ninguém} e \emph{O silêncio}.


\section{Sobre o gênero}

%55 caracteres
\paragraph{O gênero} O gênero deste livro é \textit{poesia}. 


Para uma primeira definição de poesia enquanto gênero literário, poder"-se"-ia recorrer à definição do professor Domingos Paschoal Cegalla, para quem ``poesia é a linguagem subjetiva, carregada de emoção e sentimento, com ritmo melódico constante, bela e indefinível como o mundo interior do poeta visa a um efeito estético''.\footnote{\textsc{cegalla}, Domingos Paschoal. \textit{Novíssima Gramática da Língua Portuguesa}. São Paulo: Companhia Editora Nacional, 2008, p.\,640}

Aprofundando um pouco essa definição, o crítico Antonio Candido expande a definição de poesia ao diferenciá"-la do verso.
Para o crítico, a poesia enquanto ato criador do artista independe da forma métrica do verso, que passa a ser apenas um dos registros possíveis do poético:

\begin{quote}
A poesia não se confunde necessariamente com o verso, muito menos com o verso metrificado. Pode haver poesia em prosa e poesia em verso livre. [\ldots]
Pode ser feita em verso muita coisa que não é poesia.\footnote{\textsc{candido}, Antonio. \textit{O estudo analítico do poema}. São Paulo: Terceira leitura, 1993, p.\,13--14.}
\end{quote}

Delineada, de forma breve e geral, a forma poética, pode"-se pensar agora em seus três gêneros básicos: lírico, épico e dramático.
Para o crítico Anatol Rosenfeld, a lírica é o gênero mais subjetivo, no qual uma voz central exprime um estado de alma traduzido em orações poéticas.
Seria a expressão de emoções e experiências vividas, ``a plasmação imediata das vivências intensas de um Eu no encontro com o mundo, sem que se interponham eventos distendidos no tempo (como na Épica e na Dramática)''.\footnote{\textsc{rosenfeld}, Anatol. \textit{O teatro épico}. São Paulo: Perspectiva, 2006, p.\,22.}

Devido a essa característica central da lírica, a expressão de um estado emocional, Rosenfeld considera que o eu"-lírico, nesse gênero, não se delineia enquanto um personagem. Embora possa evocar personagens e narrar acontecimentos, a lírica entendida enquanto gênero puro afasta"-se sobremaneira da apreensão objetiva do mundo, que não existe independente da subjetividade intensa que o apreende e exprime. Assim, na lírica prevalece a fusão entre o sujeito e o objeto, que serve mais a realçar os estados profundos de alma do poeta.
Sobre os aspectos formais do gênero, Rosenfeld nota:

\begin{quote}
À intensidade expressiva, à concentração e ao caráter ``'imediato'' do poema lírico, associa"-se, como traço estilístico importante, o uso do ritmo e da musicalidade das palavras e dos versos. De tal modo se realça o valor da aura conotativa do verbo que este muitas vezes chega a ter uma função mais sonora que lógico"-denotativa. A isso se liga a preponderância da voz do presente que indica a ausência de distância, geralmente associada ao pretérito. Este caráter do imediato, que se manifesta na voz do presente, não é, porém, o de uma atualidade que se processa e distende através do tempo (como na Dramática) mas de um momento ``eterno''.\footnote{Ibidem, p.\,23.}
\end{quote}


\section{Atividades}

\subsubsection{Pré-leitura}

\BNCC{EF05LP02}
\BNCC{EF15LP15}

\subsubsection{Atividade 1}

\paragraph{Tema} A linguagem enquanto forma de ver o mundo. 

\paragraph{Conteúdo} Introdução à discussão a respeito da 
linguagem, apresentando-a enquanto instrumento para ver e
para modificar o mundo.

\paragraph{Justificativa} A poesia, como vimos com os autores especialistas,
está mais preocupada com o sentido sonoro da palavra e a multiplicidade
de sentidos lógicos que ela suscita. A palavra, para o poeta,
é seu instrumento de trabalho, como é o mármore para o escultor
e o violino para o músico. Elas são percebidas com cuidado e atenção
especiais. 

De forma mais simples e adequada à etapa de desenvolvimento dos alunos,
é imprescindível que eles percebam que a poesia produz imagens 
que são formas de ver o mundo diferentes da forma convencional.
Neste sentido, o olhar poético é sempre um olhar enriquecedor. 

\paragraph{Metodologia} Para introduzir o assunto, antes de chegar na poesia
propriamente, o professor ou a professora deve abordar a questão das
diferentes palavras usadas para a mesma \textit{coisa}. 
Prepare uma aula expositiva apresentando este conteúdo. 

Em diferentes idiomas uma mesma coisa é chamada de diferentes formas.
Por exemplo, o que chamamos de \textit{casa} em português, em inglês 
é \textit{house}, em francês é \textit{maison} e em tupi é \textit{oka}. 
Ainda que as palavras sejam diferentes, estão todas falando, em geral, 
da mesma \textit{coisa}: o lugar onde se mora. 

\Image{\textit{Oka} é o tipo de habitação tradicional dos povos Tupi e Guarani
que habitam o brasil. (CC BY 2.0)}{PNLD2023-031-02.jpg}

No entanto, há casos em que o que, em uma língua, é apenas \textit{uma coisa},
em outra, são várias. Por exemplo, para os povos esquimós, que vivem
numa região muito próxima ao Polo Norte e, por isso, muito fria e com
muito gelo, existem vários tipos de branco. Como boa parte das coisas
a sua volta são cobertas por gelo e neve, eles são capazes de distinguir 
uma tonalidade de outra, e o que para nós é simplesmente \textit{branco}
para eles tem bem mais sentidos. 

Para finalizar, dê um exemplo com a língua portuguesa: a palavra \textit{terra}.
\textit{Terra} pode ser aquela \textit{coisa} meio amarronzada que fica no 
chão, onde nós pisamos, onde nascem as árvores e plantas; ela pode 
ser mais firme ou mais solta, mais arenosa ou mais pedregosa.
Mas também pode ser o planeta onde vivemos, o Planeta \textit{Terra}. 
Duas \textit{coisas} diferentes, então, têm a mesma palavra. 
O que tem uma a ver com a outra? 

\Image{Poema e ilustração ``O chão'', página 19 do livro.}{PNLD2023-031-03.jpg}

Após a explicação, proponha um exercício aos alunos:

\begin{enumerate}
\item Procurem palavras que significam, assim como \textit{terra}, 
duas coisas ao mesmo tempo;

\item Depois, tentem explicar por que vocês acham que elas têm o mesmo nome:
o que elas têm em comum?

\item Compartilhem com a turma os resultados e descubra o que os seus
colegas pensaram.

\item Depois: e as palavras que são muito parecidas mas 
que não significam a mesma coisa? 

\item \textit{Casa} tem algo a ver com \textit{caça}?
\end{enumerate}

\paragraph{Tempo estimado} Duas aulas de 50 minutos.



\subsubsection{Atividade 1.2}

Em continuação à primeira parte da atividade, onde os alunos
começaram a perceber a riqueza do universo das palavras, 
é hora de trabalhar o aspecto da linguagem visual, também 
presente no livro por meio das ilustrações. 

Para as palavras levantadas por eles em seus exercícios,
eles devem agora, individualmente, \textbf{fazer uma ilustração},
um desenho, usando as cores e instrumentos que lhes convir: lápis de cor, 
canetinha, lápis de grafite... 

Depois de feito o desenho, devem mostrá-los aos colegas.
O importante, aqui, é perceberem como, a partir da mesma 
palavra, da mesma \textit{coisa}, eles foram capazes de criar 
imagens tão diferentes. 

Por fim, encerre a etapa de pré-leitura dizendo-lhes
que \textbf{o desenho e as palavras são tipos de linguagem},
mas ainda há mais outros.



\subsection{Leitura}

\BNCC{EF05LP02}
\BNCC{EF35LP31}

\paragraph{Atividade 1}

\paragraph{Tema} O que constitui um poema?

\paragraph{Conteúdo} Leitura do poema ``Se não (se)'' a partir de sua musicalidade e 
composição gramatical.

\paragraph{Justificativa} A leitura em voz alta de um poema é 
parte fundamental de sua experiência estética. 
É na leitura em voz alta que os aspectos sonoros
criados pelo artista na escola e posicionamento das
palavras na frase e no verso ganham sua potência integral. 
No caso deste poema, a musicalidade é algo que a atenção
e deve ser explorado pelo professor ou professora 
junto à turma. 
A similaridade fonética e mesmo ortográfica entre 
as palavras também pode ser um ótimo ensejo
para trabalhar noções de pronome e conjunção, no caso 
de \textit{se}, repetido diversas vezes no 
decorrer do poema.

\paragraph{Metodologia} Faça uma leitura em voz alta do poema 
``Se não (se)'', presente na página 79. 
Depois, peça para que alguns alunos e alunas leiam, 
sempre em voz alta.
Chame a atenção para a sonoridade do poema e os aspectos formais
das \textbf{rimas}, como em perde/ pede e procura/ segura.

\textbf{Quais são as palavras que rimam?}


Depois, chame a atenção para a palavra \textit{se} e seus múltiplos significados.
Pode tratar-se de um pronome ou de uma conjunção. 
Com o cuidado de não adiantar conteúdo de outra série,
explique a diferença básica entre os dois casos 
apresentando exemplos de cada um.

\begin{itemize}
	\item A gente \textit{se} viu ontem.
	\item Eu só vou \textit{se} você for.
	\item etc.
\end{itemize}

Por fim, após a elucidação acerca dos aspectos gramaticais,
repita a leitura com a turma, agora deixando que a 
sonoridade proposta pelo mesmo ganhe mais espaço.
Peça que acompanhem com palmas o \textbf{ritmo} do poema. 
Deixe claro que o ritmo já está no poema indicado pelas
\textbf{repetições} (``se perde'', ``se não'', ``se ganha'', ``se...'', ``se não'', ``se...'', 
e assim sucessivamente até o fim).


\paragraph{Tempo estimado} Duas aulas de 50 minutos.

\paragraph{Atividade 2}

\paragraph{Tema} As ciências e a poesia. 

\paragraph{Conteúdo} Noções de espaço a partir da leitura dos poemas ``Os lugares'' e ``O céu''.

\paragraph{Justificativa} Como já falamos nas definições do gênero poético,
a poesia está mais interessada no que é móvel do que no que é fixo. 
Por extensão, no que é relativo do que no que é incondicional. 
Noções como o posicionamento no espaço, da área da \textbf{Geografia}, 
e as diferenças da vida nas diversas partes do mundo, portanto,
são naturalmente, por sua própria constituição relativa, 
um material para o olhar poético. 
Nos dois poemas escolhidos, Arnaldo Antunes trabalha 
com estes temas e os alunos e alunas poderão perceber 
as aproximações entre o olhar científico e o olhar poético. 

\paragraph{Metodologia} Comece com a leitura do poema ``O céu'', da página 53.
Faça um comentário gramatical a respeito da diferença de \textit{em cima}
e \textit{encima} --- o primeiro indica uma posição acima (\textit{em} é uma preposição de lugar
e \textit{cima} significa ``a parte mais elevada''); já o segundo é o
verbo \textit{encimar} conjugado na terceira pessoa do singular e significa
``estar acima''.

Depois, continue a leitura em voz alta com a turma. 
Pergunte:

\begin{itemize}
\item Quais são as noções espaciais que existem além de \textit{em cima} e \textit{em volta}?
\item Quais outras coisas podem variar de acordo com a posição que ocupam?
\item Por exemplo, a pessoa mais alta dessa turma é necessariamente a mais alta da escola inteira?
\end{itemize}

Para continuar a discussão do tema, façam a leitura conjunta e em voz alta
do poema ``Os lugares'', da página 87.

\Image{A Terra do Fogo, na Patagônia, está muito próxima do Polo Sul e, por isso, é uma região fria. (Foto de Ulrich Peters. CC BY 2.0)}{PNLD2023-031-04.jpg}

Faça as seguintes perguntas para provocar a reflexão:

\begin{itemize}
\item Por que ``a água gira em sentido anti-horário no japão''?
\item Por que ``os carrinhos de aeroporto nos estados unidos são puxados?''
\end{itemize}

	Então, releiam juntos o poema e deixe que os alunos e alunas
	fiquem à vontade para comentar o que quiserem. 
	Eles podem, se preferir, fazer desenhos inspirados neste
	poema e compartilhar ao fim da aula. 

\paragraph{Tempo estimado} Duas aulas de cinquenta minutos.


\subsection{Pós-leitura}

\BNCC{EF15AR24}
\BNCC{EF03ER03}
\BNCC{EF03LP13}


\paragraph{Tema} Música e poesia.

\paragraph{Conteúdo} Produção musical a partir de um poema do livro ou autoral.

\paragraph{Justificativa} A interdisciplinaridade artística é elemento
constitutivo do trabalho de Arnaldo Antunes. De certa forma, 
é impossível dissociar seus versos escritos dos versos cantados
ou grafados em seus vídeos. Por isso, a recepção de seu trabalho
não posso deixar de levar em conta este aspecto, não apenas
na apreciação passiva, como na experimentação do processo
da parte dos alunos. 

\paragraph{Metodologia} Retomando o exercício de acompanhamento
sonoro por meio de palmas com a leitura do poema ``Se não (se)'',
mostre aos alunos algumas músicas de Arnaldo Antunes feitas a partir de
poemas do livro \textit{As coisas}, como ``As coisas'', ``Cultura'' e
``O fogo''. Os links estão indicados nas \textbf{Sugestões de referências complementares}.

Então, é a vez dos alunos \textbf{criarem} uma música a partir dos poemas 
ou a partir de seus próprios poemas autorais. 
Eles podem trabalhar em grupos ou individualmente,
conforme a melhor disposição da turma. 
Ao fim, é interessante que seus trabalhos sejam compartilhados
com a turma e, eventualmente, com as outras turmas da escola.

\paragraph{Tempo estimado} Quatro aulas de 50 minutos.


\section{Sugestões de referências complementares}

\paragraph{Música}

\begin{itemize}
\item ``As coisas''. Música de Arnaldo Antunes do álbum \textit{Qualquer}, de 2006. Disponível em: \url{https://www.youtube.com/watch?v=JF4MruZSwzg}. Último acesso em 21 de dezembro de 2021.

\item ``Cultura''. Música de Arnaldo Antunes do álbum \textit{Nome}, de 1993. Disponível em: \url{https://www.youtube.com/watch?v=Aguu_QzCQy8}. Último acesso em 21 de dezembro de 2021.

\item ``O fogo''. Música de Arnaldo Antunes do álbum \textit{Disco}, de 2015. Disponível em: \url{https://www.youtube.com/watch?v=kUgUNHj2VlE}. Último acesso em 21 de dezembro de 2021.
\end{itemize}


\paragraph{Livros}

\begin{itemize}
\item \textsc{diegues júnior}, Daniel. \textit{Literatura popular em verso}. Estudos. Belo Horizonte: Itatiaia, 1986. 

\item \textsc{marco}, Haurélio. \textit{Breve história da literatura de cordel}. São Paulo: Claridade, 2010.

\item \textsc{nuvens}, Plácido Cidade. \textit{Patativa e o universo fascinante
do sertão}. Fortaleza: Fundação Edson Queiroz, 1995.

\item \textsc{tavares}, Braulio. \textit{Contando histórias em versos. Poesia e romanceiro popular no Brasil}. São Paulo: 34, 2005.

\item \textsc{tavares}, Braulio. \textit{Os martelos de trupizupe}. Natal: Edições Engenho de Arte, 2004.
\end{itemize}

\section{Bibliografia comentada}

\subsection{Livros}

\begin{itemize}
\item \textsc{brasil}. Ministério da Educação. Base Nacional Comum Curricular. Brasília, 2018.

Consultar a \textsc{bncc} é essencial para criar atividades para a turma. Além de especificar 
quais habilidades precisam ser desenvolvidas em cada ano, é fonte de informações sobre 
o processo de aprendizagem infantil. 

 
\item \textsc{van der linden}, Sophie. Para ler o livro ilustrado. São Paulo: Cosac Naify, 2011.

Livro sobre as particularidades do livro ilustrado, que apresenta as diferenças entre o livro ilustrado e o livro com ilustração. 
\end{itemize}

\end{document}
