\chapter{Apresentação}

Esta antologia é um convite para conhecer a obra de Monteiro Lobato
voltada para o público infantil, a que não faltam o espírito crítico, o
humor, o suspense, reviravoltas e a sabedoria do desejo de compreender
os conflitos humanos.

Com os textos de Monteiro Lobato, conhecemos melhor a realidade social
brasileira e experienciamos os dramas e momentos poéticos de personagens
como o comprador de fazendas, o Jeca Tatu, o estafeta, Negrinha, o
jardineiro Timóteo, o galo Peva, o menino Pedrinho. Impressiona a
atualidade da matéria e dos conflitos configurados pela arte de Lobato,
que nos possibilita entender melhor as iniquidades da sociedade
brasileira de origem colonial e escravocrata, rir e ou chorar das
histórias narradas, algumas das quais foram adaptadas para o cinema.

\chapter{Em férias}

Quando naquela tarde Pedrinho voltou da escola e disse à Dona Tonica que
as férias iam começar dali uma semana, a boa senhora perguntou:

--- E onde quer passar as férias deste ano, meu filho?

O menino riu-se.

--- Que pergunta, mamãe! Pois onde mais, senão no sítio de vovó?

Pedrinho não podia compreender férias passadas em outro lugar que não
fosse no Sítio do Picapau Amarelo, em companhia de Narizinho, do Marquês
de Rabicó, do Visconde de Sabugosa e da Emília. E tinha de ser assim
mesmo, porque Dona Benta era a melhor das vovós; Narizinho, a mais
galante das primas; Emília, a mais maluquinha de todas as bonecas; o
Marquês de Rabicó, o mais rabicó de todos os marqueses; e o Visconde de
Sabugosa, o mais ``cômodo'' de todos os viscondes.

E havia ainda Tia Nastácia, a melhor quituteira deste e de todos os
mundos que existem. Quem comia uma vez os seus bolinhos de polvilho não
podia nem sequer sentir o cheiro de bolos feitos por outras cozinheiras.

Pedrinho tinha recebido carta de sua prima, dizendo: ``Nosso grupo vai
este ano completar século e meio de idade e é preciso que você não deixe
de vir pelas férias a fim de comemorarmos o grande acontecimento''.

Esse século e meio de idade era contado assim: Dona Benta, 64 anos; Tia
Nastácia, 66; Narizinho, 8; Pedrinho, 9. Emília, o Marquês e o Visconde,
1 cada um. Ora, 64 mais 66 mais 8 mais 9 mais 1 mais 1 mais 1, fazem 150
anos, ou seja, um século e meio.

Logo que recebeu essa carta, Pedrinho fez a conta num papel para ver se
a pilhava em erro; mas não pilhou.

--- É uma danada aquela Narizinho! --- disse ele. --- Não há meio de
errar em contas.

%* ``Em férias'', capítulo 1 de \textit{O Saci} (1921).

\chapter{A pílula falante}

No outro dia a menina levantou-se muito cedo para levar a boneca ao
consultório do doutor Caramujo. Encontrou-o com cara de quem havia
comido um urutu recheado de escorpiões.

--- Que há, doutor?

--- Há que encontrei o meu depósito de pílulas saqueado. Furtaram-me
todas\ldots{}

--- Que maçada! --- exclamou a menina aborrecidíssima. --- Mas não pode
fabricar outras? Se quiser, ajudo a enrolar.

--- Impossível. Já morreu o besouro boticário que fazia as pílulas, sem
haver revelado o segredo a ninguém. A mim só me restava um cento, das
mil que comprei dos herdeiros. O miserável ladrão só deixou uma --- e
imprópria para o caso porque não é pílula falante.

--- E agora?

--- Agora, só fazendo uma certa operação. Abro a garganta da boneca muda
e ponho dentro uma falinha, respondeu o doutor, pegando na sua faca de
ponta para amolar. Já providenciei tudo.

Nesse momento ouviu-se grande barulheira no corredor.

--- Que será? --- indagou a menina surpresa.

--- É o papagaio que vem vindo --- declarou o doutor.

--- Que papagaio, homem de Deus? Que vem fazer aqui esse papagaio?

Mestre Caramujo explicou que como não houvesse encontrado suas pílulas
mandara pegar um papagaio muito falador que havia no reino. Tinha de
matá-lo para extrair a falinha que ia pôr dentro da boneca.

Narizinho, que não admitia que se matasse nem formiga, revoltou-se
contra a barbaridade.

--- Então não quero! Prefiro que Emília fique muda toda a vida a
sacrificar uma pobre ave que não tem culpa de coisa nenhuma.

Nem bem acabou de falar, e os ajudantes do doutor, uns caranguejos muito
antipáticos, surgiram à porta, arrastando um pobre papagaio de bico
amarrado. Bem que resistia ele, mas os caranguejos podiam mais e eram
murros e mais murros.

Furiosa com a estupidez, Narizinho avançou de sopapos e pontapés contra
os brutos.

--- Não quero! Não admito que judiem dele! --- berrou vermelhinha de
cólera, desamarrando o bico do papagaio e jogando as cordas no nariz dos
caranguejos.

O doutor Caramujo desapontou, porque sem pílulas nem papagaios era
impossível consertar a boneca. E deu ordem para que trouxessem o segundo
paciente.

Apareceu então o sapo num carrinho. Teve de vir sobre rodas por causa do
estufamento da barriga; parece que as pedras haviam crescido de volume
dentro. Como ainda estivesse vestido com a saia e a touca da Emília,
Narizinho viu-se obrigada a tapar a boca para não rir-se em momento tão
impróprio.

O grande cirurgião abriu com a faca a barriga do sapo e tirou com a
pinça de caranguejo a primeira pedra. Ao vê-la à luz do sol sua cara
abriu-se num sorriso caramujal.

--- Não é pedra, não! --- exclamou contentíssimo. --- É uma das minhas
queridas pílulas! Mas como teria ela ido parar na barriga deste
sapo?\ldots{}

Enfiou de novo a pinça e tirou nova pedra. Era outra pílula! E assim foi
indo até tirar lá de dentro noventa e nove pílulas.

A alegria do doutor foi imensa. Como não soubesse curar sem aquelas
pílulas, andava com medo de ser demitido de médico da corte.

--- Podemos agora curar a senhora Emília --- declarou ele depois de
costurar a barriga do sapo.

Veio a boneca. O doutor escolheu uma pílula falante e pôs-lhe na boca.

--- Engula duma vez! --- disse Narizinho, ensinando à Emília como se
engole pílula. E não faça tanta careta que arrebenta o outro olho.

Emília engoliu a pílula, muito bem engolida, e começou a falar no mesmo
instante. A primeira coisa que disse foi: ``Estou com um horrível gosto
de sapo na boca!''. E falou, falou, falou mais de uma hora sem parar.
Falou tanto que Narizinho, atordoada, disse ao doutor que era melhor
fazê-la vomitar aquela pílula e engolir outra mais fraca.

--- Não é preciso --- explicou o grande médico. --- Ela que fale até
cansar.

Depois de algumas horas de falação, sossega e fica como toda gente. Isto
é ``fala recolhida'', que tem de ser botada para fora.

E assim foi. Emília falou três horas sem tomar fôlego. Por fim calou-se.

--- Ora graças! --- exclamou a menina. --- Podemos agora conversar como
gente e saber quem foi o bandido que assaltou você na gruta. Conte o
caso direitinho.

Emília empertigou-se toda e começou a dizer na sua falinha fina de
boneca de pano:

--- Pois foi aquela diaba da dona Carocha. A coroca apareceu na gruta
das cascas\ldots{}

--- Que cascas, Emília? Você parece que ainda não está regulando\ldots{}

--- Cascas, sim --- repetiu a boneca teimosamente.

--- Dessas cascas de bichos moles que você tanto admira e chama conchas.
A coroca apareceu e começou a procurar aquele boneco\ldots{}

--- Que boneco, Emília?

--- O tal Polegada que furava bolos e você escondeu numa casca bem lá no
fundo. Começou a procurar e foi sacudindo as cascas uma por uma para ver
qual tinha boneco dentro. E tanto procurou que achou. E agarrou na casca
e foi saindo com ela debaixo do cobertor\ldots{}

--- Da mantilha, Emília!

--- Do \textsc{cobertor}.

--- Mantilha, boba!

--- \textsc{cobertor}. Foi saindo com ela debaixo do \textsc{cobertor} e eu vi e pulei
para cima dela. Mas a coroca me unhou a cara e me bateu com a casca na
cabeça, com tanta força que dormi. Só acordei quando o doutor Cara de
Coruja\ldots{}

--- Doutor Caramujo, Emília!

--- Doutor \textsc{cara de coruja}. Só acordei quando o doutor \textsc{cara de
corujíssima} me pregou um liscabão.

--- Beliscão --- emendou Narizinho pela última vez, enfiando a boneca no
bolso. Viu que a fala da Emília ainda não estava bem ajustada, coisa que
só o tempo poderia conseguir. Viu também que era de gênio teimoso e
asneirenta por natureza, pensando a respeito de tudo de um modo especial
todo seu.

--- Melhor que seja assim, --- filosofou Narizinho. --- As ideias de
vovó e tia Nastácia a respeito de tudo são tão sabidas que a gente já as
adivinha antes que elas abram a boca. As ideias de Emília hão de ser
sempre novidades.

E voltou para o palácio, onde a corte estava reunida para outra festa
que o príncipe havia organizado. Mas assim que entrou na sala de baile,
rompeu um grande estrondo lá fora --- o estrondo duma voz que dizia:

--- Narizinho, vovó está chamando\ldots{} Tamanho susto causou aquele
trovão entre os personagens do reino marinho, que todos se sumiram, como
por encanto.

Sobreveio então uma ventania muito forte, que envolveu a menina e a
boneca, arrastando-as do fundo do oceano para a beira do ribeirãozinho
do pomar. Estavam no sítio de dona Benta outra vez. Narizinho correu
para casa. Assim que a viu entrar, dona Benta foi dizendo:

--- Uma grande novidade, Lúcia. Você vai ter agora um bom companheiro
aqui no sítio para brincar. Adivinhe quem é?

A menina lembrou-se logo do Major Agarra, que prometera vir morar com
ela.

--- Já sei vovó! É o Major Agarra-e-não-larga-mais. Ele bem me falou que
vinha.

Dona Benta fez cara de espanto.

--- Você está sonhando, menina. Não se trata de major nenhum.

--- Se não é o sapo, então é o papagaio! --- continuou Narizinho,
recordando-se de que também o papagaio prometera vir visitá-la.

--- Qual sapo, nem papagaio, nem elefante, nem jacaré. Quem vem passar
uns tempos conosco é o Pedrinho, filho da minha filha Antonica.

Lúcia deu três pinotes de alegria.

--- E quando chega o meu primo? --- indagou.

--- Deve chegar amanhã de manhã. Apronte-se. Arrume o quarto de hóspedes
e endireite essa boneca. Onde se viu uma menina do seu tamanho andar com
uma boneca em fraldas de camisa e de um olho só?

--- Culpa dela, dona Benta! Narizinho tirou minha saia para vestir o
sapão rajado --- disse Emília falando pela primeira vez depois que
chegara ao sítio.

Tamanho susto levou dona Benta, que por um triz não caiu de sua
cadeirinha de pernas serradas. De olhos arregaladíssimos, gritou para a
cozinha:

--- Corra, Nastácia! Venha ver este fenômeno\ldots{}

A negra apareceu na sala, enxugando as mãos no avental.

--- Que é, sinhá? --- perguntou.

--- A boneca de Narizinho está falando!\ldots{} A boa negra deu uma
risada gostosa, com a beiçaria inteira.

--- Impossível, sinhá! Isso é coisa que nunca se viu. Narizinho está
mangando com mecê.

--- Mangando o seu nariz! --- gritou Emília furiosa. --- Falo, sim, e
hei de falar. Eu não falava porque era muda, mas o doutor Cara de Coruja
me deu uma bolinha de barriga de sapo e eu engoli e fiquei falando e hei
de falar a vida inteira, sabe?

A negra abriu a maior boca do mundo.

--- E fala mesmo, sinhá!\ldots{} --- exclamou no auge do assombro.

--- Fala que nem uma gente! Credo! O mundo está perdido\ldots{}

E encostou-se à parede para não cair.

%* ``A pílula falante'', de \textit{A menina do nariz arrebitado} (1921); \textit{Reinações de Narizinho} (1931).

\chapter{O passarinho-ninho}

A resposta foi um ``Aqui!'' vindo do pomar. Correndo no rumo da voz, a
menina encontrou Emília tão entretida com um passarinho que nem sequer a
olhou. Estava afundando as costas dum tico-tico. Todos os passarinhos
têm costas ``convexas'', isto é, arredondadas para cima. Emília estava
fazendo um passarinho de costas ``côncavas'', isto é, com um afundamento
redondo nas costas. A Rã ficou a olhar para aquilo sem entender coisa
nenhuma, até que Emília explicou.

--- Estou fazendo o passarinho-ninho. A boba da Natureza arruma as
coisas às tontas, sem raciocinar. Os passarinhos, por exemplo. Ela os
ensina a fazer ninhos nas árvores. Haverá maior perigo? Os ovos e os
filhotes ficam sujeitos à chuva, às cobras, às formigas, às ventanias. O
ano passado deu por aqui um pé-de-vento que derrubou o ninho deste
tico-tico, ali da minha pitangueira --- e lá se foram três ovos tão
bonitinhos, todos sardentinhos. E mais uma vez me convenci da
``tortura'' das coisas. Comecei a reforma da Natureza por este
passarinho.

A Rã não entendeu que reforma era aquela e perguntou:

--- Para que esse afundamento aí nas costas do tico-tico?

--- Pois é o ninho --- respondeu Emília. --- Faço o ninho dele aqui nas
costas e pronto. Para onde ele for, lá vão também os ovos ou os filhotes
--- e não há perigo de cobra, nem de ventania, nem de chuva.

--- De chuva há --- disse a Rãzinha. --- Nos ninhos em árvores a fêmea
está sempre em cima dos ovos.

Mas aí\ldots{}

Emília fez um muxoxo de superioridade.

--- Já previ todas as hipóteses --- disse ela. --- Faço a caudinha dele
bem móvel, de modo que possa virar para trás e cobrir os ovos quando for
preciso, como se fosse um telhadinho.

A Rã deu-se por satisfeita e com a maior atenção acompanhou o preparo do
primeiro passarinho ninho do mundo.

--- Pronto! --- exclamou Emília por fim. --- Passam só os ovos. Corra
ali e me traga o tico-tico fêmea que está na gaiola.

A Rã foi e trouxe o passarinho. Emília pegou-o com muito jeito e
espremeu-o de modo que saíssem três ovinhos sardentos, os quais
depositou com muito cuidado no ninho de penas feito nas costas do
tico-tico macho --- e soltou os dois, pelo ar.

Emília estava radiante.

--- Lá se foram! --- exclamou. --- Acabaram-se as inquietações, os medos
de cobra, formiga ou vento. E também se acabou o desaforo de todo o
trabalho de botar e chocar os ovos caber só à fêmea. Os homens sempre
abusaram das mulheres. Dona Benta diz que nos tempos antigos, e mesmo
hoje entre os selvagens, os marmanjos ficam no macio, pitando nas redes,
ou só se ocupam dos divertimentos da caça e da guerra, enquanto as
pobres mulheres fazem toda a trabalheira, e passam a vida lavando e
cozinhando e varrendo e aturando os filhos. E se não andam muito
direitinhas, levam pau no lombo.

Os machos sempre abusaram das fêmeas, mas agora as coisas vão mudar.
Este tico-tico, por exemplo, tem que tomar conta dos ovos. A fêmea fica
com o trabalho de botá-los, mas o macho tem que tomar conta deles.

--- Mas assim os ovos não chocam --- objetou a Rãzinha.

--- Para que choquem é preciso que as fêmeas fiquem uma porção de dias
sentadas sobre eles. As galinhas levam 21 dias no choco.

--- Já ``previ a hipótese'' --- disse Emília --- e reformei esse ponto.
No meu sistema de passarinho ninho quem choca não é a fêmea e sim o sol,
como acontece com os ovos dos jacarés, tartarugas, lagartixas e cobras.

--- E quando não houver sol? Às vezes passam-se dias sem o sol aparecer.

--- Nesse caso os ovos que tenham paciência e esperem que o sol apareça.
Para que pressa?

A Rã não teve mais nada a dizer. Estava certo. Só então é que Emília se
lembrou de cumprimentá-la e saber como iam todos lá da casa. Também lhe
examinou as mãos para ver se as unhas estavam de luto.

E fê-la voltar-se de perfil e de costas, e dar três pulos. Era a
primeira vez que as duas se encontravam pessoalmente.

--- Estou gostando do seu físico --- disse Emília no fim do exame. ---
Tive medo de que não correspondesse à ideia que fiz. Muitas vezes a
gente imagina uma pessoa e sai o contrário.

Gostei muito da sua última carta sobre a reforma das cidades e das
gentes. Adoro você, Rã, porque você não concorda.

--- Ah, não concordo mesmo! --- exclamou a Rãzinha. --- Vivo não
concordando. Em nós, gente, por exemplo, quanta coisa errada! Por que
dois olhos na frente e nenhum na nuca?

Eu, se fosse reformar as criaturas, punha um olho na testa e outro na
nuca. Desse modo eu dobrava a segurança das criaturas.

--- Pois eu aumentava o número de olhos --- disse Emília.

--- Por que dois só? Assim como temos dez dedos podíamos ter dez olhos.
Eu punha quatro na cabeça, a norte, sul, leste e oeste. Eu punha dois
nos dedões dos pés, para evitar as topadas. Outro dia Pedrinho deu uma
topada num tijolo que quase arrancou a unha. Com um olho em cada dedão
não há perigo de topadas --- nem de espinhos e estrepes. E eu também
dava olhos a cada dedo minguinho. O minguinho é um verdadeiro vagabundo
nas mãos. Não faz nada. Fica o tempo todo assistindo ao trabalho dos
outros. Ora, se o ``mingo'' tivesse um olhinho na ponta, podia prestar
bons serviços. Às vezes a gente quer enxergar numa cova de dente ou ver
se há cera no ouvido e não pode. Com o olho do ``mingo'', nada mais
fácil.

--- E esse olho do minguinho --- ajuntou a Rã --- podia ser como os
microscópios, capaz de enxergar coisinhas invisíveis aos olhos comuns.
Mas haveria um inconveniente, Emília. As mãos lidam com tudo, trabalham
muito, e esses olhos do minguinho haviam de viver se enchendo de cisco
ou se arranhando --- e que dor!

--- Nada mais fácil do que evitar isso --- lembrou Emília.

--- Basta que usem dedaizinhos. Ficam cobertos quando não tiverem o que
fazer. Mas por enquanto não podemos reformar gente, porque não há gente
aqui. Todos os humanos do sítio foram para a Europa.

--- E Rabicó?

--- Esse é desumano e quadrupedíssimo. Já pensei muito na reforma de
Rabicó. Podemos transformá-lo em bípede e\ldots{}

--- E acabar com aquela mania de comer tudo quanto encontra ---
continuou a Rã. --- Eu faria assim: no focinho punha uma espécie de
ratoeira, sempre armada; quando ele avançasse num doce ou em qualquer
coisa séria, como aquela coroa do casamento de Narizinho, a ratoeira
desarmava e segurava-lhe o focinho.

E também dava-lhe pernas de tartaruga, para que não pudesse fugir quando
Pedrinho o perseguisse com o bodoque.

Emília olhou para a Rã com ar desconfiado. Aquelas ideias pareceram-lhe
absurdas. A ratoeira impediria Rabicó de comer não só cocadas e
coroinhas como tudo mais, e ele morreria de fome.

--- ``Bissurdo'', Rã! --- disse ela. --- A sua ratoeira acabava matando
Rabicó e Dona Benta ficava danada.

--- Você não me entendeu, Emília. A ratoeira só funcionaria quando ele
quisesse comer coroinhas.

Para abóbora, milho, mandioca e o resto, não.

--- Mas como a ratoeira podia saber quando era coroinha?

--- Pelo cheiro. Eu punha um bom nariz na ratoeira.

Emília olhou para a Rã com o rabo dos olhos. Aquela menina estava com
jeito de ser maluca\ldots{}

Apesar disso encarregou-a de reformar Rabicó. A Rã mudou de assunto.

--- Na carta que você me escreveu, Emília, encontrei a palavra
``bissolutamente'' em vez de ``absolutamente'' e agora você disse
``bissurdo'' em vez de ``absurdo''. Está reformando as palavras também?

--- Ainda não, mas já pensei nisso. Por enquanto me limito a cortar uma
ou outra letra com a qual me implico. O ``a'' de certas palavras me
obriga a abrir muito a boca --- e meu queixo pode cair, como o da filha
de Nhá Veva. Experimente dizer absurdo sem abrir a boca.

A Rã experimentou e não conseguiu, mas ``bissurdo'' ela disse quase de
boca fechada.

--- Pois aí está! --- tornou Emília. --- Tudo errado, até o ``a'' de
certas palavras. O mundo é uma grande trapalhada. Para que, por exemplo,
caudinha em Rabicó? Na vaca Mocha a cauda tem razão de ser --- serve
para espantar as moscas. É um espanador. Mas em Rabicó? Para que serve
aquele caracolzinho pelado?

--- Para enfeite do fim --- lembrou a Rã.

--- Que fim?

--- O fim de Rabicó. Todos os fins têm caudinhas. É o remate. Mamãe diz
que é feio comer e deixar o prato limpo, ou beber um cálice de licor sem
deixar um bocadinho no fundo. São caudinhas. São os enfeites da boa
educação.

Emília estava cada vez mais desconfiada da Rãzinha. Parecia a Alice do
País das Maravilhas. Só vinha com disparates. E disse:

--- Enfeites são inutilidades. Não quero saber de enfeites nas minhas
reformas. Tudo há de ter uma razão científica. Aquela ideia da carta
sobre a reforma do Quindim me pareceu maluca. Acho que você quer brincar
com a Natureza, menina. Eu quero corrigir a Natureza, quero melhorá-la,
entende? Não se trata de nenhuma brincadeira. Negócio sério. Aí está a
diferença entre nós. Na última carta você falou em substituir o couro do
Quindim por um veludo. Isso é asneira.

--- Mas que necessidade tem Quindim dum couro duríssimo, aqui no Picapau
Amarelo, onde não há espinhos africanos?

--- Concordo. Poderá ter um couro mais fino, assim como a camurça; mas
de veludo, Rã, é demais. Às vezes penso que você está sabotando a minha
ideia de reforma da Natureza\ldots{}

%* ``O passarinho-ninho'', capítulo 3 de \textit{A reforma da natureza} (1941).

\chapter{Reformas na Europa e nas pulgas}

Depois falaram da viagem de Dona Benta à Europa. A Rã achou que ela não
conseguiria nada porque os homens são errados de nascença. Emília
discordou.

--- Eu conheço as ideias de ``vovó'' --- disse ela. --- A primeira coisa
que vai fazer na Conferência é transformar o mundo numa Confederação
Universal. Todos os países ficarão fazendo parte dessa confederação,
como os Estados dos Estados Unidos. E vai acabar com os exércitos e as
marinhas, com os canhões e as metralhadoras.

A Rã, que entendia um pouco de política, achou que as grandes nações
eram muito orgulhosas para se sujeitarem a ser simples Estados dum
grande Estados Unidos.

--- Pois se não se sujeitarem, pior para elas --- declarou Emília. ---
Dona Benta acha que os homens devem formar no mundo uma coisa assim como
as formigas. Elas são de muitas raças, ruivas, pretas, saúvas,
sará-sarás, quenquéns etc. mas vivem perfeitamente lado a lado umas das
outras, sem se guerrearem, sem se destruírem. Se as formigas conseguem
isso, por que os homens não conseguirão o mesmo?

--- Mas acha que os grandes de lá --- os reis, os ditadores, os homens
importantes --- vão seguir os conselhos de Dona Benta e tia Nastácia?

--- E que remédio? --- respondeu Emília. --- Enquanto eles se guiaram
pelas suas próprias cabeças só saiu piolho: desgraças e mais desgraças,
destruições sem fim. Eles devem estar convencidos de que, apesar de toda
a importância, não passam duns tremendos pedaços de asnos.

A Rã concordou.

À noite, quando foram dormir, ficaram as duas na mesma cama conversando
até tarde da noite. O assunto era sempre o mesmo: reformas e mais
reformas. Em certo momento uma pulga mordeu Emília. Ela acendeu a luz e
pôs-se a caçá-la na brancura do lençol. Pegou-a, afinal. Enrolou-a bem
enrolada entre os dedos e largou-a ``para ver''. E o que viu foi a pulga
reviver e escapar aos pulos.

Emília danou.

--- É sempre o que me acontece! Esfrego, enrolo as pulgas e elas se
desesfregam, se desenrolam e saem pulando.

Tenho também de reformar as pulgas.

--- Como?

--- Poderei fazê-las molinhas como qualquer mosca. Já reparou que, para
o tamanho que têm, as pulgas são a coisa mais rija que existe no mundo?
Mais rijas que borracha\ldots{} E também vou mudar a velocidade do pulo
das pulgas. Paço pulos em câmara lenta, de modo que a gente possa
pegá-las no ar com a maior facilidade, como se estivéssemos colhendo uma
bolotinha.

A Rã lembrou um ``melhoramento'' ainda melhor.

--- E se cortássemos o pulo das pulgas pelo meio? --- disse ela.

Emília não entendeu.

--- Cortar, como?

--- A pulga pula. Quando chega no ponto mais alto do pulo, para. Fica
paradinha no ar, como um ponto final. E a gente, sossegadamente, a pega
e a estala entre as unhas. Gosto muito de ouvir o estalinho das pulgas.
É o único inseto que tem essa habilidade.

--- As baratas também sabem estalar --- lembrou Emília.

--- Cada vez que Narizinho pisa numa, ela estala. É a linguagem das
pulgas e das baratas. E também dos chicotes.

Pedrinho tem um chicote que é mestre em estalos.

%* ``Reformas na Europa e nas pulgas'', capítulo 6 de \textit{A reforma da natureza} (1941).

\chapter{No dia seguinte}

No dia seguinte pularam da cama muito cedo e retomaram a obra de reforma
da Natureza. Tudo era examinado e reformado no que lhes parecia torto. A
Rãzinha continuava com as ideias mais absurdas, de verdadeira maluca.

A reforma do Quindim, por exemplo, que a Rã fez sozinha, era a coisa
mais esquisita que se possa imaginar. Em vez do famoso chifre sobre o
nariz, que é característico de todos os rinocerontes, a Rã botou uma
flecha de Cupido com um coração assado na ponta. Assado, imaginem! E
ornamentou os cascos de Quindim com pinturas: Branca de Neve com todos
os seus anões. E trocou as quatro pernas do rinoceronte por quatro
pernas diferentes --- uma de veado, outra de ganso, outra de jacaré,
outra de pau. E substituiu aquele couro duríssimo por um revestimento
muito bem trançado de palhinha de cadeira. Cauda, botou duas; depois
três, depois dez, depois cem; deixou-o com um verdadeiro varal de caudas
dando volta inteira em redor do pobre animal.

A reforma do Quindim saiu um tal disparate que nem andar ele podia ---
uma perninha não acompanhava a outra, e havia a tremenda atrapalhação de
tantas caudas, todas diferentes, umas com bolas na ponta, outras com
espinhos de ouriço, outras com campainhas.

Quando Emília foi ver a ``obra'', não pôde deixar de rir se. Aquilo era
o ``bissurdo dos bissurdos''. Quindim estava transformado num verdadeiro
destampatório.

--- Isso não é reformar, Rãzinha! --- disse ela. --- Isso é escangalhar
com uma pobre criatura. Ele já não é rinoceronte, nem nenhum bicho
possível. Virou quarto de badulaques, baú de mascate. Que
judiação!\ldots{}

--- E você deixa que ele fique assim? --- implorou a Rã, com medo que
Emília desmanchasse aquela obra-prima do disparate humano.

--- Deixo por enquanto --- respondeu Emília --- como castigo da
preguiça, da velhice e neurastenia que ele anda mostrando duns tempos
para cá. No dia do plebiscito sobre o tamanho Quindim me traiu ---
recusou-se a votar. A falta desse voto deu vitória ao Tamanho e eu saí
lograda. Agora que aguente. Mais tarde vou reformá-lo de novo, mas com
critério científico\ldots{}

A Rã ou era mesmo maluca ou estava ``sabotando'' a obra reformatória da
Emília. Todas as ideias que apresentava eram tontas, como aquela da
mudança dos morros. A Rã tomou um lápis e traçou um desenho assim:

--- Que é isso? --- perguntou Emília.

--- Ah, isto é uma das reformas que acho mais necessárias: as reformas
dos morros. Sempre que tenho de subir um morro, fico cansada e sem
fôlego. E então imaginei uma coisa assim: os picos serão para baixo, em
vez de serem para cima, de modo que quando a gente tem de ir ao pico dum
morro, desce, em vez de subir\ldots{}

Emília ficou a olhar, ora para a Rã ora para o desenho. Era uma reforma
que deixava tudo na mesma.

Quando alguém que descesse ao pico do morro tivesse de voltar, teria de
subir para o vale\ldots{}

--- Não. Essa ideia está boba. Muito melhor fazermos os morros bem
baixinhos, de modo que não canse a gente; ou então deixarmos os morros
em paz.

Para que subir morro?

%* ``No dia seguinte'', capítulo 8 de \textit{A reforma da natureza} (1941).

\chapter{O livro comestível}

A maior parte das ideias da Rã eram desse tipo. Pareciam brincadeiras, e
isso irritava Emília, que estava tomando muito a sério o seu programa de
reforma do mundo. Emília sempre foi uma criaturinha muito séria e
convencida. Não fazia nada de brincadeira.

--- Parece incrível, Rã! --- disse ela. --- Chamei você para me ajudar
com ideia na reforma, mas até agora não saiu dessa cabecinha uma só
coisa aproveitável --- só ``desmoralizações''\ldots{}

--- Isso não! A ideia das tetas com torneiras na Mocha foi minha e você
gostou muito. A da pulga também.

--- Só essas. Todas as outras eu tive de jogar no lixo. Vamos ver mais
uma coisa. Que acha que devemos fazer para a reforma dos livros?

A Rãzinha pensou, pensou e não se lembrou de nada.

--- Não sei. Parecem-me bem como estão.

--- Pois eu tenho uma ideia muito boa --- disse Emília. --- Fazer o
livro comestível.

--- Que história é essa?

--- Muito simples. Em vez de impressos em papel de madeira, que só é
comestível para o caruncho, eu farei os livros impressos em um papel
fabricado de trigo e muito bem temperado. A tinta será estudada pelos
químicos --- uma tinta que não faça mal para o estômago. O leitor vai
lendo o livro e comendo as folhas; lê uma, rasga-a e come. Quando chega
ao fim da leitura está almoçado ou jantado. Que tal?

A Rãzinha gostou tanto da ideia que até lambeu os beiços.

--- Ótimo, Emília! Isto é mais que uma ideia-mãe. E cada capítulo do
livro será feito com papel de um certo gosto. As primeiras páginas terão
gosto de sopa; as seguintes terão gosto de salada, de assado, de arroz,
de tutu de feijão com torresmos. As últimas serão as da sobremesa ---
gosto de manjar branco, de pudim de laranja, de doce de batata.

--- E as folhas do índice --- disse Emília --- terão gosto de café ---
serão o cafezinho final do leitor. Dizem que o livro é o pão do
espírito. Por que não ser também pão do corpo? As vantagens seriam
imensas. Poderiam ser vendidos nas padarias e confeitarias, ou entregues
de manhã pelas carrocinhas, juntamente com o pão e o leite.

--- Nem precisaria mais pão, Emília! O velho pão viraria livro. O
Livro-Pão, o Pão-Livro. Quem soube ler, lê o livro e depois come; quem
não souber ler come-o só, sem ler. Desse modo o livro pode ter entrada
em todas as casas, seja dos sábios, seja dos analfabetos. Otimíssima
ideia, Emília!

--- Sim --- disse esta muito satisfeita com o entusiasmo da Rã. ---
Porque, afinal de contas, isso de fazer os livros só comíveis para o
caruncho é bobagem --- podemos fazê-los comíveis para nós também.

--- E quem deu a você essa ideia, Emília?

--- Foi o raciocínio. O livro existe para ser lido, não é?

Mas depois que o lemos e ficamos com toda a história na cabeça, o livro
se torna uma inutilidade na casa. Ora, tornando se comestível,
diminuímos uma inutilidade.

--- E quando a gente quiser reler um livro?

--- Compra outro, do mesmo modo que compramos outro pão todos os dias.

A ideia, depois de discutida em todos os seus aspectos, foi aprovada, e
Emília reformou toda a biblioteca de Dona Benta.

Fez um papel gostosíssimo e de muito fácil digestão, com sabor e cheiro
bastante variados, de modo que todos os paladares se satisfizessem. Só
não reformou os dicionários e outros livros de consulta. Emília pensava
em tudo.

Também reformou muita coisa na casa. Por meio de cordas e carretilhas as
camas subiam para o forro de manhã, depois de desocupadas, a fim de
aumentar o espaço dos cômodos. As fechaduras não precisavam de chaves;
bastava que as pessoas pusessem a boca no buraco e dissessem:

``Sésamo, abre-te'' e elas se abriam por si mesmas.

--- E os mudos? --- perguntou a Rãzinha. --- Como vão arrumar-se? Só se
eles andarem com uma vitrola no bolso, que pronuncie por eles a palavra
Sésamo.

Emília atrapalhou-se com o caso dos mudos e deixou-o para resolver
depois.

O leite a ferver ao fogo dava um assobio quando chegava no ponto, de
modo a avisar ao fogo, o qual imediatamente parava de agir. O mesmo com
todas as comidas --- e dessa maneira acabou-se a desagradável história
do ``feijão com bispo''.

E tanta e tanta coisa as duas fizeram, que se fôssemos contar metade
teríamos de encher dois volumes. Lá pelo fim da semana o Sítio do
Picapau estava totalmente transformado, não dando a menor ideia do
antigo. Foi por essa ocasião que chegou carta de Dona Benta anunciando a
volta.

--- ``Já concluímos o nosso serviço na Europa'' --- dizia ela.

--- ``Deixamos o continente transformado num perfeito sítio --- com tudo
direitinho e todos contentes e felizes. A Comissão que nos trouxe vai
reconduzir-nos para aí novamente. Devemos chegar na próxima
segunda-feira e espero encontrar tudo em ordem.''

Emília leu a carta para a Rãzinha, dizendo:

``É uma danada, esta velha! Foi lá e fez o que todos aqueles ditadores e
reis não conseguiram. Temos agora de preparar a casa para recebê-la.''

%* ``O livro comestível'', capítulo 9 de \textit{A reforma da natureza} (1941).

\chapter{A menina do leite}

Laurinha, no seu vestido novo de pintas vermelhas, chinelos de bezerro,
treque, treque, treque, lá ia para o mercado com uma lata de leite à
cabeça --- o primeiro leite da sua vaquinha mocha. Ia contente da vida,
rindo-se e falando sozinha.

--- Vendo o leite --- dizia --- e compro uma dúzia de ovos. Choco os
ovos e antes de um mês já tenho uma dúzia de pintos. Morrem\ldots{}
dois, que seja, e crescem dez --- cinco frangas e cinco frangos. Vendo
os frangos e crio as frangas, que crescem, viram ótimas botadeiras de
duzentos ovos por ano cada uma. Cinco mil ovos! Choco tudo e lá me vêm
quinhentos galos e mais outro tanto de galinhas. Vendo os galos. A 2
cruzeiros cada um --- 2 vezes 5, 10\ldots{} 1.000 cruzeiros!\ldots{}
Posso então comprar doze porcas de cria e mais uma cabrita. As porcas
dão-me, cada uma, seis leitões. Seis vezes 12\ldots{}

Estava a menina neste ponto quando tropeçou, perdeu o equilíbrio e, com
lata e tudo, caiu um grande tombo no chão.

Pobre Laurinha!

Ergueu-se chorosa, com um ardor de esfoladura no joelho; e enquanto
espanejava as roupas sujas de pó viu sumir-se, embebido pela terra seca,
o primeiro leite da sua vaquinha mocha e com ele os doze ovos, as cinco
botadeiras, os quinhentos galos, as doze porcas de cria, a cabritinha
--- todos os belos sonhos da sua ardente imaginação\ldots{}

Emília bateu palmas.

--- Viva! Viva a Laurinha!\ldots{} No nosso passeio ao País das Fábulas
tivemos ocasião de ver essa história formar-se --- mas o fim foi
diferente. Laurinha estava esperta e não derrubou o pote de leite,
porque não carregava o leite em pote nenhum, e sim numa lata de metal
bem fechada. Lembra-se, Narizinho?

A menina lembrava-se.

--- Sim --- disse ela. --- Lembro-me muito bem. A Laurinha não derramou
o leite e deixou a fábula errada. O certo é como vovó acaba de contar.

--- Está claro, minha filha --- concordou Dona Benta. --- É preciso que
Laurinha derrame o leite para que possamos extrair uma moralidade da
história.

--- Que é moralidade, vovó?

--- É a lição moral da história. Nesta fábula da menina do leite a
moralidade é que não devemos contar com uma coisa antes de a termos
conseguido\ldots{}

%* ``A menina do leite'', de \textit{Fábulas} (1922).

\chapter{O carreiro e o papagaio}

Vinha um carreiro à frente dos bois, cantarolando pela estrada sem fim.
Estrada de lama.

Em certo ponto o carro atolou.

O pobre homem aguilhoa os bois, dá pancadas, grita; nada consegue e
põe-se a lamentar a sorte.

--- Desgraçado que sou! Que fazer agora, sozinho neste deserto? Se ao
menos São Benedito tivesse dó de mim e me ajudasse\ldots{}

Um papagaio escondido entre as folhas condoeu-se dele e, imitando a voz
de santo, começou a falar:

--- Os céus te ouviram, amigo, e Benedito em pessoa aqui está para o
ajutório que pedes.

O carreiro, num assombro, exclama:

--- Obrigado, meu santo! Mas onde estás que não te vejo?

--- Ao teu lado. Não me vês porque sou invisível. Mas, vamos, faze o que
mando. Toma da enxada e cava aqui. Isso. Agora a mesma coisa do outro
lado. Isso. Agora vais cortar uns ramos e estivar o sulco aberto. Isso.
Agora vais aguilhoar os bois.

O carreiro fez tudo como o papagaio mandou e com grande alegria viu
desatolar-se o carro.

--- Obrigado, meu santo! --- exclamou ele de mãos postas. --- Nunca me
hei de esquecer do grande socorro prestado, pois que sem ele eu ficaria
aqui toda a vida.

O papagaio achou muita graça na ingenuidade do homem e papagueou, como
despedida, um velho rifão popular:

--- Ajuda-te, que o céu te ajudará.

--- Como são sabidinhos esses bichos das fábulas! Esse papagaio, então,
está um suco!

--- Suco de quê, minha filha? --- perguntou Dona Benta.

--- De sabedoria, vovó! O meio de a gente se sair de uma dificuldade é
sempre esse --- lutar, lutar\ldots{}

--- Eu sei de outro muito melhor --- disse Emília. --- Dez vezes
melhor\ldots{}

A menina admirou-se.

--- Qual é, Emília?

--- É quando todos estão desesperados e tontos, sem saber o que fazer,
voltarem-se para mim e: ``Emília, acuda!'', e eu vou e aplico o faz de
conta e resolvo o problema. Aqui nesta casa ninguém luta para resolver
as dificuldades; todos apelam para mim\ldots{}

--- E você manda o Visconde. Sem o faz de conta e o Visconde ela não se
arranja.

--- Mas o caso é que os problemas se resolvem. É ou não?

Narizinho teve de concordar com ela.

%* ``O carreiro e o papagaio'', de \textit{Fábulas} (1922).

\chapter{Os dois burrinhos}

Muito lampeiros, dois burrinhos de tropa seguiam trotando pela estrada
além. O da frente conduzia bruacas de ouro em pó; e o de trás, simples
sacos de farelo. Embora burros da mesma igualha, não queria o primeiro
que o segundo lhe caminhasse ao lado.

--- Alto lá! --- dizia ele. --- Não se emparelhe comigo, que quem
carrega ouro não é do mesmo naipe de quem conduz farelo. Guarde cinco
passos de distância e caminhe respeitoso como se fosse um pajem.

O burrinho do farelo submetia-se e lá trotava na traseira, de orelhas
murchas, roendo-se de inveja do fidalgo.

De repente\ldots{}

--- Oah! oah!\ldots{}

São ladrões da montanha que surgem de trás de um toco e agarram os
burrinhos pelos cabrestos.

Examinam primeiramente a carga do burro humilde:

--- Farelo! --- exclamam desapontados. --- O demo o leve! Vejamos se há
coisa de mais valor no da frente.

--- Ouro, ouro! --- gritam, arregalando os olhos. E atiram-se ao saque.

Mas o burrinho resiste. Desfere coices e dispara pelo campo afora. Os
ladrões correm-lhe atrás, cercam-no e dão-lhe em cima, de pau e pedra.
Afinal saqueiam-no.

Terminada a festa, o burrinho do ouro, mais morto que vivo e tão surrado
que nem se suster em pé podia, reclama o auxílio do outro que muito
fresco da vida tosava o capim sossegadamente.

--- Socorro, amigo! Venha acudir-me, que estou descadeirado\ldots{}

O burrinho do farelo respondeu zombeteiramente:

--- Mas poderei por acaso aproximar-me de Vossa Excelência?

--- Como não? Minha fidalguia estava toda dentro da bruaca e lá se foi
nas mãos daqueles patifes.

Sem as bruacas de ouro no lombo, sou uma pobre besta igual a
você\ldots{}

--- Bem sei. Você é como certos grandes homens do mundo que só valem
pelo cargo que ocupam.

No fundo, simples bestas de carga, eu, tu, eles\ldots{}

E ajudou-o a regressar para casa, decorando, para uso próprio, a lição
que ardia no lombo do vaidoso.

Eis aqui, meus filhos, outra fábula bem boa --- disse Dona Benta. --- O
mundo está cheio de orgulhosos deste naipe\ldots{}

--- Que é ``naipe''? --- quis saber Narizinho.

--- É um termo usado para as cartas de jogar. Há quatro naipes --- ouro,
espadas, copas e paus.

--- Então naipe quer dizer ``qualidade'', ``tipo''? ``Do mesmo naipe''
quer dizer ``do mesmo tipo''?

--- Exatamente.

--- E ``igualha'', vovó?

--- É sinônimo de naipe.

--- Então por que a senhora não diz logo ``qualidade'' em vez de
``naipe'' e ``igualha''?

--- Para variar, minha filha. Estou contando estas fábulas em estilo
literário, e uma das qualidades do estilo literário é a variedade.

Pedrinho observou que o Coronel Teodorico fizera tal qual o burrinho do
ouro. Quando se encheu de dinheiro, arrotou grandeza; mas depois que
perdeu tudo nos maus negócios ficou de orelhas murchas e convencido de
que era realmente uma perfeita cavalgadura.

``Os dois burrinhos'', de \textit{Fábulas} (1922).

\chapter{A raposa e as uvas}

Certa raposa esfaimada encontrou uma parreira carregadinha de lindos
cachos maduros, coisa de fazer vir água à boca. Mas tão altos que nem
pulando.

O matreiro bicho torceu o focinho.

--- Estão verdes --- murmurou. --- Uvas verdes, só para cachorro.

E foi-se.

Nisso deu o vento e uma folha caiu.

A raposa, ouvindo o barulhinho, voltou depressa e pôs-se a
farejar\ldots{}

Quem desdenha quer comprar.

--- Que coisa certa, vovó! --- exclamou a menina. --- Outro dia eu vi
esta fábula em carne e osso. A filha do Elias Turco estava sentada à
porta da venda. Eu passei no meu vestidinho novo de pintas cor-de-rosa,
e ela fez um muxoxo. ``Não gosto de chita cor-de-rosa.'' Uma semana
depois lá a encontrei toda importante num vestido cor-de-rosa igualzinho
ao meu, namorando o filho do Quindó\ldots{}

%* ``A raposa e as uvas'', de \textit{Fábulas} (1922).

\chapter{O cão e o lobo}

Um lobo muito magro e faminto, todo pele e ossos, pôs-se um dia a
filosofar sobre as tristezas da vida. E nisso estava quando lhe surge
pela frente um cão --- mas um cão e tanto, gordo, forte, de pelo fino e
lustroso.

Espicaçado pela fome, o lobo teve ímpeto de atirar-se a ele. A
prudência, entretanto, cochichou-lhe ao ouvido: ``Cuidado! Quem se mete
a lutar com um cão desses sai perdendo''.

O lobo aproximou-se do cão com toda a cautela e disse:

--- Bravos! Palavra de honra que nunca vi um cão mais gordo nem mais
forte. Que pernas rijas, que pelo macio! Vê-se que o amigo se
trata\ldots{}

--- É verdade! --- respondeu o cão. --- Confesso que tenho tratamento de
fidalgo. Mas, amigo lobo, suponho que você pode levar a mesma boa vida
que levo\ldots{}

--- Como?

--- Basta que abandone esse viver errante, esses hábitos selvagens e se
civilize, como eu.

--- Explique-me lá isso por miúdo --- pediu o lobo com um brilho de
esperança nos olhos.

--- É fácil. Eu apresento você ao meu senhor. Ele, está claro,
simpatiza-se e dá a você o mesmo tratamento que dá a mim: bons ossos de
galinha, restos de carne, um canil com palha macia. Além disso, agrados,
mimos a toda hora, palmadas amigas, um nome.

--- Aceito! --- respondeu o lobo. --- Quem não deixará uma vida
miserável como esta por uma de regalos assim?

--- Em troca disso --- continuou o cão --- você guardará o terreiro, não
deixando entrar ladrões nem vagabundos. Agradará ao senhor e à sua
família, sacudindo a cauda e lambendo a mão de todos.

--- Fechado! --- resolveu o lobo e emparelhando-se com o cachorro partiu
a caminho da casa. Logo, porém, notou que o cachorro estava de coleira.

--- Que diabo é isso que você tem no pescoço?

--- É a coleira.

--- E para que serve?

--- Para me prenderem à corrente.

--- Então não é livre, não vai para onde quer, como eu?

--- Nem sempre. Passo às vezes vários dias preso, conforme a veneta do
meu senhor. Mas que tem isso, se a comida é boa e vem à hora certa?

O lobo entreparou, refletiu e disse:

--- Sabe do que mais? Até logo! Prefiro viver magro e faminto, porém
livre e dono do meu focinho, a viver gordo e liso como você, mas de
coleira ao pescoço. Fique-se lá com a sua gordura de escravo que eu me
contento com a minha magreza de lobo livre.

E afundou no mato.

--- Fez muito bem! --- berrou Emília. --- Isso de coleira, o diabo
queira\ldots{}

Narizinho bateu palmas.

--- E não é que ela fez um versinho, vovó? ``Isso de coleira, o diabo
queira\ldots{}'' Bonito, hein?\ldots{}

--- Bonito e certo --- continuou Emília. --- Eu sou como esse lobo.
Ninguém me segura. Ninguém me bota coleira. Ninguém me governa. Ninguém
me\ldots{}

--- Chega de ``mes'', Emília. Vovó está com cara de querer falar sobre a
liberdade.

--- Talvez não seja preciso, minha filha. Vocês sabem tão bem o que é
liberdade que nunca me lembro de falar disso.

--- Nada mais certo, vovó! --- gritou Pedrinho. --- Este seu sítio é o
suco da liberdade; e se eu fosse refazer a natureza, igualava o mundo a
isto aqui. Vida boa, vida certa, só no Picapau Amarelo.

--- Pois o segredo, meu filho, é um só: liberdade. Aqui não há coleiras.
A grande desgraça do mundo é a coleira. E como há coleiras espalhadas
pelo mundo!

%* ``O cão e o lobo'', de \textit{Fábulas} (1922).

\chapter{O galo que logrou a raposa}

Um velho galo matreiro, percebendo a aproximação da raposa,
empoleirou-se numa árvore. A raposa, desapontada, murmurou consigo:
``Deixe estar, seu malandro, que já te curo!\ldots{}''. E em voz alta:

--- Amigo, venho contar uma grande novidade: acabou-se a guerra entre os
animais. Lobo e cordeiro, gavião e pinto, onça e veado, raposa e
galinhas, todos os bichos andam agora aos beijos, como namorados. Desça
desse poleiro e venha receber o meu abraço de paz e amor.

--- Muito bem! --- exclamou o galo. --- Não imagina como tal notícia me
alegra! Que beleza vai ficar o mundo, limpo de guerras, crueldades e
traições! Vou já descer para abraçar a amiga raposa, mas\ldots{} como lá
vêm vindo três cachorros, acho bom esperá-los, para que também eles
tomem parte na confraternização.

Ao ouvir falar em cachorro, Dona Raposa não quis saber de histórias e
tratou de pôr-se ao fresco, dizendo:

--- Infelizmente, amigo Có-ri-có-có, tenho pressa e não posso esperar
pelos amigos cães. Fica para outra vez a festa, sim? Até logo.

E raspou-se.

Contra esperteza, esperteza e meia.

--- Pilhei a senhora num erro! --- gritou Narizinho. ---A senhora disse:
``Deixe estar que já te curo!''.

Começou com o ``você'' e acabou com o ``tu'', coisa que os gramáticos
não admitem. O ``te'' é do ``tu'', não é do ``você''\ldots{}

--- E como queria que eu dissesse, minha filha?

--- Para estar bem com a gramática, a senhora devia dizer: ``Deixa estar
que eu já te curo''.

--- Muito bem. Gramaticalmente é assim, mas na prática não é. Quando
falamos naturalmente, o que nos sai da boca é ora o ``você'', ora o
``tu'' --- e as frases ficam muito mais jeitosinhas quando há essa
combinação do ``você'' e do ``tu''. Não acha?

--- Acho, sim, vovó, e é como falo. Mas a gramática\ldots{}

--- A gramática, minha filha, é uma criada da língua, e não uma dona. O
dono da língua somos nós, o povo --- e a gramática o que tem a fazer é,
humildemente, ir registrando o nosso modo de falar. Quem manda é o uso
geral, e não a gramática. Se todos nós começarmos a usar o ``tu'' e o
``você'' misturados, a gramática só tem uma coisa a fazer\ldots{}

--- Eu sei o que é que ela tem a fazer, vovó! --- gritou Pedrinho. --- É
pôr o rabo entre as pernas e murchar as orelhas\ldots{}

Dona Benta aprovou.

%* ``O galo que logrou a raposa'', de \textit{Fábulas} (1922).

\chapter{Os animais e a peste}

Em certo ano terrível de peste entre os animais, o leão, mais
apreensivo, consultou um mono de barbas brancas.

--- Esta peste é um castigo do céu --- respondeu o mono ---, e o remédio
é aplacarmos a cólera divina sacrificando aos deuses um de nós.

--- Qual? --- perguntou o leão.

--- O mais carregado de crimes.

O leão fechou os olhos, concentrou-se e, depois de uma pausa, disse aos
súditos reunidos em redor:

--- Amigos! É fora de dúvida que quem deve se sacrificar sou eu. Cometi
grandes crimes, matei centenas de veados, devorei inúmeras ovelhas e até
vários pastores. Ofereço-me, pois, para o sacrifício necessário ao bem
comum.

A raposa adiantou-se e disse:

--- Acho conveniente ouvir a confissão das outras feras. Porque, para
mim, nada do que Vossa Majestade alegou constitui crime. Matar veados
--- desprezíveis criaturas; devorar ovelhas --- mesquinhos bichos de
nenhuma importância; trucidar pastores --- raça vil, merecedora de
extermínio! Nada disso é crime. São coisas até que muito honram o nosso
virtuosíssimo rei leão.

Grandes aplausos abafaram as últimas palavras da bajuladora --- e o leão
foi posto de lado como impróprio para o sacrifício.

Apresenta-se em seguida o tigre e repete-se a cena. Acusa-se ele de mil
crimes, mas a raposa prova que também o tigre era um anjo de inocência.

E o mesmo aconteceu com todas as outras feras.

Nisto chega a vez do burro. Adianta-se o pobre animal e diz:

--- A consciência só me acusa de haver comido uma folha de couve na
horta do senhor vigário.

Os animais entreolhavam-se. Era muito sério aquilo. A raposa toma a
palavra.

--- Eis, amigos, o grande criminoso! Tão horrível o que ele nos conta,
que é inútil prosseguirmos na investigação. A vítima a sacrificar-se aos
deuses não pode ser outra, porque não pode haver crime maior do que
furtar a sacratíssima couve do senhor vigário.

Toda a bicharia concordou e o triste burro foi unanimemente eleito para
o sacrifício.

Aos poderosos tudo se desculpa; aos miseráveis nada se perdoa.

--- Viva! Viva!\ldots{} Esta é a fábula do Burro Falante. --- E Pedrinho
recordou todos os incidentes daquele dia lá no País das Fábulas. ---
Esta história estava se desenvolvendo, e no instante em que as feras iam
matar o pobre burro, o Peninha derrubou do alto do morro uma enorme
pedra sobre as fuças do leão.

--- Salvamos o Conselheiro --- disse Emília ---, mas o fabulista pegou
um segundo burro para poder completar a fábula. Pobre segundo
burro!\ldots{} --- E Emília suspirou.

--- Esta fábula me parece muito boa, vovó --- opinou Narizinho.

--- E é, minha filha. Retrata as injustiças da justiça humana. A tal
justiça humana é implacável contra os fracos e pequeninos --- mas não é
capaz de pôr as mãos num grande, num poderoso.

--- Falta um Peninha que dê com pedras do tamanho do Corcovado no
focinho do Leão da injustiça\ldots{}

%* ``Os animais e a peste'', de \textit{Fábulas} (1922).

\chapter{A floresta}

Pois assim é --- continuou o Saci. --- A lei da floresta é a lei de quem
pode mais: ou por ter mais força, ou por ser mais ágil, ou por ser mais
astuto. A astúcia, principalmente, é uma grande coisa na floresta. Está
vendo ali aquele galhinho seco?

--- Sim. Um galhinho como outro qualquer --- respondeu o menino.

--- Pois está muito enganado --- replicou o Saci. --- Não é galho
nenhum, sim um bichinho que finge de galho seco para não ser atacado
pelos inimigos.

Pedrinho não quis acreditar, mas cutucando o galhinho viu que ele se
mexia. Ficou assombrado da esperteza.

--- Bem diz vovó que a mata é perigosa! Um que não sabe há de levar cada
logro aqui\ldots{}

--- E aquilo? --- perguntou o Saci apontando para uma folha. --- Que
parece a você que aquilo é?

Pedrinho olhou; viu bem que era uma folha de árvore; mas como já estava
ficando sabido nas traições da floresta, piscou para o Saci e disse:

--- Desta vez não caio na esparrela. Parece que é uma folha, mas com
certeza é outro bichinho que se disfarça em folha. --- E cutucou-a para
ver se mexia. A folha, porém, não se mexeu.

--- É folha mesmo, bobinho! --- disse o Saci dando uma risada. --- Inda
é muito cedo para você ``ler'' a mata. Isto é livro que só nós, que aqui
nascemos e vivemos toda vida, somos capazes de interpretar. Um menino da
cidade, como você, entende tanto da natureza como eu entendo de grego.

--- Realmente, Saci! Estou vendo que aqui na mata sou um perfeito
bobinho. Mas deixe estar que ainda ficarei tão sabido como você.

--- Sim, com o tempo e muita observação. Quem observa e estuda, acaba
sabendo. Aqui, porém, nós não precisamos estudar. Nascemos sabendo.
Temos o instinto de tudo. Qualquer desses bichinhos que você vê, mal sai
do casulo e já se mostra espertíssimo, não precisando dos conselhos dos
pais. Bem consideradas as coisas, Pedrinho, parece que não há animal
mais estúpido e lerdo para aprender do que o homem, não acha?

O orgulho do menino ofendeu-se com aquela observação. Um miserável saci
a fazer pouco-caso do rei dos animais! Era só o que faltava\ldots{}

--- O que você está dizendo --- replicou Pedrinho --- é tolice pura sem
mistura. O homem é o rei dos animais. Só o homem tem inteligência. Só
ele sabe construir casas de todo jeito, e máquinas, pontes, e
aeroplanos, e tudo quanto há. Ah, o homem! Você não sabe o que o homem
é, Saci! Era preciso que tivesse lido os livros que eu li em casa da
vovó\ldots{}

%* ``A floresta'', capítulo 10 de \textit{O} \textit{Saci} (1921).

\chapter{Discussão}

O saci deu uma gargalhada.

--- Que gabolice! --- exclamou. --- Casas? Qual é o bichinho que não
constrói sua casa na perfeição? Veja a das abelhas, ou a das formigas,
ou os casulos. Poderão existir habitações mais perfeitas? Todos aqui na
mata moram. Cada um inventa o seu jeito de morar. Todos moram. Todos,
portanto, têm suas casinhas, onde ficam muito mais bem abrigados do que
os homens lá nas casas deles. O caramujo, esse então até inventou o
sistema de carregar a casa às costas. É o mais esperto. Vai andando.
Assim que o perigo se aproxima, arreia a casa e mete-se dentro.

--- Casa, vá lá --- disse Pedrinho meio convencido. --- Mas aeroplano?
Que bichinho daqui seria capaz de construir aviões como nós homens os
construímos?

Outra risada do Saci.

--- Olhe, Pedrinho, você está-me saindo tão bobo que até me causa dó.
Aviões! Pois não vê que o avião é a mais atrasada máquina de voar que
existe? Aqui os bichinhos de asas estão de tal modo adiantados que
nenhum precisa de mostrengos como o tal avião. Todos possuem no corpo um
aparelho de voar aperfeiçoadíssimo. Não vê que voam, bobo? Outro dia
assisti a uma cena muito interessante. Eu estava perto duma lagoa cheia
de patos, quando um avião passou voando por cima das nossas cabeças. Os
patos entreolharam-se e riram-se. Você sabe, Pedrinho, que bicho
estúpido é o pato. Pois mesmo assim um deles disse com muita sabedoria:
``Parece incrível que os homens se gabem de ter inventado uma coisa que
nós já usamos há tantos milhares de anos\ldots{}''.

--- Sim --- continuou Pedrinho ---, mas nós sabemos ler e vocês não
sabem.

--- Ler! E para que serve ler? Se o homem é a mais boba de todas as
criaturas, de que adianta saber ler? Que é ler? Ler é um jeito de saber
o que os outros pensaram. Mas que adianta a um bobo saber o que outro
bobo pensou?

Era demais aquilo. Pedrinho encheu-se de cólera.

--- Não continue, Saci! Você está me ofendendo. O homem não é nada do
que você diz. O homem é a glória da natureza.

--- Glória da natureza! --- exclamou o capetinha com ironia. --- Ou está
repetindo como papagaio o que ouviu alguém falar ou então você não
raciocina. Inda ontem ouvi Dona Benta ler num jornal os horrores da
guerra na Europa. Basta que entre os homens haja isso que eles chamam
guerra para que sejam classificados como as criaturas mais estúpidas que
existem. Para que guerra?

--- E vocês aqui não usam guerras também? Não vivem a perseguir e comer
uns aos outros?

--- Sim; um comer o outro é a lei da vida. Cada criatura tem o direito
de viver e para isso está autorizada a matar e comer o mais fraco. Mas
vocês homens fazem guerra sem ser movidos pela fome. Matam o inimigo e
não o comem. Está errado. A lei da vida manda que só se mate para comer.
Matar por matar é crime. E só entre os homens existe isso de matar por
matar --- por esporte, por glória, como eles dizem. Qual, Pedrinho, não
se meta a defender o bicho homem que você se estrepa. E trate de fazer
como Peter Pan, que embirrou de não crescer para ficar sempre menino,
porque não há nada mais sem graça do que gente grande. Se todos os
meninos do mundo fizessem greve, como Peter Pan, e nenhum crescesse, a
humanidade endireitaria. A vida lá entre os homens só vale enquanto
vocês se conservam meninos. Depois que crescem, os homens viram uma
calamidade, não acha? Só os homens grandes fazem guerra. Basta isso. Os
meninos apenas brincam de guerra.

Pedrinho nada respondeu. Estava um tanto abalado pelas estranhas ideias
do Saci. Quando voltasse para casa iria consultar Dona Benta para saber
se era assim mesmo ou não.

%* ``Discussão'', capítulo 11 de \textit{O} \textit{Saci} (1921).

\chapter{A cartinha do Polegar}

O sítio de Dona Benta foi se tornando famoso tanto no mundo de verdade
como no chamado Mundo de Mentira. O Mundo de Mentira, ou Mundo da
Fábula, é como a gente grande costuma chamar a terra e as coisas do País
das Maravilhas, lá onde moram os anões e os gigantes, as fadas e os
sacis, os piratas como o Capitão Gancho e os anjinhos como Flor das
Alturas. Mas o Mundo da Fábula não é realmente nenhum mundo de mentira,
pois o que existe na imaginação de milhões e milhões de crianças é tão
real como as páginas deste livro. O que se dá é que as crianças logo que
se transformam em gente grande fingem não mais acreditar no que
acreditavam.

--- Só acredito no que vejo com meus olhos, cheiro com o meu nariz, pego
com minhas mãos ou provo com a ponta da minha língua, dizem os adultos
--- mas não é verdade. Eles acreditam em mil coisas que seus olhos não
veem, nem o nariz cheira, nem os ouvidos ouvem, nem as mãos pegam.

--- Deus, por exemplo --- disse Narizinho. --- Todos creem em Deus e
ninguém anda a pegá-lo, cheirá-lo, apalpá-lo.

--- Exatamente. E ainda acreditam na Justiça, na Civilização, na Bondade
--- em mil coisas invisíveis, incheiráveis, impegáveis, sem som e sem
gosto. De modo que se as coisas do Mundo da Fábula não existem, então
também não existem nem Deus, nem a Justiça, nem a Bondade, nem a
Civilização --- nem todas as coisas abstratas.

--- Eu sei o que quer dizer ``abstrato'' --- disse Emília. --- É tudo
quanto a gente não vê, nem cheira, nem ouve, nem prova, nem pega --- mas
sente que há.

--- Muito bem. Logo, o Mundo da Fábula existe, com todos os seus
maravilhosos personagens.

--- E tanto existe --- declarou Dona Benta --- que tenho aqui uma carta
muito interessante, recebida hoje.

--- É de mamãe, já sei! --- murmurou Pedrinho, aborrecido, com medo que
fosse carta de Dona Antonica chamando-o para a cidade.

--- Errou, meu filho. A cartinha que recebi é do Pequeno Polegar\ldots{}

Ao ouvir tal notícia, a criançada pulou de contente. Os olhos de
Narizinho molharam-se de ternura. O Pequeno Polegar era, de fato, a
maior das galantezas.

--- Mostre, mostre a cartinha dele, vovó!

Dona Benta pôs os óculos e tirou da bolsa uma coisinha dobrada,
pequeniníssima --- uma pétala de rosa!

--- É o papelzinho em que ele escreve --- e escreve sem tinta, com a
ponta de um espinho. Só poderei ler o que está aqui se Pedrinho me
trouxer a lente do binóculo.

Pedrinho coçou a cabeça. Onde andaria a lente do binóculo desmanchado?

--- A lente sumiu, vovó --- disse ele --- mas há os célebres olhos da
Emília, mais penetrantes que todas as lentes do mundo.

Até uma pulga no pelo do dragão de S. Jorge, lá na lua, ela já
``detectou''.

--- Ótimo! Nesse caso, venha a Emília ler a cartinha do nosso amigo.

Muito orgulhosa do seu papel, Emília aproximou-se rebolando. Tomou a
pétala dobrada, cheirou-a: ``Ah! É rosa Bela Helena!''. Abriu-a e leu
com a maior facilidade:

``Prezadíssima Senhora Dona Benta Encerrabodes de Oliveira:

Saudações. Tem esta por fim comunicar a V. Ex.ª que nós, os habitantes
do Mundo da Fábula, não aguentamos mais as saudades do Sítio do Picapau
Amarelo, e estamos dispostos a mudar-nos para aí definitivamente. O
resto do mundo anda uma coisa das mais sem graça. Aí é que é o bom.

Em vista disso, mudar-nos-emos todos para sua casa --- se a senhora der
licença, está claro\ldots{}''

O assanhamento da criançada subiu a 100 graus, que é o ponto de fervura
da água. Ficaram todos borbulhantes de alegria.

Pedrinho disparou a fazer projetos de brincadeiras com Aladino e o
Príncipe Codadade. Narizinho queria conversas de não acabar mais com
Branca de Neve e a menina da Capinha Vermelha. Até o Visconde lambeu os
beiços, ansioso por uma discussão científica com Mr. de La Fontaine, o
famoso fabulista encontrado na viagem feita ao ``País da
Fábula''.\footnote{\textit{Reinações de Narizinho}.}

--- Que suco vai ser vovó! Todos aqui, imagine! Será que também vem D.
Quixote?

--- Eu o que quero é lidar com os anões de Branca de Neve! O Dunga, o
Zangado\ldots{} Ah, gostosura!

Mas Dona Benta estava incerta. A população do Mundo da Fábula era
grande; como acomodá-la toda ali num sítio que não tinha mais de cem
alqueires de terra?

--- Aumenta-se o sítio, vovó --- propôs Pedrinho. --- A senhora compra
as fazendas dos vizinhos. Para que serve dinheiro? Depois que saiu o
petróleo, a senhora ficou empanturrada de dinheiro a ponto de enjoar e
nem permitir que se fale em dinheiro nesta casa. Aumenta-se o sítio. Tão
fácil\ldots{}

Dona Benta refletiu ainda uns instantes; depois concordou.

--- É o jeito. Podemos comprar a Fazenda do Taquaral e mais a do Cupim
Redondo. As duas juntas devem perfazer aí uns mil e duzentos alqueires
de terra. Ora, em mil e duzentos alqueires de terra eu imagino que cabem
todos os personagens do Mundo da Fábula.

--- E se não couberem, a senhora vai comprando mais fazendas --- isso
não oferece dificuldade. E podemos fazer uma coisa, vovó: uma cerca de
arame que separe o sítio velho das Terras Novas. Ficamos nós aqui e eles
nas Terras Novas. Que tal?

A lembrança de Pedrinho foi aprovada.

--- Sim, boa ideia. Fazemos uma cerca de arame com porteira --- e
porteira de cadeado. Confio a chave ao Visconde. Só abriremos a porteira
quando nos convier. Se não, invadem-nos isto aqui e\ldots{}

Narizinho ficou cismarenta, a imaginar a maravilhosa vida que iriam ter
com uma vizinhança daquelas.

--- Tia Nastácia é que vai ficar tonta --- lembrou Emília. --- Juro que
muitos deles hão de pular a cerca por causa dos bolinhos.

--- O remédio é fácil. Pomos Quindim de guarda, dia e noite, passeando
de cá para lá ao comprimento da cerca. Como eles nunca viram
rinoceronte, hão de respeitá-lo --- hão de ter medo daquele chifre
único.

Essa conversa ocorreu à noite, depois do chá --- e nesse dia só foram
para a cama às 11 horas, tamanha foi a discussão travada sobre o que
fazer. Ao se recolherem, até a Emília e o Visconde beijaram a mão de
Dona Benta, dizendo com a maior naturalidade: ``Sua bênção, vovó''.

%* ``A cartinha do Polegar'', capítulo 1 de \textit{O Picapau Amarelo} (1939).

\chapter{A resposta de Dona Benta}

No dia seguinte Dona Benta respondeu à carta do Pequeno Polegar. Que
viessem todos. Ela iria comprar mais terrenos vizinhos, de modo que o
Mundo da Fábula inteiro coubesse lá.

Mas na hora de subscritar o envelope a boa velhinha atrapalhou-se. Viu
que Polegar havia esquecido de mandar o endereço.

--- Que cabecinha de vento! Escreve-me uma carta e não indica para onde
devo remeter a resposta.

E cheia de indecisão estava a pensar naquilo, quando Narizinho se
aproximou com um ``Que é, vovó?''. Ao saber da falta de endereço, riu-se
regaladamente.

--- Ora, dá-se! Parece incrível que a senhora se aperte por tão pouca
coisa. É só chamar a Emília\ldots{} Emília! \textsc{emília}!\ldots{}

Emília surgiu rebolando.

--- Venha resolver um caso que está atrapalhando vovó.

Polegar escreveu, mas esqueceu de botar o endereço. Vovó não sabe para
onde mandar a resposta.

Emília deu uma risada gostosa.

--- Ah, meu Deus! Que bicho bobo é gente grande!\ldots{}

Morrem de lidar com as maravilhas e não aprendem nada --- não aprendem
essa coisa tão simples que é o ``faz de conta''. Me dá aqui a carta.

Dona Benta, de boca entreaberta e olhar admirado, foi-lhe entregando a
cartinha de resposta. Emília agarrou-a e leu-a\ldots{}

--- Isso é falta de delicadeza, Emília, ler carta dos outros ---
observou Narizinho.

--- É falta de delicadeza quando a carta vai pelo correio.

--- No sistema do ``faz de conta'' não é, porque faz de conta que não
li. Li para ver se ``vovó'' não nos traiu\ldots{}

--- Emília, Emília! --- gritou Narizinho com severidade. --- Como se
atreve a fazer semelhante juízo de vovó?!

--- Minha cara --- respondeu Emília com o maior desplante --- eu já
virei uma Floriana Peixota: confio desconfiando\ldots{}

--- Já se viu que diabinha? --- murmurou Dona Benta filosoficamente.

Emília chegou à janela e gritou:

--- Ventos e brisas daquém e dalém

Passarinhos e borboletas

Esta resposta ao Polegar levade,

Depressa, depressa, se não\ldots{}

E lançou a cartinha ao vento.

--- Se não o quê, Emília? --- perguntou Narizinho.

--- Se não, nada. O se não é só para meter medo.

Tia Nastácia vinha entrando da cozinha para ver o que Dona Benta queria
no almoço. Ao saber da cartinha do Polegar e da licença para que viessem
morar ali, exclamou, erguendo as mãos para o céu:

--- Nossa Senhora! Isto vai virar ``hospiço''. Sinhá não se lembra
daquela vez que eles entupiram a casa de reizinhos e príncipes e
princesas? Nossa Senhora, onde iremos parar?

--- Fazemos uma cerca, Nastácia. Eles ficam morando para lá da cerca. Só
virão aqui quando quisermos.

--- Cercas, Sinhá? Pois se até muro com caco de vidro em cima gente
pula, quanto mais eles, que são uns demoninhos\ldots{}

--- Quindim fica de guarda à cerca, passeando de cá para lá --- e na
porteira eu boto um cadeado. O Visconde tomará conta da chave.

A negra deu uma grande risada.

--- Ché, Sinhá! Tudo é muito bonito e fácil no ``papé''. Mas eu quero
ver! O Visconde chaveiro, ha, ha, ha!

Dona Benta, que estava com preguiça de discutir, mudou de assunto.

--- Olhe, para o almoço faça um frango assado com farofa --- e uns
pastéis de palmito.

--- O palmito acabou, Sinhá. ``Seu'' Pedrinho gastou ontem o último para
fazer uma tal de bica d'água.

--- Mande buscar meia dúzia no Elias Turco.

--- O palmito do Elias é falsificado, Sinhá. Só casca.

--- Então, então\ldots{}

%* ``A resposta de Dona Benta'', capítulo 2 de \textit{O Picapau Amarelo} (1939).

\chapter{O plano da Emília}

Dona Benta mandou chamar os donos das fazendas vizinhas para propor-lhes
a compra das propriedades. Nenhum quis vender. Eram fazendas que não
valiam nada, mas como Dona Benta tinha fama de muito dinheiro, todos
trataram de aproveitar-se. Dona Benta, porém, não era das que ``vão na
onda''. Não aumentou a oferta. Depois que os homens se retiraram, chamou
os meninos.

--- Esses homens querem aproveitar-se da situação; fingem não querer
vender as fazendas --- tudo para me explorar. Que acha que devo fazer,
Pedrinho?

Pedrinho, danado com os fazendeiros, imaginou logo uma solução violenta:
comprar as fazendas à força. Agarrá-los e obrigá-los a assinar as
escrituras, com um revólver encostado na nuca de cada um.

--- Absurdo, meu filho. Os processos violentos nunca dão bons
resultados. Temos que estudar um meio jeitoso.

--- Isto de jeito é cá comigo --- gritou a Marquesa de Rabicó.

--- Entregue-me o caso que num instante estará resolvido.

--- Pois vamos ver --- disse Dona Benta. --- Fale com os homens e dê lá
os jeitinhos necessários. Não ofereço mais de 400 mil cruzeiros pelas
duas fazendas --- e é muito. Não valem nem 200.

Emília chamou de parte o Visconde para a discussão do assunto. Em
seguida foram os dois ao pasto, montaram no burro falante e saíram em
procura dos fazendeiros.

Encontraram-nos na venda do Elias, bebendo cerveja entre grandes
risadas. --- ``Desta vez a velha nos paga'' --- diziam eles.

``Havemos de lhe arrancar couro e cabelo.''

Emília e o Visconde entraram, sentaram-se atrás deles numa mesinha dos
fundos, e pediram meia garrafa de cerveja e duas cocadas queimadas. E
puseram-se a conversar com ares misteriosos. Aquilo imediatamente
intrigou os fazendeiros.

--- Pois é --- dizia a Marquesa para o Visconde, muito baixinho mas de
modo que os homens ouvissem; --- a bicharia já está embarcada: duzentos,
cem machos e cem fêmeas --- e rinocerontes dos mais ferozes, caçados de
fresco no Uganda, lá no sul da África\ldots{}

--- Mas então é verdade? --- perguntou o Visconde com o maior
fingimento. --- Pensei que fosse brincadeira\ldots{}

--- Brincadeira, nada! Dona Benta não brinca. Vai fazer aqui a maior
criação de feras do mundo. Chegam agora esses 200 rinocerontes
ferocíssimos. Depois vêm os leões que estão sendo caçados --- 300 leões!
E mais 150 tigres-de-bengala, daqueles que só se alimentam de gente. E
há as panteras negras --- 100. Isso sem falar nos ursos brancos do Pólo,
nem nos lobos da Rússia, nem naquelas cobras da Índia que têm capelo
venenosíssimas.

Os fazendeiros, de boca aberta, entreolhavam-se assustados.

Emília continuou:

--- A criação de feras de Dona Benta será a maior do mundo, mas os
vizinhos vão sofrer com isso. Toda gente sabe que os animais caseiros,
burros, bois, cavalos etc., têm um verdadeiro horror pelas grandes
feras. Adivinham-nas de longe pelo cheiro e morrem de medo --- fogem com
quantas pernas têm. Estou com dó dessas fazendas. Vão ficar a pé, sem um
cavalo, um burro, um boi. O vento leva para lá o cheiro dos leões e dos
tigres e a animalada mansa dispara\ldots{}

--- Mas esses vizinhos podem processar Dona Benta --- lembrou o
Visconde.

--- Ela já mandou estudar esse ponto por um bom advogado do Rio. As leis
não tratam do assunto. Se eles processarem Dona Benta, perdem as
demandas e ainda têm de pagar às custas. Não tem lei nenhuma que proíba
cheiros em fazendas.

A conversa prosseguiu nesse tom cochichado até o fim da cervejinha.
Depois pagaram a conta e saíram, muito lampeiros.

Os dois homens ficaram com caras de asno, a olhar um para o outro.

--- E esta, compadre! Se o raio da velha vai mesmo fazer isso, nossas
fazendas, que já pouco valem, ficarão valendo ainda menos. Aquilo que o
pelotinho de gente disse é certo. Os animais têm um medo horrível às
onças e outras feras. Assim que farejam alguma por perto, fogem loucos
de terror. Isso é coisa das mais sabidas.

--- Pensando bem --- disse o segundo --- o verdadeiro é aceitarmos a
proposta da velha. Quatrocentos mil cruzeiros pelas duas fazendas é até
muito dinheiro --- porque não valem 200 --- nem 180. O café está de
rastos --- porco não dá nada --- algodão, o curuquerê come\ldots{}
Quanto mais eu trabalho em minha fazenda, mais endividado fico. O melhor
é aceitarmos a proposta da velha.

Nesse mesmo dia voltaram os dois ao Picapau Amarelo.

O preço que a senhora nos oferece --- disse um deles --- é fraquinho,
mas nós cansados já desta vida de fazenda, resolvemos ceder. Pode marcar
o dia para as escrituras.

--- Amanhã --- foi a resposta de Dona Benta, muito admirada da
reviravolta operada neles. Que seria que os fez mudar assim tão
repentinamente? Dona Benta interpelou-os.

Os fazendeiros coçaram a cabeça.

--- Coisas da vida, minha senhora. Fomos à venda do Turco beber uma
cervejinha e pensamos melhor sobre o caso. Estamos velhos e cansados, é
isso.

Nesse instante Dona Benta ouviu o ringido da porteira.

Olhou. Vinham chegando a Emília e o Visconde, montados no Burro Falante.
--- ``Hum! Estou compreendendo!'' --- murmurou ela consigo. ``Aqui anda
o dedinho da Emília.''

Depois que os fazendeiros se retiraram, Dona Benta a chamou.

--- Que foi que você fez, diabinha, para mudar desse modo a opinião dos
dois homens?

--- Nada, Dona Benta. Apenas comemos uns doces na bodega do Elias e
tomamos uma cervejinha. Por sinal que estou tonta, tonta\ldots{}

E estava mesmo. Tão tontinhos, ela e o Visconde, que caíram na rede e
ferraram no sono.

Dona Benta ficou a cismar: --- ``Que será que Emília botou na cabeça
deles?'' Mas por mais que cismasse, nada adivinhou.

%* ``O plano da Emília'', capítulo 3 de \textit{O Picapau Amarelo} (1939).

\chapter{O Visconde e a Quimera}

Enquanto no castelo de Branca de Neve os meninos se extasiavam diante
dos maravilhosos diamantes extraídos do seio da terra pelos anões, o
pobre Visconde conversava sem medo nenhum com o monstro. O ``sabinho''
nunca teve medo das feras --- só tremia diante de vacas e galinhas. Quem
tem alma de sabugo é assim.

Quando os meninos fugiram, ele sentou-se, a segurar o pé destroncado, e
só então viu diante de si o estranho monstro de três cabeças. Sua
curiosidade de sábio espicaçou-o. De que ``mitologia'' era aquele
monstro? Há muitas mitologias, isto é, coleção de fábulas --- uma para
cada civilização. Há a mitologia grega, a mais rica de todas; há a
mitologia da Índia; há a mitologia dos povos nórdicos; há até a
mitologia do Brasil, na qual vemos o Saci, o Caipora, a Mula-sem-cabeça,
a Iara. Mas aquele monstro? Em qual dessas mitologias figurava? ---
resolveu perguntar.

--- Perdoe a minha indiscrição, senhor monstro, mas eu muito desejava
saber a que mitologia o senhor pertence. Poderá tirar-me da dúvida?

O monstro parecia um poço de estupidez. Não entendia coisa nenhuma e
muito menos o que quisesse dizer ``mitologia''.

Olhou para o Visconde com os seis olhos ao mesmo tempo, com ar de
galinha que olha para a gente.

--- Sim --- continuou o Visconde. --- Desejo saber se o senhor é grego,
hindu ou nórdico.

O monstro continuava galinha.

--- Onde nasceu? Na Grécia?

Os seis olhos do monstro brilharam. Havia afinal compreendido qualquer
coisa. E uma voz rouquenta saiu de sua cabeça de cabra.

--- Sou da Lícia.

O ``sabinho'' franziu a testa. ``Lícia?'' Deu busca à memória; vagamente
recordou-se dum reino da Lícia que existiu antigamente na Ásia Menor.

--- Hum --- exclamou. --- Sei, sei, a Lícia\ldots{} E seu nome como é,
senhor monstro?

--- Quimera.

Os olhinhos do Visconde cintilaram.

--- Ora viva! Lembro-me perfeitamente. A Quimera, sim, o monstro que o
herói Belerofonte venceu em combate. Mas pelo que sei esse monstro
vomitava fogo pela boca das três cabeças. Nós também temos por aqui
qualquer coisa desse gênero --- a Mula-sem-cabeça que vomita fogo pelas
ventas. Muito curioso, não? Sem cabeça e com ventas! Que maravilha é
esse mundo das maravilhas! Mas, diga-me Senhora Quimera, ainda sai fogo
das suas três goelas ou não?

O monstro, como resposta, espremeu-se, e das três bocarras saiu uma
fumacinha à toa.

O Visconde refletiu consigo que estava diante dum monstro muito velho,
de milhares de anos e já extinto --- como os vulcões que apenas fumegam.
Examinando-o melhor, confirmou-se nessa ideia. O bicho apresentava todos
os sinais duma tremenda velhice: pelo escasso e branco, rugas, olhos
lacrimosos e tremores nas pernas. Parecia o papagaio caduco do tio
Barnabé, que tinha cem anos e só dez penas no corpo enrugado. Sim, ele
estava diante da terrível Quimera que fora o pavor da antiguidade ---
mas já inofensiva, sem dentes, sem fogo, sem pelos --- caduca. E o
Visconde sentiu um grande dó daquela decadência. Coitada!

Quando lhe pediu fogo, ela, com o maior esforço, só pôde dar
fumacinhas\ldots{}

--- É curioso esse fenômeno de sair fumaça das suas entranhas --- disse
ele. --- Parece contrário a todas as leis da fisiologia.

--- Que é fisiologia? --- perguntou a Quimera. --- A rainha deste reino?

O Visconde riu-se com superioridade de sábio.

--- Fisiologia é a ciência que estuda o funcionamento do corpo dos
animais.

--- Mas eu não sou animal --- disse a Quimera --- apesar de minha
aparência de leão, cabra e serpente.

--- Que é então?

--- Sou uma fábula grega, como você me parece uma fábula moderna.

O Visconde ficou admiradíssimo da resposta. A Quimera não estava tão
caduca como ele pensou. Raciocinava e muito bem. O interesse dela,
entretanto, resumia-se em saber quem mandava naquele reino.

--- Quem é o rei ou rainha daqui? --- perguntou.

--- Não há disso por cá. Somos uma democracia. Há Dona Benta, que é a
Tesoureira, ou a Dona. Há dois príncipes herdeiros: Narizinho e
Pedrinho. Há a Lambeta-Mor, que é uma tal Marquesa de Rabicó. Há o
Ministro da Defesa Nacional, que é o Marechal Quindim. Há a Provedora
Mor das Comidas, que é tia Nastácia. Há o Sábio dos Sábios, que é o
ilustríssimo Senhor Visconde de Sabugosa\ldots{}

Pobre Visconde! A dor do pé destroncado o ia levando ao delírio dos
febrentos. Falava de modo que a Quimera nada podia entender.

--- E que está fazendo aqui? --- perguntou esta.

--- Estou pagando os meus pecados, senhora. Fui vítima dum litígio
arbóreo e joanesco --- o choque do Polegar com os joões.

--- Litígio? --- repetiu a Quimera. --- Que quer dizer litígio?

--- Um conflito de direito --- o choque de dois direitos, um
direito-torto e um direito-direito. Polegar julgou-se com o direito
torto de ocupar a casa dos dois joões; os dois joões estavam no
direito-direito de resistir. Começou a luta --- bicadas de lá, botadas
de cá. Vai então o príncipe herdeiro e me manda subir à árvore como o
anjo da paz. Quem se mete entre dois litigantes acaba apanhando. Foi o
que me sucedeu. Apanhei botada pelas ventas. Perdi o prumo. A força da
gravidade atraiu-me para o centro da terra, isto é, fez-me cair. O
tornozelo esquerdo não aguentou o choque --- saiu do lugar. Apareceu
então uma célebre peste chamada Dor --- e a Dor está doendo. É
isso\ldots{}

A Quimera estava com as três bocas abertas, sem entender coisa nenhuma.
O Visconde resumiu o caso numa sentença: ``Foi o Polegar que me derrubou
lá de cima do pau''.

--- E quem é esse Polegar?

--- Um garoto que vem da Idade Média e anda nos livros de Andersen,
Perrault e Grimm. Dona Benta caiu na asneira de mudar para cá o tal
Mundo da Fábula --- e a primeira consequência foi esta: o meu pé
destroncado.

--- Por que não se recolhe à sua casa?

--- De que modo? Como poderei andar, com este pé assim?

Narizinho, Pedrinho e Emília estão longe, escondidos nalgum daqueles
castelos. Quem me há de levar ao sítio?

--- Eu. Eu o levo --- e com o maior prazer.

Apesar da dor que sentia, o Visconde riu-se à ideia de aparecer no
terreiro de Dona Benta montado na Quimera! O espanto de tia
Nastácia!\ldots{}

--- Pois aceito --- disse ele --- e semelhante ato de caridade não
ficará sem recompensa.

Nesse momento soou na galharia do Cedro Grande um barulho. Oito olhos
ergueram-se para lá --- os seis da Quimera e os dois do Visconde. O
casal de João-de-barro havia voltado e reiniciado o ataque ao intruso
que lhes invadira a casinha.

Polegar, sem as botas, não tinha com que defender-se. Foi obrigado a
sair do ninho. Assim que o pilharam fora, as duas enfurecidas aves
deram-lhe tal surra de bicanca, que ele fez como o Visconde: perdeu o
equilíbrio e caiu --- plaf! no chão. E quebrou a perna --- ai, ai, ai!

Gemia, gemia\ldots{}

--- Ai, ai, ai Que vai ser de mim agora, sem botas e de perna quebrada?

O Visconde procurou acalmá-lo.

--- Tudo se há de arranjar, amigo. Aqui a Senhora Quimera nos vai levar
ao sítio de Dona Benta.

Só então Polegar deu com o monstro de três cabeças. Que susto! Seu
coraçãozinho pulou. O sangue fugiu-lhe das faces.

--- Não tenha medo --- disse o Visconde. --- A madama aqui é velha,
mansíssima, e de tão boa paz como o Quindim. Vai levar-nos montados em
seu lombo.

Polegar foi sossegando.

Há coisas fáceis de dizer e bem duras de fazer. Custa muito aos dois
estropiados colocarem-se sobre o lombo da Quimera --- mas a Necessidade
sabe operar prodígios. Montaram, afinal, e lá foram. Chegados à cerca, o
monstro parou.

--- Como atravessar estes malditos arames espinhentos?

--- Pela porteira --- respondeu o Visconde. --- Tenho a chave aqui no
bolso. Sou o chaveiro.

Outra dificuldade. A Quimera não sabia lidar com chaves, de modo que o
Visconde gemendo, gemendo, teve de apear para abrir.

O monstro grego ficou assombrado de uma chavinha tão pequena abrir uma
porteira tão grande. Evidentemente tratava se de um talismã
encantado\ldots{}

Quando aquela esquisitíssima trempe surgiu no terreiro, tia Nastácia
vinha entrando na varanda com duas moringas d'água.

``Credo'' --- urrou a pobre negra largando tudo no chão --- e caiu
desmaiada.

%* ``O Visconde e a Quimera'', capítulo 7 de \textit{O Picapau Amarelo} (1939).

\chapter{A Fênix}

Do rouxinol a conversa passou para outras aves e por fim recaiu sobre a
célebre fênix.

--- Oh, a fênix! --- exclamou Hércules.

--- Já ouvi falar. Dizem que vive séculos.

Tem o tamanho da águia e na cabeça um topete dum vermelho vivíssimo. As
penas do corpo, também vermelhas, com exceção das do pescoço que são
douradas.

--- E as da cauda?

--- Essas são brancas, entremeadas de algumas cor de sangue.

--- Que linda deve ser! --- exclamou Pedrinho.

Já era noite quase fechada. Hércules ajeitou-se por ali mesmo para
dormir, e os pica-pauzinhos procuraram o abrigo duma gruta de pedra.
Meioameio deitou-se na entrada da gruta. Era ele o guarda-noturno dos
seus amigos do século 20.

Os sonhos daquela noite foram sonhos ``ornitológicos'', como disse no dia
seguinte o Visconde, e foi explicando: ``Ornitologia é a ciência que
estuda as aves. Logo, quem sonha com passarinho tem um sonho
ornitológico\ldots{}''.

Ao retornarem à viagem para os montes do Erimanto, a conversa voltou ao
mesmo assunto da noite anterior: aves.

--- Conte mais alguma coisa da fênix, Lelé! --- pediu Emília --- e o
herói contou.

--- O que me disseram foi o que narrei ontem e mais isto: a fênix tem
olhos brilhantes como estrelas\ldots{}

--- Que lindo!\ldots{}

--- E quando sente que a hora da morte está chegando, começa a juntar no
mato ramos de plantas cheirosas, resinas e gravetos; e com aquilo tudo
faz uma espécie de ninho dentro do qual se acomoda. Isso antes do carro
de Apolo aparecer no horizonte. Quando aparece e seus raios começam a
esquentar, aquele ninho resinoso pega fogo e vira uma grande fogueira na
qual a fênix é completamente consumida, só ficando um montinho de
cinzas. E aí então é que acontece o prodígio: no meio daquela cinza
aparece um ovo, do qual logo sai uma nova fênix.

Essa fênix junta toda aquela cinza e vai depositá-la no altar do Sol, na
cidade de Heliópolis.

--- Que lindo! --- exclamou Emília. A fênix renasce de suas próprias
cinzas! --- E não há nenhuma fênix aqui por esta Grécia, Lelé?

--- Às vezes aparece alguma, vinda de outras terras. Mas não é ave
grega.

Minutos depois dessa conversa Emília gritou: ``Alto!\ldots{}'' e todos
pararam. Ela trepou ao ombro de Meioameio e ali de pé, com a mão em
viseira, pôs-se a sondar a distância. E ia falando:

--- Estou vendo muito longe uma ave a amontoar um ninhofogueira\ldots{}
Belíssima, sim\ldots{} Toda cor de pitanga, com topete muito vivo e rabo
branco\ldots{}

--- Será uma fênix? --- exclamou Pedrinho, já assanhado --- e Emília
continuou:

--- Não sei, mas está fazendo direitinho como Lelé disse. Traz para o
ninho-fogueira plantas odoríferas\ldots{}

O Visconde suspirou. Estava achando aquilo um pouco demais. Que daquela
distância Emília visse a ave trazer plantas para o ninho, ainda vá lá.
Mas declarar que as plantas eram odoríferas? Seria possível que além dos
olhinhos de telescópio ela possuísse teleolfato?

--- Está pronto o ninho-fogueira! --- continuou Emília. ---

Agora a ave ajeitou-se no meio daqueles ``combustíveis'' e está rezando
de mãos postas, à espera de que um raio de sol venha incendiá-la\ldots{}

Embora Hércules acreditasse cegamente no que a ex-boneca dizia, também
começou a achar aquilo ``demais'' --- e deu ordem a Meioameio para
correr até lá e ver se era assim mesmo.

O centaurinho partiu no galope, com o Visconde no lombo, porque os
verdadeiros sábios nunca perdem ensejo de verificar o que podem. E
enquanto Meioameio galopava na direção da fênix, Emília continuava a ver
``coisas'', mas já preparando uma escapatória.

--- Uma vez no Deserto do Saara disse a marotinha --- eu vi uma coisa
linda: um chafariz lá muito longe. Não podia haver encontro mais lindo
no Saara do que o de um chafariz, para gente que estava morrendo de
sede, como nós\ldots{}

Pedrinho pensou em desmascarar a ex-boneca, dizendo que tudo aquilo era
invenção. Emília jamais havia estado em Saara nenhum; mas de dó dela
limitou-se a dizer:

--- Esse chafariz devia ser uma das chamadas ``miragens'' tão frequentes
nos desertos. Os viajantes sedentos veem oásis e coisas onde não há
oásis nem coisa nenhuma.

Hércules ficou na mesma, porque na terra grega não havia desertos, nem
oásis, nem miragens. Emília continuou.

--- E bem pode ser que aquela fênix seja uma miragem\ldots{} Não! Não
é!\ldots{} Esperem, esperem um pouco\ldots{} Está mas é pegando fogo!
Pronto! O ninho-fogueira pegou fogo!\ldots{} A fênix está se consumindo
nas chamas\ldots{}

O centaurinho acabava de chegar ao ponto indicado e por mais que olhasse
não percebeu fênix nenhuma. O Visconde sorriu consigo, murmurando:
``Aquela Emília\ldots{}''. E como nada achassem, voltaram.

--- Não encontramos ave nenhuma --- disse Meioameio ao chegar. --- Eu e
o Visconde demos uma volta por lá e nem sinal.

Hércules, já meio desconfiado, olhou para Emília, a qual botou as mãos
na cintura e deu uma gargalhada gostosa.

--- Nunca vi dois sarambés maiores! Quando chegaram lá, a fênix já havia
sido devorada pelo fogo. Em vez de procurarem uma ``ave'', deviam ter
procurado uma ``cinzinha'', mas aposto que nem pensaram nisso.

Meioameio olhou muito desapontado para o Visconde.

Realmente, eles não tinham tido a ideia de procurar cinzinha
nenhuma\ldots{}

--- Pois, meus grandes bobos, o que se deu foi isto: enquanto vocês
galopavam para lá, a fênix desapareceu consumida pelas chamas e ficou
reduzida a um punhadinho de cinzas.

Querendo tirar a prova daquilo, Hércules deu ordem a Meioameio para
voltar e verificar a existência da cinzinha. Meioameio partiu, e
enquanto galopava para lá Emília continuou a ver.

--- Que beleza! --- exclamou fazendo cara de admiração. --- Estou vendo
a maravilha das maravilhas\ldots{} A cinza está se juntando\ldots{} está
tomando forma\ldots{} É a fênix que renasce de suas próprias cinzas.
Pronto! Está formadinha\ldots{} Agora começou a experimentar as asas.
Vai voar\ldots{} Voou!\ldots{}

Hércules estava de boca aberta. Que maravilha, aquela criaturinha!
Enquanto isso Meioameio e o Visconde chegaram novamente ao ponto
indicado e puseram-se a procurar cinzinhas.

Nem sombra! Não havia nem cheiro de cinza --- e voltaram desapontados.

--- Nada encontramos, Hércules --- disse Meioameio ao chegar; e o
Visconde confirmou: --- Não há lá nem sequer sombra de nenhuma cinzinha.

Emília deu nova gargalhada.

--- Os bobos!\ldots{} Como poderiam ter encontrado cinza, se quando
vocês estavam no meio do caminho a fênix renasceu e lá se foi pelos
ares? Queriam que ela ficasse parada, à espera dos dois sarambés?

Desse modo Emília embaçou a todos com a sua prodigiosa esperteza e até
Pedrinho ficou na dúvida. ``Quem sabe se é mesmo verdade tudo quanto ela
disse?'' Apenas um não duvidou da Emília: Hércules. Não duvidou naquele
momento nem nunca.

Ficara tão escravo daquela criaturinha, que era Emília dizer, era ele
jurar em cima, como se ela fosse o próprio escudo da deusa Palas.

O incidente foi o assunto da conversa entre Pedrinho e Hércules, num
momento em que os dois se afastaram do resto do bando.

--- Emília faz coisas que atrapalham a gente --- disse Pedrinho. ---
Aquela história da pulga que ela viu nas escamas do dragão de S. Jorge
parece caçoada pura --- mas quem sabe? Tudo é possível neste mundo. Esse
caso da fênix, hoje. Ela veria mesmo a fênix incendiar-se e renascer das
cinzas ou estava nos enganando? Impossível saber.

Hércules, porém, já não tinha a menor dúvida.

--- Na minha opinião, viu. Ela contou tudo tão certinho\ldots{}

--- Ah, Hércules, você não conhece a Emília. É um dos maiores mistérios
dos tempos modernos. Nasceu boneca de pano, feia e muda, feita lá pela
tia Nastácia, e foi indo, foi ``evoluindo'', até ficar no que é.

Hércules não tinha vergonha de perguntar o que era quando não entendia
alguma palavra, e perguntou o que queria dizer ``evoluindo''.

--- Evoluir é mudar com aperfeiçoamento. Uma coisa que muda mas não se
aperfeiçoa, não está evoluindo. A água dum rio está sempre mudando de
lugar, mas não evolui; porque muda sem aperfeiçoar-se, entendeu?

Hércules fez um esforço para entender e parece que entendeu, pois disse:

--- Nesse caso, eu também estou evoluindo. Minhas ideias estão mudando.

--- Para melhor ou para pior?

--- Para melhor\ldots{}

%* ``A Fênix'', capítulo 4 de \textit{Os doze trabalhos de Hércules} (1944).

\chapter{A Terra vista da Lua}

--- Mas o mais bonito da Lua --- disse depois São Jorge --- é a Terra, a
nossa Terra que daqui vemos perpetuamente no céu, girando sobre si
mesma. Olhe como está linda!

Parece incrível, mas só naquele momento os meninos ergueram os olhos
para o céu e lá viram a Terra. Tão entretidos desde a chegada estiveram
com as coisas do chão, que só naquele instante deram com o espetáculo
mais belo da Lua --- a Terra vista de lá.

--- Que beleza! --- exclamou Narizinho. --- Só para ver este espetáculo
vale a pena vir à Lua\ldots{}

A Terra é a lua da Lua. Mora permanentemente no céu da Lua, sempre
girando sobre si mesma e a mostrar os seus continentes e mares. Um
verdadeiro relógio. Quem quer saber das horas é só olhar para a Terra em
seu giro sem fim e ver que continentes vão aparecendo.

Naquele momento a face que a Terra exibia estava completamente escura,
porque era dia de eclipse do Sol. Mas depois de findo o eclipse, quando
o Sol voltou a iluminar a Terra, os meninos se regalaram. Lá estava bem
visível, como num mapa, o continente americano, composto de dois grandes
``\textsc{vv}'', um em cima do outro. No alto do \textsc{v} de cima aparecia uma brancura
vivíssima --- as terras de gelo do polo norte; e igual brancura aparecia
embaixo do segundo \textsc{v} --- as terras de gelo do polo sul. E apareciam umas
imensidades escuras --- os oceanos. E também grandes zonas de verdura.

--- Aquela verdura enorme --- disse Pedrinho --- é o Brasil e os países
que ficam perto dele --- Argentina, Uruguai, Paraguai, Chile, Peru,
Bolívia, etc. Está vendo aquelas minhocas que varam o continente de
ponta a ponta, com brancura em certos trechos do dorso? Pois são os
Andes, a grande cordilheira cheia de picos de neves eternas, e a
cordilheira do México e as montanhas Rochosas. E lá em cima estão o
Canadá, os Estados Unidos, o México e a América Central\ldots{} Aqueles
pontinhos de outra cor na imensidão do mar são as ilhas --- Cuba e
tantas outras\ldots{}

São Jorge não estava entendendo coisa nenhuma, porque todos aqueles
nomes lhe eram novidade.

--- Meu Deus! --- exclamou em certo momento. --- Será possível que haja
no mundo tantos países novos que eu não conheça?

--- Se há! --- exclamou Pedrinho. --- Isso de países é como broto de
árvore. Uns secam, apodrecem e caem --- e surgem brotos novos. Quais
eram os países do seu tempo?

São Jorge suspirou.

--- Ah, no meu tempo o mundo era bem menor. Havia Roma, a grande Roma,
cabeça do Império Romano --- e o Império Romano era tudo. Quase todos os
povos da Europa estavam dominados pelos romanos --- como a Espanha, a
Aquitânia, a Bretanha, a Macedônia, a Grécia, a Trácia, a Panônia, a
Arábia Petréia, a Galácia, a Cilícia, a Mauritânia lá na costa da
África\ldots{}

--- E a tal Capadócia onde o senhor nasceu? --- perguntou a menina.

--- A minha Capadócia ficava entre um país de nome Ponto e outro de nome
Cilícia --- junto da Mesopotâmia.

Pedrinho contou que estava tudo muito mudado. O tal Império Romano já
não existia; em vez dele surgira o Império Britânico, cuja cabeça era a
Grã-Bretanha.

Ao ouvir falar em Grã-Bretanha São Jorge arregalou os olhos. Percebeu
que era a mesma Bretanha do seu tempo, um país que na era dos romanos
não valia nada. E também muito se admirou quando Pedrinho se referiu à
Rússia como o maior país do mundo, e à China, e à índia e ao Japão.

--- Onde fica a tal Rússia? --- perguntou ele.

Pedrinho explicou como pôde, e por fim São Jorge descobriu que a famosa
Rússia devia ser numas terras muito desconhecidas dos romanos e às quais
vagamente eles chamavam Sarmácia. Da China e do Japão o santo não tinha
a mais leve ideia.

--- Como tudo está mudado! --- exclamou ele. --- Se eu voltar à Terra,
não reconhecerei coisa nenhuma.

--- Também acho --- concordou Pedrinho. --- Há continentes inteiros que
no seu tempo eram totalmente ignorados, como as Américas e o continente
australiano. As

Américas foram descobertas mais ou menos ali em redor do ano 1.500, e a
Austrália em redor do ano 1.800.

--- Onde fica essa Austrália?

--- Nos confins do Judas! --- berrou Emília. --- Nem queira saber.
Existem lá uns tais cangurus que carregam os filhotes numa bolsa da
barriga. E há o bumerangue, que a gente joga e ele volta para cima da
gente.

A ignorância de São Jorge era natural, visto como vivera no tempo de
Diocleciano, cujo reinado fora entre os anos 284 e 313. De modo que fez
muitas perguntas a Pedrinho, grandemente se assombrando com as
respostas.

Emília estava com cara de quem quer dizer uma coisa, mas não se atreve.
Por fim afastou-se de Narizinho (para evitar o beliscão) e de repente
disse:

--- Santo, desculpe o meu intrometimento --- mas lá no sítio, quando
alguém quer dizer que um gajo não presta, e é vadio ou malandro, sabe
como diz? Diz que é um capadócio!\ldots{}

Narizinho fuzilou-a com os olhos, mas São Jorge não se zangou, até
sorriu, e foi suspirando que explicou:

--- Meus patrícios lá da Capadócia sempre tiveram má fama --- e fama
exatamente disso, de mandriões, de fanfarrões, de mentirosos. Mas o que
admira é que apesar de tantos séculos, a palavra ``capadócio'' ainda
esteja em uso até num país que nem existia no meu tempo\ldots{}

--- Pois existe --- continuou Emília sempre com o olho em Narizinho ---
e acho que o senhor não deve andar dizendo que é um capadócio, porque
não há o que desmoralize mais\ldots{}

--- Emília!\ldots{} --- gritou a menina ameaçando-a com um tapa. Mas São
Jorge acalmou-a e, chamando Emília para o seu colo, alisou-lhe a cabeça.

--- Vou seguir o seu conselho, bonequinha. Não contarei nem ao dragão
que sou um capadócio\ldots{}

%* ``A Terra vista da Lua'', capítulo 8 de \textit{Viagem ao céu} (1932).

\chapter{Discussões em Atenas}

Enquanto se desenvolvia a conversa de Dona Benta com os dois gregos, os
meninos examinavam as estátuas, os móveis, as pinturas das paredes. ---
Coitados! --- exclamou Narizinho. --- Estão completamente tontos. Não
entendem nada do que vovó diz.

--- Pois eu estou entendendo tudo nesta casa, e estou até adivinhando
que ali dentro é o lugar dos comes --- cochichou Emília, apontando para
uma sala vizinha. --- Vamos espiar?

Espiaram pela porta. Sala das refeições, sim. Uma escrava punha à mesa o
almoço de Péricles --- pão, queijo, mel, vinho, uvas e figos ---
daqueles maduríssimos que fazem vir água à boca.

Narizinho e Pedrinho lamberam os beiços. A servente sorriu e
aproximou-se com três figos na mão, um para cada um.

--- Quem no mundo vai acreditar --- disse Narizinho, abrindo o seu ---
que já comemos figos na casa de Péricles? E que bom está! Um
melado\ldots{}

A escrava não entendeu --- nem podia entender, mas levou-os para dentro,
a mostrar a casa. Móveis lindos, mas discretos. Tudo muito elegante e
sóbrio.

Pedrinho achou graça nas lâmpadas de azeite.

--- Isto é o tal candeeiro que vovó conta que havia na casa do pai dela.
Aqui a gente põe o azeite; aqui é a mecha. Engraçado, não?

--- Não é assim também na terra onde vocês moram? --- perguntou a
escrava.

--- Foi assim --- respondeu Pedrinho. --- Hoje temos a eletricidade ---
a luz elétrica.

--- É algum azeite especial? Pedrinho deu uma risada gostosa e bobeou-a:

--- Sim, é um azeite feito de vibrações do éter.

A pobre escrava ficou na mesma.

Narizinho estranhou muito o sistema de mesas dali.

Baixinhas, tendo em redor, em vez de cadeiras, coxins.

Os gregos comiam reclinados em coxins.

--- Só que não vejo talheres. Que é dos talheres, escrava? --- perguntou
Emília.

A escrava não entendeu --- nem podia entender, porque naquele tempo
todos comiam com as mãos.

Ao saber disso, a ex-boneca berrou:

--- Ché! Quando vem à mesa um peru assado, como se arranjam?

A escrava não sabia nada sobre o peru, que Pedrinho lembrou ser
originário da América do Norte, e, portanto, desconhecido dos gregos.

--- Mas pavões há por aqui --- disse Narizinho, vendo uma pena de pavão
espetada na parede.

--- Sim --- disse a escrava --- e lindos. Querem ver? E levou-os a
visitar a meia dúzia de lindos pavões do aviário de Péricles.

Lá no pátio o grande heleno continuava de prosa com a velha. Discutiam
política.

--- Vencemos a aristocracia, minha senhora --- dizia ele. --- Hoje a
Grécia é positivamente governada pelo povo. Sólon revelou gênio ao
conceber a nossa forma de governo. Não há imposição dum homem. O
governante é escolhido pelo povo. Eu, por exemplo, executo o que o povo
deseja --- e por isso me reelegem.

--- O senhor é um caso excepcional --- argumentou Dona Benta --- diz que
segue os desejos do povo, mas na realidade a sua inteligência e os seus
excelentes discursos é que fazem o povo desejar isto ou aquilo. Quem
realmente governa é o senhor, não o povo.

--- Vejo que a senhora possui um alto descortino psicológico --- disse
Péricles sorrindo. --- O povo tem muito das crianças. Quer ser conduzido
--- mas com aparências de que é ele quem de fato conduz e manda. O meu
sistema, entretanto, é nada querer em contrário aos interesses do povo.
Sou o intérprete desses interesses --- e o esclarecedor da cidade. Esta
minha ideia de fazer de Atenas uma obra-prima de arte é hoje a ideia de
todos os atenienses. Consegui passá-la de meu cérebro para o de todos
--- e sinto grande satisfação ao ver o orgulho dos atenienses quando os
visitantes se deslumbram com a nossa cidade.

--- Noto um erro nas suas palavras quando se refere a ``povo'', Senhor
Péricles. Não é o povo quem governa Atenas, sim a pequena classe dos
cidadãos. Povo é a população inteira e aqui há 400 mil escravos que não
têm o direito de voto. Isto é injusto e será fatal à Grécia.

Péricles muito se admirou daquele modo de ver.

--- Mas eles são escravos, minha senhora! Escravo é escravo.

--- Engano seu, Senhor Péricles. Pelo fato de ser escravo, um homem não
deixa de ser homem; e uma sociedade que divide os homens em livres e
escravos, está condenada a desaparecer.

Essa ideia fez o grego sorrir.

--- Acha então que pode haver uma sociedade sem escravos e senhores?
Quem fará os trabalhos pesados?

--- Uma sociedade justa não pode ter escravos, Senhor Péricles, e nela
todos os trabalhos serão feitos por homens livres. Assim é lá no mundo
moderno donde vim. Bem sei que aqui todos pensam como o senhor, e até o
grande filósofo Aristóteles, que para os gregos de hoje ainda vai nascer
daqui a 54 anos, dirá na abertura de seu tratado sobre a Política, estas
palavras absurdas:

``Os homens dividem-se naturalmente em escravos e senhores.'' Está
errado. Artificialmente é que é assim; naturalmente, não. Já no meu país
também tivemos escravos, até o ano de 1888 que foi quando a Princesa
Isabel, a Redentora, promulgou a Lei 13 de Maio, também chamada Lei
Áurea. Foi o fim da escravidão no Brasil.

Péricles abriu a boca. Ele julgava perfeita a forma social de Atenas e
aquela misteriosa criatura tinha o topete de dizer que não\ldots{}

Dona Benta mudou de assunto.

--- Pois, Senhor Péricles, saiba que o problema de governar os povos
talvez não seja resolvido nunca. Na era em que vivo, a 2377 anos daqui,
o problema continua cada vez mais ameaçador. Fazem-se experiências de
toda sorte. Uns povos se inclinam para a democracia, que é como chamam
esta forma grega de governar; outros estão sob as unhas da ditadura;
outros tentam um comunismo que nada tem com o que Platão sonhou.

--- Que Platão?

--- Um grande filósofo que ainda está no calcanhar da avó e irá nascer
justamente no ano da sua morte, Senhor Péricles, em 429 A. E. Esse
filósofo sonhou uma forma de governo adiantada demais para criaturas tão
imperfeitas como os homens, mas mesmo assim os modernos do meu tempo
tentam pô-la em prática. Outros povos experimentam uma coisa chamada
``totalitarismo'', em que o Estado é tudo e nós, as pessoas, menos que
moscas. Neste regime o indivíduo não passa de grão de areia do Estado.

--- Mas não há Estado, minha senhora! --- disse Péricles. Isso é uma
ideia abstrata. O que há são criaturas humanas com interesses em
conflito; a política não passa da arte de harmonizar esses interesses
individuais com um máximo de benefício geral. O meu governo não é mais
que isso.

--- O sonho é esse, Senhor Péricles, mas a realidade para a qual
caminhamos afastar-se-á muito dessa sensatíssima concepção. A pobre
humanidade, depois de tremendas lutas para escapar à escravização aos
reis, caiu na escravização, pior ainda, ao Estado --- à palavra Estado.

--- Quer dizer que no futuro os reis de carne e osso serão substituídos
por um ``som'' --- o som ``Estado?''

--- Sim, e isso virá fazer mais mal ao mundo do que todos os velhos reis
reunidos, somados e multiplicados uns pelos outros. Esta forma
democrática de Atenas tropicará no meio do caminho. Será destruída pela
palavra ``Estado'', que crescerá e dominará tudo até chegar à forma
``totalitária'' em que o som ``Estado'' é o total, e nós, os indivíduos,
simples pulgas.

Péricles ficou meditativo. Aquela revelação vinha contrariar as suas
ideias sobre a continuidade do progresso humano.

--- Então\ldots{} então a prova provada de que uma forma de governo é
boa não tem valor nenhum? O progresso não é uma consolidação de
conquista?

--- Nem na arte é assim, Senhor Péricles. Ao ver aqui em sua casa estas
maravilhas da escultura grega, sinto pontadas no fígado.

--- Por que, minha senhora?

--- Porque o futuro vai afastar-se disto\ldots{}

--- Como? Não admite então que nestas estátuas há o máximo de beleza que
os escultores já conseguiram?

--- Admito, sim --- mas ``sei'' que no futuro isto será moteiado, e esta
beleza substituída por outra, isto é, pelo horrendo grotesco que para os
meus modernos constituirá a última palavra da beleza. Como prova do que
estou dizendo vou mostrar um papel que por acaso tenho aqui na bolsa ---
e Dona Benta tirou da bolsa uma página de ``arte moderna'', onde havia a
reprodução dumas esculturas e pinturas cubistas e futuristas.

Péricles olhou para aquilo com espanto, e mostrou-o a Fídias.

--- Mas é simplesmente grotesco, minha senhora! --- disse depois. ---
Estas esculturas lembram-me obras rudimentares dos bárbaros da Ásia e
das regiões núbias abaixo do Egito\ldots{}

--- Pois não são. São as maravilhas que embasbacam os povos mais cultos
do meu tempo --- a 2377 anos daqui\ldots{}

Os dois gregos ficaram literalmente tontos, sem saber o que pensar. As
revelações da estranha velhota vinham opor-se a todas as suas ideias
sobre a marcha indefinida do progresso humano. Totalitarismo, cubismo,
futurismo\ldots{} Pobre humanidade!

%* ``Discussões em Atenas'', capítulo 5 de \textit{O Minotauro} (1939).

\chapter{Emília forma palavras}

--- Pois é isso --- continuou a velha, ainda tonta da sapequice
gramatical da Emília. --- A Raiz das palavras não muda; de modo que,
para formar palavras novas, a gente faz como o jardineiro: poda o que
não é Raiz e enxerta o Sufixo.

--- Em vez de enxertar o Sufixo no fim, não é possível enxertá-lo no
começo? --- quis saber Narizinho.

--- Não --- respondeu a velha. --- Os Sufixos, assim como os rabos dos
animais, só se usam na parte traseira. Há, porém, os irmãos dos Sufixos
que servem justamente para enxertos no começo da Raiz.

--- E como se chamam?

--- \textsc{prefixos}; \textsc{pre} quer dizer antes. \textsc{re}, \textsc{trans}, \textsc{a} e \textsc{com}, por exemplo, são
Prefixos. Se tomarmos o Verbo \textsc{formar} e grudarmos na frente dele esses
Prefixos, teremos os novos Verbos \textsc{reformar}, \textsc{transformar} e \textsc{conformar},
todos com sentido diferente.

--- Voltemos aos Sufixos, que são mais engraçadinhos --- propôs Emília.
--- Diga uma porção deles, Dona Timótea.

A velha, que já estava cansada de tanto falar, tomou mais um gole de chá
e prosseguiu, apontando para um armário:

--- Os Sufixos estão todos nas gavetas daquele armário. Vá lá e mexa com
eles quanto quiser, mas não me chame mais de Timótea, ouviu?

Emília não esperou segunda ordem. Correu ao armário, abriu as gavetas e
tirou de dentro um punhado de Sufixos. Depois espalhou-os sobre a mesa
para aprender a usá-los. Pedrinho e a menina vieram tomar parte no
brinquedo.

--- Olhe, Narizinho --- disse a boneca ---, ali está uma caixa de
Substantivos. Traga-me um; e você, Pedrinho, agarre aquela faca.

Os dois meninos assim fizeram. Narizinho depôs sobre a mesa um
Substantivo pegado ao acaso --- \textsc{pedra}.

--- Segure-o bem, senão ele escapa --- recomendou Emília ---; e agora,
Pedrinho, corte a Desinência deste Substantivo de um só golpe. Vá!

--- Mas esta faca será capaz de cortar \textsc{pedra}? --- indagou o menino de
brincadeira, só para ver o que a boneca dizia. A diabinha, porém, estava
tão interessada na operação cirúrgica que apenas gritou:

--- Corte e não amole!

Apesar da recomendação, o menino amolou a faca na sola do sapato e só
depois disso é que --- Zás!\ldots{} --- atorou a Desinência de \textsc{pedra}, a
qual deu um gritinho agudo.

--- Pronto! --- exclamou Emília. --- O pobre Substantivo está reduzido a
uma simples Raiz. Venha ver, Dona \textsc{etimologia}, como é engraçadinha esta
Raiz.

Mas a velha, que andava farta e refarta de lidar com Raízes, nem se
mexeu do lugar. Emília, então, tomou um dos Sufixos tirados da gaveta,
justamente o \textsc{aria}, e fez a ligação com um pouco de cuspo. Imediatamente
surgiu a palavra \textsc{pedraria}.

--- Viva! Viva! --- gritou ela batendo palmas. --- Deu certinho!

--- Venha ver, Dona Eufrásia! Com uma Raiz e um Sufixo fabricamos uma
palavra nova, que quer dizer muitas pedras. Deixe esse chá sem graça e
venha brincar.

Mas a velha estava muito velha para brincadeiras e limitou-se a tomar um
novo gole de chá.

--- Vamos ver outro Sufixo --- propôs Narizinho.

Emília pegou outro, o Sufixo \textsc{ada}, e experimentou a ligação. Deu a
palavra \textsc{pedrada}.

--- Ótimo! Este também dá certinho. \textsc{pedrada}, todos sabemos o que é.
Vamos ver outro.

Emília pegou um terceiro Sufixo --- \textsc{eria} --- e experimentou a ligação.
Deu a palavra \textsc{pedreria}, que não tinha sentido.

--- Este não presta --- gritou Pedrinho. --- Não dá nada que se entenda.
Veja outro, Emília, esse \textsc{eiro}.

Ligou bem. Deu \textsc{pedreiro}.

--- Ótimo! --- exclamou a boneca. --- Vamos ver este cá, \textsc{alha}.

Deu \textsc{pedralha}, que eles não sabiam o que era, mas estava ---com jeito de
ser qualquer coisa. Depois experimentaram os Sufixos \textsc{ulho}, \textsc{ena}, \textsc{io},
\textsc{dade}, \textsc{ame}, \textsc{uje} e \textsc{al}, com resultados variáveis. Uns deram, outros não
deram nada. \textsc{ulho}, com um \textsc{eg} no meio, deu uma beleza --- \textsc{pedregulho}. O
Sufixo \textsc{dade} deu asneira ---\textsc{pedrade}.

Emília olhou para o rótulo da gaveta e viu que estava usando Sufixos de
Coleção.

Na gaveta imediata estavam os Sufixos de Aumento: \textsc{ão}, \textsc{zarrão}, \textsc{az}, \textsc{aço} etc.

Na terceira gaveta estavam os Sufixos de Diminuição: \textsc{inho}, \textsc{zinho}, \textsc{ito},
\textsc{ebre}, \textsc{ilho} e mais uns trinta.

Na quarta estavam os Sufixos de Agente: \textsc{dor}, \textsc{nte}, \textsc{ario}, \textsc{aria}, \textsc{eiro},
\textsc{eira}.

Na quinta, os Sufixos designativos de Ação ou de resultado da Ação: \textsc{ção},
\textsc{mento}, \textsc{ada}.

Na sexta, os Sufixos designativos de Lugar: \textsc{ario}, \textsc{aria}, \textsc{eiro}, \textsc{eira},
\textsc{douro}, \textsc{doura}, \textsc{orio}.

Na sétima estavam os Sufixos designativos de Estado: \textsc{ura}, \textsc{eza}, \textsc{idade},
\textsc{dade}, \textsc{ice}, Ê\textsc{ncia}, \textsc{tura}, \textsc{ite}.

E na oitava estavam os Sufixos de Dignidade e Profissão: \textsc{ado} e \textsc{ato}.

Apesar de serem muitos, os meninos fizeram experiência naquela Raiz com
quase todos os Sufixos, conseguindo formar as seguintes palavras
derivadas de \textsc{pedra}: \textsc{pedraria}, \textsc{pedrada}, \textsc{pedral}, \textsc{pedragem}, \textsc{pedreiro},
\textsc{pedrama}, \textsc{pedrame}, \textsc{pedrume}, \textsc{pedregulho} (neste caso foi preciso intercalar
um E e um G para dar certo), \textsc{pedrão}, \textsc{pedraço}, \textsc{pedraça}, \textsc{pedrázio},
\textsc{pedralha}, \textsc{pedrorra}, \textsc{pedrinha}, \textsc{pedrita}, \textsc{pedrete}, \textsc{pedrote}, \textsc{pedrilha},
\textsc{pedrica}, \textsc{pedrisco}, \textsc{pedracho} e \textsc{pedreira}, ou seja, 24 ao todo.

--- Vinte e quatro! --- exclamou o menino. --- Agora estou compreendendo
por que há tantas palavras na língua. Pois se somente com esta
porqueirinha de Raiz nós pudemos formar 24 palavras diversas, imagine
quantas não formaríamos usando todas as raízes que existem!

--- E isso lidando só com os Sufixos próprios para fazer Substantivos
--- disse Dona \textsc{etimologia} aproximando-se ---, porque há ainda os Sufixos
que servem para fabricar Verbos, como, por exemplo, \textsc{gotejar}, que é a
Raiz do Substantivo \textsc{gota} ligada ao Sufixo \textsc{ejar}. Com esse \textsc{ejar}, e ainda
com \textsc{ear}, \textsc{izar}, \textsc{entar}, \textsc{ecer}, \textsc{itar}, \textsc{inhar}, \textsc{icar} e outros, a Emília pode
passar dias e dias brincando de transformar em Verbos todos os
Substantivos que houver lá no sítio.

--- A senhora dá licença de eu levar para lá uma coleção de Sufixos? ---
pediu a boneca.

--- Dou --- respondeu a velha ---; mas primeiro trate de consertar a
palavra \textsc{pedra} e de juntar do chão todos esses Sufixos espalhados. Quero
tudo direitinho como estava.

Emília recolou a Desinência da palavra \textsc{pedra} e varreu todos os Sufixos
que tinham caído no chão. Depois arrumou-os, muito bem-arrumadinhos, nas
respectivas gavetas.

Dona \textsc{etimologia} ofereceu chá aos meninos, e enquanto eles o tomavam teve
ocasião de explicar que a palavra \textsc{chá} viera da China, onde significava a
bebida feita de certa planta, o \textit{Thea sinensis}; depois a palavra
passou a ser usada para designar a infusão de folhas ou raízes de
qualquer planta. Que velha sabida! Parecia Dona Benta.

--- Bom, bom, bom --- exclamou ela ao terminar o lanche e sem se erguer
da mesa. --- Isso de que falei, e com que vocês estiveram brincando,
chama-se a Derivação Própria das Palavras, porque há também a Derivação
Imprópria.

--- Já sei! --- adivinhou Emília. --- A tal Derivação Imprópria é a que
se faz sem Sufixo, nem Prefixo.

A boa velha assombrou-se.

--- Como sabe, bonequinha?

--- Esperteza --- disse Emília piscando um olho. --- Eu muitas vezes
arrisco opiniões que dão certo. Tia Nastácia diz que quem não arrisca
não petisca\ldots{}

%* ``Emília forma palavras'', capítulo 15 de \textit{Emília no País da Gramática} (1934).

\chapter{O museu da Emília}

\castpagenamed{Personagens}

\cast{dona benta}
\cast{narizinho}
\cast{pedrinho}
\cast{tia nastácia}
\cast{visconde de sabugosa}
\cast{emília, a menina da capinha vermelha e um lobo}
\noindent\paren{como são imaginados nas \textit{Reinações de Narizinho}, do mesmo autor.}

\newscenenamed{Ato único}

\stagedir{Cenário: Sala de jantar de uma modesta casa de fazenda: o Sítio da
Dona Benta. Dona Benta está em cena examinando várias peças de roupa de
Pedrinho, amontoadas sobre a mesa. Há uma cadeira comum, de pernas
serradas --- cadeira de velha.}

\siderepl{dona benta} Não sei o que Pedrinho faz dos botões. Com certeza os
come. Toda a semana levo consertando a roupa dele e pregando botões.
Olha só este paletó, sem um só botão e sem bolso também. Parece
incrível! \paren{toma aquele paletó e senta-se na cadeira de pernas
serradas junto à qual está a cesta de costura} Nastácia! \paren{pausa}
Nastácia!\ldots{}

\siderepl{uma voz de dentro} Já ouvi, Sinhá. Já vou indo. \paren{Nastácia
aparece enxugando as mãos no avental}

\siderepl{tia nastácia} Que é Sinhá???

\siderepl{dona benta} Onde andam os meninos? Saíram cedo e até agora nem sinal.

\siderepl{tia nastácia} Onde andam!\ldots{} Por esses mundos afora, Sinhá,
fazendo quanta estripulia há. Aqueles diabretes são ``capaz'' de tudo,
Sinhá. Depois que deram comigo na Lua, e me deixaram lá cozinhando para
São Jorge e com aquele horrível Dragão que me espiava e lambia os beiços
com a língua de ponta de flecha, eu não acho nada impossível. Credo! Só
de me lembrar disso, sinto ainda um arrepio no corpo\ldots{}

\siderepl{dona benta} A Emília foi também?

\siderepl{tia nastácia} Se foi! A Emília está virando a Lampiãozinha do bando
depois que se apanhou dona daquele boi dum chifre só na testa, o tal
ri\ldots{} ri\ldots{}

\siderepl{dona benta} Rinoceronte.

\siderepl{tia nastácia} É isso. Depois que se apanhou dona e amiga íntima do
tal do ``rinoceronte'', está que está uma rainha de tão mandona. Diz
cada desaforo para mim, Sinhá, que só vendo. Me destrata de
``anarfabeta'' e ``inguinorante'' para baixo, a pestinha, como se não
fosse eu que fizesse ela.

\siderepl{dona benta} Que a fizesse, Nastácia. Olhe a gramática.

\siderepl{tia nastácia} \paren{suspirando} Inda mais essa agora, a tal
gramática, como se não fosse a minha trabalheira na cozinha, com o raio
do ri\ldots{} ``rinoceronte'' no quintal me espiando o tempo inteiro,
tal qual o dragão. Eu é porque\ldots{}

\siderepl{dona benta} Espere! Que horas são?

\siderepl{tia nastácia} \paren{que tem vista curta, trepa a uma cadeira para ver as
horas no relógio da parede} Quatro e meia, quase hora de jantar.
Sinhá, não se assuste que a cambadinha não tarda aí. Garanto. Quando a
fome aperta, vêm todos ventando nem que seja da Lua. Nesses países
encantados onde costumam passear ou aventurar, como eles dizem, parece
que há de tudo: Fadas, Príncipes que viram ursos, castelo de ouro e
marfim, tudo, menos comida.

\siderepl{dona benta} O que está fazendo para o jantar?

\siderepl{tia nastácia} Sopa de batata, salsa, galinha ensopada\ldots{}

\siderepl{dona benta} Que galinha matou?

\siderepl{tia nastácia} Aquela franga sura de seis dedos no pé direito.

\siderepl{dona benta} \paren{recordando-se} Espere. Essa franga parece que era
da Emília, não? Ouvi aí um negócio entre a Emília e Narizinho a
propósito dessa sura.

\siderepl{tia nastácia} Eu sei, Sinhá. A franga era de Narizinho, mas Narizinho
vendeu o pé da franga a Emília e me deu o resto, de modo que matei a
franga e guardei o pé de seis dedos para a Emília que anda agora com
mania de fazer um museu de coisas esquisitas. Tem cada uma\ldots{}

\paren{Barulhada no terreiro vem interromper a conversa. É o bandinho que
chega. Entram Narizinho, Pedrinho, Emília e o Visconde,
atropeladamente.}

\siderepl{narizinho} Bom dia ou boa tarde, vovó. \paren{corre a beijar a mão da
velha. Pedrinho faz o mesmo}

\siderepl{dona benta} Então? Por onde andaram?

\siderepl{pedrinho e narizinho} \paren{ao mesmo tempo --- ao atropelo, um
dizendo uma frase e o outro a outra} Nem queira saber, vovó! Tivemos
uma aventura das mais perigosas. Na floresta dos Tucanos Amarelos. Sim,
lá perto da casa da menina da Capinha Vermelha. Ela não estava em casa.

\siderepl{dona benta} \paren{levando as mãos aos ouvidos} Parem! Vocês me
deixam tonta, tonta. Cada um fale por sua vez. Vamos, comece, Narizinho.

\siderepl{emília} \paren{que está a um canto mostrando qualquer coisa para o
Visconde} Não seja bobo! Eu sei o que faço \paren{e começa a
cochichar-lhe ao ouvido}.

\siderepl{narizinho} Pois é isso, vovó. Fomos parar bem perto da casa da
Capinha Vermelha. Mas sabe quem encontramos? O lobo! Aquele horrível
lobo que comeu a avó dela!\ldots{}

\siderepl{tia nastácia} \paren{que já se ia retirando para a cozinha, entrepara ao
ouvir a palavra ``lobo'' e persigna-se} Credo!

\siderepl{narizinho} \paren{continuando} E sabe o que a Emília fez? Desafiou o
lobo! ``Vá lá no Sítio comer a Dona Benta pra ver o que acontece, seu
cara de coruja seca!'' Disse ela bem no focinho dele, imagine! Eu,
Pedrinho e o Visconde, assim que vimos o lobo, não quisemos saber de
histórias e trepamos numa árvore bem alta. Ele estava com cara de fome
de velha. Mas a Emília nem se mexeu. Plantou-se diante dele com as
mãozinhas na cintura, a dizer os maiores desaforos que um lobo jamais
ouviu! Até de analfabeto o xingou!

\siderepl{dona benta} \paren{para Emília} Emília, como é que você faz isso?
Nunca devemos ofender os passantes, e principalmente um passante
perigoso como esse lobo. Por vingança é bem capaz de vir rondar aqui a
casa e no mínimo me apanhar umas galinhas.

\siderepl{emília} Pois que venha, é isso mesmo que eu quero. Provoquei-o de
propósito e já botei o meu rinoceronte de guarda na porteira, de chifre
armado. Assim que o lobo aparecer com aquele focinho e começar a farejar
o ar para ver se tem avó de menina aqui dentro\ldots{}

\siderepl{dona benta} Que história é essa de avó de menina?

\siderepl{emília} Esse lobo se alimenta de avós de meninas. Comeu a avó da
Capinha e gostou do petisco.

\siderepl{dona benta} Que bobagem, Emília! Você bem sabe que o Lobo que comeu a
avó da Capinha foi morto a machadadas por um lenhador.

\siderepl{emília} O lenhador não o matou bem matado, e o lobo reviveu outra
vez, com mais fome de velha ainda! Cheguei bem pertinho e vi no corpo
dele os sinais das machadadas.

\siderepl{pedrinho} Não perca tempo com essa boba, vovó. Ela não viu sinal
nenhum. Era um lobo à toa, como outro qualquer. Pergunte ao Visconde.

\siderepl{visconde} Na minha opinião\ldots{}

\paren{Emília finca as mãos na cintura e encara o Visconde com tais
olhos que ele treme e diz justamente o contrário do que ia
dizendo}\ldots{}

\siderepl{visconde} Esse lobo era exatamente o mesmo que comeu a avó de Capinha
\paren{Emília vitoriosa põe a língua para Pedrinho --- Ahn!}.

\paren{Barulho fora. Alguém bate com aflição na porta. ``Abram!''
Pedrinho corre a abrir, mas primeiro espia quem é pelo buraco da
fechadura. ``Uma menina!'', exclamou. Abre a porta.
Aparece a menina da Capinha Vermelha.}

\siderepl{capinha} \paren{Entra de ímpeto e fecha a porta, ficando a escorá-la,
enquanto volta-se para os demais, de olhos arregalados} --- O lobo! O
lobo que comeu a Vovó!\ldots{}

\paren{Grande pânico. Dona Benta abana-se com o paletó de Pedrinho, está
com as pernas tão moles que não pode erguer-se da cadeira. O Visconde
trepa em cima da mesa, tira dum prego o binóculo de Dona Benta e põe-se
a espiar o terreiro pelo vão da janela}.

\siderepl{visconde} Não vejo lobo nenhum. Foi rebate falso. Esperem\ldots{}
Estou vendo sim\ldots{} Lá longe\ldots{} uma coisa. Parece o cachorro do
compadre Teodorico! Muito longe. Um quilômetro daqui.

\siderepl{emília} O Visconde é um idiota. Não sabe ver lobo. \paren{corre para
ele e toma-lhe o binóculo e olha} É lobo sim. Cachorro do compadre
Teodorico é o nariz dele. Lobíssimo! Com dois olhos que são duas tochas.
E dentes arreganhados. Vem babando de fome. Já percebeu que aqui há avó
de menina. Pedrinho, feche Dona Benta dentro do guarda-comida!

\siderepl{dona benta} Nossa Senhora da Aparecida! Esta criança acaba me pondo
maluca. Não há mais sossego neste Sítio? Cada dia é uma coisa? Ou é
rinoceronte, ou é dragão de São Jorge, ou é lobo\ldots{} \paren{abana-se
aflita}.

\siderepl{capinha} Chamem o homem do machado!

\siderepl{narizinho} Nesta casa o único homem é o Pedrinho, que só usa bodoque.

\siderepl{capinha} Então não sei como vai ser, porque sem homem com machado não
há meio de vencer esse lobo.

\siderepl{emília} \paren{sempre a observar pelo binóculo} Lá vem vindo ele!
Vem lambendo os beiços. Já farejou duas velhas aqui dentro. É exatamente
o mesmo que comeu a avó da Capinha. Estou vendo as cicatrizes das
machadadas e até estou vendo um pedaço de machado que ficou na testa
dele\ldots{}

\siderepl{dona benta} Que olhos Emília tem!

\siderepl{emília} \paren{continuando} Já passou a porteira\ldots{} Está no
terreiro. Vem vindo, vem vindo\ldots{} Parou para farejar o mastro de
São João. Vai comer o mastro!\ldots{}

\siderepl{dona benta} \paren{consigo} Será possível?

\siderepl{emília} \paren{continuando} Não comeu o mastro, não. Vem vindo para
a varanda. Chegou \paren{pula na mesa e tranca a janela.}

\paren{Todos se agitam, menos Dona Benta, que não consegue despegar-se da
cadeira, embora o tente. Pedrinho empurra um móvel para escorar a porta.
``Ajuda garotada, que o negócio é sério.'' Emília ajuda, levando a
vassoura para fazer peso na porta. Narizinho reflete, de mão no queixo.
Nisto ouve-se um arranhar de tábua. É o lobo arranhando a porta.}

\siderepl{emília} Pronto! Está aí ele arranhando a porta e quero ver agora como
vocês se arranjam! \paren{Para Dona Benta} E a senhora que é tão
sabida, de tantos livros e dicionários que leu, quero ver como se salva.
Veja no seu dicionário, que ensina tudo quanto se quer, se ensina jeitos
de espantar lobos. Eu não preciso ir ao dicionário. Sei um jeito que é
tiro e queda.

\siderepl{narizinho} Então diga logo. Tenha dó da aflição de vovó.

\siderepl{pedrinho} \paren{fazendo muxoxo} Não dou um vintém pelo tal jeito.

\siderepl{emília} \paren{muito lampeira} Só direi se Narizinho me der uma
coisa\ldots{}

\siderepl{narizinho} Já sei. Quer que eu dê a minha coleção de potinhos de
barro para você botar no museu, não é? Dou, sim, diga o jeito agora.

\siderepl{emília} \paren{mais lampeira ainda} Todos são testemunhas de que
Narizinho me deu os potinhos, não é mesmo? Muito bem. Nesse caso direi
que meu rinoceronte já está avisado de tudo e logo que eu der um assobio
ele investe contra o lobo e o espeta, bem espetado, no seu chifre
pontudo.

\siderepl{dona benta} Pois dê logo esse assobio, Emília, e não nos atormente
mais. Não seja tão mazinha.

\siderepl{emília} Esperem. Para dar o assobio eu quero que o Pedrinho me dê
aquele\ldots{}

\siderepl{pedrinho} O cavalinho sem rabo, não é? Pois não dou. Não tenho medo
de lobo. Você é uma cigana, mas comigo não tira farinha \paren{o lobo
arranha com mais força a porta e dá uns roncos terríveis, pondo-se em
seguida a uivar. Pedrinho amedronta-se}. Isto é, só dou se vovó mandar.
Por mim não dou. Mas se vovó mandar é outra coisa\ldots{}

\siderepl{dona benta} \paren{sempre aflita} Dê, Pedrinho. Dê tudo quanto ela
quiser. A Emília já tomou conta desta casa\ldots{}

\siderepl{emília} \paren{vitoriosa} E o Visconde também tem que me dar\ldots{}
É uma tristeza isto de fidalgos arruinados. Ele nunca tem nada para dar.
Só a cartolinha, que é tudo quanto o Visconde possui\ldots{}

\siderepl{narizinho} Ande, Emília! Chega de amolar! Assobia logo! Tenha dó da vovó, coitada!

\paren{O lobo atira-se contra a porta. Ouve-se um estalo de madeira
rachada. Emília assusta-se e leva os dois dedos à boca. Assobia. Há uma
pausa. Todos ficam à escuta do que se passa lá fora. Emília assobia de
novo. Nada acontece lá fora, e o lobo continua a despedaçar a porta.
Emília assobia pela terceira vez.}

\siderepl{emília} Que será que aconteceu?

\siderepl{narizinho} Com certeza o rinoceronte ficou surdo com a chuva desta
noite.

\siderepl{pedrinho} \paren{zombeteiro} Fiem-se numa boneca.

\siderepl{capinha} Não ficou surdo, não. O que ele está é dormindo. Quando vim
para cá encontrei-o atravessado na porteira, roncando. Até pulei por
cima, sem medo nenhum. Juro que está dormindo ainda.

\siderepl{emília} \paren{embaraçada} Com essa não contei. Dormindo, ladrão.
Nesse caso temos de acordá-lo. Mas como?

\siderepl{pedrinho} Se ele não tivesse o couro tão duro eu o acordava com uns
pelotaços de bodoque no focinho. Mas pelotada de bodoque em couro de
rinoceronte é o mesmo que beijo de mosquito.

\paren{Tia Nastácia, que está fora de cena desde o começo, entra da
cozinha com uma colher de pau na mão. Não sabe nada do que se passa.}

\siderepl{tia nastácia} Que barulhada é essa, gente? Sosseguem, que é hora de
arrumar a mesa.

\siderepl{dona benta} É o lobo, Nastácia!

\siderepl{tia nastácia} Que lobo, Sinhá? Mecê parece que está caducando. Onde
já se viu lobo a estas horas por aqui? Lobo, nada. Os meninos estão
empulhando mecê. \paren{Percebe a presença de Capinha} Ué? A menina da
Capinha Vermelha por aqui! Que novidade é essa?

\siderepl{capinha} O lobo que comeu vovó me perseguiu na floresta e corri a
esconder-me aqui. Está na porta, arranhando e despedaçando as tábuas.

\paren{O lobo dá um uivo prolongado e arranha as tábuas com mais fúria.
Tia Nastácia compreende tudo e põe-se a tremer de medo. A colher de pau
cai da sua mão.}

\siderepl{tia nastácia} Credo! Figa, rabudo! É o lobo mesmo. E agora, Sinhá,
que vai ser de nós?

\siderepl{dona benta} Emília foi quem arranjou isso e tinha combinado a defesa
com o rinoceronte. Era só dar um assobio que ele avançava com o chifre
contra o lobo e o varava de lado a lado. Mas já assobiou três vezes e
nada. Diz Capinha que ele está ferrado no sono, lá na porteira. Estamos
pensando num jeito de acordá-lo.

\siderepl{tia nastácia} Pois é mandar o Visconde fazer esse serviço. Para que
serve um visconde tão importante em casa senão para esses serviços
perigosos? \paren{voltando-se para o Visconde} Ande, Visconde!, vá lá
na carreira e acorde o ri\ldots{} o ``rinoceronte''. Mexa-se.

\siderepl{capinha} E se O lobo comer o Visconde?

\siderepl{tia nastácia} Não come nada, menina. O Visconde é de sabugo e os
lobos são ``carnivo''.

\siderepl{dona benta} Bela ideia! Vá, Visconde. Vá numa carreira acordar o
rinoceronte.

\siderepl{capinha} \paren{sempre com dó do Visconde} O lobo pode não comer o
Visconde, mas é bem capaz de espedaçá-lo. Não existe lobo mais malvado
que esse.

\siderepl{tia nastácia} Não se incomode, menina. O Visconde é de sabugo e foi
feito por estas mãos aqui. Se levar a breca faço outro ainda mais bonito
hoje mesmo. Vamos, ``seo'' Visconde. Que está esperando? Pule. Salve a
família.

\paren{O Visconde prepara-se para sair, mas antes disso vai espiar o
terreiro e vê a vaca mocha de Dona Benta mascando umas palhas ali por
perto.}

\siderepl{visconde} \paren{apavorado} Não posso ir. Ela está no caminho\ldots{}

\siderepl{narizinho} Que ela é essa, medroso?

\siderepl{visconde} A vaca mocha! De lobo, de dragão, de rinoceronte eu não
tenho medo, mas de vaca tenho e hei de ter sempre. Foi a mocha quem
comeu meu pai e minha mãe e todos os meus irmãos e parentes. Essa peste
de vaca não perdoa a um só sabugo. Assim que vê um colhe-o com aquele
linguão vermelho e o vai mascando com a maior sem-cerimônia. Não vou,
não vou e não vou.

\paren{O lobo continua a uivar e arranhar a porta. Dá um grande urro.
Capinha, muito pálida, vacila.}

\siderepl{narizinho} Acudam! Capinha está desmaiando!\ldots{}

\paren{Pedrinho corre para Capinha e a sustenta nos braços. Leva-a para o
colo de Dona Benta. Depois corre ao bodoque e, de cima da mesa, através
do vão da janela, prega umas pelotas na vaca. A vaca foge para o pasto.}

\siderepl{pedrinho} Para a mocha, bodoque! ``Zum!'' Acertei uma na anca.
``Zum!'' Outra na orelha. Lá vai ela fugindo. ``Zum!'' Mais uma na teta.
Esta valeu! Pronto, Visconde! O caminho está desimpedido. \paren{sempre
a espiar} Lá vai ele com muito medo, a olhar de todos os lados. É o
eterno medo que o Visconde sempre teve da vaca mocha porque ele é sabugo
e vaca não perdoa a sabugos. Come mesmo. Agora fez um rodeio para não
passar perto do galo. Medo que o galo coma os três grãos de milho que
ele ainda conserva no peito. Chegou\ldots{} Está berrando no ouvido de
Quindim\ldots{} Mas Quindim não dá pela coisa. O Visconde berra
inutilmente. Mudou de lugar. Foi berrar no outro ouvido. Nada! Agora
está sapateando em cima de Quindim, mas não há meio. Quindim não acorda.

\siderepl{narizinho} \paren{torcendo as mãos} Nossa Senhora! Que será de nós!

\siderepl{pedrinho} \paren{sempre a espiar} A vaca aparece lá longe e o
Visconde disparou. Vem vindo na volada. Tropeçou numa casca. Vem
chegando \paren{o menino recolhe-o}. E então, senhor mensageiro?

\siderepl{visconde} \paren{enxugando o rosto suado com as palhas do pescoço}
Impossível acordar aquele dorminhoco. Parece que morreu. Fiz tudo.
Beijei-lhe no ouvido. Sapateei em cima. Nada. Não acorda.

\siderepl{emília} É um estafermo este Senhor Visconde! Não presta nem para
acordar rinoceronte.

\paren{O lobo solta novo uivo de cólera e arranca mais uma tábua da
porta. Enfia pelo buraco o seu horrível focinho.}

\siderepl{tia nastácia} Nossa Senhora! É lobo mesmo, do ``legite!''.

\siderepl{dona benta} \paren{prestes a desmaiar também, pendendo a cabeça para o
encosto da cadeira} Legítimo, Nastácia\ldots{}

\siderepl{pedrinho e narizinho} \paren{pulando da mesa para o chão} E agora?

\paren{Emília corre à cozinha e volta com o vidro de pimenta em pó. Trepa
à mesa e salta para fora, dizendo antes de pular:}

\siderepl{emília} Esperem que eu arranjo tudo. Quero ver se o sono do Quindim
resiste a esta pimenta. Vou mostrar ao sarambé do Visconde como é que se
acorda rinoceronte.

\paren{Pedrinho volta ao seu posto de observação em cima da mesa. Tia
Nastácia faz cruzes no peito e reza em voz baixa. Narizinho vai para
junto de Dona Benta, que ainda tem ao colo a menina da Capinha
Vermelha.}

\siderepl{pedrinho} Lá vai a sirigaita muito lampeira com o vidro de pimenta.
Pimenta precisa ela. Vai correndo sem olhar para trás. Chegou junto ao
rinoceronte. Está abrindo o vidro de pimenta. Abriu. Despejou-o quase
inteiro nos olhos do pobre animal. Malvada! O rinoceronte fez uma
careta. Sacudiu a cabeça. Ergueu-se. Emília o está descompondo de
cachorro para baixo, como se ele tivesse culpa de dormir, com um sol
quente destes. Agora ela está apontando para o lobo --- está atiçando
Quindim contra o lobo\ldots{}

\siderepl{narizinho} \paren{correndo a espiar também} É uma danada a Emília!
Acordou o rinoceronte e tais coisas lhe disse que ele vem vindo que nem
uma bala de canhão. Já estou ficando com dó do lobo\ldots{}

\paren{O focinho do lobo desaparece do buraco da porta. Logo em seguida
ouve-se um grande berro de lobo espetado em chifre de rinoceronte.
Grande alegria na sala. Tia Nastácia suspira com alivio. Pedrinho não
esconde o seu desapontamento diante do sucesso da Emília.}

\siderepl{dona benta} Tragam água fria para eu acordar a esta menina.

\paren{Todos rodeiam a menina desmaiada no colo de Dona Benta. Narizinho
traz um copo d'água, e a borrifa no rosto de Capinha. Ela começa a abrir
os olhos, ainda tonta, como a sair dum pesadelo.}

\siderepl{capinha} O lobo já comeu Dona Benta?

\siderepl{tia nastácia} Comeu nada, menina. Pois não está vendo ela aí na sua
frente? Desta vez quem foi comido foi o lobo. O rinoceronte deu cabo
dele com uma chifrada na barriga.

\paren{Capinha suspira aliviada e esfrega os olhos. Volta a si
completamente e desce do colo de Dona Benta. Dona Benta também se ergue
da cadeira, esfregando as pernas ainda meio moles.}

\siderepl{dona benta} Que susto! O perigo desta vez foi grande. Vi que o lobo
entrava mesmo e me comia.

\siderepl{capinha} E quem salvou a situação?

\siderepl{dona benta} Quem mais senão a Emília? Está ficando mais sabida que
todos os outros. Foi lá e acordou o rinoceronte, coisa que nem com um
sapateado em cima dele o Visconde conseguiu. E sabe como? Despejando-lhe
pimenta-do-reino nos olhos. Lembranças assim, só mesmo da Emília.

\siderepl{capinha} \paren{suspirando} Ora graças que vou viver sossegada
de hoje em diante! Esse lobo malvado comeu vovó, e como o homem do
machado não o matou bem matado, ele sarou e vivia rondando minha casa
para me comer também. Tanto medo eu tinha dele que me conservava sempre
fechada lá dentro. Por isso nunca vinha aqui ao sítio, apesar dos
convites de Narizinho. Agora virei sempre.

\siderepl{narizinho} Mas como veio hoje?

\siderepl{capinha} É que espiei pela janela e não vi o lobo lá por perto; e
então criei coragem e saí para colher no campo uns malmequeres. E
naquilo me distraí e fui me afastando de casa. De repente, o lobo! Como
aqui ficava mais perto que minha casa, corri para aqui. Foi isso.

\siderepl{pedrinho} \paren{que fora ao quarto e voltara} Tome lá. Aí está o
seu cavalo, ciganinha.

\paren{Emília examina o cavalo e vê que estava sem uma das orelhas.}

\siderepl{emília} Dispenso. Presente sem orelha, eu dispenso.

\siderepl{tia nastácia} Tenho coisa muito melhor para você Emília. \paren{tira
do bolso do avental o pé da franga}. Uma coisa mesmo de museu --- um pé
de galinha de seis dedos, um verdadeiro ``felomeno''.

\siderepl{dona benta} Fenômeno, Nastácia.

\siderepl{tia nastácia} Não sei como se diz, mas que tem seis dedos, isso tem.
Veja, conte.

\siderepl{emília} \paren{examinando o pé de galinha com toda a atenção} Muito
bem. Vai para o meu museuzinho de curiosidades. Vai fazer\ldots{} Como é
que se diz, Dona Benta, quando uma coisa faz parelha com outra?

\siderepl{dona benta} Diz-se em francês fazer \textit{pendant}.

\siderepl{emília} É isso. Este pé de galinha vai fazer \textit{pendant} com outra
coisa que está lá.

\paren{Todos se voltaram para ela, curiosos.}

\siderepl{emília} \paren{mordendo os lábios, toda ironia} A inveja de Pedrinho\ldots{}

\paren{Pedrinho disfarça o desapontamento enquanto Emília o olha firme. O
Visconde leva a mão à boca para esconder uma risadinha espremida. Dona
Benta volta-se para Tia Nastácia.}

\siderepl{dona benta} Vê, Nastácia, até irônica está ficando\ldots{}

\siderepl{tia nastácia} Credo! (\textit{e benze-se})

\begin{center}
\textsc{fim}
\end{center}

%* \textit{O museu da Emília}, de \textit{Histórias diversas} (1947): Peça de teatro escrita por Monteiro Lobato para ser representada na Biblioteca Infantil Municipal de São Paulo, em 1938.


\chapter{Fontes}

Os 24 textos que compõem esta coletânea foram selecionados de dez livros de Monteiro Lobato:

\begin{enumerate}
\item
  ``Em férias'', capítulo 1 de \textit{O Saci} (1921).
\item
  ``A pílula falante'', de \textit{A menina do nariz arrebitado} (1921);
  \textit{Reinações de Narizinho} (1931).
\item
  ``O passarinho-ninho'', capítulo 3 de \textit{A reforma da natureza} (1941).
\item
  ``Reformas na Europa e nas pulgas'', capítulo 6 de \textit{A reforma da natureza} (1941).
\item
  ``No dia seguinte'', capítulo 8 de \textit{A reforma da natureza} (1941).
\item
  ``O livro comestível'', capítulo 9 de \textit{A reforma da natureza} (1941).
\item
  ``A menina do leite'', de \textit{Fábulas} (1922).
\item
  ``O carreiro e o papagaio'', de \textit{Fábulas} (1922).
\item
  ``Os dois burrinhos'', de \textit{Fábulas} (1922).
\item
  ``A raposa e as uvas'', de \textit{Fábulas} (1922).
\item
  ``O cão e o lobo'', de \textit{Fábulas} (1922).
\item
  ``O galo que logrou a raposa'', de \textit{Fábulas} (1922).
\item
  ``Os animais e a peste'', de \textit{Fábulas} (1922).
\item
  ``A floresta'', capítulo 10 de \textit{O} \textit{Saci} (1921).
\item
  ``Discussão'', capítulo 11 de \textit{O} \textit{Saci} (1921).
\item
  ``A cartinha do Polegar'', capítulo 1 de \textit{O Picapau Amarelo} (1939).
\item
  ``A resposta de Dona Benta'', capítulo 2 de \textit{O Picapau Amarelo} (1939).
\item
  ``O plano da Emília'', capítulo 3 de \textit{O Picapau Amarelo} (1939).
\item
  ``O Visconde e a Quimera'', capítulo 7 de \textit{O Picapau Amarelo} (1939).
\item
  ``A Fênix'', capítulo 4 de \textit{Os doze trabalhos de Hércules} (1944).
\item
  ``A Terra vista da Lua'', capítulo 8 de \textit{Viagem ao céu} (1932).
\item
  ``Discussões em Atenas'', capítulo 5 de \textit{O Minotauro} (1939).
\item
  ``Emília forma palavras'', capítulo 15 de \textit{Emília no País da Gramática} (1934).
\item
  \textit{O museu da Emília}, de \textit{Histórias diversas} (1947): Peça de
  teatro escrita por Monteiro Lobato para ser representada na Biblioteca
  Infantil Municipal de São Paulo, em 1938.
\end{enumerate}