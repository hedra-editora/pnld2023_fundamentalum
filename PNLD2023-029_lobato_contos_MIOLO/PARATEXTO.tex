\chapter{Vida e obra de Monteiro Lobato}

\section{Sobre o livro}
\textit{O plano da Emília e outros textos} contém uma seleção de 24 contos e capítulos extraídos das principais obras de Monteiro Lobato. Esses textos funcionam como narrativas independentes, com enredos e estilo  envolventes e temas atuais, desde a figura do saci e outros animais típicos das fábulas, até as conhecidas personagens do \textit{Sítio do Picapau Amarelo}, como Narizinho e Emília. Além dos temas brasileiros, o autor não deixou de investigar grandes mitos e narrativas da cultura ocidental, como é possível ler em \textit{O Minotauro} e \textit{Os doze trabalhos de Hércules}. Com esse panorama, o jovem leitor é introduzido no universo literário de Monteiro Lobato e tem a oportunidade de ler e fruir as principais narrativas desse escritor.

\section{Sobre o autor}

Fazendeiro, escritor, editor, empresário,defensor do petróleo nacional: 
a intensidade com que Monteiro Lobato experienciou as várias faces 
de sua vida transparece na vitalidade de
seus contos, frutos de sua sensibilidade, observação crítica,
conhecimentos literários e trabalho intelectual e artístico.

José Bento Renato Monteiro Lobato nasceu em Taubaté, São Paulo, a 18 de
abril de 1882, que ficou consagrado como Dia Nacional do Livro Infantil,
e faleceu em São Paulo, a 4 de julho de 1948.

Foi escritor de literatura para crianças e adolescentes, contista, jornalista, editor,
tradutor, pintor e fotógrafo. Cursou a Faculdade de Direito do Largo São Francisco, em São Paulo, por imposição
do avô, o Visconde de Tremembé. Escreveu para diversos jornais, sempre envolvido em polêmicas.  
Em 1918, estreou com o livro de contos \textit{Urupês}, que esgotou 30 mil
exemplares entre 1918 e 1925. Nesse mesmo ano, comprou a \textit{Revista do Brasil},
lançando as bases da indústria editorial no país, que jamais seria a mesma depois de sua atuação. 
Monteiro Lobato revolucionou o mercado livreiro, com diversas iniciativas. Em 1920, fundou a
editora Monteiro Lobato \& Cia, que publicou obras de Lima Barreto e Oswald de Andrade, entre muitos outros autores. No mesmo ano, lançou \textit{A menina do Narizinho Arrebitado}, primeira da série
de histórias com que Lobato tem formado gerações de leitores até hoje. 

\section{Sobre a organizadora}

A organizadora Ieda Lebensztayn é crítica literária, pesquisadora e ensaísta, preparadora e revisora de livros. Mestre em Teoria Literária e Literatura Comparada e doutora em Literatura Brasileira pela Universidade de São Paulo. Fez dois pós-doutorados: o primeiro no Instituto de Estudos Brasileiros, IEB-USP, e o segundo na Biblioteca Brasiliana Mindlin / Faculdade de Filosofia, Letras e Ciências Humanas, BBM/FFLCH-USP. 


\section{Sobre o gênero}

Como \textit{contar uma história}? Um livro pode ser contato como um conto, uma crônica,
uma novela, um romance romance ou uma fábula. Cada jeito de contar 
uma história ganha um nome diferente. 

O conto é uma narrativa curta. Como todas as coisas que são rápidas, o conto causa 
uma impressão muito forte na gente: as personagens principais se metem numa confusão 
e tem que resolver esse problema. Na maioria das vezes elas são criativas e inteligentes, 
às vezes não. Tem outra coisa: tudo depende também de quem está contando a história, 
o narrador. Será que o leitor sempre pode confiar nele? Será que o narrador não esconde 
alguns detalhes para deixar a história mais interessante? Além disso, o lugar e o 
momento em que as aventuras acontecem também podem ajudar ou complicar a v
ida das personagens. Finalmente: o conto é um texto curto, muito bem amarrado, 
que te prende, porque tem uma história envolvente e um final surpreendente.