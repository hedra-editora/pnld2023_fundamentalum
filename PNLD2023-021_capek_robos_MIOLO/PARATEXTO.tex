\chapter{Vida e obra de Karel Tchápek}

\section{Sobre o livro}

\textit{A Fábrica de Robôs} causou alarde quando foi encenada
pela primeira vez, em 1921. Divida em três atos, a peça fala de 
algo incomum à época: a profunda crise causada pelo avanço
científico-tecnológico, a qual põe em risco a espécie humana. 
Após ganharem vida, máquinas semelhantes a seres humanos passam a exercer
todas as atividades braçais, o que levou o autor a usar o termo ``robô''
--- palavra que em tcheco significa servidão e trabalho forçado. Embora
mais eficientes, as tais máquinas
desconhecem a criatividade e os sentimentos, o que acarreta
consequências tristes à humanidade. A peça traz a história de um cientista que descobre uma
fórmula capaz de dar vida a máquinas de aparência humana, gerando um
desequilíbrio radical e tornando a mão de obra
humana obsoleta. Essas criaturas artificiais, desprovidas de sentimentos
e criatividade, passam a exercer todas as atividades braçais, com
consequências negativas para os homens. A palavra \textit{robô}, cujo
significado em tcheco é \textit{servidão, trabalho forçado}, e que seria
incorporada em quase todas línguas, foi cunhada e usada pela primeira
vez na peça de Tchápek.

\section{Sobre o autor}

Karel Capek (1890--1938) foi um escritor e filósofo tcheco,
nascido em Malé Svatonovice, Boêmia, então parte do Império
Austro-Húngaro. Suas obras de ficção denunciaram os perigos do confronto
entre o homem e os avanços tecnológicos, os perigos que ameaçavam o
mundo moderno se este se deixasse levar pelos excessos do materialismo e
do mecanicismo. Estudou filosofia em diversas cidades europeias até se
estabelecer em Praga (1917), onde trabalhou como escritor e jornalista.
Na literatura, deve sua
popularidade sobretudo a suas obras de ficção, suas utopias satíricas e
filosóficas, traduzidas para muitos idiomas. Por alguns anos escreveu em
parceria com o irmão Josef, como por exemplo em \textit{Krakonosova zahrada}
(1918), coletânea de contos e narrativas de grande interesse.

Nessa obra, numa estória em que humanidade se achava ameaçada por uma
máquina de sua invenção, o \textit{robot}, cunhou a palavra que posteriormente
popularizou-se pelo mundo inteiro como nome de uma unidade cibernética.
Morreu em Praga e entre outras obras importantes ficaram a peça
dramática \textit{R.U.R.} (1920), \textit{Hordubal} (1933), \textit{Povetron} (1934) e
\textit{Obycejny zivot} (1934), as famosas sátiras \textit{Valka smloky} (1936) e a
peça realista \textit{Bilá nemoc} (1937), em que conclamava o povo à
solidariedade e à resistência contra o nazismo.

\section{Sobre o gênero}
As peças de teatro não são escritas, incialmente, para serem lidas. Elas servem de roteiro para a montagem da peça no palco. Diretor, atores, contra-regra, iluminadores, figurinistas, sonoplastas --- toda a equipe do grupo de teatro usa o texto como guia para criar o espetáculo ao qual o público vai assistir. Mas é claro que nós podemos comprar esse roteiro e lê-lo como se ele tivesse sido escrito para nosso desfrute pessoal. Muitos dramaturgos gostam mesmo de escrever suas peças pensando não apenas na montagem, mas também na experiência da leitura.  

Os textos das peças de teatro têm algumas características específicas. Tradicionalmente eles começam com uma lista completa das personagens que vão aparecer no palco, mesmo que elas não tenham nenhuma fala e sejam apenas figurantes. As peças são divididas em \textit{atos}, que servem como capítulos. Cada ato costuma ser dividido em \textit{cenas}, que são unidades menores. 

Muitas peças de teatro não têm narrador. Nelas as personagens interagem umas com as outras principalmente por meio de diálogos, nos quais se desenrolam os conflitos entre elas. É claro que não é apenas pela fala que as coisas acontecem: a expressão corporal dos atores também contribui muito para o andamento das ações. Para explicitá-la, o autor de uma peça usa as chamadas \textit{rubricas}, pequenos comentários entre os diálogos que servem de guia para a equipe e para os leitores, em que se explica o que a personagem está fazendo, em que lugar do palco ela está localizada etc. As rubricas também podem indicar mudanças de cenário, avanços no tempo e até intensidade da luz, de acordo com a necessidade dessas alterações para a narrativa.

Levando em consideração que a peça será apresentada ao grande público, a linguagem dos textos para teatro tende a se aproximar da fala cotidiana, mas, como a língua está sempre mudando, é muito comum que as trupes de teatro adaptem textos mais antigos para a fala do presente. Esse é um grande desafio dos autores de teatro: escrever um texto cuja força expressiva, associada à fala, se mantenha atual, apesar da passagem do tempo e de um certo envelhecimento de algumas expressões que caem em desuso.